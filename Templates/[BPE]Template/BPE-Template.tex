% 若编译失败,且生成 .synctex(busy) 辅助文件,可能有两个原因:
% 1. 需要插入的图片不存在:Ctrl + F 搜索 'figure' 将这些代码注释/删除掉即可
% 2. 路径/文件名含中文或空格:更改路径/文件名即可

% --------------------- 文章宏包及相关设置 --------------------- %
% >> ------------------ 文章宏包及相关设置 ------------------ << %
% 设定文章类型与编码格式
\documentclass[UTF8]{article}		

% 物理实验报告所需的其它宏包
\usepackage{ulem}   % \uline 下划线支持
\usepackage{circuitikz} % 电路图 tikz 支持
\usepackage{pdfpages}   % 用于导入 pdf 文件

% 本 .tex 专属的宏定义
    \def\V{\ \mathrm{V}}
    \def\uV{\ \mu\mathrm{V}}
    \def\mV{\ \mathrm{mV}}
    \def\K{\ \mathrm{K}}
    \def\kV{\ \mathrm{KV}}
    \def\KV{\ \mathrm{KV}}
    \def\MV{\ \mathrm{MV}}
    \def\uA{\ \mu\mathrm{A}}
    \def\mA{\ \mathrm{mA}}
    \def\A{\ \mathrm{A}}
    \def\kA{\ \mathrm{KA}}
    \def\KA{\ \mathrm{KA}}
    \def\MA{\ \mathrm{MA}}
    \def\O{\ \Omega}
    \def\mO{\ \Omega}
    \def\kO{\ \mathrm{K}\Omega}
    \def\KO{\ \mathrm{K}\Omega}
    \def\MO{\ \mathrm{M}\Omega}
    \def\Hz{\ \mathrm{Hz}}
    \def\uF{\ \mu\mathrm{F}}
    \def\mF{\ \mathrm{mF}}
    \def\F{\ \mathrm{F}}

% 自定义宏定义
    \def\N{\mathbb{N}}
    \def\F{\mathbb{F}}
    \def\Z{\mathbb{Z}}
    \def\Q{\mathbb{Q}}
    \def\R{\mathbb{R}}
    \def\C{\mathbb{C}}
    \def\T{\mathbb{T}}
    \def\S{\mathbb{S}}
    %\def\A{\mathbb{A}}
    \def\I{\mathscr{I}}
    \def\d{\mathrm{d}}
    \def\p{\partial}


% 导入基本宏包
    \usepackage[UTF8]{ctex}     % 设置文档为中文语言
    \usepackage{hyperref}  % 宏包:自动生成超链接 (此宏包与标题中的数学环境冲突)
    \hypersetup{
        colorlinks=true,    % false:边框链接 ; true:彩色链接
        citecolor={blue},    % 文献引用颜色
        linkcolor={blue},   % 目录 (我们在目录处单独设置),公式,图表,脚注等内部链接颜色
        urlcolor={orange},    % 网页 URL 链接颜色,包括 \href 中的 text
        % cyan 浅蓝色 
        % magenta 洋红色
        % yellow 黄色
        % black 黑色
        % white 白色
        % red 红色
        % green 绿色
        % blue 蓝色
        % gray 灰色
        % darkgray 深灰色
        % lightgray 浅灰色
        % brown 棕色
        % lime 石灰色
        % olive 橄榄色
        % orange 橙色
        % pink 粉红色
        % purple 紫色
        % teal 蓝绿色
        % violet 紫罗兰色
    }
    % \usepackage{docmute}    % 宏包:子文件导入时自动去除导言区,用于主/子文件的写作方式,\include{./51单片机笔记}即可。注:启用此宏包会导致.tex文件capacity受限。
    \usepackage{amsmath}    % 宏包:数学公式
    \usepackage{mathrsfs}   % 宏包:提供更多数学符号
    \usepackage{amssymb}    % 宏包:提供更多数学符号
    \usepackage{pifont}     % 宏包:提供了特殊符号和字体
    \usepackage{extarrows}  % 宏包:更多箭头符号 
    \usepackage{multicol}   % 宏包:支持多栏 

% 文章页面margin设置
    \usepackage[a4paper]{geometry}
        \geometry{top=0.75in}
        \geometry{bottom=0.75in}
        \geometry{left=0.75in}
        \geometry{right=0.75in}   % 设置上下左右页边距
        \geometry{marginparwidth=1.75cm}    % 设置边注距离(注释、标记等)

% 配置数学环境
    \usepackage{amsthm} % 宏包:数学环境配置
    % theorem-line 环境自定义
        \newtheoremstyle{MyLineTheoremStyle}% <name>
            {11pt}% <space above>
            {11pt}% <space below>
            {\kaishu}% <body font> 默认使用正文字体, \kaishu 为楷体
            {}% <indent amount>
            {\bfseries}% <theorem head font> 设置标题项为加粗
            {:\ \ }% <punctuation after theorem head>
            {.5em}% <space after theorem head>
            {\textbf{#1}\thmnumber{#2}\ \ (\,\textbf{#3}\,)}% 设置标题内容顺序
        \theoremstyle{MyLineTheoremStyle} % 应用自定义的定理样式
        \newtheorem{LineTheorem}{Theorem.\,}
    % theorem-block 环境自定义
        \newtheoremstyle{MyBlockTheoremStyle}% <name>
            {11pt}% <space above>
            {11pt}% <space below>
            {\kaishu}% <body font> 使用默认正文字体
            {}% <indent amount>
            {\bfseries}% <theorem head font> 设置标题项为加粗
            {:\\ \indent}% <punctuation after theorem head>
            {.5em}% <space after theorem head>
            {\textbf{#1}\thmnumber{#2}\ \ (\,\textbf{#3}\,)}% 设置标题内容顺序
        \theoremstyle{MyBlockTheoremStyle} % 应用自定义的定理样式
        \newtheorem{BlockTheorem}[LineTheorem]{Theorem.\,} % 使用 LineTheorem 的计数器
    % definition 环境自定义
        \newtheoremstyle{MySubsubsectionStyle}% <name>
            {11pt}% <space above>
            {11pt}% <space below>
            {}% <body font> 使用默认正文字体
            {}% <indent amount>
            {\bfseries}% <theorem head font> 设置标题项为加粗
            {:\\ \indent}% <punctuation after theorem head>
            {0pt}% <space after theorem head>
            {\textbf{#3}}% 设置标题内容顺序
        \theoremstyle{MySubsubsectionStyle} % 应用自定义的定理样式
        \newtheorem{definition}{}

%宏包:有色文本框(proof环境)及其设置
    \usepackage{xcolor}    %设置插入的文本框颜色
    \usepackage[strict]{changepage}     % 提供一个 adjustwidth 环境
    \usepackage{framed}     % 实现方框效果
        \definecolor{graybox_color}{rgb}{0.95,0.95,0.96} % 文本框颜色。修改此行中的 rgb 数值即可改变方框纹颜色,具体颜色的rgb数值可以在网站https://colordrop.io/ 中获得。(截止目前的尝试还没有成功过,感觉单位不一样)(找到喜欢的颜色,点击下方的小眼睛,找到rgb值,复制修改即可)
        \newenvironment{graybox}{%
        \def\FrameCommand{%
        \hspace{1pt}%
        {\color{gray}\vrule width 2pt}%
        {\color{graybox_color}\vrule width 4pt}%
        \colorbox{graybox_color}%
        }%
        \MakeFramed{\advance\hsize-\width\FrameRestore}%
        \noindent\hspace{-4.55pt}% disable indenting first paragraph
        \begin{adjustwidth}{}{7pt}%
        \vspace{2pt}\vspace{2pt}%
        }
        {%
        \vspace{2pt}\end{adjustwidth}\endMakeFramed%
        }

% 外源代码插入设置
    % matlab 代码插入设置
    \usepackage{matlab-prettifier}
        \lstset{style=Matlab-editor}    % 继承 matlab 代码高亮 , 此行不能删去
    \usepackage[most]{tcolorbox} % 引入tcolorbox包 
    \usepackage{listings} % 引入listings包
        \tcbuselibrary{listings, skins, breakable}
        \newfontfamily\codefont{Consolas} % 定义需要的 codefont 字体
        \lstdefinestyle{MatlabStyle_inc}{   % 插入代码的样式
            language=Matlab,
            basicstyle=\footnotesize\ttfamily\codefont,    % ttfamily 确保等宽 
            breakatwhitespace=false,
            breaklines=true,
            captionpos=b,
            keepspaces=true,
            numbers=left,
            numbersep=15pt,
            showspaces=false,
            showstringspaces=false,
            showtabs=false,
            tabsize=2,
            xleftmargin=15pt,   % 左边距
            %frame=single, % single 为包围式单线框
            frame=shadowbox,    % shadowbox 为带阴影包围式单线框效果
            %escapeinside=``,   % 允许在代码块中使用 LaTeX 命令 (此行无用)
            %frameround=tttt,    % tttt 表示四个角都是圆角
            framextopmargin=0pt,    % 边框上边距
            framexbottommargin=0pt, % 边框下边距
            framexleftmargin=5pt,   % 边框左边距
            framexrightmargin=5pt,  % 边框右边距
            rulesepcolor=\color{red!20!green!20!blue!20}, % 阴影框颜色设置
            %backgroundcolor=\color{blue!10}, % 背景颜色
        }
        \lstdefinestyle{MatlabStyle_src}{   % 插入代码的样式
            language=Matlab,
            basicstyle=\small\ttfamily\codefont,    % ttfamily 确保等宽 
            breakatwhitespace=false,
            breaklines=true,
            captionpos=b,
            keepspaces=true,
            numbers=left,
            numbersep=15pt,
            showspaces=false,
            showstringspaces=false,
            showtabs=false,
            tabsize=2,
        }
        \newtcblisting{matlablisting}{
            %arc=2pt,        % 圆角半径
            % 调整代码在 listing 中的位置以和引入文件时的格式相同
            top=0pt,
            bottom=0pt,
            left=-5pt,
            right=-5pt,
            listing only,   % 此句不能删去
            listing style=MatlabStyle_src,
            breakable,
            colback=white,   % 选一个合适的颜色
            colframe=black!0,   % 感叹号后跟不透明度 (为 0 时完全透明)
        }
        \lstset{
            style=MatlabStyle_inc,
        }

% table 支持
    \usepackage{booktabs}   % 宏包:三线表
    \usepackage{tabularray} % 宏包:表格排版
    \usepackage{longtable}  % 宏包:长表格

% figure 设置
    \usepackage{graphicx}  % 支持 jpg, png, eps, pdf 图片 
    \usepackage{svg}       % 支持 svg 图片
        \svgsetup{
            % 指向 inkscape.exe 的路径
            inkscapeexe = C:/aa_MySame/inkscape/bin/inkscape.exe, 
            % 一定程度上修复导入后图片文字溢出几何图形的问题
            inkscapelatex = false                 
        }
    \usepackage{subcaption} % 用于子图和小图注  

% 图表进阶设置
    \usepackage{caption}    % 图注、表注
        \captionsetup[figure]{name=图}  
        \captionsetup[table]{name=表}
        \captionsetup{
            labelfont=bf, % 设置标签为粗体
            textfont=bf,  % 设置文本为粗体
            font=small  
        }
    \usepackage{float}     % 图表位置浮动设置 
    \usepackage{etoolbox} % 用于保证图注表注的数学字符为粗体
        \AtBeginEnvironment{figure}{\boldmath} % 图注中的数学字符为粗体
        \AtBeginEnvironment{table}{\boldmath}  % 表注中的数学字符为粗体
        \AtBeginEnvironment{tabular}{\unboldmath}   % 保证表格中的数学字符不受额外影响

% 圆圈序号自定义
    \newcommand*\circled[1]{\tikz[baseline=(char.base)]{\node[shape=circle,draw,inner sep=0.8pt, line width = 0.03em] (char) {\bfseries #1};}}   % TikZ solution

% 列表设置
    \usepackage{enumitem}   % 宏包:列表环境设置
        \setlist[enumerate]{
            label=(\arabic*) ,   % 设置序号样式为加粗的 (1) (2) (3)
            ref=\arabic*, % 如果需要引用列表项,这将决定引用格式(这里仍然使用数字)
            itemsep=0pt, parsep=0pt, topsep=0pt, partopsep=0pt, leftmargin=3.5em} 
        \setlist[itemize]{itemsep=0pt, parsep=0pt, topsep=0pt, partopsep=0pt, leftmargin=3.5em}
        \newlist{circledenum}{enumerate}{1} % 创建一个新的枚举环境  
        \setlist[circledenum,1]{  
            label=\protect\circled{\arabic*}, % 使用 \arabic* 来获取当前枚举计数器的值,并用 \circled 包装它  
            ref=\arabic*, % 如果需要引用列表项,这将决定引用格式(这里仍然使用数字)
            itemsep=0pt, parsep=0pt, topsep=0pt, partopsep=0pt, leftmargin=3.5em
        }  

% 其它设置
    % 脚注设置
        \renewcommand\thefootnote{\ding{\numexpr171+\value{footnote}}}
    % 参考文献引用设置
        \bibliographystyle{unsrt}   % 设置参考文献引用格式为unsrt
        \newcommand{\upcite}[1]{\textsuperscript{\cite{#1}}}     % 自定义上角标式引用
    % 文章序言设置
        \newcommand{\cnabstractname}{序言}
        \newenvironment{cnabstract}{%
            \par\Large
            \noindent\mbox{}\hfill{\bfseries \cnabstractname}\hfill\mbox{}\par
            \vskip 2.5ex
            }{\par\vskip 2.5ex}

% 文章默认字体设置
    \usepackage{fontspec}   % 宏包:字体设置
        \setmainfont{SimSun}    % 设置中文字体为宋体字体
        \setCJKmainfont[AutoFakeBold=3]{SimSun} % 设置加粗字体为 SimSun 族,AutoFakeBold 可以调整字体粗细
        \setmainfont{Times New Roman} % 设置英文字体为Times New Roman

% 各级标题自定义设置
    \usepackage{titlesec}   
        % section标题自定义设置 
        \titleformat{\section}[hang]{\normalfont\Large\bfseries\boldmath}{\thesection}{8pt}{}
        % subsection 标题自定义设置
        \titleformat{\subsection}[hang]{\normalfont\large\bfseries\boldmath}{\thesubsection}{8pt}{}
        \titlespacing*{\subsection}{0pt}{6pt}{3pt} % 控制上下间距


% --------------------- 文章宏包及相关设置 --------------------- %
% >> ------------------ 文章宏包及相关设置 ------------------ << %


% ------------------------ 文章信息区 ------------------------ %
% ------------------------ 文章信息区 ------------------------ %
% 页眉页脚设置
\usepackage{fancyhdr}   %宏包:页眉页脚设置
    \pagestyle{fancy}
    \fancyhf{}
    \cfoot{\thepage}
    \renewcommand\headrulewidth{1pt}
    \renewcommand\footrulewidth{0pt}
    \rhead{\bfseries \large {\color{red} 分组序号: 2-05}}    
    \chead{《基础物理实验》实验报告,\ 小明,\ 1111K8009000099}
    \lhead{Ex.8 磁滞回线 (2099.99.99)}
\begin{document}
\begin{center}\large
    \vspace*{-0.8cm}
    \noindent{\huge\bfseries《\ \ 基\ \ 础\ \ 物\ \ 理\ \ 实\ \ 验\ \ \ 》\ \ 实\ \ 验\ \ 报\ \ 告 }
    \\\vspace{0.1cm}
    \noindent{
    {\bfseries 
    实验名称:\uline{\hspace{1.05cm} 实验名称实验名称实验名称 \hspace{1.05cm}}
    }\hspace{0.4cm}
    指导教师:\uline{\hspace{0.3cm}朱中柱\ \ zhuzz@ihep.ac.cn\hspace{0.3cm}}
    }
    \\\vspace{0.1cm}
    \noindent
    {
    姓名:\uline{\,\,\,小明\,\,\,}\hspace{0.2cm}
    学号:\uline{\,\,\,{1111K8009000099}\,\,\,}\hspace{0.2cm}
    班级/专业:\uline{\,\,\,{2999/专业名字}\,\,\,}\hspace{0.2cm}
    分组序号:\uline{\,\,\,{2-05}\,\,\,}
    }
    \\\vspace{0.1cm}
    \noindent{
    实验日期:\uline{\,\,{2099.99.99}\,\,}\hspace{0.2cm}
    实验地点:\uline{\,\,\,教学楼{999}\,\,\,}\hspace{0.2cm}
    是否调课/补课:\uline{\hspace{0.26cm}否 \hspace{0.26cm}}\hspace{0.2cm}
    成绩:\uline{\hspace{2cm}}
    }
\end{center}
\vspace{-0.4cm}
\noindent\rule{\textwidth}{0.075em}   % 分割线
\vspace{-1.0cm}
% ------------------------ 文章信息区 ------------------------ %
% ------------------------ 文章信息区 ------------------------ %


% 目录
\setcounter{tocdepth}{3}  % 目录深度为 2(不显示 subsubsection)
\noindent\tableofcontents\thispagestyle{fancy}   % 显示页码、页眉等

% 控制目录不换页
%\vspace{1cm}
%\setcounter{tocdepth}{2}  % 目录深度为 2(不显示 subsubsection)
%\noindent\begin{minipage}{\textwidth}
%\tableofcontents\thispagestyle{fancy}   % 显示页码、页眉等   
%\end{minipage}  

\newpage
\rhead{\bfseries 分组序号: 2-05}



% >> --------------------- 下面是正文内容 --------------------- << %
% ------------------------ 下面是正文内容 ------------------------ %
% ------------------------ 下面是正文内容 ------------------------ %
% ------------------------ 下面是正文内容 ------------------------ %
% ------------------------ 下面是正文内容 ------------------------ %
% >> --------------------- 下面是正文内容 --------------------- << %



\section{实验目的}

\begin{enumerate}
\item 掌握利用示波器测量铁磁材料动态磁滞回线的方法;
\item 掌握利用霍尔传感器测量铁磁材料(准)静态磁滞回线的方法;
\item 了解铁磁性材料的磁化特性;
\item 了解磁滞、磁滞回线和磁化曲线的概念,加深对饱和磁化强度、剩余磁化强度、矫顽力等物理量的理解。
\end{enumerate}


\section{实验仪器}

\subsection{DH4516 磁特性综合测量实验仪等}

DH4516 磁特性综合测量实验仪(包括正弦波信号源,待测样品及绕组,积分电路所用的电阻和电容)、双通道示波器、直流电源、电感、数字多用表。

其中,磁特性综合测量实验仪的主要技术参数如下表所示:

\begin{table}[H]\centering
    %\renewcommand{\arraystretch}{1.5} % 调整行间距为 1.5 倍
    %\setlength{\tabcolsep}{1.5mm} % 调整列间距
    \caption{磁特性综合测量实验仪主要技术参数}
    \label{磁特性综合测量实验仪主要技术参数}
\begin{tabular}{cccccccccc}\toprule
    样品 & 磁滞损耗 & 平均磁路长度 $l$ & 截面面积 $S$ & 线圈匝数 $N$  \\
    \midrule
    样品 1(锰锌铁氧体)  & 较小 &  0.130 m & 1.24 $\times 10^{-4}\  \mathrm{m^2}$ & $N_1 = N_2 = N_3 = 150$\\
    样品 2(EI 型硅钢片) & 较大 &  0.075 m & 1.20 $\times 10^{-4}\  \mathrm{m^2}$ & $N_1 = N_2 = N_3 = 150$\\
    \bottomrule
\end{tabular}
\end{table}

\begin{figure}[H]\centering
\begin{subfigure}[b]{0.48\columnwidth}\centering
    这里可以放实验仪器的图片这里可以放实验仪器的图片这里可以放实验仪器的图片这里可以放实验仪器的图片这里可以放实验仪器的图片这里可以放实验仪器的图片这里可以放实验仪器的图片这里可以放实验仪器的图片这里可以放实验仪器的图片这里可以放实验仪器的图片这里可以放实验仪器的图片这里可以放实验仪器的图片这里可以放实验仪器的图片
    %\includegraphics[height=170pt]{assets/DH4516 磁特性综合测量实验仪.jpg}
    \caption{DH4516 磁特性综合测量实验仪}
\end{subfigure}\hfill
\begin{subfigure}[b]{0.48\columnwidth}\centering
    这里可以放实验仪器的图片这里可以放实验仪器的图片这里可以放实验仪器的图片这里可以放实验仪器的图片这里可以放实验仪器的图片这里可以放实验仪器的图片这里可以放实验仪器的图片这里可以放实验仪器的图片这里可以放实验仪器的图片这里可以放实验仪器的图片这里可以放实验仪器的图片这里可以放实验仪器的图片这里可以放实验仪器的图片
    %\includegraphics[height=170pt]{assets/示波器和可编程电源.jpg}
    \caption{双通道示波器和可编程电源}
\end{subfigure}
\caption{第一部分实验的主要实验仪器}
\end{figure}

另外,信号源的频率 $f$ 在 $ 20\ \mathrm{Hz} \sim 200\ \mathrm{Hz} $ 间可调;可调标准电阻 $ R_1,R_2 $ 均为线性无感交流电阻,$ R_1 $ 的调节范围为 $ 0.1 \ \Omega \sim 11\ \Omega $,$ R_2 $ 的调节范围为 $ 1 \kO \sim 100 \kO $;标准电容 $C$ 在 $ 0.1 \uF \sim 11 \uF $ 间可调。

\subsection{FD-BH-I 霍尔传感器磁滞回线和磁化曲线测定仪等}

FD-BH-I 霍尔传感器磁滞回线和磁化曲线测定仪(包括数字式特斯拉计、恒流源、磁性材料样品、磁化线圈、双刀双掷开关、霍耳探头移动架、双叉头连接线、箱式实验平台)。

其主要技术指标如下:
\begin{enumerate}
\item 数字式特斯拉计: 四位半LED 显示,量程 $0 \sim 2.000 $ T;分辨率 $0.1$ mT,带霍耳探头;
\item 恒流源:四位半 LED 显示,可调恒定电流 $0\sim 600.0$ mA;
\item 磁性材料样品:条状矩形结构,截面长 $2.00$ cm;宽 $2.00$ cm;间隙宽(隙隔) $l_g = 2.00$ mm;平均磁路长度 $l =0.240$ m(样品与固定螺丝为同种材料);
\item 磁化线圈总匝数 $N=2000$。
\end{enumerate}

这里是实验仪器这里是实验仪器这里是实验仪器这里是实验仪器这里是实验仪器这里是实验仪器这里是实验仪器这里是实验仪器这里是实验仪器这里是实验仪器这里是实验仪器这里是实验仪器这里是实验仪器这里是实验仪器这里是实验仪器这里是实验仪器这里是实验仪器这里是实验仪器这里是实验仪器这里是实验仪器这里是实验仪器这里是实验仪器这里是实验仪器这里是实验仪器

这里是实验仪器这里是实验仪器这里是实验仪器这里是实验仪器这里是实验仪器这里是实验仪器这里是实验仪器这里是实验仪器这里是实验仪器这里是实验仪器这里是实验仪器这里是实验仪器这里是实验仪器这里是实验仪器这里是实验仪器这里是实验仪器这里是实验仪器这里是实验仪器

\begin{figure}[H]\centering
    这里可以放实验仪器的图片这里可以放实验仪器的图片这里可以放实验仪器的图片这里可以放实验仪器的图片这里可以放实验仪器的图片这里可以放实验仪器的图片
%\includegraphics[width=0.4\columnwidth]{}
\caption{这里可以放实验仪器的图片}\label{示例图片}
\end{figure}

\section{实验原理}
\subsection{铁磁材料的磁化特性}
把物体放在外磁场$ H $中,物体就会被磁化,在其内部产生磁场。设其内部磁化强度为$ M $,磁感应强度为$ B $,即可定义磁化率 $ \chi_m $ 和相对磁导率 $ \mu_r $: 
\begin{equation}
    \chi_m=\frac MH,\quad \mu_r=\frac{B}{\mu_0 H}
\end{equation}

其中$ \mu_0=4\pi\times10^{-7}\,\mathrm{N\cdot A^{-2}} $是真空磁导率。又由于 $ B=\mu_0(M+H) $,所以 $ \mu_r=1+\chi_m $。物质的磁性按磁化率可分为抗磁性($\chi_m < 0$)、顺磁性($\chi_m > 0$ 且较小)和铁磁性($\chi_m >0$ 且较大)三种,其中铁磁性物质的磁化率通常大于 1,远大于前两类物质。

这里是实验原理,这里是实验原理,这里是实验原理,这里是实验原理,这里是实验原理,这里是实验原理,这里是实验原理,这里是实验原理,这里是实验原理,这里是实验原理,这里是实验原理,这里是实验原理,这里是实验原理,这里是实验原理,这里是实验原理,这里是实验原理,这里是实验原理,这里是实验原理,这里是实验原理,这里是实验原理,这里是实验原理,这里是实验原理,这里是实验原理,这里是实验原理,这里是实验原理,这里是实验原理,这里是实验原理,这里是实验原理,这里是实验原理,这里是实验原理,这里是实验原理。

除磁导率高这一特点外,铁磁材料还具有特殊的磁化规律。对一个处于磁中性状态(即完全退磁,外磁场 $H=0$ 时有 $B = 0$)的铁磁材料加上由小变大的磁场$ H $进行磁化时,磁感应强度$ B $随$ H $的变化曲线大致分为三个阶段:(1)可逆磁化阶段;(2)不可逆化阶段;(3)饱和磁化阶段。

如果磁场 $H$ 在某个范围 $ [-H_0,H_0] $ 间作循环变化,那么$ B $也会作循环变化,从而$ B-H $图像成为一个闭合的回线,称为磁滞回线。当 $H_0$ 比较大时,回线会有一段明显的饱和区(平稳区),此时得到的回线称为饱和磁滞回线,如图 \ref{铁磁材料的动态磁滞回线和动态磁化曲线示意图} 所示,$ H_S$ 和 $B_S $分别称为饱和磁场强度与饱和磁感应强度。

这里是实验原理,这里是实验原理,这里是实验原理,这里是实验原理,这里是实验原理,这里是实验原理,这里是实验原理,这里是实验原理,这里是实验原理,这里是实验原理,这里是实验原理,这里是实验原理,这里是实验原理,这里是实验原理,这里是实验原理,这里是实验原理,这里是实验原理,这里是实验原理,这里是实验原理,这里是实验原理,这里是实验原理,这里是实验原理,这里是实验原理,这里是实验原理,这里是实验原理,这里是实验原理,这里是实验原理,这里是实验原理,这里是实验原理,这里是实验原理,这里是实验原理。


这里是实验原理,这里是实验原理,这里是实验原理,这里是实验原理,这里是实验原理,这里是实验原理,这里是实验原理,这里是实验原理,这里是实验原理,这里是实验原理,这里是实验原理,这里是实验原理,这里是实验原理,这里是实验原理,这里是实验原理,这里是实验原理,这里是实验原理,这里是实验原理,这里是实验原理,这里是实验原理,这里是实验原理。


\subsection{动态磁滞回线的测量}
测量动态磁滞回线的原理电路如图 \ref{用示波器测量动态磁滞回线电路图} 所示。环形铁芯上绕有三组线圈。线圈 1 为交流励磁线圈,接交流正弦信号源;线圈 2 为感应线圈,接 RC 积分电路;线圈 3 为直流励磁线圈,用于在测有直流偏置磁场下的可逆磁导率时接直流电源。将$ u_{R_1} $和$ u_C $从示波器两通道输入,在示波器 X-Y 显示模式下即可看到动态磁滞回线。

这里是实验原理,这里是实验原理,这里是实验原理,这里是实验原理,这里是实验原理,这里是实验原理,这里是实验原理,这里是实验原理,这里是实验原理,这里是实验原理,这里是实验原理,这里是实验原理,这里是实验原理,这里是实验原理,这里是实验原理,这里是实验原理,这里是实验原理,这里是实验原理,这里是实验原理,这里是实验原理,这里是实验原理。

\begin{gather}
H = \frac{N_1}{l}i_1 = \frac{N_1}{lR_1}\cdot u_{R_1} \Longrightarrow \boxed{
    H = \frac{N_1}{lR_1}\cdot u_{R_1} \propto u_{R_1}
} 
\\ 
u_2 = - N_2 \frac{\mathrm{d} \Phi }{\mathrm{d} t } = - N_2S \frac{\mathrm{d} B }{\mathrm{d} t } 
\\ 
u_C = \frac{Q}{C} = u_C|_{t = 0} + \frac{1}{CR_2} \int_{0}^{t} u_{R_2}\mathrm{d}t \approx \frac{1}{CR_2} \int_{0}^{t} u_{R_2}\mathrm{d}t \quad  (R_2 C \gg T) 
\\
\Longrightarrow \boxed{
    B = \frac{R_2 C}{N_2 S}\cdot u_C \propto u_C
}
\end{gather}

因此,只需利用双通道示波器对 $u_{R_1}$ 和 $u_C$ 进行测量,便和等价地得到 $H$ 和 $B$,进而绘制出 $H$-$B$ 图像,得到动态磁化曲线或磁滞回线。

\begin{figure}[H]\centering
\begin{subfigure}[b]{0.48\columnwidth}\centering
    这里是图片这里是图片这里是图片这里是图片这里是图片这里是图片这里是图片这里是图片这里是图片这里是图片这里是图片这里是图片这里是图片这里是图片
    %\includegraphics[height=120pt]{}
    \caption{取一个好听的名字}
\end{subfigure}\hfill
\begin{subfigure}[b]{0.48\columnwidth}\centering
    这里是图片这里是图片这里是图片这里是图片这里是图片这里是图片这里是图片这里是图片这里是图片这里是图片这里是图片这里是图片这里是图片这里是图片这里是图片这里是图片
    %\includegraphics[height=120pt]{}
    \caption{再取一个好听的名字}
\end{subfigure}
\caption{取一个更好听的名字}
\end{figure}



\subsection{(准)静态磁化曲线和(准)静态磁滞回线的测量}

这里是实验原理,这里是实验原理,这里是实验原理,这里是实验原理,这里是实验原理,这里是实验原理,这里是实验原理,这里是实验原理,这里是实验原理,这里是实验原理,这里是实验原理,这里是实验原理,这里是实验原理,这里是实验原理,这里是实验原理,这里是实验原理,这里是实验原理,这里是实验原理,这里是实验原理,这里是实验原理,这里是实验原理。


\section{实验内容与步骤}

\subsection{第一部分}

\subsubsection{实验一:观测样品 1(铁氧体)的饱和动态磁滞回线}

这里是实验步骤

\begin{enumerate}
\item 这里是实验步骤,这里是实验步骤,这里是实验步骤,这里是实验步骤,这里是实验步骤,这里是实验步骤,这里是实验步骤,这里是实验步骤,这里是实验步骤,这里是实验步骤,这里是实验步骤,这里是实验步骤,这里是实验步骤,这里是实验步骤,这里是实验步骤,这里是实验步骤,这里是实验步骤,这里是实验步骤。
\item 这里是实验步骤,这里是实验步骤,这里是实验步骤,这里是实验步骤,这里是实验步骤,这里是实验步骤,这里是实验步骤,这里是实验步骤,这里是实验步骤,这里是实验步骤,这里是实验步骤,这里是实验步骤,这里是实验步骤,这里是实验步骤,这里是实验步骤,这里是实验步骤,这里是实验步骤,这里是实验步骤。
\item 这里是实验步骤,这里是实验步骤,这里是实验步骤,这里是实验步骤,这里是实验步骤,这里是实验步骤,这里是实验步骤,这里是实验步骤,这里是实验步骤,这里是实验步骤,这里是实验步骤,这里是实验步骤,这里是实验步骤,这里是实验步骤,这里是实验步骤,这里是实验步骤,这里是实验步骤,这里是实验步骤。
\end{enumerate}

\begin{figure}[H]\centering
    这里是图片,这里是图片,这里是图片,这里是图片,这里是图片,这里是图片,这里是图片。
%\includegraphics[width=0.6\columnwidth]{assets/1.1/第一部分实验一接线图.jpg}
\caption{第一部分实验一接线图}\label{第一部分实验一接线图}
\end{figure}

\subsubsection{实验二:测量样品 1(铁氧体)的动态磁化曲线}

这里是实验步骤

\begin{enumerate}
\item 这里是实验步骤,这里是实验步骤,这里是实验步骤,这里是实验步骤,这里是实验步骤,这里是实验步骤,这里是实验步骤,这里是实验步骤,这里是实验步骤,这里是实验步骤,这里是实验步骤,这里是实验步骤,这里是实验步骤,这里是实验步骤,这里是实验步骤,这里是实验步骤,这里是实验步骤,这里是实验步骤。
\item 这里是实验步骤,这里是实验步骤,这里是实验步骤,这里是实验步骤,这里是实验步骤,这里是实验步骤,这里是实验步骤,这里是实验步骤,这里是实验步骤,这里是实验步骤,这里是实验步骤,这里是实验步骤,这里是实验步骤,这里是实验步骤,这里是实验步骤,这里是实验步骤,这里是实验步骤,这里是实验步骤。
\item 这里是实验步骤,这里是实验步骤,这里是实验步骤,这里是实验步骤,这里是实验步骤,这里是实验步骤,这里是实验步骤,这里是实验步骤,这里是实验步骤,这里是实验步骤,这里是实验步骤,这里是实验步骤,这里是实验步骤,这里是实验步骤,这里是实验步骤,这里是实验步骤,这里是实验步骤,这里是实验步骤。
\end{enumerate}


\subsubsection{实验三:观察不同频率下样品 2(硅钢)的动态磁滞回线}

这里是实验步骤,这里是实验步骤,这里是实验步骤,这里是实验步骤,这里是实验步骤,这里是实验步骤,这里是实验步骤,这里是实验步骤,这里是实验步骤,这里是实验步骤,这里是实验步骤,这里是实验步骤,这里是实验步骤,这里是实验步骤,这里是实验步骤,这里是实验步骤,这里是实验步骤,这里是实验步骤。


\subsubsection{实验四:测量样品 1(铁氧体)在不同直流偏置磁场下的可逆磁导率}

这里是实验步骤,这里是实验步骤,这里是实验步骤,这里是实验步骤,这里是实验步骤,这里是实验步骤,这里是实验步骤,这里是实验步骤,这里是实验步骤,这里是实验步骤,这里是实验步骤,这里是实验步骤,这里是实验步骤,这里是实验步骤,这里是实验步骤,这里是实验步骤,这里是实验步骤,这里是实验步骤。



\subsection{第二部分}

\subsubsection{实验一:测量模具钢的(准)静态起始磁化曲线}

\subsubsection{实验二:测量模具钢的(准)静态磁滞回线}


\newpage
\section{实验结果与数据处理}


下面是数据处理需要用到的公式,在此一并列出:
\begin{align}
\boxed{
\begin{aligned}
    \text{第一部分中 $H$ 和 $B$ 的测量原理:}& H = \frac{N_1}{l_{1, k}R_1}\cdot u_{R_1},\quad B = \frac{R_2 C}{N_2 S_k}\cdot u_C
    \\
    \text{直流偏置下的测量原理:}&  H = \frac{N_3}{l_{1, k}}\cdot I,\quad \mu_R=\lim_{\Delta H\to 0}\frac{\Delta B}{\mu_0\Delta H}
    \\
    \text{第二部分中 $H$ 和 $H_{\text{re}}$ 的测量原理:}& H = \frac{N}{l_2}\cdot I,\quad H_{\text{re}} = \frac{N}{l_2}\cdot I - \frac{l_g}{\mu_0 l_2}\cdot B
\end{aligned}
}
\end{align}
实验中可能需要的常量如下所示:
\begin{gather}
    \mu_0 = 4\pi \times 10^{-7}\  \mathrm{H\cdot m},\ \ 
    l_{1, 1} = 0.13 \ \mathrm{m},\ \ 
    l_{1, 2} = 0.075 \ \mathrm{m}
    \\
    S_1 = 1.24 \times 10^{-4}\ \mathrm{m^2} ,\ \ 
    S_2 = 1.20 \times 10^{-4}\ \mathrm{m^2} ,\ \ 
    N_k = 150
    \\ 
    l_2 = 0.240 \ \mathrm{m},\ \ l_g = 2 \times 10^{-3} \ \mathrm{m},\ \ N = 2000
\end{gather}

对于每个小实验,我们会将对应的参数列出,给出具体的数据换算公式,然后代入计算。

\subsection{第一部分}
\subsubsection{实验一:观测样品 1(铁氧体)的饱和动态磁滞回线}\label{铁氧体}

\noindent \textbf{(1)} 本节我们使用样品 1,参数 $f = 100 \ \mathrm{Hz}$,$R_1 = 2.0 \ \Omega$,$R_2 = 50.0 \kO$,$C = 10 \uF$,于是有换算公式 (\ref{1.1换算公式})。我们共测得 13 个数据点,原始电压测量结果见表 \ref{1.1电压},换算结果见表 \ref{1.1换算后}。

数据处理数据处理数据处理数据处理数据处理数据处理数据处理数据处理数据处理数据处理数据处理数据处理数据处理数据处理数据处理数据处理数据处理数据处理数据处理数据处理数据处理数据处理数据处理数据处理数据处理。


\begin{equation}\label{1.1换算公式}
    y = x^2,\quad y = x^2,\quad y = x^2,\quad y = x^2,\quad y = x^2,\quad y = x^2,\quad y = x^2
\end{equation}

数据处理数据处理数据处理数据处理数据处理数据处理数据处理数据处理数据处理数据处理数据处理数据处理数据处理数据处理数据处理数据处理数据处理数据处理数据处理数据处理数据处理数据处理数据处理数据处理数据处理。

\begin{center}\noindent\begin{minipage}{0.35\columnwidth}
\begin{table}[H]\centering
        %\renewcommand{\arraystretch}{1.5} % 调整行间距为 1.5 倍
        %\setlength{\tabcolsep}{1.5mm} % 调整列间距
        \caption{原始电压数据点}
        \label{1.1电压}
    \begin{tabular}{cccccccccc}\toprule
$u_{R_1}$ (mV) & $u_{C, 1}$ (mV) & $u_{C, 2}$ (mV)  \\
\midrule
0.001 &-0.002 &-0.003 \\
0.001 &-0.002 &-0.003 \\
0.001 &-0.002 &-0.003 \\
0.001 &-0.002 &-0.003 \\
0.001 &-0.002 &-0.003 \\
0.001 & 0.002 &-0.003 \\
0.001 & 0.002 &-0.003 \\
0.001 & 0.002 & 0.003 \\
0.001 & 0.002 & 0.003 \\
0.001 & 0.002 & 0.003 \\
0.001 & 0.002 & 0.003 \\
0.001 & 0.002 & 0.003 \\
0.001 & 0.002 & 0.003 \\
        \bottomrule
    \end{tabular}
\end{table}
\end{minipage}\begin{minipage}{0.35\columnwidth}
\begin{table}[H]\centering
    %\renewcommand{\arraystretch}{1.5} % 调整行间距为 1.5 倍
    %\setlength{\tabcolsep}{1.5mm} % 调整列间距
    \caption{换算后的磁滞回线数据点}
    \label{1.1换算后}
    \begin{tabular}{ccc}\toprule
$H$ $\mathrm{(A\cdot m^{-1})}$ & $B_1$ (T) & $B_2$ (T)  \\
\midrule
0.001 &-0.002 &-0.003 \\
0.001 &-0.002 &-0.003 \\
0.001 &-0.002 &-0.003 \\
0.001 &-0.002 &-0.003 \\
0.001 &-0.002 &-0.003 \\
0.001 & 0.002 &-0.003 \\
0.001 & 0.002 &-0.003 \\
0.001 & 0.002 & 0.003 \\
0.001 & 0.002 & 0.003 \\
0.001 & 0.002 & 0.003 \\
0.001 & 0.002 & 0.003 \\
0.001 & 0.002 & 0.003 \\
0.001 & 0.002 & 0.003 \\
\bottomrule
    \end{tabular}
\end{table}
\end{minipage}
\begin{minipage}{0.24\columnwidth}
\begin{figure}[H]\centering
\includegraphics[width=\columnwidth]{assets/1.1/(1)/下半支.png}
\caption{拟合结果与优度}\label{1.1拟合优度}
\end{figure}
\end{minipage}
\end{center}


为了得到较准确的 $B_r$ 和矫顽力 $H_c$,我们利用 Matlab 软件对数据进行拟合,定义拟合函数为:
\begin{equation}
y = f(x) = a \arctan (b x + c) + d
\end{equation}
其中 $a, b, c, d$ 为待定常数。据此拟合磁滞回线的半支,拟合两次分别得到上下半支。拟合优度如图 \ref{1.1拟合优度} 所示。

数据处理数据处理数据处理数据处理数据处理数据处理数据处理数据处理数据处理数据处理数据处理数据处理数据处理数据处理数据处理数据处理数据处理数据处理数据处理数据处理数据处理数据处理数据处理数据处理数据处理。


依据原始数据和拟合结果,作出动态磁滞回线,如图 \ref{1.1图} 所示\footnote{Matlab 源码见附录 \ref{1.1源码}}。实验时的实际图像如图 \ref{1.1照片} 所示,结合优度参数和拟合图像,可以知道拟合效果极好,于是由此拟合结果可得 $B_r$ 和 $H_c$ :
\begin{equation}
    B_r = 99999999 \ \mathrm{T}\quad ,\quad H_c = 999999999999999 \ \mathrm{A\cdot m^{-1}}\quad
\end{equation}

\begin{figure}[H]\centering
\begin{subfigure}[b]{0.5\columnwidth}\centering
    \includegraphics[height=190pt]{assets/1.1/(1)/2024-10-23_01-06-12.pdf}
    \caption{饱和磁滞回线}
\end{subfigure}\hfill
\begin{subfigure}[b]{0.5\columnwidth}\centering
    \includegraphics[height=190pt]{assets/1.1/(1)/2024-10-23_01-06-14.pdf}
    \caption{局部放大图}
\end{subfigure}
\caption{磁感应强度 $B$ 随外磁场 $H$ 的变化情况}
\label{1.1图}
\end{figure}
\begin{figure}[H]\centering
\begin{subfigure}[b]{0.5\columnwidth}\centering
    \includegraphics[height=170pt]{assets/1.1/(1)/CamScanner 10-22-2024 18.26_06.jpg}
    \caption{饱和磁滞回线}
\end{subfigure}\hfill
\begin{subfigure}[b]{0.5\columnwidth}\centering
    \includegraphics[height=170pt]{assets/1.1/(1)/CamScanner 10-22-2024 18.26_08.jpg}
    \caption{局部放大图}
\end{subfigure}
\caption{磁感应强度 $B$ 随外磁场 $H$ 的变化情况 (实物图)}\label{1.1照片}
\end{figure}

\noindent \textbf{(2)} 仍利用公式 (\ref{1.1换算公式}) 进行数值换算,得到不同频率时的 $B_r$ 和 $H_c$,结果如下表所示:


\noindent \textbf{(3)} 改变积分常量 $R_2 C$,得到不同积分常量下的李萨如图形(动态磁滞回线),如下所示:
\begin{figure}[H]\centering
\begin{subfigure}[b]{0.33\columnwidth}\centering
    %\includegraphics[height=120pt]{assets/1.1/(2)/IMG_1796.JPG}
    \caption{$R_2C = 0.01$ s}
\end{subfigure}\hfill
\begin{subfigure}[b]{0.33\columnwidth}\centering
    %\includegraphics[height=120pt]{assets/1.1/(2)/IMG_1794.JPG}
    \caption{$R_2C = 0.05$ s}
\end{subfigure}
\begin{subfigure}[b]{0.33\columnwidth}\centering
    %\includegraphics[height=120pt]{assets/1.1/(2)/IMG_1795.JPG}
    \caption{$R_2C = 0.5$ s}
\end{subfigure}
\caption{不同积分常量下的李萨如图形}
\end{figure}

\noindent Q1. 为什么积分常量会影响 $u_{R_1}-u_c$ 李萨如图形的形状?\par



\noindent Q2. 积分常量是否会影响真实的$ B − H $磁滞回线的形状?\par


\subsubsection{实验二:测量样品 1(铁氧体)的动态磁化曲线}


\newpage


\subsubsection{实验三:观察不同频率下样品 2(硅钢)的动态磁滞回线}\label{不同频率硅钢}

\subsubsection{实验四:测量样品 1(铁氧体)在不同直流偏置磁场下的可逆磁导率}

\subsection{第二部分}

\subsubsection{实验一:测量模具钢的(准)静态起始磁化曲线}


\subsubsection{实验二:测量模具钢的(准)静态磁滞回线}


\begin{table}[H]\centering
    %\renewcommand{\arraystretch}{1.5} % 调整行间距为 1.5 倍
    %\setlength{\tabcolsep}{1.5mm} % 调整列间距
    \caption{霍尔传感器测量样品的静态磁滞回线}
    \label{2.2表}
\resizebox{\linewidth}{!}{   % 设置宽度为 \linewidth 等比例缩放
\begin{tabular}{|c|cccc|c|cccc|}\hline
    序号 & $I$ (mA) & $B$ (mT) & $H \ \mathrm{(A\cdot m^{-1})}$ &  $H_{\text{re}} \ \mathrm{(A\cdot m^{-1})}$ & 序号 & $I$ (mA) & $B$ (mT) & $H \ \mathrm{(A\cdot m^{-1})}$ &  $H_{\text{re}} \ \mathrm{(A\cdot m^{-1})}$    \\
    \hline
    1	& 0.001 &0.002 	&0.003	&0.004	&28	&0.005	&0.006	&0.007 & 0.008 \\
    2	& 0.001 &0.002 	&0.003	&0.004	&29	&0.005	&0.006	&0.007 & 0.008 \\
    3	& 0.001 &0.002 	&0.003	&0.004	&30	&0.005	&0.006	&0.007 & 0.008 \\
    4	& 0.001 &0.002 	&0.003	&0.004	&31	&0.005	&0.006	&0.007 & 0.008 \\
    5	& 0.001 &0.002 	&0.003	&0.004	&32	&0.005	&0.006	&0.007 & 0.008 \\
    6	& 0.001 &0.002 	&0.003	&0.004	&33	&0.005	&0.006	&0.007 & 0.008 \\
    7	& 0.001 &0.002 	&0.003	&0.004	&34	&0.005	&0.006	&0.007 & 0.008 \\
    8	& 0.001 &0.002  &0.003	&0.004	&35	&0.005	&0.006	&0.007 & 0.008 \\
    9	& 0.001 &0.002  &0.003	&0.004	&36	&0.005	&0.006	&0.007 & 0.008 \\
    10	& 0.001 &0.002  &0.003	&0.004	&37	&0.005	&0.006	&0.007 & 0.008 \\
    11	& 0.001 &0.002  &0.003	&0.004	&38	&0.005	&0.006	&0.007 & 0.008 \\
    12	& 0.001 &0.002  &0.003	&0.004	&39	&0.005	&0.006	&0.007 & 0.008 \\
    13	& 0.001 &0.002  &0.003	&0.004	&40	&0.005  &0.006	&0.007 & 0.008 \\
    14	& 0.001 &0.002  &0.003  &0.004	&41	&0.005  &0.006	&0.007 & 0.008 \\
    15	& 0.001 &0.002  &0.003	&0.004	&42	&0.005	&0.006	&0.007 & 0.008 \\
    16	& 0.001 &0.002  &0.003	&0.004	&43	&0.005	&0.006  &0.007 & 0.008 \\
    17	& 0.001 &0.002 	&0.003	&0.004	&44	&0.005	&0.006  &0.007 & 0.008 \\
    18	& 0.001 &0.002 	&0.003	&0.004	&45	&0.005	&0.006  &0.007 & 0.008 \\
    19	& 0.001 &0.002 	&0.003	&0.004	&46	&0.005	&0.006  &0.007 & 0.008 \\
    20	& 0.001 &0.002 	&0.003	&0.004	&47	&0.005	&0.006  &0.007 & 0.008 \\
    21	& 0.001 &0.002 	&0.003	&0.004	&48	&0.005	&0.006  &0.007 & 0.008 \\
    22	& 0.001 &0.002 	&0.003	&0.004	&49	&0.005	&0.006  &0.007 & 0.008 \\
    23	& 0.001 &0.002 	&0.003	&0.004	&50	&0.005	&0.006  &0.007 & 0.008 \\
    24	& 0.001 &0.002 	&0.003	&0.004	&51	&0.005	&0.006	&0.007 & 0.008 \\
    25	& 0.001 &0.002 	&0.003	&0.004	&52	&0.005	&0.006	&0.007 & 0.008 \\
    26	& 0.001 &0.002 	&0.003	&0.004	&53	&0.005	&0.006	&0.007 & 0.008 \\
    27	& 0.001 &0.002 	&0.003	&0.004	&	&		&	    &      &       \\
    \hline
\end{tabular}}
\end{table}

\section{思考题}

\subsection*{6.1 铁磁材料的动态磁滞回线与(准)静态磁滞回线在概念上有什么区别?铁磁材料动态磁滞回线的形状和面积受那些因素影响?}


\noindent \textbf{(1) 区别:}

\noindent \textbf{(2) 影响因素:}



\subsection*{6.2 什么叫做基本磁化曲线?它和起始磁化曲线间有何区别?}

\noindent\textbf{(1) 初始磁化曲线:}


\noindent\textbf{(2) 基本磁化曲线:}

\noindent\textbf{(3) 区别:}



\subsection*{6.3 铁氧体和硅钢材料的动态磁化特性各有什么特点?}

\noindent\textbf{(1) 磁化特性:}



\noindent\textbf{(2) 磁滞损耗:}



\subsection*{6.4 动态磁滞回线测量实验中,电路参量应怎样设置才能保证$u_{R_1}-u_C$所形成的李萨如图形正确反映材料动态磁滞回线的形状?}



\subsection*{6.5 准静态磁滞回线测量实验中,为什么要对样品进行磁锻炼才能获得稳定的饱和磁滞回线?}



\section{实验总结与心得体会}

实验总结与心得体会,实验总结与心得体会,实验总结与心得体会,实验总结与心得体会,实验总结与心得体会,实验总结与心得体会,实验总结与心得体会,实验总结与心得体会,实验总结与心得体会,实验总结与心得体会,实验总结与心得体会,实验总结与心得体会。实验总结与心得体会,实验总结与心得体会,实验总结与心得体会,实验总结与心得体会,实验总结与心得体会,实验总结与心得体会,实验总结与心得体会,实验总结与心得体会,实验总结与心得体会,实验总结与心得体会,实验总结与心得体会,实验总结与心得体会。


实验总结与心得体会,实验总结与心得体会,实验总结与心得体会,实验总结与心得体会,实验总结与心得体会,实验总结与心得体会,实验总结与心得体会,实验总结与心得体会,实验总结与心得体会,实验总结与心得体会,实验总结与心得体会,实验总结与心得体会。


实验总结与心得体会,实验总结与心得体会,实验总结与心得体会,实验总结与心得体会,实验总结与心得体会,实验总结与心得体会,实验总结与心得体会,实验总结与心得体会,实验总结与心得体会,实验总结与心得体会,实验总结与心得体会,实验总结与心得体会。
\footnote{手写预习报告、原始数据记录表和 Matlab 源代码附在附录中。}




% --------------------------- 附录 --------------------------- %
% >> ------------------------ 附录 ------------------------ << %


\newpage
\appendix
% section 标题自定义设置 
\titleformat{\section}[hang]{\normalfont\huge\bfseries\centering}{}{20pt}{}
\titlespacing*{\section}{0pt}{-25pt}{8pt} % 控制上方空白的大小
% subsection 标题自定义设置 
\titleformat{\subsection}[hang]{\normalfont\Large\bfseries\boldmath}{\thesubsection}{8pt}{}
% subsubsection 标题自定义设置 
\titleformat{\subsection}[hang]{\normalfont\normalsize\bfseries\boldmath}{\thesubsection}{8pt}{}


% 附录 A
\section*{附录 A\hspace*{20pt} 手写预习报告}
\addcontentsline{toc}{section}{附录 A\hspace*{6pt} 手写预习报告} 
\thispagestyle{fancy} 

\begin{figure}[H]\centering
    %\includepdf[pages=1, width=500pt]{assets/pdf/预习报告-2-05组-丁毅-磁滞回线-2024.10.22-朱中柱.pdf}
\end{figure}
%\includepdf[pages={2}]{assets/pdf/预习报告-2-05组-丁毅-磁滞回线-2024.10.22-朱中柱.pdf}


% 附录 B
\section*{附录 B\hspace*{20pt} 原始数据记录表}
\addcontentsline{toc}{section}{附录 B\hspace*{6pt} 原始数据记录表} 
\thispagestyle{fancy} 

\begin{figure}[H]\centering
    %\includepdf[pages=1, width=500pt]{assets/pdf/Ex.8 原始数据记录表.pdf}
\end{figure}
%\includepdf[pages={2-4}]{assets/pdf/Ex.8 原始数据记录表.pdf}


% 附录 C
\section*{附录 C\hspace*{20pt} Matlab 源码}
\addcontentsline{toc}{section}{附录 C\hspace*{6pt} Matlab 源码} 
\thispagestyle{fancy} 

% 目录不够放了,只能出此下策
{\normalfont\Large\bfseries\boldmath 
\begin{center}
    C.1 第一部分
\end{center}
}

{\noindent\normalfont\large\bfseries\boldmath C.1.1 实验一图 \ref{1.1图} 源码}\label{1.1源码}
\lstinputlisting{d:/a_RemoteRepo/GH.MatlabCodes/本科课程代码/基础物理实验/Ex_8/Ex_8_Part1_1.m}

{\noindent\normalfont\large\bfseries\boldmath C.1.2 实验二图 \ref{1.2图} 源码}\label{1.2源码}
\lstinputlisting{d:/a_RemoteRepo/GH.MatlabCodes/本科课程代码/基础物理实验/Ex_8/Ex_8_Part1_2.m}

{\noindent\normalfont\large\bfseries\boldmath C.1.3 实验四图 \ref{1.4图} 源码}\label{1.4源码}
\lstinputlisting{d:/a_RemoteRepo/GH.MatlabCodes/本科课程代码/基础物理实验/Ex_8/Ex_8_Part1_4.m}

\end{document}

% VScode 常用快捷键:

% F2:                       变量重命名
% Ctrl + Enter:             行中换行
% Alt + up/down:            上下移行
% 鼠标中键 + 移动:           快速多光标
% Shift + Alt + up/down:    上下复制
% Ctrl + left/right:        左右跳单词
% Ctrl + Backspace/Delete:  左右删单词    
% Shift + Delete:           删除此行
% Ctrl + J:                 打开 VScode 下栏(输出栏)
% Ctrl + B:                 打开 VScode 左栏(目录栏)
% Ctrl + `:                 打开 VScode 终端栏
% Ctrl + 0:                 定位文件
% Ctrl + Tab:               切换已打开的文件(切标签)
% Ctrl + Shift + P:         打开全局命令(设置)

% Latex 常用快捷键:

% Ctrl + Alt + J:           由代码定位到PDF


