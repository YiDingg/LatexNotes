% 若编译失败,且生成 .synctex(busy) 辅助文件,可能有两个原因:
% 1. 需要插入的图片不存在:Ctrl + F 搜索 'figure' 将这些代码注释/删除掉即可
% 2. 路径/文件名含中文或空格:更改路径/文件名即可

% --------------------- 文章宏包及相关设置 --------------------- %
% >> ------------------ 文章宏包及相关设置 ------------------ << %
% 设定文章类型与编码格式
\documentclass[UTF8]{report}		

% 本文专属的宏包
\usepackage{pdfpages}   % 用于导入 pdf 文件
\usepackage{titletoc}   % 用于设置 toc
\titlecontents{chapter}
              [0em]
              {\bf}%
              {}%   % \contentslabel{2.5em} 会显示宽为 2.5em 的章节编号
              {}%
              {\hfill \contentspage \vspace{10.5pt}}%
              []%



% 本 .tex 专属的宏定义
    \def\V{\ \mathrm{V}}
    \def\uV{\ \mu\mathrm{V}}
    \def\mV{\ \mathrm{mV}}
    \def\K{\ \mathrm{K}}
    \def\kV{\ \mathrm{KV}}
    \def\KV{\ \mathrm{KV}}
    \def\MV{\ \mathrm{MV}}
    \def\uA{\ \mu\mathrm{A}}
    \def\mA{\ \mathrm{mA}}
    \def\A{\ \mathrm{A}}
    \def\kA{\ \mathrm{KA}}
    \def\KA{\ \mathrm{KA}}
    \def\MA{\ \mathrm{MA}}
    \def\O{\ \Omega}
    \def\mO{\ \Omega}
    \def\kO{\ \mathrm{K}\Omega}
    \def\KO{\ \mathrm{K}\Omega}
    \def\MO{\ \mathrm{M}\Omega}
    \def\Hz{\ \mathrm{Hz}}
    \def\uF{\ \mu\mathrm{F}}
    \def\mF{\ \mathrm{mF}}
    \def\F{\ \mathrm{F}}

% 自定义宏定义
    \def\N{\mathbb{N}}
    %\def\F{\mathbb{F}}
    \def\Z{\mathbb{Z}}
    \def\Q{\mathbb{Q}}
    \def\R{\mathbb{R}}
    \def\C{\mathbb{C}}
    \def\T{\mathbb{T}}
    \def\S{\mathbb{S}}
    \def\A{\mathbb{A}}
    \def\I{\mathscr{I}}
    \def\Im{\mathrm{Im\,}}
    \def\Re{\mathrm{Re\,}}
    \def\d{\mathrm{d}}
    \def\p{\partial}

% 导入基本宏包
    \usepackage[UTF8]{ctex}     % 设置文档为中文语言
        \usepackage{hyperref}  % 宏包:自动生成超链接 (此宏包与标题中的数学环境冲突)
    \hypersetup{
        colorlinks=true,    % false:边框链接 ; true:彩色链接
        citecolor={blue},    % 文献引用颜色
        linkcolor={blue},   % 目录 (我们在目录处单独设置),公式,图表,脚注等内部链接颜色
        urlcolor={magenta},    % 网页 URL 链接颜色,包括 \href 中的 text
        % cyan 浅蓝色 
        % magenta 洋红色
        % yellow 黄色
        % black 黑色
        % white 白色
        % red 红色
        % green 绿色
        % blue 蓝色
        % gray 灰色
        % darkgray 深灰色
        % lightgray 浅灰色
        % brown 棕色
        % lime 石灰色
        % olive 橄榄色
        % orange 橙色
        % pink 粉红色
        % purple 紫色
        % teal 蓝绿色
        % violet 紫罗兰色
    }
    % \usepackage{docmute}    % 宏包:子文件导入时自动去除导言区,用于主/子文件的写作方式,\include{./51单片机笔记}即可。注:启用此宏包会导致.tex文件capacity受限。
    \usepackage{amsmath}    % 宏包:数学公式
    \usepackage{mathrsfs}   % 宏包:提供更多数学符号
    \usepackage{amssymb}    % 宏包:提供更多数学符号
    \usepackage{pifont}     % 宏包:提供了特殊符号和字体
    \usepackage{extarrows}  % 宏包:更多箭头符号


% 文章页面margin设置
    \usepackage[a4paper]{geometry}
        \geometry{top=1in}
        \geometry{bottom=1in}
        \geometry{left=0.75in}
        \geometry{right=0.75in}   % 设置上下左右页边距
        \geometry{marginparwidth=1.75cm}    % 设置边注距离(注释、标记等)

% 配置数学环境
    \usepackage{amsthm} % 宏包:数学环境配置
    %\everymath{\displaystyle}   % 设置全文数学公式都为展示样式
    % theorem-line 环境自定义
        \newtheoremstyle{MyLineTheoremStyle}% <name>
            {11pt}% <space above>
            {11pt}% <space below>
            {\kaishu}% <body font> 使用默认正文字体
            {}% <indent amount>
            {\bfseries}% <theorem head font> 设置标题项为加粗
            {:}% <punctuation after theorem head>
            {.5em}% <space after theorem head>
            {\textbf{#1}\thmnumber{#2}\ \ (\,\textbf{#3}\,)}% 设置标题内容顺序
        \theoremstyle{MyLineTheoremStyle} % 应用自定义的定理样式
        \newtheorem{LineTheorem}{Theorem.\,}
    % theorem-block 环境自定义
        \newtheoremstyle{MyBlockTheoremStyle}% <name>
            {11pt}% <space above>
            {11pt}% <space below>
            {\kaishu}% <body font> 使用默认正文字体
            {}% <indent amount>
            {\bfseries}% <theorem head font> 设置标题项为加粗
            {:\\ \indent}% <punctuation after theorem head>
            {.5em}% <space after theorem head>
            {\textbf{#1}\thmnumber{#2}\ \ (\,\textbf{#3}\,)}% 设置标题内容顺序
        \theoremstyle{MyBlockTheoremStyle} % 应用自定义的定理样式
        \newtheorem{BlockTheorem}[LineTheorem]{Theorem.\,} % 使用 LineTheorem 的计数器
    % definition 环境自定义
        \newtheoremstyle{MySubsubsectionStyle}% <name>
            {11pt}% <space above>
            {11pt}% <space below>
            {}% <body font> 使用默认正文字体
            {}% <indent amount>
            {\bfseries}% <theorem head font> 设置标题项为加粗
            {:\\ \indent}% <punctuation after theorem head>
            {0pt}% <space after theorem head>
            {\textbf{#3}}% 设置标题内容顺序
        \theoremstyle{MySubsubsectionStyle} % 应用自定义的定理样式
        \newtheorem{definition}{}

%宏包:有色文本框(proof环境)及其设置
    \usepackage{xcolor}    %设置插入的文本框颜色
    \usepackage[strict]{changepage}     % 提供一个 adjustwidth 环境
    \usepackage{framed}     % 实现方框效果
        \definecolor{graybox_color}{rgb}{0.95,0.95,0.96} % 文本框颜色。修改此行中的 rgb 数值即可改变方框纹颜色,具体颜色的rgb数值可以在网站https://colordrop.io/ 中获得。(截止目前的尝试还没有成功过,感觉单位不一样)(找到喜欢的颜色,点击下方的小眼睛,找到rgb值,复制修改即可)
        \newenvironment{graybox}{%
        \def\FrameCommand{%
        \hspace{1pt}%
        {\color{gray}\small \vrule width 2pt}%
        {\color{graybox_color}\vrule width 4pt}%
        \colorbox{graybox_color}%
        }%
        \MakeFramed{\advance\hsize-\width\FrameRestore}%
        \noindent\hspace{-4.55pt}% disable indenting first paragraph
        \begin{adjustwidth}{}{7pt}%
        \vspace{2pt}\vspace{2pt}%
        }
        {%
        \vspace{2pt}\end{adjustwidth}\endMakeFramed%
        }

% 外源代码插入设置
    % matlab 代码插入设置
    \usepackage{matlab-prettifier}
        \lstset{style=Matlab-editor}    % 继承 matlab 代码高亮 , 此行不能删去
    \usepackage[most]{tcolorbox} % 引入tcolorbox包 
    \usepackage{listings} % 引入listings包
        \tcbuselibrary{listings, skins, breakable}
        \newfontfamily\codefont{Consolas} % 定义需要的 codefont 字体
        \lstdefinestyle{MatlabStyle_inc}{   % 插入代码的样式
            language=Matlab,
            basicstyle=\small\ttfamily\codefont,    % ttfamily 确保等宽 
            breakatwhitespace=false,
            breaklines=true,
            captionpos=b,
            keepspaces=true,
            numbers=left,
            numbersep=15pt,
            showspaces=false,
            showstringspaces=false,
            showtabs=false,
            tabsize=2,
            xleftmargin=15pt,   % 左边距
            %frame=single, % single 为包围式单线框
            frame=shadowbox,    % shadowbox 为带阴影包围式单线框效果
            %escapeinside=``,   % 允许在代码块中使用 LaTeX 命令 (此行无用)
            %frameround=tttt,    % tttt 表示四个角都是圆角
            framextopmargin=0pt,    % 边框上边距
            framexbottommargin=0pt, % 边框下边距
            framexleftmargin=5pt,   % 边框左边距
            framexrightmargin=5pt,  % 边框右边距
            rulesepcolor=\color{red!20!green!20!blue!20}, % 阴影框颜色设置
            %backgroundcolor=\color{blue!10}, % 背景颜色
        }
        \lstdefinestyle{MatlabStyle_src}{   % 插入代码的样式
            language=Matlab,
            basicstyle=\small\ttfamily\codefont,    % ttfamily 确保等宽 
            breakatwhitespace=false,
            breaklines=true,
            captionpos=b,
            keepspaces=true,
            numbers=left,
            numbersep=15pt,
            showspaces=false,
            showstringspaces=false,
            showtabs=false,
            tabsize=2,
        }
        \newtcblisting{matlablisting}{
            %arc=2pt,        % 圆角半径
            % 调整代码在 listing 中的位置以和引入文件时的格式相同
            top=0pt,
            bottom=0pt,
            left=-5pt,
            right=-5pt,
            listing only,   % 此句不能删去
            listing style=MatlabStyle_src,
            breakable,
            colback=white,   % 选一个合适的颜色
            colframe=black!0,   % 感叹号后跟不透明度 (为 0 时完全透明)
        }
        \lstset{
            style=MatlabStyle_inc,
        }

% table 支持
    \usepackage{booktabs}   % 宏包:三线表
    \usepackage{tabularray} % 宏包:表格排版
    \usepackage[longtable]{multirow} % 宏包:multi 行列
    \usepackage{longtable}  % 宏包:长表格

% figure 设置
    \usepackage{graphicx}  % 支持 jpg, png, eps, pdf 图片 
    \usepackage{svg}       % 支持 svg 图片
        \svgsetup{
            % 指向 inkscape.exe 的路径
            inkscapeexe = C:/aa_MySame/inkscape/bin/inkscape.exe, 
            % 一定程度上修复导入后图片文字溢出几何图形的问题
            inkscapelatex = false                 
        }
    \usepackage{subcaption} % 支持子图

% 图表进阶设置
    \usepackage{caption}    % 图注、表注
        \captionsetup[figure]{name=图}  
        \captionsetup[table]{name=表}
        \captionsetup{
            labelfont=bf, % 设置标签为粗体
            textfont=bf,  % 设置文本为粗体
            font=small  
            }
    \usepackage{float}     % 图表位置浮动设置 
    \usepackage{etoolbox} % 用于保证图注表注的数学字符为粗体
        \AtBeginEnvironment{figure}{\boldmath} % 图注中的数学字符为粗体
        \AtBeginEnvironment{table}{\boldmath}  % 表注中的数学字符为粗体
        \AtBeginEnvironment{tabular}{\unboldmath}   % 保证表格中的数学字符不受额外影响
% 圆圈序号自定义
    \newcommand*\circled[1]{\tikz[baseline=(char.base)]{\node[shape=circle,draw,inner sep=0.8pt, line width = 0.03em] (char) {\small \bfseries #1};}}   % TikZ solution

% 列表设置
    \usepackage{enumitem}   % 宏包:列表环境设置
        \setlist[enumerate]{
            label=(\arabic*) ,   % 设置序号样式为加粗的 (1) (2) (3)
            ref=\arabic*, % 如果需要引用列表项,这将决定引用格式(这里仍然使用数字)
            itemsep=0pt, parsep=0pt, topsep=0pt, partopsep=0pt, leftmargin=3.5em} 
        \setlist[itemize]{itemsep=0pt, parsep=0pt, topsep=0pt, partopsep=0pt, leftmargin=3.5em}
        \newlist{circledenum}{enumerate}{1} % 创建一个新的枚举环境  
        \setlist[circledenum,1]{  
            label=\protect\circled{\arabic*}, % 使用 \arabic* 来获取当前枚举计数器的值,并用 \circled 包装它  
            ref=\arabic*, % 如果需要引用列表项,这将决定引用格式(这里仍然使用数字)
            itemsep=0pt, parsep=0pt, topsep=0pt, partopsep=0pt, leftmargin=3.5em
        }  
    

% 文章默认字体设置
    \usepackage{fontspec}   % 宏包:字体设置
        \setmainfont{SimSun}    % 设置中文字体为宋体字体
        \setCJKmainfont[AutoFakeBold=3]{SimSun} % 设置加粗字体为 SimSun 族,AutoFakeBold 可以调整字体粗细
        \setmainfont{Times New Roman} % 设置英文字体为Times New Roman


% 文章序言设置
    \newcommand{\cnabstractname}{序言}
    \newenvironment{cnabstract}{%
        \par\Large
        \noindent\mbox{}\hfill{\bfseries \cnabstractname}\hfill\mbox{}\par
        \vskip 2.5ex
        }{\par\vskip 2.5ex}

% 参考文献引用设置
    \bibliographystyle{unsrt}   % 设置参考文献引用格式为unsrt
    \newcommand{\upcite}[1]{\textsuperscript{\cite{#1}}}     % 自定义上角标式引用

% 各级标题自定义设置
    \usepackage{titlesec}   
    % chapter
        \titleformat{\chapter}[hang]{\normalfont\Large\bfseries\centering\boldmath}{}{10pt}{}
        %\titleformat{\chapter}[hang]{\normalfont\Large\bfseries\centering\boldmath}{Homework \thechapter :}{10pt}{}
        \titlespacing*{\chapter}{0pt}{-30pt}{10pt} % 控制上方空白的大小
    % section
        \titleformat{\section}[hang]{\normalfont\large\bfseries\boldmath}{\thesection}{8pt}{}
    % subsection
        % 设置 subsection 样式为 (1) (2) (3)
        \renewcommand{\thesubsection}{(\arabic{subsection})}    
        \titleformat{\subsection}[hang]{\normalfont\bfseries\boldmath}{\thesubsection}{8pt}{}


% --------------------- 文章宏包及相关设置 --------------------- %
% >> ------------------ 文章宏包及相关设置 ------------------ << %


% ------------------------ 文章信息区 ------------------------ %
% >> --------------------- 文章信息区 --------------------- << %
% 页眉页脚设置
\usepackage{fancyhdr}   %宏包:页眉页脚设置
    \pagestyle{fancy}
    \fancyhf{}
    \cfoot{\thepage}
    \renewcommand\headrulewidth{1pt}
    \renewcommand\footrulewidth{0pt}
    \chead{电路原理课程作业,\ 丁毅}
    \usepackage{fontawesome}    % 宏包:更多符号与图标 (用于插入 GitHub 图标等)
    \lhead{
    \href{https://github.com/YiDingg/LatexNotes}{\color{black}\faGithub\ https://github.com/YiDingg/LatexNotes}
    }
    \rhead{dingyi233@mails.ucas.ac.cn}

%文档信息设置
\title{电路原理课程作业\\ Homework of Principles of Electric Circuits }
\author{丁毅\\ \footnotesize 中国科学院大学,北京 100049\\ Yi Ding \\ \footnotesize University of Chinese Academy of Sciences, Beijing 100049, China}
\date{\footnotesize 2024.8 -- 2025.1}
% >> --------------------- 文章信息区 --------------------- << %
% ------------------------ 文章信息区 ------------------------ %

% 开始编辑文章

\begin{document}
\zihao{5}           % 设置全文字号大小

% ------------------------ 封面序言与目录 ------------------------ %
% >> --------------------- 封面序言与目录 --------------------- << %
% 封面
    \maketitle\newpage  
    \pagenumbering{Roman} % 页码为大写罗马数字
    \thispagestyle{fancy}   % 显示页码、页眉等

% 序言
    \begin{cnabstract}\normalsize 
        本文为笔者本科时的“电路原理”课程作业(Homework of Principles of Electric Circuits, 2024.9-2025.1)。所有作业课件已上传到网址 \href{https://www.123865.com/s/0y0pTd-R8Kj3}{https://www.123865.com/s/0y0pTd-R8Kj3}(包括 Homework, Simulation 和 Laboratory)。读者可在笔者的个人网站 \href{https://yidingg.github.io/YiDingg/\#/Notes/MajorCourses/CircuitTheoryNotes}{https://yidingg.github.io/YiDingg/\#/Notes/MajorCourses/CircuitTheoryNotes} 上找到课程信息、教材、教辅和作业答案等相关资料。

        由于个人学识浅陋,认识有限,文中难免有不妥甚至错误之处,望读者不吝指正。读者可以将错误发送到我的邮箱 {\color{blue}\ dingyi233@mails.ucas.ac.cn},也可以到笔者的 \href{https://github.com/YiDingg/LatexNotes}{GitHub (https://github.com/YiDingg/LatexNotes)} 上提 issue,衷心感谢。
    \end{cnabstract}
    \addcontentsline{toc}{chapter}{序言} % 手动添加到目录



%\listoftables
%\listoffigures
    \clearpage
    \setcounter{tocdepth}{0}
    \tableofcontents\thispagestyle{fancy}   % 显示页码、页眉等  
    \addcontentsline{toc}{chapter}{目录} % 手动添加到目录



% 收尾工作
    \newpage    
    %\rhead{Homework \thechapter}
    \pagenumbering{arabic} 

% >> --------------------- 封面序言与目录 --------------------- << %
% ------------------------ 封面序言与目录 ------------------------ %



\chapter{Design (Buck Circuits): 2024.11.07 - 2024.12.31} \thispagestyle{fancy}

Buck 电路是一种开关型 DC-DC 变换器,用于将输入的(高)直流电压转换为输出(低)直流电压。在本次,我们将设计一个开关频率和占空比可调的 Buck 电路。其中,开关频率主要影响输出的正弦(三角)纹波大小,占空比可以控制输出电压大小。

需要注意的是,开关频率和占空比都可能影响电路的稳定性(如抗噪性能),实际使用时应做抗干扰测试,或者进一步改进后再使用。

\section{Buck 电路设计}

\subsection{设计要求}
利用运算放大器 OPA 和 MOSFET 实现 Buck 电路,下面是具体的设计要求:
\begin{enumerate}
\item 输出电压 3.3 V,电压输出纹波比 $r = \frac{\Delta U_{\text{o}}}{U_{\text{o, ave}}} < 5\%$,$3.3\ \mathrm{V}\times 5\,\% = 165\ \mathrm{mV}$,相当于 $\pm\, 82.2\ \mathrm{mV}$;
\item 使用一个 +15 V 电源和一个 -15 V 电源供电;
\item MOSFET 型号为 2N7000;
\item 运放型号为 LM258N;
\item 整流二极管型号为 1N4001;
\item 电阻电容电感数量不限;最后的滤波电容最大值不超过 470 $\uF$,电感最大值不超过 2 mH,作为最后负载电阻不要小于 2 $\kO$;
\item 电感非理想,所以需要测试电感的 Q 值和 DCR (Direct Current Resistance),仿真时将 DCR 考虑进去;
\item 电容非理想,所以需要测试电容的 ESR (Equivalent Series Resistance) 和;
\item 实验报告里面包含原理说明、设计的电路图、仿真结果、实际电路照片、实际电路测试结果及分析。
\end{enumerate}


\subsection{Buck 电路原理}
图 \ref{} 是一个典型的 Buck 电路:
\begin{figure}[H]\centering
    %\includegraphics[width=0.4\columnwidth]{}
    \caption{典型 Buck 电路}
\end{figure}
我们只考虑电路达到稳定工作状态时的情况,即输出电压在平均值附近小范围摆动。为了推导其工作原理,先作出一些必要的假设:
\begin{enumerate}
\item 经过可接受的启动时间后,电路能够达到稳定输出状态,此时输出电压在均值附近小幅振荡;
\item MOSFET 可视为 Switch-Resistor Model,且导通电阻极小(相比于 $\KO$ 量级):这意味着电路其它电阻在 $\kO$ 级别,且 MOSFET 各级电压满足:
\begin{equation}
u_{\text{GS}} > U_T \quad \text{and} \quad u_{\text{GD}} > U_T
\end{equation}
\item 二极管可视为 Switch-Source-Resistor Model,且导通电阻极小(相比于 $\KO$ 量级);
\item MOSFET 开关周期远小于电感时间常量 $\tau$,即 $T \ll \tau$,这等价于开关频率 $f \gg \frac{1}{\tau}$,此时输出纹波可近似视为三角波(或正弦波);
\item MOSFET 开关频率不高于 500 KHz,以避免高频状态下元件性能异常,此时可不考虑高频效应。
\end{enumerate}

~\\
设脉冲频率为 $f$,占空比为 $k$,一个周期 $T$ 分为高电平 $T_{\text{ON}}$ 和低电平 $T_{\text{OFF}}$。下面作具体的推导。设 $t = 0$ 时,MOSFET 的 G 级接收到高电平脉冲信号,MOSFET 导通,等效电路如图 \ref{} 所示。此时电感电流应为最小值,设其为 $\left(i_L\right)_{\min}$,稳态值是 $ \frac{U_s}{R + R_{\text{L}} + R_{\text{ON}}}$,由三要素法,有:
\begin{gather}
i_L(t) = \left(i_L\right)_{\min}\cdot e^{-\frac{t}{\tau_{\text{ON}}}} + \frac{U_s}{R + R_{\text{L}} + R_{\text{ON}}}\cdot\left(1 - e^{-\frac{t}{\tau_{\text{ON}}}}\right) \Longrightarrow \\ 
i_L(t) = \left[ \left(i_L\right)_{\min} - \frac{U_s}{R + R_{\text{L}} + R_{\text{ON}}} \right]\cdot e^{-\frac{t}{\tau_{\text{ON}}}} + \frac{U_s}{R + R_{\text{L}} + R_{\text{ON}}},\quad \tau_{\text{ON}} = \frac{L}{R + R_{\text{L}} + R_{\text{ON}}}
\end{gather}
经过 $T_{\text{ON}}$ 时间,MOSFET 关断,则 $[0,\  T_{\text{ON}}]$ 时间段内的电感电流增量为:
\begin{equation}
\left(\Delta i_L\right)_{\text{ON}} = \left[ \left(i_L\right)_{\min} - \frac{U_s}{R + R_{\text{L}} + R_{\text{ON}}} \right]\cdot \left(1 - e^{-\frac{T_{\text{ON}}}{\tau_{\text{ON}}}} \right) ,\quad
\tau_{\text{ON}} = \frac{L}{R + R_{\text{L}} + R_{\text{ON}}}
\end{equation}
类似的思路,设 $t = 0$ 时,MOSFET 的 G 级接收到低电平脉冲信号,MOSFET 关断,等效电路如图 \ref{} 所示。此时电感电流应为最大值,设其为 $\left(i_L\right)_{\max}$,由于二极管存在导通压降 $U_\text{D}$,电流“稳态值”是 $- \frac{U_\text{D}}{R + R_{\text{L}} + R_{\text{D}}}$ ,由三要素法,有:
\begin{gather}
    i_L(t) = \left(i_L\right)_{\max}\cdot e^{-\frac{t}{\tau_{\text{OFF}}}} - \frac{U_\text{D}}{R + R_{\text{L}} + R_{\text{D}}}\cdot\left(1 - e^{-\frac{t}{\tau_{\text{OFF}}}}\right) \Longrightarrow \\ 
    i_L(t) = \left[ \left(i_L\right)_{\max} + \frac{U_\text{D}}{R + R_{\text{L}} + R_{\text{D}}} \right]\cdot e^{-\frac{t}{\tau_{\text{OFF}}}} - \frac{U_\text{D}}{R + R_{\text{L}} + R_{\text{D}}},\quad \tau_{\text{OFF}} = \frac{L}{R + R_{\text{L}} + R_{\text{D}}}
\end{gather}
经过 $T_{\text{OFF}}$ 时间,MOSFET 又导通,则 $[0,\  T_{\text{OFF}}]$ 时间段内的电感电流增量为:
\begin{equation}
    \left(\Delta i_L\right)_{\text{OFF}} = \left[ \left(i_L\right)_{\max} + \frac{U_\text{D}}{R + R_{\text{L}} + R_{\text{D}}}\right]\cdot \left(1 - e^{-\frac{T_{\text{OFF}}}{\tau_{\text{OFF}}}} \right)
\end{equation}

电路输出达到稳态,所以应有 $\left(\Delta i_L\right)_{\text{ON}} + \left(\Delta i_L\right)_{\text{OFF}} = 0$,简记 $e_{\text{ON}} = e^{-\frac{T_{\text{ON}}}{\tau_{\text{ON}}}}$ 和 $e_{\text{OFF}} = e^{-\frac{T_{\text{OFF}}}{\tau_{\text{OFF}}}}$,得到:
\begin{align}
\left(i_L\right)_{\max} &= \frac{U_s}{R + R_{\text{L}} + R_{\text{ON}}}\cdot \frac{1 - e_{\text{ON}}}{1 - e_{\text{ON}}e_{\text{OFF}}} - \frac{U_\text{D}}{R + R_{\text{L}} + R_{\text{D}}}
\\
\left(i_L\right)_{\min} &=  e_{\text{OFF}}\left(i_L\right)_{\max} =  \frac{U_s}{R + R_{\text{L}} + R_{\text{ON}}}\cdot \frac{e_{\text{OFF}} - e_{\text{ON}}e_{\text{OFF}}}{1 - e_{\text{ON}}e_{\text{OFF}}} - \frac{e_{\text{OFF}}U_\text{D}}{R + R_{\text{L}} + R_{\text{D}}}
\\
\Delta i_L &= \left(1  -e_{\text{OFF}}\right)\left(i_L\right)_{\max} = \frac{U_s}{R + R_{\text{L}} + R_{\text{ON}}}\cdot \frac{ (1 - e_{\text{ON}})(1 - e_{\text{OFF}}) }{1 - e_{\text{ON}}e_{\text{OFF}}} - \frac{ U_\text{D}(1 - e_{\text{OFF}})}{R + R_{\text{L}} + R_{\text{D}}}
\end{align}
%输出电压的波纹比为:
%\begin{equation}
%    r = \frac{\Delta U_{\text{o}}}{\left(U_{\text{out}}\right)_%{\text{average}}}  = \frac{\left(\Delta i_L\right)_{\text{ON}}}{\left(i_L\right)_{\text{average}}\cdot T_{\text{ON}}}
%\end{equation}
由于功率损耗发生在二极管导通电阻 $R_{\text{D}}$、压降 $U_\text{D}$ 和 MOSFET 导通电阻 $R_{\text{ON}}$ 上,我们可以进一步计算功率损耗和转化效率:
\begin{align}
P_{\text{loss}} 
&= k\left(i_L\right)_{\text{average}}^2R_{\text{ON}} + (1 - k)\left(i_L\right)_{\text{average}}^2R_{\text{D}}
+ (1 - k)\left(i_L\right)_{\text{average}}U_\text{D} 
\\
&= \left(i_L\right)_{\text{average}}^2 \left[ k R_{\text{ON}} + (1 - k)R_{\text{D}} \right] \\ 
\Longrightarrow \eta &= \frac{P_{\text{out}}}{P_{\text{out}} + P_{\text{loss}} } = \frac{\left(u_o\right)_\text{average}}{\left(u_o\right)_\text{average} + \left(i_o\right)_{\text{average}} \left[ k R_{\text{ON}} + (1 - k)R_{\text{D}} + (1 - k)U_\text{D} \right]}
\end{align}
如果取近似 $\left(u_o\right)_\text{average} = kU_s$,$\left(i_o\right)_{\text{average}} = \frac{kU_s}{R}$,则转化效率为:
\begin{equation}
\eta = \frac{R}{R + kR_{\text{ON}} + (1-k)R_{\text{D}} +  \frac{(1 - k)U_\text{D}}{kU_s} R }
\end{equation}



\subsection{Buck 改进方案 1}
一方面,上面的电路,输出电压纹波比 $r = \frac{\Delta U_{\text{o}}}{U_{\text{o, ave}}}$ 可能较大;另一方面,二极管的导通压降 $U_\text{D}$ 可能使电路效率明显降低(尤其在输出低电平时)。改进方案如下图:
\begin{figure}[H]\centering
    %\includegraphics[width=0.4\columnwidth]{}
    \caption{}
\end{figure}
在图 \ref{} 中,一方面,我们在输出加一个较大的电容 $C$,这可以明显降低输出纹波幅度;另一方面,我们用另一个 MOSFET 来替代二极管 D,此 MOSFET 的 G 级由脉冲信号经过反相器得到,这样相当于将二极管的导通压降 $U_\text{D}$ 降低为 $0$,同时将 $R_{\text{D}}$ 换为 $R_{\text{ON}}$,既可以避免压降 $U_{\text{D}}$ 带来的占空比失调,也可以提高电路的转化效率。


另外,我们在输出端串联了一个 $0 \Omega$ 电阻,这使得我们可以方便地串入小电阻(示波器测电压)或电流表来测量输出电流。

改进之后,开关 MOS 导通与不导通时的等效电路如下图:
\begin{figure}[H]\centering
\begin{subfigure}[b]{0.5\columnwidth}\centering
    %\includegraphics[height=120pt]{}
    \caption{}
\end{subfigure}\hfill
\begin{subfigure}[b]{0.5\columnwidth}\centering
    %\includegraphics[height=120pt]{}
    \caption{}
\end{subfigure}
\caption{}
\end{figure}

稳定工作时,可分为两种状态。记开关 MOS 为下面以输出负端 $u_{o, -}$ 为电压参考点,定性地分析两种状态下的电路行为:
\begin{enumerate}
\item 开关 MOS ($T_1$) 导通:此时二极管 MOS ($T_2$) 截止,电路如图 \ref{} (a)。$u_{L, -}$ 维持在 $u_\text{out}$ 附近,电感正端(即 $T_2$ D 极)
$u_{L, +} = u_{2, \text{DS}}$ 在 $U_S$ 附近小幅下降(近似线性),这使得 $T_1$ 的 DS 电压 $u_{1, \text{DS}}$ 在 0 附近小幅线性上升(从负到正);
\item 开关 MOS ($T_1$) 截止:此时二极管 MOS ($T_2$) 导通,电路如图 \ref{} (b)。
$u_{L, -}$ 仍维持在 $u_\text{out}$ 附近,电感正端(即 $T_2$ D 极)
$u_{L, +} = u_{2, \text{DS}}$ 在 $0$ 附近小幅线性上升(从负到正),$T_1$ 的 DS 电压 $u_{1, \text{DS}}$ 在 $U_S$ 附近小幅线性下降;
\end{enumerate}
作出两种状态下的各元件电压情况,如下图所示:


\subsection{Buck 改进方案 2}
如果想进一步降低输出纹波,可考虑下图所示的改进方案:
\begin{figure}[H]\centering
    %\includegraphics[width=0.4\columnwidth]{}
    \caption{}
\end{figure}
与第一种方案相比,这种方案只是将 $LC$ 滤波换为了 $CLC$ $\pi$ 型滤波,看似变化不大,实则从原理层面作了改变。由于二极管 MOS(用于替代二极管的 N-MOSFET)两端并联了较大的电容 $C$,它 DS 间的电压 $u_{\text{DS}}$ 无法发生突变,经过半定量推导可知 $u_{\text{DS}}$ 会在 $u_{\text{out}}$ 附近以锯齿波形小幅振荡,这便需要 MOSFET 在  $u_{\text{DS}} = u_{\text{out}}$ 附近能正常工作,对元件的要求比较高,2N7000 不能满足此要求,需要另选其他型号。



在此改进方案中,开关 MOS 导通与不导通时的等效电路如下图:

\subsection{脉冲发生器原理}
为了能产生提供给 MOSFET 的脉冲电压,我们考虑脉冲发生器 (Pulse Generator),典型的脉冲发生器如下图所示:
\begin{figure}[H]\centering
    %\includegraphics[width=0.4\columnwidth]{}
    \caption{}
\end{figure}
为 OPA 提供 $+15 V$ 和 $-15 V$ 电压,记 OPA 饱和电压为 $\pm U_{\text{sat}}$。设 $t = 0$ 时 $u_C < 0$ 位于最小值,处于上升阶段,则此时 $U_o = +U_{\text{sat}}$,当 $u_C$(也即 $u_-$)上升到 $u_+ = \frac{R_f}{R_f + R_1}U_{\text{sat}}$ 并超过它的瞬间,OPA 输出突变为 $-U_{\text{sat}}$,$u_C$ 进入下降阶段。

同理,在下降阶段,$U_o = -U_{\text{sat}}$。当 $u_C$ 下降到 $u_+ = -\frac{R_f}{R_f + R_1}U_{\text{sat}}$ 并低于它的瞬间,OPA 输出突变为 $+U_{\text{sat}}$,$u_C$ 进入上升阶段。如此循环往复,即可得到脉冲信号。由于上升和下降过程完全对称,脉冲占空比恒为 50 \% 不变,且周期为:
\begin{equation}
T = 2R_2 C_f \ln \left( 1 + \frac{2R_f}{R_1} \right),\quad f = \frac{1}{T}
\end{equation}
为了方便参考,我们也给出 OPA 正负电源分别接 $V_{\text{DD}}$ 和 $V_{\text{SS}}$ 时的脉冲周期:
\begin{equation}
T = 2R_2 C_f \ln \left( \frac{1 - \frac{V_{\text{SS}}}{V_{\text{DD}}}\cdot \frac{R_f}{R_f + R_1}}{1 - \frac{R_f}{R_f + R_1}} \right),\quad f = \frac{1}{T}
\end{equation}


\subsection{脉冲发生器改进方案}

显然,占空比恒为 50 \% 会使 Buck 电路输出电压恒为 $0.5 \,U_S$,这无法满足我们的需求,需要改进脉冲发生器。改进方案如下图所示:
\begin{figure}[H]\centering
    %\includegraphics[width=0.4\columnwidth]{}
    \caption{}
\end{figure}
在图 \ref{} 中,我们将 $R_f$ 换为滑动变阻器(可调电阻),并在 $R_2$ 与电容之间加入二极管与滑动变阻器 $R_k$ 的并联,这样便可以通过 $R_f$ 调节电容电压振荡的幅度,从而调节脉冲频率,同时又能通过 $R_k$ 与二极管的并联,实现上升下降阶段有不同的时间常量 $\tau$,以此来调节脉冲占空比。

经过改进的脉冲发生器占空比 $k$ 和周期 $T$ 为(为 OPA 提供 $+15 V$ 和 $-15 V$ 电压):
\begin{gather}
k = \frac{1}{1 + \frac{R_2 +  R_\text{D} \parallel R_k}{R_2 + R_k}},\quad 
T = \left( 2R_2 + R_k + R_\text{D} \parallel R_k \right)C_f \ln \left(1 + \frac{2 R_2}{R_1}\right)
\end{gather}
为了直观感受 $R_k$ 对占空比的调节作用和 $R_f$ 对频率的调节作用,我们分别作出 $k = k(R_k)$ 的图像。令 $R_2 = 1\kO,\ R_\text{D} = 10 \Omega$,则占空比如图 \ref{} 所示,再令 $R_k = 2 \KO$,作出频率变化如图 \ref{} 所示。
\begin{figure}[H]\centering
\begin{subfigure}[b]{0.5\columnwidth}\centering
    %\includegraphics[height=120pt]{}
    \caption{}
\end{subfigure}\hfill
\begin{subfigure}[b]{0.5\columnwidth}\centering
    %\includegraphics[height=120pt]{}
    \caption{}
\end{subfigure}
\caption{}
\end{figure}


\section{仿真电路性能测试}
如图 \ref{} 搭建仿真电路,各元件的参数见图中标注。下面进行仿真电路的性能测试。

\noindent 3.3 V 工作点性能测试:
\begin{enumerate}
\item 输出电压: 3.308 V ($\pm$ 0.766 mV, $r = 0.0232 \,\%$),见图 \ref{},此时 $R_k = 2520 \Omega$、$R_f =  3.2\KO$;
\item 转化效率: 95.2 \%,
\item 各参数对 $R_k$ 的灵敏度:包括输出电压、波纹比和转化效率见表 \ref{};
\item 各参数对 $R_f$ 的灵敏度:见表 \ref{};
\item 各参数对 $R$ 的灵敏度:见表 \ref{};
\end{enumerate}

\noindent 极限性能测试:
\begin{enumerate}
\item 最小输出电压: 1.895 V ($\pm$ 0.27 mV, $r = 0.0451 \,\%$),见图 \ref{},此时 $R_k = 0$、$R_f =  5\KO$;
\item 最大输出电压: 4.252 V ($\pm$ 8.87 mV, $r = 0.2086 \,\%$),见图 \ref{},此时 $R_k = 10 \KO$、$R_f =  2.6\KO$;
\item 最小输出电压纹波比: 0.5 \%,
\item 最大输出电压纹波比: 4.5 \%,
\end{enumerate}

\begin{figure}[H]\centering
\begin{subfigure}[b]{0.5\columnwidth}\centering
    %\includegraphics[height=120pt]{}
    \caption{最小输出电压及波纹比}
\end{subfigure}\hfill
\begin{subfigure}[b]{0.5\columnwidth}\centering
    %\includegraphics[height=120pt]{}
    \caption{最大输出电压及波纹比}
\end{subfigure}
\caption{}
\end{figure}

\section{仿真设计检查}
查找各元件的的 Data Sheet,依据仿真结果,对照 Data Sheet 判断元件是否能正常工作,如果不能,需要修改设计参数以保证元件正常工作。

\begin{enumerate}
\item 开关 MOS :仿真测得的 $u_\text{GS}$ 和 $u_{\text{DS}}$ 如图 \ref{} 所示,$u_{\text{DS}}$ 在 1.9 V 左右小幅振荡。可以大致分为两种情况,第一种,$u_{\text{GS}} = 10 \ \mathrm{V}$,MOS 管导通,依据 Data Sheet,此时 2N7000 的 $I_{\text{D}} = 1.45 \ \mathrm{A}$ 超过了 Absolute Maximun Ratings 中规定的 Maximum Drain Current (Pulsed) 500 mA。
\end{enumerate}

\begin{table}[H]\centering
    %\renewcommand{\arraystretch}{1.5} % 调整行间距为 1.5 倍
    %\setlength{\tabcolsep}{1.5mm} % 调整列间距
    \caption{元件工作状态检查}
    \label{元件工作状态检查}
\begin{tabular}{cccccccccc}\toprule
     &       & 峰值电压 & 平均电压 & 峰值电流 & 平均电流 \\
    \midrule
    开关 MOS &  &  \\
     &  &   \\
     &  &   \\
    \bottomrule
\end{tabular}
\end{table}
\begin{table}[H]\centering
    %\renewcommand{\arraystretch}{1.5} % 调整行间距为 1.5 倍
    %\setlength{\tabcolsep}{1.5mm} % 调整列间距
    \caption{元件 Data Sheet 参数}
    \label{元件 Data Sheet 参数}
\begin{tabular}{cccccccccc}\toprule
     &  &   \\
    \midrule
    开关 MOS &  &   \\
     &  &   \\
     &  &   \\
    \bottomrule
\end{tabular}
\end{table}

\section{实际电路性能测试}



\section{最终成品电路}

\section{总结}

% --------------------------- 其它作业 --------------------------- %
% >> ------------------------ 其它作业 ------------------------ << %




% >> ------------------------ 附录 ------------------------ << %
% --------------------------- 附录 --------------------------- %


\end{document}



% VScode 常用快捷键:

% F2:                       变量重命名
% Ctrl + Enter:             行中换行
% Alt + up/down:            上下移行
% 鼠标中键 + 移动:           快速多光标
% Shift + Alt + up/down:    上下复制
% Ctrl + left/right:        左右跳单词
% Ctrl + Backspace/Delete:  左右删单词    
% Shift + Delete:           删除此行
% Ctrl + J:                 打开 VScode 下栏(输出栏)
% Ctrl + B:                 打开 VScode 左栏(目录栏)
% Ctrl + `:                 打开 VScode 终端栏
% Ctrl + 0:                 定位文件
% Ctrl + Tab:               切换已打开的文件(切标签)
% Ctrl + Shift + P:         打开全局命令(设置)

% Latex 常用快捷键

% Ctrl + Alt + J:           由代码定位到PDF
% 


% Git提交规范:
% update: Linear Algebra 2 notes
% add: Linear Algebra 2 notes
% import: Linear Algebra 2 notes
% delete: Linear Algebra 2 notes