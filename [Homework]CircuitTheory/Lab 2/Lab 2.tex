% 若编译失败,且生成 .synctex(busy) 辅助文件,可能有两个原因:
% 1. 需要插入的图片不存在:Ctrl + F 搜索 'figure' 将这些代码注释/删除掉即可
% 2. 路径/文件名含中文或空格:更改路径/文件名即可

% --------------------- 文章宏包及相关设置 --------------------- %
% >> ------------------ 文章宏包及相关设置 ------------------ << %
% 设定文章类型与编码格式
\documentclass[UTF8]{article}		

% 本文所需的特殊宏包

% 本 .tex 专属的宏定义
    \def\V{\ \mathrm{V}}
    \def\uV{\ \mu\mathrm{V}}
    \def\mV{\ \mathrm{mV}}
    \def\K{\ \mathrm{K}}
    \def\kV{\ \mathrm{KV}}
    \def\KV{\ \mathrm{KV}}
    \def\MV{\ \mathrm{MV}}
    \def\uA{\ \mu\mathrm{A}}
    \def\mA{\ \mathrm{mA}}
    \def\A{\ \mathrm{A}}
    \def\kA{\ \mathrm{KA}}
    \def\KA{\ \mathrm{KA}}
    \def\MA{\ \mathrm{MA}}
    \def\O{\ \Omega}
    \def\mO{\ \Omega}
    \def\kO{\ \mathrm{K}\Omega}
    \def\KO{\ \mathrm{K}\Omega}
    \def\MO{\ \mathrm{M}\Omega}
    \def\Hz{\ \mathrm{Hz}}
    \def\uF{\ \mu\mathrm{F}}
    \def\mF{\ \mathrm{mF}}
    \def\nF{\ \mathrm{nF}}
    \def\pF{\ \mathrm{pF}}
    \def\F{\ \mathrm{F}}
    \def\Re{\mathrm{\,Re}\,}
    \def\Im{\mathrm{\,Im}\,}
    \def\sinc{\mathrm{\,sinc}\,}

% 自定义宏定义
    \def\N{\mathbb{N}}
    \def\F{\mathbb{F}}
    \def\Z{\mathbb{Z}}
    \def\Q{\mathbb{Q}}
    \def\R{\mathbb{R}}
    \def\C{\mathbb{C}}
    \def\T{\mathbb{T}}
    \def\S{\mathbb{S}}
    %\def\A{\mathbb{A}}
    \def\I{\mathscr{I}}
    \def\d{\mathrm{d}}
    \def\p{\partial}

    
% 本 .tex 专属的宏定义
\def\V{\ \mathrm{V}}
\def\uV{\ \mu\mathrm{V}}
\def\mV{\ \mathrm{mV}}
\def\K{\ \mathrm{K}}
\def\kV{\ \mathrm{KV}}
\def\KV{\ \mathrm{KV}}
\def\MV{\ \mathrm{MV}}
\def\uA{\ \mu\mathrm{A}}
\def\mA{\ \mathrm{mA}}
\def\A{\ \mathrm{A}}
\def\kA{\ \mathrm{KA}}
\def\KA{\ \mathrm{KA}}
\def\MA{\ \mathrm{MA}}
\def\O{\ \Omega}
\def\mO{\ \Omega}
\def\kO{\ \mathrm{K}\Omega}
\def\KO{\ \mathrm{K}\Omega}
\def\MO{\ \mathrm{M}\Omega}
\def\Hz{\ \mathrm{Hz}}
\def\uF{\ \mu\mathrm{F}}
\def\mF{\ \mathrm{mF}}
\def\F{\ \mathrm{F}}

% 自定义宏定义
\def\N{\mathbb{N}}
%\def\F{\mathbb{F}}
\def\Z{\mathbb{Z}}
\def\Q{\mathbb{Q}}
\def\R{\mathbb{R}}
\def\C{\mathbb{C}}
\def\T{\mathbb{T}}
\def\S{\mathbb{S}}
\def\A{\mathbb{A}}
\def\I{\mathscr{I}}
\def\Im{\mathrm{Im\,}}
\def\Re{\mathrm{Re\,}}
\def\d{\mathrm{d}}
\def\p{\partial}


% 导入基本宏包
    \usepackage[UTF8]{ctex}     % 设置文档为中文语言
    \usepackage{hyperref}  % 宏包:自动生成超链接 (此宏包与标题中的数学环境冲突)
    \hypersetup{
        colorlinks=true,    % false:边框链接 ; true:彩色链接
        citecolor={blue},    % 文献引用颜色
        linkcolor={blue},   % 目录 (我们在目录处单独设置),公式,图表,脚注等内部链接颜色
        urlcolor={magenta},    % 网页 URL 链接颜色,包括 \href 中的 text
        % cyan 浅蓝色 
        % magenta 洋红色
        % yellow 黄色
        % black 黑色
        % white 白色
        % red 红色
        % green 绿色
        % blue 蓝色
        % gray 灰色
        % darkgray 深灰色
        % lightgray 浅灰色
        % brown 棕色
        % lime 石灰色
        % olive 橄榄色
        % orange 橙色
        % pink 粉红色
        % purple 紫色
        % teal 蓝绿色
        % violet 紫罗兰色
    }
    % \usepackage{docmute}    % 宏包:子文件导入时自动去除导言区,用于主/子文件的写作方式,\include{./51单片机笔记}即可。注:启用此宏包会导致.tex文件capacity受限。
    \usepackage{amsmath}    % 宏包:数学公式
    \usepackage{mathrsfs}   % 宏包:提供更多数学符号
    \usepackage{amssymb}    % 宏包:提供更多数学符号
    \usepackage{pifont}     % 宏包:提供了特殊符号和字体
    \usepackage{extarrows}  % 宏包:更多箭头符号 
    \usepackage{multicol}   % 宏包:支持多栏 

% 文章页面margin设置
    \usepackage[a4paper]{geometry}
        \geometry{top=0.75in}
        \geometry{bottom=0.75in}
        \geometry{left=0.75in}
        \geometry{right=0.75in}   % 设置上下左右页边距
        \geometry{marginparwidth=1.75cm}    % 设置边注距离(注释、标记等)

% 配置数学环境
    \usepackage{amsthm} % 宏包:数学环境配置
    % theorem-line 环境自定义
        \newtheoremstyle{MyLineTheoremStyle}% <name>
            {11pt}% <space above>
            {11pt}% <space below>
            {\kaishu}% <body font> 默认使用正文字体, \kaishu 为楷体
            {}% <indent amount>
            {\bfseries}% <theorem head font> 设置标题项为加粗
            {:\ \ }% <punctuation after theorem head>
            {.5em}% <space after theorem head>
            {\textbf{#1}\thmnumber{#2}\ \ (\,\textbf{#3}\,)}% 设置标题内容顺序
        \theoremstyle{MyLineTheoremStyle} % 应用自定义的定理样式
        \newtheorem{LineTheorem}{Theorem.\,}
    % theorem-block 环境自定义
        \newtheoremstyle{MyBlockTheoremStyle}% <name>
            {11pt}% <space above>
            {11pt}% <space below>
            {\kaishu}% <body font> 使用默认正文字体
            {}% <indent amount>
            {\bfseries}% <theorem head font> 设置标题项为加粗
            {:\\ \indent}% <punctuation after theorem head>
            {.5em}% <space after theorem head>
            {\textbf{#1}\thmnumber{#2}\ \ (\,\textbf{#3}\,)}% 设置标题内容顺序
        \theoremstyle{MyBlockTheoremStyle} % 应用自定义的定理样式
        \newtheorem{BlockTheorem}[LineTheorem]{Theorem.\,} % 使用 LineTheorem 的计数器
    % definition 环境自定义
        \newtheoremstyle{MySubsubsectionStyle}% <name>
            {11pt}% <space above>
            {11pt}% <space below>
            {}% <body font> 使用默认正文字体
            {}% <indent amount>
            {\bfseries}% <theorem head font> 设置标题项为加粗
            {:\\ \indent}% <punctuation after theorem head>
            {0pt}% <space after theorem head>
            {\textbf{#3}}% 设置标题内容顺序
        \theoremstyle{MySubsubsectionStyle} % 应用自定义的定理样式
        \newtheorem{definition}{}

%宏包:有色文本框(proof环境)及其设置
    \usepackage{xcolor}    %设置插入的文本框颜色
    \usepackage[strict]{changepage}     % 提供一个 adjustwidth 环境
    \usepackage{framed}     % 实现方框效果
        \definecolor{graybox_color}{rgb}{0.95,0.95,0.96} % 文本框颜色。修改此行中的 rgb 数值即可改变方框纹颜色,具体颜色的rgb数值可以在网站https://colordrop.io/ 中获得。(截止目前的尝试还没有成功过,感觉单位不一样)(找到喜欢的颜色,点击下方的小眼睛,找到rgb值,复制修改即可)
        \newenvironment{graybox}{%
        \def\FrameCommand{%
        \hspace{1pt}%
        {\color{gray}\vrule width 2pt}%
        {\color{graybox_color}\vrule width 4pt}%
        \colorbox{graybox_color}%
        }%
        \MakeFramed{\advance\hsize-\width\FrameRestore}%
        \noindent\hspace{-4.55pt}% disable indenting first paragraph
        \begin{adjustwidth}{}{7pt}%
        \vspace{2pt}\vspace{2pt}%
        }
        {%
        \vspace{2pt}\end{adjustwidth}\endMakeFramed%
        }

% 外源代码插入设置
    % matlab 代码插入设置
    \usepackage{matlab-prettifier}
        \lstset{style=Matlab-editor}    % 继承 matlab 代码高亮 , 此行不能删去
    \usepackage[most]{tcolorbox} % 引入tcolorbox包 
    \usepackage{listings} % 引入listings包
        \tcbuselibrary{listings, skins, breakable}
        \newfontfamily\codefont{Consolas} % 定义需要的 codefont 字体
        \lstdefinestyle{MatlabStyle_inc}{   % 插入代码的样式
            language=Matlab,
            basicstyle=\footnotesize\ttfamily\codefont,    % ttfamily 确保等宽 
            breakatwhitespace=false,
            breaklines=true,
            captionpos=b,
            keepspaces=true,
            numbers=left,
            numbersep=15pt,
            showspaces=false,
            showstringspaces=false,
            showtabs=false,
            tabsize=2,
            xleftmargin=15pt,   % 左边距
            %frame=single, % single 为包围式单线框
            frame=shadowbox,    % shadowbox 为带阴影包围式单线框效果
            %escapeinside=``,   % 允许在代码块中使用 LaTeX 命令 (此行无用)
            %frameround=tttt,    % tttt 表示四个角都是圆角
            framextopmargin=0pt,    % 边框上边距
            framexbottommargin=0pt, % 边框下边距
            framexleftmargin=5pt,   % 边框左边距
            framexrightmargin=5pt,  % 边框右边距
            rulesepcolor=\color{red!20!green!20!blue!20}, % 阴影框颜色设置
            %backgroundcolor=\color{blue!10}, % 背景颜色
        }
        \lstdefinestyle{MatlabStyle_src}{   % 插入代码的样式
            language=Matlab,
            basicstyle=\small\ttfamily\codefont,    % ttfamily 确保等宽 
            breakatwhitespace=false,
            breaklines=true,
            captionpos=b,
            keepspaces=true,
            numbers=left,
            numbersep=15pt,
            showspaces=false,
            showstringspaces=false,
            showtabs=false,
            tabsize=2,
        }
        \newtcblisting{matlablisting}{
            %arc=2pt,        % 圆角半径
            % 调整代码在 listing 中的位置以和引入文件时的格式相同
            top=0pt,
            bottom=0pt,
            left=-5pt,
            right=-5pt,
            listing only,   % 此句不能删去
            listing style=MatlabStyle_src,
            breakable,
            colback=white,   % 选一个合适的颜色
            colframe=black!0,   % 感叹号后跟不透明度 (为 0 时完全透明)
        }
        \lstset{
            style=MatlabStyle_inc,
        }

% table 支持
    \usepackage{booktabs}   % 宏包:三线表
    \usepackage{tabularray} % 宏包:表格排版
    \usepackage{longtable}  % 宏包:长表格

% figure 设置
    \usepackage{graphicx}  % 支持 jpg, png, eps, pdf 图片 
    \usepackage{svg}       % 支持 svg 图片
        \svgsetup{
            % 指向 inkscape.exe 的路径
            inkscapeexe = C:/aa_MySame/inkscape/bin/inkscape.exe, 
            % 一定程度上修复导入后图片文字溢出几何图形的问题
            inkscapelatex = false                 
        }
    \usepackage{subcaption} % 用于子图和小图注  

% 图表进阶设置
    \usepackage{caption}    % 图注、表注
        \captionsetup[figure]{name=图}  
        \captionsetup[table]{name=表}
        \captionsetup{
            labelfont=bf, % 设置标签为粗体
            textfont=bf,  % 设置文本为粗体
            font=small  
        }
    \usepackage{float}     % 图表位置浮动设置 
    \usepackage{etoolbox} % 用于保证图注表注的数学字符为粗体
        \AtBeginEnvironment{figure}{\boldmath} % 图注中的数学字符为粗体
        \AtBeginEnvironment{table}{\boldmath}  % 表注中的数学字符为粗体
        \AtBeginEnvironment{tabular}{\unboldmath}   % 保证表格中的数学字符不受额外影响

% 圆圈序号自定义
    \newcommand*\circled[1]{\tikz[baseline=(char.base)]{\node[shape=circle,draw,inner sep=0.8pt, line width = 0.03em] (char) {\bfseries #1};}}   % TikZ solution

% 列表设置
    \usepackage{enumitem}   % 宏包:列表环境设置
        \setlist[enumerate]{
            label=(\arabic*) ,   % 设置序号样式为加粗的 (1) (2) (3)
            ref=\arabic*, % 如果需要引用列表项,这将决定引用格式(这里仍然使用数字)
            itemsep=0pt, parsep=0pt, topsep=0pt, partopsep=0pt, leftmargin=3.5em} 
        \setlist[itemize]{itemsep=0pt, parsep=0pt, topsep=0pt, partopsep=0pt, leftmargin=3.5em}
        \newlist{circledenum}{enumerate}{1} % 创建一个新的枚举环境  
        \setlist[circledenum,1]{  
            label=\protect\circled{\arabic*}, % 使用 \arabic* 来获取当前枚举计数器的值,并用 \circled 包装它  
            ref=\arabic*, % 如果需要引用列表项,这将决定引用格式(这里仍然使用数字)
            itemsep=0pt, parsep=0pt, topsep=0pt, partopsep=0pt, leftmargin=3.5em
        }  

% 其它设置
    % 脚注设置
        \renewcommand\thefootnote{\ding{\numexpr171+\value{footnote}}}
    % 参考文献引用设置
        \bibliographystyle{unsrt}   % 设置参考文献引用格式为unsrt
        %\bibliographystyle{ieeetr}   % 设置参考文献引用格式为unsrt
        \newcommand{\upcite}[1]{\textsuperscript{\cite{#1}}}     % 自定义上角标式引用
    % 文章序言设置
        \newcommand{\cnabstractname}{序言}
        \newenvironment{cnabstract}{%
            \par\Large
            \noindent\mbox{}\hfill{\bfseries \cnabstractname}\hfill\mbox{}\par
            \vskip 2.5ex
            }{\par\vskip 2.5ex}

% 文章默认字体设置
    \usepackage{fontspec}   % 宏包:字体设置
        \setmainfont{SimSun}    % 设置中文字体为宋体字体
        \setCJKmainfont[AutoFakeBold=3]{SimSun} % 设置加粗字体为 SimSun 族,AutoFakeBold 可以调整字体粗细
        \setmainfont{Times New Roman} % 设置英文字体为Times New Roman

% 各级标题自定义设置
    \usepackage{titlesec}   
        % section标题自定义设置 
        \titleformat{\section}[hang]{\normalfont\Large\bfseries\boldmath}{\thesection}{8pt}{}
        % subsection 标题自定义设置
        \titleformat{\subsection}[hang]{\normalfont\large\bfseries\boldmath}{\thesubsection}{8pt}{}
        \titlespacing*{\subsection}{0pt}{10pt}{6pt} % 控制上下间距


% --------------------- 文章宏包及相关设置 --------------------- %
% >> ------------------ 文章宏包及相关设置 ------------------ << %


% ------------------------ 文章信息区 ------------------------ %
% ------------------------ 文章信息区 ------------------------ %
% 页眉页脚设置
\usepackage{fancyhdr}   %宏包:页眉页脚设置
\usepackage{fontawesome}    % 宏包:更多符号与图标 (用于插入 GitHub 图标等)
    \pagestyle{fancy}
    \fancyhf{}
    \cfoot{\thepage}
    \renewcommand\headrulewidth{1pt}
    \renewcommand\footrulewidth{0pt}
    \chead{Lab 2: 2024.11.21 - 2024.12.12, 丁毅}
    \lhead{\small
    \href{https://github.com/YiDingg/LatexNotes}{\color{black}\faGithub\ https://github.com/YiDingg/LatexNotes}
    }
    \rhead{dingyi233@mails.ucas.ac.cn}

% 开始编辑文章

\begin{document}
\begin{center}\large
    \vspace*{-0.8cm}
    \noindent{\huge\bfseries Laboratory \ \ 2 :\ \  2024.11.21 - 2024.12.12}\\\vspace{0.2cm}
    \noindent{\Large\bfseries 《\ 电\ 路\ 原\ 理》\ \ \ 实\ 验\ 报\ 告 }\\
    %\noindent{\large\bfseries 2024.11.07 - 2024.12.31}
\end{center}
\vspace{-0.5cm}
\noindent\rule{\textwidth}{0.075em}   % 分割线
\vspace{-1.0cm}

% 目录
%\setcounter{tocdepth}{2}  % 目录深度为 2(不显示 subsubsection)
%\noindent\tableofcontents\thispagestyle{fancy}   % 显示页码、页眉等
%\newpage
% ------------------------ 文章信息区 ------------------------ %
% ------------------------ 文章信息区 ------------------------ %


%% 下面是正文内容

\captionsetup[figure]{name=Figure}  
\captionsetup[table]{name=Table}

\section{Pre-Lab}

\subsection{Voltage-Voltage Characteristics of Inverting Amplifier}
The output voltage $V_{\text{out}}$ as a function of input voltage $V_\text{in}$ is shown in Fig.\ref{Voltage-Voltage Characteristics}, where $V_\text{T}$ is the threshold voltage of teh MOSFET.
\begin{figure}[H]\centering
    \includegraphics[width=0.7\columnwidth]{assets/Voltage-Voltage Characteristics of Inverting Amplifier.pdf}
    \caption{Voltage-Voltage Characteristics of Inverting Amplifier}
    \label{Voltage-Voltage Characteristics}
\end{figure}
For easy reference, we summarize the formula as:
\begin{gather}
V_{\text{out}} = 
\begin{cases}
\begin{aligned}
    &V_\text{S}, && V_\text{GS} \in [0, V_\text{T}) && \text{(Cut-Off Region)} \\ 
    &V_S - \frac{KR_L}{2}\left(V_\text{GS} - V_\text{T}\right)^2, && V_\text{GS} \in [V_\text{T}, V_0) &&\text{(Saturation Region)} \\ 
    &V_\text{GS} - V_\text{T} + \frac{1 - \sigma}{KR_L}, && V_\text{GS} \in [V_0, V_0 + \Delta V) &&\text{(Transition Region)} \\ 
    &\frac{V_\text{S}}{KR_L (V_\text{GS} - V_\text{T})},\quad && V_\text{GS} \in [V_0 + \Delta V, V_{\max}] &&\text{(Linear Region)} \\
\end{aligned}
\end{cases}
\end{gather}
Where $V_0$ is the second solution $V_{\text{GS}, 2}$ of equations:
\begin{equation}
\begin{cases}
    V_\text{GS} = V_S - \frac{KR_L}{2}\left(V_\text{GS} - V_\text{T}\right)^2 \\ 
    V_\text{GD} = V_\text{T}
\end{cases}\Longrightarrow 
V_0 = \frac{\sqrt{ 2KR_LV_S + 1 } + KR_L V_T - 1}{K R_L}
\end{equation}
And $\sigma$ is the discriminant of the quadratic equation:
\begin{equation}
\sigma = \sqrt{
K^2R_L^2 V_\text{GS}^2 + 2KR_L\left(1 - KR_L V_T\right) V_\text{GS} + \left[ K^2R_L^2 V_\text{T}^2 - 2KR_L( V_\text{T} + V_\text{S}) + 1\right] 
}
\end{equation}
When the input voltage $V_\text{in} (V_{\text{GS}}) > V_0$ is large enough, i.e. $V_\text{in} \geqslant V_0 + \Delta V$, an approximation can be made:
\begin{equation}
I_\text{DS} \approx K \left(V_\text{GS} - V_\text{T} \right) V_\text{DS} \Longrightarrow V_{\text{out}} = \frac{V_S}{KR_L (V_\text{GS} - V_\text{T})}
\end{equation}


\subsection{Small Voltage Gain of Inverting Amplifier}
We have alread driven the small signal voltage gain during the last homework. Assuming the small AC input voltage is $u_{\text{in}}$, and MOS is biased into saturation region, it follows that:
\begin{equation}
A = \frac{u_{\text{out}}}{u_{\text{in}}} = -g_m R_L = - K \left(V_\text{GS} - V_\text{T}\right)R_L
\end{equation}


\subsection{RC Transient Process}
By the three-element method, we can obtain:
\begin{equation}
V_\text{out} = \frac{R_2 V_\text{I}}{R_1 + R_2} \left( 1 - e^{-\frac{t}{\tau}}\right),\quad \tau = \left(R_1 \parallel R_2 \right) C = \frac{R_1 R_2}{R_1 + R_2} C
\end{equation}
\subsection{Transient Time}
Given $V_\text{T}$ in the range $[0, V_\text{S}]$ ($ V_\text{S} = \frac{R_2 V_{\text{I}}}{R_1 + R_2}$), the time where $V_\text{out}$ reaches $V_\text{T}$ is:
\begin{equation}
\Delta t = \tau \,\ln \left( \frac{V_\text{S}}{V_\text{S} - V_\text{T}} \right),\quad \tau = \frac{R_1 R_2}{R_1 + R_2} C,\ \ V_\text{S} = \frac{R_2 V_{\text{I}}}{R_1 + R_2}
\end{equation}

\section{In-Lab}
\subsection{Static Input-Output Relationship of Inverting Amplifier}
\subsubsection{Measure In-Out Voltage Relationship}\label{sec.voltage}
Construct the circuit in Fig.\ref{In-Lab 2-1}, then we can obtain the voltage relationship shown in Fig.\ref{Operational Characteristics}.
\begin{figure}[H]\centering
    \includegraphics[width=0.45\columnwidth]{assets/In-Lab 2-1.png}
    \caption{Measure In-Out Voltage Relationship}
    \label{In-Lab 2-1}
\end{figure}
\begin{figure}[H]\centering
    \includegraphics[width=0.9\columnwidth]{assets/Voltage-Voltage Characteristics of Inverting Amplifier (Experimental) 2.pdf}
    \caption{Operational Characteristics of Inverting Amplifier}
    \label{Operational Characteristics}
\end{figure}

\subsubsection{The Threshold of The MOSFET}
With $V_\text{S} = 5.0185 \ \mathrm{V},\ R_L = 1 \KO\ (998\ \Omega)$ and the data obtained in the last section, we can get the threshold voltage of the MOSFET:
\begin{gather}
\boldmath
V_T = 1.85 \ \mathrm{V},\quad V_0 = 2.3934 \ \mathrm{V},\quad \Delta V = 0.2066 \ \mathrm{V}
\end{gather}

\subsubsection{Tables of Input-Output Voltage}
\begin{table}[H]\centering
    %\renewcommand{\arraystretch}{1.5} % 调整行间距为 1.5 倍
    %\setlength{\tabcolsep}{1.5mm} % 调整列间距
    \caption{Input-Output Voltage Relationship}
    \label{Input-Output Voltage Relationship}
\begin{tabular}{cccccccccc}\toprule
    Output (V) & 5 & 4 & 3 & 2 & 1 & 0.0116\\
    \midrule
    Input (V) & 0.2765 & 2.1767 & 2.2686 & 2.3238 & 2.3712 & 4.9599 \\
    \bottomrule
\end{tabular}
\end{table}


\subsection{Small Signal Voltage Gain}
\subsubsection{Voltage Gain of Inverting Amplifier}\label{sec.voltage gain}
With the data measured in section \ref{sec.voltage}, we can derived the voltage gain $A_v = \frac{\mathrm{d} V_\text{DS} }{\mathrm{d} V_\text{GS} }$ via matrix difference (see Fig.\ref{Voltage Gain}).
\begin{figure}[H]\centering
    \includegraphics[width=0.9\columnwidth]{assets/Voltage Gain of Inverting Amplifier.pdf}
    \caption{Small Signal Voltage Gain of Inverting Amplifier}
    \label{Voltage Gain}
\end{figure}
Since transconductance satisfies $A = -g_m R_L$, we can also get the transconductance by dividing the limiting resistance $R_L = 1 \KO$ (998 $\Omega$).

Construct the circuit to measure the voltage gain where the output voltage is 2 V and the sine wave has 50 mV amplitude (from -50 mV to + 50 mV). The measured result and the data in Fig.\ref{Voltage Gain} is:
\begin{equation}
\left(A_v\right)_{\text{meas}} = \frac{-1.0056 \ \mathrm{V}}{50 \ \mathrm{mV}} = -20.1120,\quad \left(A_v\right)_{\text{fig}}  = -20.1513
\end{equation}
As we can see, almost no deviation.

\subsubsection{Clipping Distortion}
Set the amplitude of sine wave to 50 mV ($[-50 \ \mathrm{mV}, +50 \ \mathrm{mV}]$), the lower and upper bias limits are (see Fig.\ref{The Lower Limit} and Fig.\ref{The Upper Limit}):
\begin{equation}
V_{\text{bias}, \min} = 2.00 \ \mathrm{V},\quad V_{\text{bias}, \max} = 2.39 \ \mathrm{V}
\end{equation}
\begin{figure}[H]\centering
    \includegraphics[width=0.99\columnwidth]{assets/lower bias limit.png}
    \caption{The Lower Limit of The Input Bias Voltage}
    \label{The Lower Limit}
\end{figure}
\begin{figure}[H]\centering
    \includegraphics[width=0.99\columnwidth]{assets/upper bias limit.png}
    \caption{The Upper Limit of The Input Bias Voltage}
    \label{The Upper Limit}
\end{figure}

\subsection{The Delay of Inverting Amplifier as a Digital Inverter}
\subsubsection{Measure The Gate to Source Capacitance of Inverting Amplifier}
Use a 500 $\KO$ resistor and a 30 $\KO$ resistor to measure the gate to source capacitance $C_{\text{GS}}$ of the MOSFET, respectively. The measured results are:
\begin{gather}
R_L = 500 \KO ,\quad \lim_{t \to 0^+} \frac{\mathrm{d} V_{\text{out}} }{\mathrm{d} t } = 57070.3 \ \mathrm{V\cdot s^{-1}},\quad V_{\text{steady}} =  3.3198 \ \mathrm{V} 
\Longrightarrow C = \frac{V_0}{R_L k_{0^+}} = 116.3407 \pF
\\
R_L = 30.0 \KO ,\lim_{t \to 0^+} \frac{\mathrm{d} V_{\text{out}} }{\mathrm{d} t } = 948889.8 \ \mathrm{V\cdot s^{-1}},\quad V_{\text{steady}} = 4.9032 \ \mathrm{V} 
\Longrightarrow C = \frac{V_0}{R_L k_{0^+}} = 172.2434 \pF
\end{gather}
\begin{figure}[H]\centering
    \includegraphics[width=0.88\columnwidth]{assets/C_GS, 500 KOhm.png}
    \caption{Gate to Source Capacitance of 2N7000, $R_L = 500 \KO$}
    \label{500 KOhm}
\end{figure}
\begin{figure}[H]\centering
    \includegraphics[width=0.88\columnwidth]{assets/C_GS, 30.0 KOhm.png}
    \caption{Gate to Source Capacitance of 2N7000, $R_L = 30 \KO$}
\end{figure}
It is puzzling that $C = $ with $R_L = 500 \KO$. Actually, with the resistance $R_L = 500 \KO$, the steady voltage $V_0$ is limited to 3.3198 V (see Fig.\ref{500 KOhm}), which is not the expected value. Why is it like that? Because the oscilloscope input resistance is not large enough. We keep this to the Post-Lab, 

\begin{figure}[H]\centering
\begin{subfigure}[b]{0.5\columnwidth}\centering
    \includegraphics[height=175pt]{assets/C_GS, 30.0 KOhm Circuit.png}
    \caption{The Gate to Source Capacitance of Inverting Amplifier}
\end{subfigure}\hfill
\begin{subfigure}[b]{0.5\columnwidth}\centering
    \includegraphics[height=175pt]{assets/Delay.png}
    \caption{The Delay of Inverting Amplifier as a Digital Inverter}
\end{subfigure}
\caption{In-Lab 3.1 and In-Lab 3.2}
\label{In-Lab 3.1 and In-Lab 3.2}
\end{figure}

\subsubsection{Measure The Delay of Inverting Amplifier}
With $R_{L, 1} = 30 \KO$ and $R_{L, 2} = 1 \KO$, construct the circuit in Fig.\ref{In-Lab 3.1 and In-Lab 3.2} (b), obtain the delay time of the inverting amplifier as a digital inverter (see Fig.\ref{Starts to Fall} and Fig.\ref{Reaches to Low}).
\begin{equation}
\text{start to fall:\ \ } \Delta t_1 = 1.504 \ \mathrm{\mu s},\quad \text{reach low:\ \ } \Delta t_2 = 3.799 \ \mathrm{\mu s}
\end{equation}
\begin{figure}[H]\centering
    \includegraphics[width=\columnwidth]{assets/Start to fall.png}
    \caption{The Output Voltage Starts to Fall}
    \label{Starts to Fall}
\end{figure}
\begin{figure}[H]\centering
    \includegraphics[width=\columnwidth]{assets/Reach low.png}
    \caption{The Output Voltage Reaches to Low}
    \label{Reaches to Low}
\end{figure}

It is interesting that resonance phenomena occurs, see Fig.\ref{Resonance 1} and Fig.\ref{Resonance 2}.
\begin{figure}[H]\centering
    \includegraphics[width=\columnwidth]{assets/Resonance 1.png}
    \caption{Input Voltage (Yellow) and Output Voltage (Blue) of The First MOS}
    \label{Resonance 1}
\end{figure}
\begin{figure}[H]\centering
    \includegraphics[width=\columnwidth]{assets/Resonance 2.png}
    \caption{Input Voltage (Yellow) and Output Voltage (Blue) of The Second MOS}
    \label{Resonance 2}
\end{figure}

\section{Post-Lab}
\subsection{Voltage-Voltage Characteristics Comparison}
\subsubsection{Transconductance Sensitivity $K$}

Use the voltage-voltage characteristics data in section \ref{sec.voltage} to calculate the transconductance sensitivity $K$ of the MOSFET:
\begin{equation}
    \begin{cases}
        I_\text{DS} = \frac{K}{2} \left(V_\text{GS} - V_\text{T}\right)^2 \\ 
        V_\text{DS} = V_\text{S} - I_\text{DS} R_L
    \end{cases}
\Longrightarrow K = \frac{2 \left(V_\text{S} - V_\text{DS}\right)}{R_L \left(V_\text{GS} - V_\text{T}\right)^2}
\end{equation}
Where $V_\text{T} = 1.85 \ \mathrm{V}$, $R_L = 1 \KO\ (998\ \Omega)$ and $V_\text{S} = 5.0185 \ \mathrm{V}$. Plot the curve $K = K(V_\text{GS})$, as shown in Fig.\ref{Transconductance Sensitivity} (a):
\begin{figure}[H]\centering
\begin{subfigure}[b]{0.5\columnwidth}\centering
    \includegraphics[height=175pt]{assets/K.pdf}
    \caption{Transconductance Sensitivity $K$ as a Function of $V_\text{GS}$}
\end{subfigure}\hfill
\begin{subfigure}[b]{0.5\columnwidth}\centering
    \includegraphics[height=175pt]{assets/I_DS-V_GS fit K.pdf}
    \caption{Drain to Source Current as a Function of $V_\text{GS}$ }
\end{subfigure}
\caption{Transconductance Sensitivity}
\label{Transconductance Sensitivity}
\end{figure}
As we can see, $K$ is not a ideal constant value. So we try fitting $I_\text{DS}$ as a function of $V_\text{DS}$, and the result is shown in Fig.\ref{Transconductance Sensitivity} (b), $R^2 = 0.9511$.
\begin{equation}
I_\text{DS} = \frac{K}{2}\left(V_\text{GS} - V_\text{T}\right)^2,\quad \mathbf{K = 0.02572 \ A\cdot V^{-2}},\quad V_\text{T} = 1.85 \ \mathrm{V}
\end{equation}
It is funny that the fitting result is exllecent if we use $I_\text{DS} = \frac{K}{2}\left(V_\text{GS} - V_\text{T}\right)^3$, which has a high $R^2$ = 0.9977.

\subsubsection{Comparison of Operational Characteristics}

With the four parameters $V_T = 1.85 \ \mathrm{V}$, $K = 0.02572 \ \mathrm{A\cdot V^{-2}}$, $V_\text{S} = 5 \ \mathrm{V}$ and $R_L = 1 \KO\ (998 \ \Omega)$, we can compute and plot the theoretical operational characteristics of the inverting amplifier, as shown in Fig.\ref{Operational Characteristics Comparison}. Below are the other parameters for our theoretical model:
\begin{gather}
V_0 = \frac{\sqrt{ 2KR_LV_S + 1 } + KR_L V_T - 1}{K R_L} = 2.4376 \ \mathrm{V},\quad \Delta V = 1.5\, (V_0 - V_\text{T}) = 0.8814 \mathrm{V}
\end{gather}
\begin{figure}[H]\centering
    \includegraphics[width=0.9\columnwidth]{assets/Theoretical model.pdf}
    \caption{Operational Characteristics Comparison}
    \label{Operational Characteristics Comparison}
\end{figure}


\subsection{Small Signal Voltage Gain}
The voltage gain $A_v$ measured during In-Lab 2-2 (section \ref{sec.voltage gain}), by the operational characteristics obtained in In-Lab 2-1 (section \ref{sec.voltage}) and the theoretical module from Pre-Lab are respectively (at Output $V_\text{GS} = 2 \ \mathrm{V}$):
\begin{table}[H]\centering
    %\renewcommand{\arraystretch}{1.5} % 调整行间距为 1.5 倍
    %\setlength{\tabcolsep}{1.5mm} % 调整列间距
    \caption{Voltage Gain at Output 2 V by Different Methods}
    \label{Voltage Gain by Different Methods}
\begin{tabular}{cccccccccc}\toprule
    By Experimental Measurement & By Operational Characteristics & By Theoretical Op Characteristics \\
    \midrule
    -20.1120 & -20.1513 & -12.1670  \\
    \bottomrule
\end{tabular}
\end{table}

\subsection{Capacitance and Delay Analysis}
\subsubsection{Total Input Capacitance}
In Fig.\ref{500 KOhm}, we have seen that the output voltage drops to 3.3198 V, when $R_L = 500 \KO$ and $V_\text{S} = 5$ V. It follows that:
\begin{equation}
V_\text{out} = \frac{R_{\text{osci}}}{R_{\text{osci}} + R_L}V_\text{S} \Longrightarrow R_{\text{osci}} = \frac{R_L}{\frac{V_\text{S}}{V_\text{out}} - 1} = 987.9 \KO \approx 1 \mathrm{M}\Omega
\end{equation}
Assuming $V_\text{source} = 5$ V, we can obtain the total input capacitance, including GS capacitance $C_{\text{GS}}$ and oscilloscope input capacitance $C_{\text{osci}}$:
\begin{equation}
    \begin{cases}
        k_{0^+} = \frac{V_\text{steady}}{\tau}\\
        V_\text{steady} = \frac{R_{\text{osci}}}{R_{\text{osci}} + R_L}\cdot V_\text{source}\\
        \tau = \left( R_{\text{osci}} \parallel R_L \right) \left(C_{\text{GS}} + C_{\text{osci}} \right)
    \end{cases}
    \Longrightarrow 
    C_{\text{GS}} + C_{\text{osci}}= \frac{V_\text{S}}{k_{0^+}R_L} = 175.2225 \pF
\end{equation}
If oscilloscope input capacitance $C_{\text{osci}}$ is about 15 pF, then we have $C_{\text{GS}} \approx 160 \pF$.

\subsubsection{The Delay Used as a Digital Inverter} 
Let $C_{\text{GS}} \approx 160 \pF$, $V_\text{T} = 1.85 \ \mathrm{V}$, $V_0 + \Delta V = 3.22 \ \mathrm{V}$ $V_\text{I} = 5 \ \mathrm{V}$,  yielding:
\begin{equation}
    \begin{cases}
        \Delta t = \tau \,\ln \left(\frac{V_\text{S}}{V_\text{S}} - V_\text{T}\right) \\ 
        \tau = \frac{R_1 R_2}{R_1 + R_2} C \\ 
        V_\text{S} = \frac{R_2}{R_1 + R_2} \cdot V_\text{I}
    \end{cases}\Longrightarrow 
\text{start to fall:\ } \Delta t_1 = 0.1287 \mathrm{\mu s},\quad \text{reach low:\ } \Delta t_2 = 1.3040  \mathrm{\mu s}
\end{equation}

\captionsetup[figure]{name=图}  
\captionsetup[table]{name=表}



\end{document}

% VScode 常用快捷键:

% F2:                       变量重命名
% Ctrl + Enter:             行中换行
% Alt + up/down:            上下移行
% 鼠标中键 + 移动:           快速多光标
% Shift + Alt + up/down:    上下复制
% Ctrl + left/right:        左右跳单词
% Ctrl + Backspace/Delete:  左右删单词    
% Shift + Delete:           删除此行
% Ctrl + J:                 打开 VScode 下栏(输出栏)
% Ctrl + B:                 打开 VScode 左栏(目录栏)
% Ctrl + `:                 打开 VScode 终端栏
% Ctrl + 0:                 定位文件
% Ctrl + Tab:               切换已打开的文件(切标签)
% Ctrl + Shift + P:         打开全局命令(设置)

% Latex 常用快捷键:

% Ctrl + Alt + J:           由代码定位到PDF


