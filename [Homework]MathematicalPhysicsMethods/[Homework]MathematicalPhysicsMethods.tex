% 若编译失败,且生成 .synctex(busy) 辅助文件,可能有两个原因:
% 1. 需要插入的图片不存在:Ctrl + F 搜索 'figure' 将这些代码注释/删除掉即可
% 2. 路径/文件名含中文或空格:更改路径/文件名即可

% --------------------- 文章宏包及相关设置 --------------------- %
% >> ------------------ 文章宏包及相关设置 ------------------ << %
% 设定文章类型与编码格式
\documentclass[UTF8]{report}		

% 本文的特殊宏定义
    \def\Im{\mathrm{\,Im\,}}
    \def\Re{\mathrm{\,Re\,}}
    \def\Ln{\mathrm{\,Ln\,}}
    \def\Arg{\mathrm{\,Arg\,}}
    \def\Arccos{\mathrm{\,Arccos\,}}
    \def\Arcsin{\mathrm{\,Arcsin\,}}
    \def\Arctan{\mathrm{\,Arctan\,}}
    \def\res{\mathrm{res\,}}

% 通用宏定义
    \def\N{\mathbb{N}}
    \def\F{\mathbb{F}}
    \def\Z{\mathbb{Z}}
    \def\Q{\mathbb{Q}}
    \def\R{\mathbb{R}}
    \def\C{\mathbb{C}}
    \def\T{\mathbb{T}}
    \def\S{\mathbb{S}}
    \def\A{\mathbb{A}}
    \def\I{\mathscr{I}}
    \def\d{\mathrm{d}}
    \def\p{\partial}

% 导入基本宏包
    \usepackage[UTF8]{ctex}     % 设置文档为中文语言
        \usepackage{hyperref}  % 宏包:自动生成超链接 (此宏包与标题中的数学环境冲突)
    \hypersetup{
        colorlinks=true,    % false:边框链接 ; true:彩色链接
        citecolor={blue},    % 文献引用颜色
        linkcolor={blue},   % 目录 (我们在目录处单独设置),公式,图表,脚注等内部链接颜色
        urlcolor={magenta},    % 网页 URL 链接颜色,包括 \href 中的 text
        % cyan 浅蓝色 
        % magenta 洋红色
        % yellow 黄色
        % black 黑色
        % white 白色
        % red 红色
        % green 绿色
        % blue 蓝色
        % gray 灰色
        % darkgray 深灰色
        % lightgray 浅灰色
        % brown 棕色
        % lime 石灰色
        % olive 橄榄色
        % orange 橙色
        % pink 粉红色
        % purple 紫色
        % teal 蓝绿色
        % violet 紫罗兰色
    }
    % \usepackage{docmute}    % 宏包:子文件导入时自动去除导言区,用于主/子文件的写作方式,\include{./51单片机笔记}即可。注:启用此宏包会导致.tex文件capacity受限。
    \usepackage{amsmath}    % 宏包:数学公式
    \usepackage{mathrsfs}   % 宏包:提供更多数学符号
    \usepackage{amssymb}    % 宏包:提供更多数学符号
    \usepackage{pifont}     % 宏包:提供了特殊符号和字体
    \usepackage{extarrows}  % 宏包:更多箭头符号


% 文章页面margin设置
    \usepackage[a4paper]{geometry}
        \geometry{top=1in}
        \geometry{bottom=1in}
        \geometry{left=0.75in}
        \geometry{right=0.75in}   % 设置上下左右页边距
        \geometry{marginparwidth=1.75cm}    % 设置边注距离(注释、标记等)

% 配置数学环境
    %\everymath{\displaystyle}   % 设置全文数学公式都为展示样式
    \usepackage{amsthm} % 宏包:数学环境配置
    % theorem-line 环境自定义
        \newtheoremstyle{MyLineTheoremStyle}% <name>
            {11pt}% <space above>
            {11pt}% <space below>
            {\kaishu}% <body font> 使用默认正文字体
            {}% <indent amount>
            {\bfseries}% <theorem head font> 设置标题项为加粗
            {:}% <punctuation after theorem head>
            {.5em}% <space after theorem head>
            {\textbf{#1}\thmnumber{#2}\ \ (\,\textbf{#3}\,)}% 设置标题内容顺序
        \theoremstyle{MyLineTheoremStyle} % 应用自定义的定理样式
        \newtheorem{LineTheorem}{Theorem.\,}
    % theorem-block 环境自定义
        \newtheoremstyle{MyBlockTheoremStyle}% <name>
            {11pt}% <space above>
            {11pt}% <space below>
            {\kaishu}% <body font> 使用默认正文字体
            {}% <indent amount>
            {\bfseries}% <theorem head font> 设置标题项为加粗
            {:\\ \indent}% <punctuation after theorem head>
            {.5em}% <space after theorem head>
            {\textbf{#1}\thmnumber{#2}\ \ (\,\textbf{#3}\,)}% 设置标题内容顺序
        \theoremstyle{MyBlockTheoremStyle} % 应用自定义的定理样式
        \newtheorem{BlockTheorem}[LineTheorem]{Theorem.\,} % 使用 LineTheorem 的计数器
    % definition 环境自定义
        \newtheoremstyle{MySubsubsectionStyle}% <name>
            {11pt}% <space above>
            {11pt}% <space below>
            {}% <body font> 使用默认正文字体
            {}% <indent amount>
            {\bfseries}% <theorem head font> 设置标题项为加粗
            {:\\ \indent}% <punctuation after theorem head>
            {0pt}% <space after theorem head>
            {\textbf{#3}}% 设置标题内容顺序
        \theoremstyle{MySubsubsectionStyle} % 应用自定义的定理样式
        \newtheorem{definition}{}

%宏包:有色文本框(proof环境)及其设置
    \usepackage[dvipsnames,svgnames]{xcolor}    %设置插入的文本框颜色
    \usepackage[strict]{changepage}     % 提供一个 adjustwidth 环境
    \usepackage{framed}     % 实现方框效果
        \definecolor{graybox_color}{rgb}{0.95,0.95,0.96} % 文本框颜色。修改此行中的 rgb 数值即可改变方框纹颜色,具体颜色的rgb数值可以在网站https://colordrop.io/ 中获得。(截止目前的尝试还没有成功过,感觉单位不一样)(找到喜欢的颜色,点击下方的小眼睛,找到rgb值,复制修改即可)
        \newenvironment{graybox}{%
        \def\FrameCommand{%
        \hspace{1pt}%
        {\color{gray}\small \vrule width 2pt}%
        {\color{graybox_color}\vrule width 4pt}%
        \colorbox{graybox_color}%
        }%
        \MakeFramed{\advance\hsize-\width\FrameRestore}%
        \noindent\hspace{-4.55pt}% disable indenting first paragraph
        \begin{adjustwidth}{}{7pt}%
        \vspace{2pt}\vspace{2pt}%
        }
        {%
        \vspace{2pt}\end{adjustwidth}\endMakeFramed%
        }

% 外源代码插入设置
    % matlab 代码插入设置
    \usepackage{matlab-prettifier}
        \lstset{style=Matlab-editor}    % 继承 matlab 代码高亮 , 此行不能删去
    \usepackage[most]{tcolorbox} % 引入tcolorbox包 
    \usepackage{listings} % 引入listings包
        \tcbuselibrary{listings, skins, breakable}
        \newfontfamily\codefont{Consolas} % 定义需要的 codefont 字体
        \lstdefinestyle{MatlabStyle_inc}{   % 插入代码的样式
            language=Matlab,
            basicstyle=\small\ttfamily\codefont,    % ttfamily 确保等宽 
            breakatwhitespace=false,
            breaklines=true,
            captionpos=b,
            keepspaces=true,
            numbers=left,
            numbersep=15pt,
            showspaces=false,
            showstringspaces=false,
            showtabs=false,
            tabsize=2,
            xleftmargin=15pt,   % 左边距
            %frame=single, % single 为包围式单线框
            frame=shadowbox,    % shadowbox 为带阴影包围式单线框效果
            %escapeinside=``,   % 允许在代码块中使用 LaTeX 命令 (此行无用)
            %frameround=tttt,    % tttt 表示四个角都是圆角
            framextopmargin=0pt,    % 边框上边距
            framexbottommargin=0pt, % 边框下边距
            framexleftmargin=5pt,   % 边框左边距
            framexrightmargin=5pt,  % 边框右边距
            rulesepcolor=\color{red!20!green!20!blue!20}, % 阴影框颜色设置
            %backgroundcolor=\color{blue!10}, % 背景颜色
        }
        \lstdefinestyle{MatlabStyle_src}{   % 插入代码的样式
            language=Matlab,
            basicstyle=\small\ttfamily\codefont,    % ttfamily 确保等宽 
            breakatwhitespace=false,
            breaklines=true,
            captionpos=b,
            keepspaces=true,
            numbers=left,
            numbersep=15pt,
            showspaces=false,
            showstringspaces=false,
            showtabs=false,
            tabsize=2,
        }
        \newtcblisting{matlablisting}{
            %arc=2pt,        % 圆角半径
            % 调整代码在 listing 中的位置以和引入文件时的格式相同
            top=0pt,
            bottom=0pt,
            left=-5pt,
            right=-5pt,
            listing only,   % 此句不能删去
            listing style=MatlabStyle_src,
            breakable,
            colback=white,   % 选一个合适的颜色
            colframe=black!0,   % 感叹号后跟不透明度 (为 0 时完全透明)
        }
        \lstset{
            style=MatlabStyle_inc,
        }

% table 支持
    \usepackage{booktabs}   % 宏包:三线表
    \usepackage{tabularray} % 宏包:表格排版
    \usepackage{longtable}  % 宏包:长表格

% figure 设置
    \usepackage{graphicx}  % 支持 jpg, png, eps, pdf 图片 
    \usepackage{svg}       % 支持 svg 图片
        \svgsetup{
            % 指向 inkscape.exe 的路径
            inkscapeexe = C:/aa_MySame/inkscape/bin/inkscape.exe, 
            % 一定程度上修复导入后图片文字溢出几何图形的问题
            inkscapelatex = false                 
        }

% 图表进阶设置
    \usepackage{caption}    % 图注、表注
        \captionsetup[figure]{name=图}  
        \captionsetup[table]{name=表}
        \captionsetup{
            labelfont=bf, % 设置标签为粗体
            textfont=bf,  % 设置文本为粗体
            font=small  
        }
    \usepackage{float}     % 图表位置浮动设置 
    \usepackage{etoolbox} % 用于保证图注表注的数学字符为粗体
        \AtBeginEnvironment{figure}{\boldmath} % 图注中的数学字符为粗体
        \AtBeginEnvironment{table}{\boldmath}  % 表注中的数学字符为粗体
        \AtBeginEnvironment{tabular}{\unboldmath}   % 保证表格中的数学字符不受额外影响
% 圆圈序号自定义
    \newcommand*\circled[1]{\tikz[baseline=(char.base)]{\node[shape=circle,draw,inner sep=0.8pt, line width = 0.03em] (char) {\small \bfseries #1};}}   % TikZ solution

% 列表设置
    \usepackage{enumitem}   % 宏包:列表环境设置
        \setlist[enumerate]{
            label=\bfseries(\arabic*) ,   % 设置序号样式为加粗的 (1) (2) (3)
            ref=\arabic*, % 如果需要引用列表项,这将决定引用格式(这里仍然使用数字)
            itemsep=0pt, parsep=0pt, topsep=0pt, partopsep=0pt, leftmargin=3.5em} 
        \setlist[itemize]{itemsep=0pt, parsep=0pt, topsep=0pt, partopsep=0pt, leftmargin=3.5em}
        \newlist{circledenum}{enumerate}{1} % 创建一个新的枚举环境  
        \setlist[circledenum,1]{  
            label=\protect\circled{\arabic*}, % 使用 \arabic* 来获取当前枚举计数器的值,并用 \circled 包装它  
            ref=\arabic*, % 如果需要引用列表项,这将决定引用格式(这里仍然使用数字)
            itemsep=0pt, parsep=0pt, topsep=0pt, partopsep=0pt, leftmargin=3.5em
        }  
        

% 文章默认字体设置
    \usepackage{fontspec}   % 宏包:字体设置
        \setmainfont{SimSun}    % 设置中文字体为宋体字体
        \setCJKmainfont[AutoFakeBold=3]{SimSun} % 设置加粗字体为 SimSun 族,AutoFakeBold 可以调整字体粗细
        \setmainfont{Times New Roman} % 设置英文字体为Times New Roman

% 其它设置
    % 脚注设置
    \renewcommand\thefootnote{\ding{\numexpr171+\value{footnote}}}
    % 参考文献引用设置
        \bibliographystyle{unsrt}   % 设置参考文献引用格式为unsrt
        \newcommand{\upcite}[1]{\textsuperscript{\cite{#1}}}     % 自定义上角标式引用
    % 文章序言设置
        \newcommand{\cnabstractname}{序言}
        \newenvironment{cnabstract}{%
            \par\Large
            \noindent\mbox{}\hfill{\bfseries \cnabstractname}\hfill\mbox{}\par
            \vskip 2.5ex
            }{\par\vskip 2.5ex}

% 各级标题自定义设置
    \usepackage{titlesec}   
    % chapter
    %    \titleformat{\chapter}[hang]{\normalfont\Large\bfseries\centering\boldmath}{Homework \thechapter :}{10pt}{}
        \titleformat{\chapter}[hang]{\normalfont\Large\bfseries\centering\boldmath}{}{10pt}{}
        \titlespacing*{\chapter}{0pt}{-30pt}{10pt} % 控制上方空白的大小
    % section
        \titleformat{\section}[hang]{\normalfont\large\bfseries\boldmath}{\thesection}{8pt}{}
    % subsection
        % 设置 subsection 样式为 (1) (2) (3)
        \renewcommand{\thesubsection}{(\arabic{subsection})}    
        \titleformat{\subsection}[hang]{\normalfont\bfseries\boldmath}{\thesubsection}{8pt}{}    


% --------------------- 文章宏包及相关设置 --------------------- %
% >> ------------------ 文章宏包及相关设置 ------------------ << %

% --------------------- 文章信息区 --------------------- %
% --------------------- 文章信息区 --------------------- %
% 页眉页脚设置
\usepackage{fancyhdr}   %宏包:页眉页脚设置
    \pagestyle{fancy}
    \fancyhf{}
    \cfoot{\thepage}
    \renewcommand\headrulewidth{1pt}
    \renewcommand\footrulewidth{0pt}
    \chead{数学物理方法课程作业,\ 丁毅}
    \rhead{dingyi233@mails.ucas.ac.cn}
    % \thechapter 是本页 chapter 序号
    % \leftmark 是本页 chapter 标题
    % \rightmark 是本页 section 标题
    
    %\rhead{\small dingyi233@mails.ucas.ac.cn}
    \usepackage{fontawesome}    % 宏包:更多符号与图标 (用于插入 GitHub 图标等)
    \lhead{
    \href{https://github.com/YiDingg/LatexNotes}{\color{black}\faGithub\ https://github.com/YiDingg/LatexNotes}
    }
    
%文档信息设置
\title{数学物理方法课程作业\\ Homework of Mathematical Physics Methods}
\author{丁毅\\ \footnotesize 中国科学院大学,北京 100049\\ Yi Ding \\ \footnotesize University of Chinese Academy of Sciences, Beijing 100049, China}
\date{\footnotesize 2024.8 -- 2025.1}
% --------------------- 文章信息区 --------------------- %
% --------------------- 文章信息区 --------------------- %

% 开始编辑文章

\begin{document}\zihao{5}   % 设置字号

\maketitle  % 封面

\newpage    % 序言、目录页
\pagenumbering{Roman}

\thispagestyle{fancy}   % 显示页码、页眉等
\begin{cnabstract}
\normalsize 本文为笔者本科时的“数学物理方法”课程作业(Homework of Mathematical Physics Methods, 2024.9-2025.1)。由于个人学识浅陋,认识有限,文中难免有不妥甚至错误之处,望读者不吝指正,在此感谢。\par 
我的邮箱是 dingyi233@mails.ucas.ac.cn。
\end{cnabstract}\addcontentsline{toc}{chapter}{序言} % 手动添加为目录

% 控制目录不换页
\setcounter{tocdepth}{0}
\noindent\rule{\textwidth}{0.1em}   % 分割线
\noindent\begin{minipage}{\textwidth}\centering 
    \vspace{1cm}
    \tableofcontents\thispagestyle{fancy}   % 显示页码、页眉等   
\end{minipage}  
\addcontentsline{toc}{chapter}{目录} % 手动添加为目录

\newpage
\pagenumbering{arabic} 
\rhead{\leftmark}

\chapter{Homework 1: 2024.8.26 - 2024.9.1}\thispagestyle{fancy}

\section{计算}

\begin{enumerate}
\item $(\frac{1+ \mathrm{i} }{2- \mathrm{i} })^2$
\begin{equation*}
    \left(\frac{1+ \mathrm{i} }{2- \mathrm{i} }\right)^2 
    = \left(\frac{(1+i)(2+i)}{5}\right)^2
    = \left(\frac{1+3i}{5}\right)^2 = \frac{-8 + 6i}{25}
\end{equation*}

\item $(1+i)^n + (1-i)^n$

首先得到:
\begin{gather*}
    1+i = \sqrt{2}e^{i\frac{\pi}{4}},\ 1-i = \sqrt{2}e^{i(-\frac{\pi}{4})}\\ \Longrightarrow
    I = 2^{\frac{n}{2}}\left( e^{i\frac{n\pi}{4}} + e^{-i\frac{n\pi}{4}} \right)
\end{gather*}
于是有:
\begin{equation*}
I = 
\begin{cases}
\begin{aligned}
    2^{\frac{n}{2}+1} &, && n = 0 + 4k \\
    2^{\frac{n+1}{2}} &, && n = 1 + 4k \\
    0  \hspace{0.4cm} &, && n = 2 + 4k \\
    -2^{\frac{n}{2}+1}&, && n = 3 + 4k \\
\end{aligned}
\end{cases}\ ,\ \ k \in \N
\end{equation*}
{\par\color{gray}\small
习题课补:
\begin{align*}
I &= 
2^{\frac{n}{2}}\left( e^{i\frac{n\pi}{4}} + e^{-i\frac{n\pi}{4}} \right)  \\ 
&= 2^{\frac{n}{2}}  \left( \cos(\frac{n\pi}{4}) +i\sin\frac{n\pi}{4} +  \cos(-\frac{n\pi}{4}) +i\sin(-\frac{n\pi}{4}) \right)  \\ 
&= 2^{\frac{n}{2}+1} \cos(\frac{n\pi}{4})
\end{align*}
\par}


\item $\sqrt[4]{1+i}$
\begin{equation*}
    \sqrt[4]{1+i} = \left(\sqrt{2}e^{i\frac{\pi}{4}}\right)^{\frac{1}{4}} = 2^{\frac{1}{8}}e^{i\frac{\pi}{16}}  
\end{equation*}
\end{enumerate}
{\par\color{gray}\small
习题课补:在复数域中,开根号是多值函数,这里四次根在复数域中应有四个复根,设 $x = \sqrt[4]{1+i}$,则原式等价于方程:
\begin{equation*}
x^4 = 1+i = \sqrt{2}e^{i\frac{\pi}{4}} \Longrightarrow | x | = 2^{\frac{1}{8}},\quad \arg x = \frac{\pi}{16} + k\frac{\pi}{2}, k = 0,1,2,3 
\end{equation*}
\par}


\section{将复数化为三角或指数形式}

\begin{enumerate}
\item $\frac{5}{-3 + i}$
\begin{equation*}
    \frac{5}{-3 + i} = \frac{5e^{i0}}{\sqrt{10}e^{i (\arctan (-\frac{1}{3}) + \pi)}} = \sqrt{\frac{5}{2}}\cdot e^{-i (\arctan (-\frac{1}{3}) + \pi)}
\end{equation*}
\item $\left(\frac{2+i}{3-2i}\right)^2$ 
\begin{equation*}
    \left(\frac{2+i}{3-2i}\right)^2 
    = \left( \frac{\sqrt{5}e^{i \arctan(\frac{1}{2})}}{\sqrt{13}e^{i \arctan(-\frac{2}{3})}} \right)^2 
    = \frac{5}{13} e^{2i\left(\arctan(\frac{1}{2}) - \arctan(-\frac{2}{3})\right) }
\end{equation*}
\end{enumerate}

\section{求极限 $\lim_{z\to i} \frac{1+z^6}{1+z^{10}}$}
作不完全因式分解:

\begin{equation*}
    1+z^6 = z^6 - i^6 
    = (z^3 - i^3)(z^3 + i^3) = (z - i)(z^2 + iz +i^2)(z^3 + i^3)
\end{equation*}
\begin{equation*}
    1+z^{10} 
    = z^{10} - i^{10} = (z^5-i^5)(z^5+i^5) 
    = (z-i)(z^4 + iz^3 + i^2z^2 + i^3z + i^4 )(z^5+i^5)
\end{equation*}
\begin{equation*}
    \begin{aligned}
        \Longrightarrow L 
        = \lim_{z\to i} \frac{1+z^6}{1+z^{10}} 
        &= \lim_{z\to i} \frac{(z - i)(z^2 + iz +i^2)(z^3 + i^3)}{(z-i)(z^4 + iz^3 + i^2z^2 + i^3z + i^4 )(z^5+i^5)} \\
        &= \lim_{z\to i} \frac{(z^2 + iz +i^2)(z^3 + i^3)}{(z^4 + iz^3 + i^2z^2 + i^3z + i^4 )(z^5+i^5)} \\
        &= \frac{(-3)\times (-2i)}{ 5 i} = \frac{3}{5}
        \end{aligned}
\end{equation*}

事实上,实数域上的洛必达法则(L'Hospital)可以推广到复数域的解析函数,下面给出 $\frac{0}{0}$ 型的证明。
设复变函数 $f(z), g(z)$ 在 $z = z_0$ 解析,且 $f(z_0) = g(z_0) = 0$,则有:
\begin{equation*}
\lim_{z\to z_0} \frac{f(z)}{g(z)} 
= \lim_{z\to z_0} \frac{f(z) - f(z_0)}{g(z) - g(z_0)} 
= \lim_{z\to z_0} \frac{\frac{f(z) - f(z_0)}{z-z_0}}{\frac{g(z) - g(z_0)}{z-z_0}} 
= \lim_{z\to z_0} \frac{f'(z)}{g'(z)}
\end{equation*}

特别地,若 $f'(z_0)$ 与 $g'(z_0)$ 存在且不为零,就有 $\lim_{z\to z_0} \frac{f(z)}{g(z)} = \frac{f'(z_0)}{g'(z_0)}$

\section{讨论函数在原点的连续性}

\begin{enumerate}
\item $f(z) = \begin{cases}
    \frac{1}{2i}(\frac{z}{z^*} - \frac{z^*}{z}), & z \neq 0 \\ 
    0, & z = 0
\end{cases}$

令 $z = x + iy, x,y \in \R$,则 $\forall\ (x,y) \ne (0,0)$ :
\begin{equation*}
f(x,y) 
= \frac{1}{2i} \left(\frac{x+iy}{x-iy} - \frac{x-iy}{x+iy}\right) 
= \frac{1}{2i} \cdot \frac{ 4ixy }{x^2 + y^2} = \frac{2xy}{x^2 + y^2} 
\end{equation*}
令 $k = \frac{y}{x}$,则:
\begin{equation*}
L = \lim_{(x,y) \to (0,0)} f(x,y) 
= \lim_{(x,y) \to (0,0)} \frac{2k}{1+k^2}
\end{equation*}
显然,$L$ 随着 $k$ 的变化而变化,因此极限不存在,$f(z)$ 在 $0$ 处不连续。

\item $f(z) = \begin{cases}
    \frac{\Im z}{1 + | z |}, & z \neq 0 \\ 
    0, & z = 0
\end{cases}$

令 $z = x + iy$ 和 $k = \frac{y}{x}$,则$\forall\ (x,y) \ne (0,0)$ :
\begin{equation*}
    f(x,y) = \frac{y}{1+\sqrt{x^2+y^2}} \Longrightarrow \lim_{(x,y) \to (0,0)} f(x,y) = \frac{0}{1+0} = 0 = f(0,0) 
\end{equation*}
因此 $f(z)$ 在 0 处连续。

\item  $f(z) = \begin{cases}
    \frac{\Re z^2}{| z^2 |}, & z \neq 0 \\ 
    0, & z = 0
\end{cases}$

同理令 $z = x + iy$ 和 $k = \frac{y}{x}$,则$\forall\ (x,y) \ne (0,0)$ :
\begin{equation*}
f(x,y) = \frac{x^2 - y^2}{ x^2 + y^2 } = \frac{1-k^2}{1+k^2} 
\end{equation*}

因此 $f(z)$ 在 0 处不连续。

\end{enumerate}



\section{恒等式证明(附加题)}

\begin{equation*}
\left| \sum_{i=1}^{n} a_ib_i \right| ^2 
= \sum_{i=1}^{n} | a_i |^2 \cdot \sum_{i=1}^{n} | b_i |^2 - \sum_{1 \leqslant i <j \leqslant n} \left|  a_ib_j^* - a_jb_i^*  \right|^2  
\end{equation*}



\chapter{Homework 2: 2024.9.2 - 2024.9.8}\thispagestyle{fancy}


\section{下列函数在何处可导,何处解析}
\begin{enumerate}
\item $ f(z) = z\cdot \Re z$

设 $z = x + iy$,则 $f(z) = u(x,y) + iv(x,y) = x^2 + ixy$。
$\forall\  z \in C$,$u(x,y) = x^2$ 和 $v(x,y) = xy$ 在 $\C$ 上有连续一阶偏导,下面考虑 C-R 条件:
\begin{gather}
\frac{\partial u }{\partial x } = 2x, \quad \frac{\partial u }{\partial y } = 0 \\ 
\frac{\partial v }{\partial x } = y, \quad \frac{\partial v }{\partial y } = x
\end{gather}

联立 C-R 条件,得 $(x,y) = (0,0)$,因此 $f$ 在 $(0,0)$ 处可导,在 $\C$ 上不解析。
不在点 $(0,0)$ 上解析是因为在某点解析是指在此点的有心邻域上解析,显然这里不满足,因此 $(0,0)$ 为奇点。


{\par\color{gray}\small
后补:
\begin{equation*}
u,v\ \text{有一阶连续偏导且满足 C-R 条件} \Longrightarrow  u,v\ \text{可微且满足 C-R 条件} \Longleftrightarrow f\ \text{可微} \Longleftrightarrow 
f\ \text{可导}
\end{equation*}
\par}


%\begin{gather*}
%\frac{ f(z + \Delta z) - f(z)}{\Delta z} 
%= \frac{2x\Delta x + (\Delta x)^2 + iy\Delta x + ix \Delta y + \Delta x \Delta y}{\Delta x + i \Delta y} 
%= \frac{2x + \Delta x + iy + ikx + \Delta y}{1+ ik} \\ 
%\Longrightarrow 
%\lim_{\Delta z \to (0,0)} \frac{ f(z + \Delta z) - f(z)}{\Delta z} 
%= \lim_{\Delta z \to (0,0)} \frac{2x  + iy + ikx }{1+ ik} = x + \lim_{\Delta z \to (0,0)} \frac{x+iy}{1+ik} 
%\end{gather*}
%显然,随着 $k = \frac{\Delta x}{\Delta y}$ 的变化,极限值也在变化,因此 $\forall\ z \in C$,$f(z)$ 不可导。

\item $f(x,y) = (x-y)^2 + 2i(x+y)$

$\forall\  z \in C$,$u(x,y) = (x-y)^2$ 和 $v(x,y) = 2(x+y)$ 在 $\C$ 上有连续一阶偏导,下面验证 C-R 条件:
\begin{gather}
\frac{\partial u }{\partial x } = 2(x-y), \quad \frac{\partial u }{\partial y } = -2(x-y) \\ 
\frac{\partial v }{\partial x } = 2, \quad \frac{\partial v }{\partial y } = 2
\end{gather}

联立 C-R 条件后无解,因此 $f$ 在 $\C$ 上不可导,在 $\C$ 上不解析。



%\begin{gather*}
%\frac{ f(z + \Delta z) - f(z)}{\Delta z} 
%= \frac{2x\Delta x + (\Delta x)^2 - 2y\Delta x -2x \Delta y - \Delta x \Delta y - 2y\Delta y - \Delta y^2 + 2i\Delta x + 2i\Delta y}{\Delta x + i \Delta y} 
%\end{gather*}
%\begin{align*}
%\Longrightarrow 
%\lim_{\Delta z \to (0,0)}\frac{ f(z + \Delta z) - f(z)}{\Delta z} 
%&= \lim_{\Delta z \to (0,0)}\frac{  2x + \Delta x - 2y -2xk - \Delta y - 2yk - k\Delta y +2i +2ik }{1+ik} \\ 
%&= \lim_{\Delta z \to (0,0)}   \\
%&= \lim_{\Delta z \to (0,0)}   \\
%\end{align*}
\end{enumerate}



\section{求下列函数的解析区域}
\begin{enumerate}
\item $f(z) = xy+iy$

\begin{gather*}
\frac{\partial u }{\partial x } = y, \quad \frac{\partial u }{\partial y } = x ,\quad 
\frac{\partial v }{\partial x } = 0, \quad \frac{\partial v }{\partial y } = 1
\end{gather*}
欲满足 C-R 条件,则:
\begin{equation*}
y=1,\  x = 0 \Longrightarrow \text{$f$ 在全平面不解析}
\end{equation*}
不在点 $(0,1)$ 上解析是因为在某点解析是指在此点的有心邻域上解析,显然这里不满足。

\item  $ f(z) = \begin{cases}
    | z | \cdot z, &| z | <1 \\ 
    z^2,  &|z|  \geqslant 1
\end{cases} $ 

设 $z = x +iy$,则:
\begin{gather*}
f(z) = u(x,y) + iv(x,y) = 
\begin{cases}
    (x\sqrt{x^2+y^2}) + i(y\sqrt{x^2+y^2}), &\sqrt{x^2+y^2} < 1 \\ 
    (x^2 - y^2) + i(2xy), &\sqrt{x^2+y^2} \geqslant 1
\end{cases} \\ 
\Longleftrightarrow 
u(x,y) = 
\begin{cases}
    x\sqrt{x^2+y^2},&  \sqrt{x^2+y^2} < 1 \\ 
    x^2 - y^2, & \sqrt{x^2+y^2} \geqslant 1
\end{cases} ,\quad
v(x,y) = 
\begin{cases}
    y\sqrt{x^2+y^2}, & \sqrt{x^2+y^2} < 1 \\ 
    2xy, & \sqrt{x^2+y^2} \geqslant 1
\end{cases} 
\end{gather*}

分别求偏导得到:
\begin{gather}
    \begin{cases}
        \frac{\partial u }{\partial x } = \frac{2x^2 + y^2}{\sqrt{x^2 + y^2}}
        ,\quad \frac{\partial u }{\partial y } = \frac{xy}{\sqrt{x^2 + y^2}} \\
        \frac{\partial v }{\partial x } = \frac{xy}{\sqrt{x^2 + y^2}}
        ,\quad \frac{\partial v }{\partial y } = \frac{x^2 + 2y^2}{\sqrt{x^2 + y^2}}
    \end{cases},\quad \sqrt{x^2+y^2} < 1  
\end{gather}
\begin{gather*}
\begin{cases}
    \frac{\partial u }{\partial x } = 2x
    ,\quad 
    \frac{\partial u }{\partial y } = -2y \\
\frac{\partial v }{\partial x } = 2y
,\quad \frac{\partial v }{\partial y } = 2x
\end{cases}
,\quad \sqrt{x^2+y^2} \geqslant 1
\end{gather*}

偏导要满足 C-R 条件,代入得到:
\begin{gather*}
x^2 = y^2,\  2xy = 0, \quad \forall\  \sqrt{x^2+y^2} < 1, x^2 + y^2 \ne 0  \\
2x = 2x,\ -2y = -2y, \quad \forall\ \sqrt{x^2+y^2} \geqslant 1 \\ 
\Longrightarrow \text{$f(z)$ 在 $\{ z\in \C \mid |z| \geqslant 1 \}$ 上解析}
\end{gather*}
不在点 $(0,0)$ 上解析是因为在某点解析是指在此点的有心领域上解析,显然这里不满足。

{\par\color{gray}\small
后补:{\color{red} 解析区域必须是开集}(因为受“有心邻域”限制),$f$ 的解析区域应为 $\{ z\mid | z | {\color{red} >} 1 \}$。另外,$| z | = 1$ 代表的圆周上也不可微,这是因为 $f$ 在 $| z | =1$ 上不连续(内部是一倍幅角,外部是二倍幅角),所以可微区域也为 $\{z \mid | z | >1\}$。
\par}


\end{enumerate}


\section{已知解析函数 $f(z)$ 的实部如下,求 $f(z)$}
\begin{enumerate}
\item $u(x,y) = x^2-y^2 + x$
\begin{gather*}
v_x' = -u_y' = 2y,\quad v_y' = u_x' = 2x + 1\\ 
\Longrightarrow 
v(x,y) = \int 2y \ \mathrm{d}x + \int \mathrm{d}y = 2xy + y + C \\ 
\Longrightarrow f(x,y)= (x^2 + y^2 + x )+ i(2xy + y) + C,\ C \in \R
\end{gather*}

\item $u(x,y) = e^y\cos x$
\begin{gather*}
v_x' = -u_y' = -e^y\cos x,\quad v_y' = u_x' = -e^y\sin x\\ 
\Longrightarrow
v(x,y) = \int -e^y\cos x\  \mathrm{d}x + \int 0\ \mathrm{d}y = - e^y\sin x  + C \\ 
\Longrightarrow f(x,y) = (e^y\cos x) + i(-e^y\sin x ) + C ,\ C \in \R
\end{gather*}
\end{enumerate}


\section{$f$ 解析,且 $u - v = (x-y)(x^2 + 4xy + y^2)$,求 $f(z)$}
两边分别对 $x,\ y$ 求导,得到:
\begin{gather*}
\frac{\partial u }{\partial x } - \frac{\partial v }{\partial x }  = 3x^2 + 6xy -3y^2, \quad 
\frac{\partial u }{\partial y } - \frac{\partial v }{\partial y } = 3x^2 - 6xy -3y^2
\end{gather*}
联立 C-R 条件,可以解出:
\begin{gather*}
v_x' = -3x^2 + 3y^2, \quad v_y' = 6xy \\ 
u_x' = 6xy,\quad u_y' = 3x^2 - 3y^2 \\ 
\Longrightarrow v(x,y) = -x^3 + 3xy^2 + C, \quad u(x,y) = 3x^2y - y^3 + C \\ 
\Longrightarrow f(x,y) = (3x^2y - y^3) + i(-x^3 + 3xy^2) + C(1+i),\ C \in \R
\end{gather*}

{\par\color{gray}\small
后补:$u$ 和 $v$ 中的实常数 $C$ 其实是同一个!这是因为题目中 $u-v$ 没有常数项,说明两者积分常数相同。
\par}


\section{极坐标 C-R 条件}
证明极坐标下的 C-R 条件为:
\begin{equation*}
\frac{\partial u }{\partial r } = \frac{1}{r}\frac{\partial v }{\partial \theta }, \quad \frac{\partial v }{\partial r } = -\frac{1}{r}\frac{\partial u }{\partial \theta } 
\end{equation*}

极坐标变换:
\begin{gather*}
    x = x(r,\theta ) = r\cos \theta, \quad y = y(r,\theta ) = r\sin \theta  \\ \Longrightarrow 
    \frac{\partial x }{\partial r } = \cos \theta,\ \frac{\partial x }{\partial \theta } = -r\sin \theta,\ \frac{\partial y }{\partial r } = \sin \theta,\ \frac{\partial y }{\partial \theta } = r\cos \theta
\end{gather*}

由复合函数的求导法则:
\begin{gather*}
\frac{\partial  }{\partial r }u\left(x(r, \theta),y(r, \theta)\right) 
= \frac{\partial u(x,y) }{\partial x }\cdot\frac{\partial x}{\partial r } + \frac{\partial u(x,y) }{\partial y}\cdot\frac{\partial y}{\partial r } 
= \frac{\partial u(x,y) }{\partial x }\cdot \cos \theta + \frac{\partial u(x,y) }{\partial y}\cdot \sin \theta 
\\
\frac{\partial  }{\partial \theta }v\left(x(r, \theta),y(r, \theta)\right) 
= \frac{\partial v(x,y) }{\partial x }\cdot\frac{\partial x}{\partial \theta } + \frac{\partial v(x,y) }{\partial y}\cdot\frac{\partial y}{\partial \theta } 
= - \frac{\partial v(x,y) }{\partial x } \cdot r\sin \theta + \frac{\partial v(x,y) }{\partial y}\cdot r \cos \theta 
\end{gather*}



联立 C-R 条件,化简得到:
\begin{equation*}
v_r' = -u_y'\cos \theta + u_x' \sin \theta = \frac{1}{r}u_\theta'
\end{equation*}

同理,由偏导关系:
\begin{gather*}
\frac{\partial  }{\partial \theta }u\left(x(r, \theta),y(r, \theta)\right) 
= \frac{\partial u(x,y) }{\partial x }\cdot\frac{\partial x}{\partial \theta } + \frac{\partial u(x,y) }{\partial y}\cdot\frac{\partial y}{\partial \theta } 
= - \frac{\partial u(x,y) }{\partial x } \cdot r\sin \theta + \frac{\partial u(x,y) }{\partial y}\cdot r \cos \theta 
\\ 
\frac{\partial  }{\partial r }v\left(x(r, \theta),y(r, \theta)\right) 
= \frac{\partial v(x,y) }{\partial x }\cdot\frac{\partial x}{\partial r } + \frac{\partial v(x,y) }{\partial y}\cdot\frac{\partial y}{\partial r } 
= \frac{\partial v(x,y) }{\partial x }\cdot \cos \theta + \frac{\partial v(x,y) }{\partial y}\cdot \sin \theta 
\end{gather*}
联立 C-R 条件,化简得到:
\begin{equation*}
u_r' = u_x'\cos \theta + u_y' \sin \theta = \frac{1}{r}v_\theta'
\end{equation*}

反之也可以化为原 C-R 条件,因此 C-R 条件在极坐标下的形式为:
\begin{equation*}
    \frac{\partial u }{\partial r } = \frac{1}{r}\cdot\frac{\partial v }{\partial \theta }, \quad \frac{\partial v }{\partial r } = -\frac{1}{r}\cdot\frac{\partial u }{\partial \theta }\quad \square
\end{equation*}

\section{证明 $f(z)$ 和 $\overline{f(\bar{z})}$ 同解析或同不解析 }
\begin{enumerate}
\item $f(z)$ 解析 $\Longrightarrow \overline{f(\bar{z})}$ 解析

假设 $f(z)$ 在点 $z = z_0$ 解析,即 $f(z) = u(x,y) + iv(x,y)$ 在有心邻域 $U_{\delta}(z_0)$ 上解析,这等价于 $f(z)$ 有一阶导,且在邻域内满足 C-R 条件。设 $g(z) = \overline{f(\bar{z})} = u(x,-y) - iv(x,-y)$,也即:
\begin{equation*}
g(z) = u_g(x,y) + iv_g(x,y),\quad u_g(x,y) = u(x,-y),\ v_g(x,y) = -v(x,-y)
\end{equation*}
容易验证 $g(z)$ 有一阶偏导,下面验证 C-R 条件:
\begin{gather*}
\frac{\partial u_g }{\partial x } = \frac{\partial u}{\partial x }(x,-y) ,\quad
\frac{\partial u_g }{\partial y } = \frac{\partial u(x,-y) }{\partial (-y) }\cdot \frac{\partial (-y) }{\partial y } = -\frac{\partial u }{\partial y }(x,-y) \\ 
\frac{\partial v_g }{\partial x } = -\frac{\partial v}{\partial x }(x,-y) ,\quad
\frac{\partial v_g }{\partial y } = -\frac{\partial v(x,-y) }{\partial (-y) }\cdot \frac{\partial (-y) }{\partial y } = \frac{\partial v }{\partial y }(x,-y) 
\end{gather*}
联立 $u$ 和 $v$ 的 C-R 条件,得到:
\begin{gather*}
\frac{\partial u_g }{\partial x } - \frac{\partial v_g }{\partial y } = \frac{\partial u}{\partial x }(x,-y) - \frac{\partial v }{\partial y }(x,-y) = 0 \Longrightarrow \frac{\partial u_g }{\partial x } = \frac{\partial v_g }{\partial y }
\\ 
\frac{\partial u_g }{\partial y } + \frac{\partial v_g }{\partial x } = -\left[ \frac{\partial u }{\partial y }(x,-y) + \frac{\partial v}{\partial x }(x,-y)\right] = 0
\Longrightarrow 
\frac{\partial u_g }{\partial y } = -\frac{\partial v_g }{\partial x }
\end{gather*}
因此 $g(z) = \overline{f(\bar{z})}$ 也解析。

\item $f(z)$ 解析 $\Longleftarrow \overline{f(\bar{z})}$ 解析

假设 $\overline{f(\bar{z})}$ 解析,令 $g(z) = \overline{f(\bar{z})}$,则 $f(z) = \overline{g(\bar{z})}$,由 (1) 的结论,$g(z)$ 解析 $\Longrightarrow$ $f(z) = \overline{g(\bar{z})}$ 也解析。

证毕。$\square$
\end{enumerate}


\chapter{Homework 3: 2024.9.9 - 2024.9.15}\thispagestyle{fancy}

\section{若 $f(z)$ 解析,$\arg f(z)$ 是否为调和函数?}

{\par\color{gray}\small
注:下面的过程仅讨论了 $\arg f(z)$ 的解析性,未能揭示其调和性,正确的解答见后文补充的灰色小字。
\par}


\begin{enumerate}
\item 当 $f(z) = C \in \C, \forall\ z \in G$,也即 $f(z)$ 恒为常量时:$\arg f(z)$ 也为常量,设 $\arg f(z) = a + ib$,则 $a = \arg f(z) \in R$ 而 $b=0$,自然满足 $\Delta\, a = \Delta\, b =0$,因此 $\arg f(z)$ 为调和函数。
\item 当 $f(z)$ 是非常量函数时:

由 $\ln z = \ln | z | + i\arg z$,移项,并作映射 $z \to f(z)$,则有:
\begin{equation*}
\arg f(z) = \frac{1}{i}\left(\ln f(z) - \ln \rho\right)
\end{equation*}
函数 $\ln$ 在 $\C \setminus \{0\}$ 上解析,但对于函数 $\rho = \rho(z)$:
\begin{gather}
\rho = \sqrt{u^2 + v^2}\Longrightarrow u_\rho = \sqrt{u^2+v^2}, v_\rho = 0  \\ 
\frac{\partial u_\rho }{\partial x }
= \frac{uu_x'}{\sqrt{u^2+v^2} } + \frac{vv_x'}{\sqrt{u^2+v^2} },\quad \frac{\partial u_\rho }{\partial y } = \frac{uu_y'}{\sqrt{u^2+v^2} } + \frac{vv_y'}{\sqrt{u^2+v^2} }
\end{gather}

假设 $\rho$ 满足 C-R 条件,代入得到:
\begin{equation*}
\begin{cases}
    uu_x' + vv_x' = 0 \\ 
    uu_y' + vv_y' = 0 \\ 
    \sqrt{u^2+v^2} \ne 0
\end{cases}
\end{equation*}
由于 $f(z)$ 解析,满足 C-R 条件 $\frac{\partial u }{\partial x } = \frac{\partial v }{\partial y },\ \frac{\partial u }{\partial y } = - \frac{\partial v }{\partial x }$,代入后整理得到:
\begin{equation*}
\begin{cases}
    v(v_y'^2 - u_y'^2) = 0 \\ 
    u(u_y'^2 + v_y'^2) = 0 \\ 
\end{cases}
\end{equation*}
$f(z)$ 非常量,因此 $u,v$ 非常量,因此只能有:
\begin{equation*}
    v_y' = u_y' = 0 \Longrightarrow  u_x' = v_x' = 0 \Longrightarrow  \text{$u$ 和 $v$ 为常量函数}
\end{equation*}
这使得 $f(z) = u +iv$ 是常量,矛盾!因此 $\arg f(z)$ 不解析(这能否推出不调和?解析是调和的充分条件,但是充要的吗?事实上并不是,因此并不能揭示调和性)。
\end{enumerate}

{\par\color{gray}\small
后补:即使仅从解析性的角度来看,上面的过程也没有抓到主要矛盾,是舍本逐末了。因为无论 $f(z)$ 的性质如何,$\arg f(z)$ 始终是 $\C \longrightarrow \R$ 的函数,这表明 $\arg f(z)$ 是实部是它本身而虚部恒为 0,因此,由 C-R 条件可知 $\arg f(z)$ 解析的必要条件是实部为常数,而这也是充分条件。 

对 $\arg f(z)$ 的调和性,我们有如下推导:
\begin{equation}
\arg f(z) = \arctan \frac{u(x,y)}{v(x,y)} + A,\quad A \in \{ 0, \pi\}
\end{equation}
令 $g(z) = \arg f(z)$,则有:
\begin{equation}
g_x' = \frac{uv_x' - u_x'v}{u^2 + v^2},\quad g_{xx}'' = \frac{1}{(u^2 + v^2)^2}\left[
    (u^2 + v^2)(uv_{xx}'' + u_{xx}'' v) - 2 uv(v_x^2-u_x^2) - 2(u^2- v^2)u_x'v_x' 
\right]
\end{equation}
对 $y$ 求导也是同理,只需将上面的角标 $x$ 换为 $y$,于是有 $\Delta\, g$: 
\begin{align*}
    \Delta\, g 
    &= g_{xx}'' + g_{yy}'' \\ 
    &= \frac{1}{(u^2 + v^2)^2}\left[
    (u^2 + v^2)(u(v_{xx}'' + v_{yy}'') + (u_{xx}'' + u_{yy}'') v)- 2 uv(v_x^2 + v_y^2-u_x^2-u_y^2) - 2(u^2- v^2)(u_x'v_x' + u_y'v_y')  
\right]
\end{align*}
$f$ 解析意味着 $u, v$ 构成一对共轭调和函数,有 $\Delta\, u = \Delta\,  v = 0$,代入上式,再代入 C-R 条件,容易验证右边为 0,也即证明了 $\Delta\, g = 0$,因此 $\arg f(z)$ 为调和函数。对 $u^2 + v^2 = 0$ 的情况,我们不再赘述,只关心普遍结论。
\par}


\section{从已知的实虚部求出解析函数 $f(z)$}

\begin{enumerate}
\item $u = e^x(x\cos y - y\sin y) + 2\sin x \cdot \sinh y + x^3 - 3xy^2 +y$

\begin{gather}
u_x' = e^x(x\cos y -y\sin y + \cos y) + 2\cos x \sinh y +3x^2 - 3y^2 \\ 
u_y' = e^x(-x\sin y - \sin y - y\cos y ) + 2\sin x \cosh y - 6xy +1
\end{gather}
由 C-R 条件,$v_x' = -u_y',\ v_y' = u_x'$,于是得到:
\begin{align}
v(x,y) 
&= \int (-u_y')\mathrm{d}x + \int (-3y^2) \mathrm{d}y \\
&= (x-1)e^x\sin y  + (\sin y +y \cos y) e^x + 2\cos x \cosh y + 3x^2y - x - y^3 + C\\ 
&= (x\sin y + y\cos y)e^x + 2\cos x \cosh y + 3x^2y -x -y^3 + C,\ C \in \R
\end{align}

令 $(x,y) = (z,0)$,得到:
\begin{gather}
u(z,0) = ze^z + z^3,\quad v(z,0) = 2\cos z - z +C,\ C \in \R
\end{gather}
于是得到 $f(x,y)$:
\begin{equation*}
f(z) = \left[u(x,y) + iv(x,y)\right]_{x=z,y=0}
= 
(ze^z + z^3) + i(2\cos z -z +C),\ C \in \R
\end{equation*}


\item $v = \ln (x^2 + y^2) + x -2y$

\begin{equation*}
v_x' = \frac{2x}{x^2 + y^2} + 1, \quad v_y' = \frac{2y}{x^2 + y^2} -2
\end{equation*}
由 C-R 条件,$u_x' = v_y',\ u_y' = -v_x'$,于是得到:
\begin{align}
u(x,y) 
= \int v_y' \mathrm{d}x +  \int (-1)\mathrm{d}y = 2\arctan \frac{x}{y} - 2x -y + C
\end{align}
\begin{equation*}
f(x,y) = u + iv = (2\arctan \frac{x}{y} - 2x -y + C) + i(\ln (x^2 + y^2) + x -2y),\ C \in \R
\end{equation*}
\end{enumerate}
{\par\color{gray}\small
后补,这里之所以没有令 $(x,y) = (z,0)$ 得到 $f(z)$,是因为函数 $\arctan \frac{x}{y}$ 在实轴附近是不连续的,例如在正实轴 $x>0$ 附近,$\lim_{y \to 0^+}$ 时趋于 $+\infty$ 而 $\lim_{y \to 0^-}$ 时趋于 $-\infty$。而映射 $(x,y) = (z,0)$ 的必要条件是解析域中包含实轴,这涉及到解析延拓的内容,我们不提。只需要写到 $f(x,y)$ 的形式就这样放着即可。
\par}


\section{求下列函数的值}
\begin{enumerate}
\item $\cos(2+i)$

由 $\cos z = \frac{e^{iz} + e^{-iz}}{2}$,可得:
\begin{align*}
\cos(2+i) 
&= \frac{1}{2}\left[  e^{i(2+i)} + e^{i(2-i)}  \right] = \frac{1}{2}\left[ e^{2i -1} + e^{1-2i} \right] \\
&= \frac{1}{2}\left[ (\frac{1}{e} + e)\cos 2 + i(\frac{1}{e}-e)\sin 2\right]
\end{align*}

\item $\Ln(2-3i)$

由 $\Ln z = \ln | z | + i\Arg z$,可得:
\begin{align*}
\Ln (2-3i) = \ln | 2-3i | + i\Arg (2-3i)
= \frac{1}{2}\ln 13 + i\left(\arctan (-\frac{3}{2}) + 2k\pi\right),\quad k \in \mathbb{Z}
\end{align*}

\item $\Arccos (\frac{3+i}{4})$

$\arccos z = -i \ln (z + \sqrt{z^2 -1} )$,于是:
\begin{align*}
\Arccos (\frac{3+i}{4}) 
&= -i \Ln\left(\frac{3+i}{4} + \sqrt{(\frac{3+i}{4})^2 -1} \right) = -i \Ln \left( \frac{3+i}{4} + \frac{\sqrt{-8 + 6i} }{4} \right) \\ 
&= -i \Ln \left( \frac{3+i}{4} \pm \frac{1 + 3i}{4} \right) = -i \Ln(1+i)\  \text{或}\ -i \Ln(\frac{1-i}{2}) \\ 
&= (\frac{\pi}{4} + 2k\pi) - i\frac{\ln 2}{2} \ \text{或}\ -(\frac{\pi}{4} + 2k\pi) + i\frac{\ln 2}{2},\quad k \in \mathbb{Z}
\end{align*}

\item $\Arctan (1+2i)$

由 $\Arctan z = \frac{1}{2i} \Ln \frac{1+iz}{1-iz}$,得:
\begin{align*}
\Arctan (1+2i) 
&= \frac{1}{2i}\Ln \left( \frac{1+i(1+2i)}{1-i(1+2i)} \right)
= \frac{1}{2i}\Ln \left( \frac{-1 + i}{3-i} \right) \\
&= \frac{1}{2i}\left( \Ln (-2+i) - \ln 5 \right) 
= \frac{1}{2i}\left[ -\frac{\ln 5}{2}  + i\left( \pi - \arctan(-\frac{1}{2})  + 2k\pi\right) \right] \\ 
&= \frac{1}{2}\left(\pi - \arctan(-\frac{1}{2})  + 2k\pi\right) + i\frac{\ln 5}{4},\quad k \in \mathbb{Z}
\end{align*}
\end{enumerate}

\section{判断下列函数是单值还是多值函数}
\begin{enumerate}
\item $\sin \sqrt{z}$

多值函数。$\sqrt{z}$ 为双值函数,$a^2 = z \Longrightarrow  \sqrt{z} = \pm a $,而 $\sin $ 为奇函数,$\sin a \ne \sin (-a)$,故为多值函数。

\item $\frac{\sin \sqrt{z} }{\sqrt{z} }$

单值函数。$\frac{\sin a}{a} = \frac{\sin (-a)}{-a}$,因此为单值函数。

\item $\frac{\cos \sqrt{z} }{z }$

单值函数。$\frac{\cos a}{a^2} =\frac{\cos (-a)}{(-a)^2}$,故为单值函数。
\end{enumerate}

\section{解方程:$2\cosh^2 z - 3\cosh z + 1 = 0$}
原方程等价于:
\begin{gather}
(2\cosh z - 1)(\cosh z - 1) = 0 \Longrightarrow  \cosh z = \frac{1}{2} \ \text{或}\  1 
\\ 
\overset{\cosh z = \frac{e^{z} + e^{-z}}{2}}{\Longrightarrow } 
e^z = \frac{1 \pm \sqrt{3}i }{2}\  \text{或}\ 1 
\\ 
\Longrightarrow
z = i(\pm \frac{\pi}{3} + 2k\pi) \ \text{或}\ i(0 + 2k\pi),\quad  k \in \Z
\end{gather}

\section{求下列多值函数的分支点}

\textbf{(1)\ } $\sqrt{1 - z^3}$ 的分支点:$1, -\frac{1}{2} + \frac{\sqrt{3}}{2}i, -\frac{1}{2} - \frac{\sqrt{3}}{2}i, \infty$

宗量 $1 - z^3$ 不妨记为 $1 - z^3 = (z_1 - z)(z_2 - z)(z_3 - z)$。支点仅可能在宗量的零点、奇点处出现,下面分别考察 $z_1,z_2,z_3, \infty$ 四点。

对 $z_1$,取仅包含点 $z_1$ 的简单闭合曲线,曲线上一点 $z$ 沿逆时针绕一圈回到原处,因子 $(z_1 - z)$ 的幅角增加了 $2\pi$,因子 $(z_2 - z)$ 和 $z_3 - z$ 的幅角增加了 0,因此整个宗量的幅角增加 $2\pi$,开根后,函数值幅角增加 $\pi$,前后不相等。因此点 $z_1$ 是分支点。同理可得 $z_2$ 和 $z_3 $ 是分支点。

对 $\infty$,取包含点 $z_1, z_2, z_3 $ 的简单闭合曲线,曲线上一点 $z$ 沿顺时针(不是逆时针)绕一圈回到原处,整个宗量的幅角增加了 $-6\pi$,开根后函数值幅角增加 $-3\pi$,因此 $\infty$ 也是分支点。

\textbf{(2)\ } $\Ln \cos z$ 的分支点:$\infty,\ \frac{\pi}{2} + k\pi,\ k \in \Z$。

可以证明,$\Ln f(z)$ 的分支点等价于方程 $f(z) = 0$ 和 $f(z) = \infty$ 的解\footnote{这是助教在习题课上给出的结论,并未给出具体证明。但是我们可以证明 $\Ln z$ 的分支点为 $0$ 和 $\infty$,这是因为 $\Ln z = \ln | z | + i\Arg z$,当 $z$ 绕原点逆时针转一圈时,$\Arg z$ 增加 $2\pi$ 而不是回到原来的函数值,因此 0 为分支点;无穷点同理。}。于是分别令 $\cos z = \frac{e^{iz} + e^{-iz}}{2}$ 为 $0$ 和 $\infty$,解得:
\begin{equation}
z = \frac{\pi}{2} + k\pi, \quad k \in \Z \text{ 或 } z = \infty
\end{equation}

\textbf{(3)\ } $\sqrt{\frac{z}{(z-1)(z-2)}} $ 的分支点:$0, 1, 2, \infty$

考虑点 $0,1,2$,取仅包含点 $0$ 的简单闭合曲线,曲线上一点 $z$ 逆时针绕一圈后,宗量整体幅角增加 $2\pi$,函数值幅角增加 $\pi$,因此点 $0$ 是分支点。同理点 $1$ 和 $2$ 也是分支点。

对 $\infty$,取包含点 $0,1,2$ 的简单闭合曲线,曲线上一点 $z$ 顺时针绕一圈后,宗量整体幅角增加 $-2\pi$,函数值也不发生变化,$\infty$ 不是分支点。

\textbf{(4)\ } $\Ln \frac{(z-a)(z-b)}{(z-c)}$ 的分支点:$a,b,c, \infty$

与 (2) 同理,考虑宗量 $\frac{(z-a)(z-b)}{(z-c)}$ 的零点和无穷点,得到 $z = a,b,c, \infty$,即为所求分支点。

\chapter{Homework 4: 2024.9.16 - 2024.9.22}\thispagestyle{fancy}

\section{计算下列积分}

\textbf{(1)\ \ } $\displaystyle \oint_{| z + i | = 1} \frac{e^z}{1 + z^2} \ \mathrm{d} z$

被积函数 $\frac{e^z}{1 + z^2}$ 在圆周 $| z + i | = 1$ 内有且仅有 $z = -i$ 一个奇点,由 Cauthy 定理和 Cauthy 积分公式:
\begin{equation}
I =  \oint_{| z + i | = 1} \frac{e^z}{1 + z^2} \ \mathrm{d} z = 2 \pi i\left[\frac{ e^z}{z - i} \right]_{z = -i} = - \pi e^{-i}
\end{equation}
结果化简到上面一步即可。

\textbf{(2)\ \ } $\displaystyle \oint_{| z -a | = a} \frac{ z}{z^4 - 1} \ \mathrm{d} z,\ a > 1$

被积函数 $\frac{z}{z^4 - 1}$ 在圆周 $| z - a | = a$ 内有且仅有 $z = 1$ 一个奇点,由 Cauthy 定理和 Cauthy 积分公式:
\begin{equation*}
I = \oint_{| z - 1 | = \delta } \frac{z}{z^4 - 1} \ \mathrm{d}z = 2 \pi i \cdot \left[ \frac{z}{z^3 + z^2 +z +1} \right]_{z = 1} = \frac{\pi i }{2}
\end{equation*}

\textbf{(3)\ \ } $\displaystyle \oint_{| z | = 2} \frac{z^2 - 1}{z^2 + 1} \ \mathrm{d} z$

被积函数在圆周 $| z | =2$ 内有且仅有 $z = \pm i$ 两个奇点,由 Cauthy 定理和 Cauthy 积分公式:
\begin{equation*}
    I = \oint_{| z  + i| = \delta_1} \frac{z^2 - 1}{z^2 + 1} \ \mathrm{d} z + \oint_{| z  - i | = \delta_2} \frac{z^2 - 1}{z^2 + 1} \ \mathrm{d} z =  2 \pi i \cdot \left[ \frac{z^2 - 1}{z - i} \right]_{z = -i} + 2 \pi i \cdot \left[ \frac{z^2 - 1}{z + i} \right]_{z = i} 
    = 0
\end{equation*}

\textbf{(4)\ \ } $\displaystyle \oint_{| z | = 2} \frac{1}{z^2(z^2 + 16)} \ \mathrm{d} z$

被积函数在圆周 $| z | =2$ 内有且仅有 $z = 0$ 一个奇点,由 Cauthy 定理和 Cauthy 积分公式:
\begin{equation*}
    I 
    = \oint_{| z | = \delta} \frac{1}{z^2(z^2 + 16)} \ \mathrm{d} z 
    = 2 \pi i \cdot \left[ \frac{1}{z^2 + 16} \right]^{(1)}_{z = 0} 
    = 2 \pi i \cdot \left[ -\frac{2z}{(z^2 + 16)^2} \right]_{z = 0} 
    = 0
\end{equation*}

\section{计算下列积分}

\textbf{(1)\ \ }  $\displaystyle \oint_{| z - 1 | = 1 } \frac{\sin \frac{\pi z}{4}}{z^2 - 1}  \ \mathrm{d} z$

被积函数在圆周 $| z | = R$ 内有且仅有 $z = 1$ 一个奇点,则:
\begin{equation*}
I 
=  2 \pi i \cdot \left[ \frac{\sin \frac{\pi z}{4}}{z + 1}  \right]_{z = 1} = \frac{\sqrt{2} \pi i }{2}
\end{equation*}

\textbf{(2)\ \ } $\displaystyle \lim_{R \to +\infty}\oint_{| z | = R } \frac{\sin \frac{\pi z}{4}}{z^2 - 1}  \ \mathrm{d} z$

被积函数在圆周 $| z | = R$ 内有且仅有 $z = \pm 1$ 两个奇点,则:
\begin{equation*}
I 
= 2 \pi i \cdot \left[ \frac{\sin \frac{\pi z}{4}}{z - 1}  \right]_{z = -1} +  2 \pi i \cdot \left[ \frac{\sin \frac{\pi z}{4}}{z + 1}  \right]_{z = 1} = \sqrt{2} \pi i 
\end{equation*}

\textbf{(3)\ \ }$\displaystyle \oint_{| z + 1 | = \frac{1}{2} } \frac{\sin \frac{\pi z}{4}}{z^2 - 1}  \ \mathrm{d} z$

被积函数在圆周 $| z | = R$ 内有且仅有 $z = -1$ 一个奇点,则:
\begin{equation*}
I 
= 2 \pi i \cdot \left[ \frac{\sin \frac{\pi z}{4}}{z - 1}  \right]_{z = -1}
= \frac{\sqrt{2} \pi i }{2}
\end{equation*}



\section{计算积分 $\displaystyle \int_{L} \frac{1}{(z - a)^n} \ \mathrm{d}z $,其中 $L$ 为以 $a$ 为圆心,$r$ 为半径的上半圆周}

作变换 $z \to z + a$,则原积分化为 $\int_{L'} \frac{1}{z^n} \ \mathrm{d}z$,其中 $L'$ 是以 0 为圆心,$r$ 为半径的上半圆周。当 $n = 1$,时,$\frac{1}{z}$ 在 $\C \setminus \{ 0 \} $ 内解析,$I(n) = \left[ \ln z \right]_{z = r}^{z = -r} = \ln(-1) = i \pi$;当 $n \in \Z \setminus \{ 1 \}$ 时,$\frac{1}{z^n}$ 在 $\C \setminus \{ 0 \} $ 内解析,$I(n) = \left[ \frac{z^{1-n}}{1-n} \right]_{z = r}^{z = -r}  =  \frac{1}{1 - n}\left[  (-r)^{1-n} - r^{1-n} \right]$。综上,我们有:
\begin{equation*}
I(n) = \int_{L} \frac{1}{(z - a)^n} \ \mathrm{d}z
= 
\begin{cases}
    i\pi, & n = 1 \\
    \left[ (-1)^{1-n} -1 \right]\cdot\frac{r^{1-n}}{1-n}, & n \in \Z \setminus \{ 1 \}
\end{cases}
\end{equation*}

\section{计算积分 $\displaystyle \oint_{| z | = R} \frac{1}{(z-a)^n(z-b)}  \ \mathrm{d}z $,其中 $a, b$ 不在圆周 $| z | = R$ 上,$n$ 为正整数}

令 $G = \{ z \mid | z | = R\}$,共有四种情况,总结如下:
\begin{equation*}
I = \oint_{| z | = R} \frac{1}{(z-a)^n(z-b)}  \ \mathrm{d}z
= 
\begin{cases}
    0, & a, b \notin G \\ 
    \frac{(-1)^{n-1}2 \pi i }{(a-b)^n}, & a \in G, b \notin G \\ 
    \frac{2 \pi i}{(b-a)^n}, & b \in G, a \notin G \\ 
    0, & a, b \in G
\end{cases}
\end{equation*}

\section{(附加题)$f(z)$ 在 $| z | < R$ 内解析,求证 $\displaystyle I(r) = \int_{0}^{2\pi} f(r\cdot e^{i \theta}) \ \mathrm{d}\theta$ 与 $r$ 无关,$\forall\  r \in (0, R)$}

设 $f(z)$ 在 $G$ 内解析,由 Cauthy 积分公式:
\begin{equation*}
f(a) = \frac{1}{2\pi i} \oint_{\partial G} \frac{f(z)}{z - a}\ \mathrm{d}z
\end{equation*}
在上式中,取 $G = \{ z \mid |z - a| = r,\ r \in (0, R)\}$,也即以 $a$ 为圆心,$r$ 为半径的圆周,则有 $z - a = r\cdot e^{i \theta},\ \mathrm{d}z = ire^{i \theta}\ \mathrm{d}\theta $,代入即得:
\begin{gather*}
f(a) 
= \frac{1}{2\pi i} \oint_{\partial G} \frac{f(z)}{r\cdot e^{i \theta}}\ ire^{i \theta}\ \mathrm{d}\theta 
= \frac{1}{2 \pi} \oint_{0}^{2\pi}  f(z)\ \mathrm{d}\theta
= \frac{1}{2 \pi} \oint_{0}^{2\pi}  f(a + r\cdot e^{i \theta})\ \mathrm{d}\theta \\ 
\end{gather*}
上式中令 $a = 0$,即得:
\begin{equation}
    I = I(r) = \oint_{0}^{2\pi} f(r\cdot e^{i \theta})\ \mathrm{d}\theta = 2 \pi f(0),\quad \forall\ r \in (0, R)   \quad\square
\end{equation}
因此积分的值与 $r$ 无关。

\chapter{Homework 5: 2024.9.23 - 2024.9.29}\thispagestyle{fancy}

{\color{red}\large !!!不要忘了 $2 \pi i$ !!!}

\section{求积分 $\displaystyle \oint_C \frac{\sin \frac{\pi z}{4}}{z^2 - 1} \mathrm{d} z$,$C:\ x^2 + y^2 -2x = 0$}

$C:\  (x - 1)^2 + y^2 =1$,因此:
\begin{gather}
I 
= 2 \pi i \left[ \frac{\sin \frac{\pi z}{4}}{z + 1} \right]_{z = 1} = \frac{\sqrt{2}}{2} \pi i
\end{gather}

\section{求下列积分的值,积分路径均沿直线}

\begin{enumerate}
\item $\displaystyle \int_{0}^{i} \frac{z}{z + 1} \  \mathrm{d} z$
\begin{gather}
I 
= \int_{0}^{i} \left( 1 - \frac{1}{z+1} \right) \  \mathrm{d} z = \left[ z - \ln (z+1) \right]_{0}^{i} = i - \ln (1+i) = - \frac{\ln 2}{2} + i\left( 1 - \frac{\pi}{4} \right)
\end{gather}

\item $\displaystyle \int_{0}^{1+i} z^2 \sin z \ \mathrm{d} z$
\begin{align*}
I &= \left[ - z^2 \cos z + 2z \sin z + 2 \cos z \right]_{0}^{1+i} 
\\ 
&= (2 - 2i)\cdot \frac{1}{2} \cdot \left[ \left(\frac{1}{e} + e \right)\cos 1 + \left(\frac{1}{e} - e \right)\sin 1 \right] + 2( 1 + i)\cdot\frac{1}{2i}\cdot  \left[ \left(\frac{1}{e} - e \right)\cos 1 + \left(\frac{1}{e} + e \right)\sin 1 \right] - 2
\\ 
&= \frac{2 (1 - i)}{e}(\cos 1 + i \sin 1) - 2
\end{align*}



\item $\displaystyle \int_{-1}^{i} \frac{1}{z^2 + z-2} \ \mathrm{d}z$

\begin{align}
I 
= \frac{1}{3} \int_{-1}^{i} \left( \frac{1}{z - 1} - \frac{1}{ z + 2} \right) \ \mathrm{d}z 
= \frac{1}{3} \left[ \ln (z - 1) - \ln (z + 2) \right]_{-1}^{i} 
= -\frac{1}{3}\left[  \frac{\ln 10}{2} + i\left( \arctan \frac{1}{2} + \frac{\pi}{4} \right) \right]
\end{align}

\end{enumerate}

\section{讨论下列各积分的值,其中积分路径是圆周 $| z | = r$}

\begin{enumerate}
\item $\displaystyle \oint_{| z | = r} \frac{z^3}{(z - 1)(z^2 + 2z +3)} \ \mathrm{d}z $

记 $z^2 + 2z + 3 = 0$ 的两个根分别为 $z_1 = -1 + i\sqrt{2} ,\ z_2 = -1 - i\sqrt{2} $,先考虑 $r \in (\sqrt{3},\ +\infty)$,此时积分围道内有三个奇点 $1,\ z_1,\ z_2$。由 Cauthy 定理,可得:
\begin{gather}
I = 2 \pi i \left\{ \left[\frac{z^3}{(z - z_1)(z - z_2)}\right]_{z = 1} + \left[\frac{z^3}{(z - 1)(z - z_2)}\right]_{z = z_1} + \left[\frac{z^3}{(z - 1)(z - z_1)}\right]_{z = z_2}  \right\} \\ 
= 2 \pi i \left[ \frac{1}{6} + \left( -\frac{1}{4}\cdot \frac{7 - i4 \sqrt{2} }{3} \right) + \left( -\frac{1}{4}\cdot \frac{7 + i4 \sqrt{2} }{3} \right)  \right] = - 2\pi i
\end{gather}

当 $r \in (0, 1)$ 时,无奇点,$I = 0$;当 $r \in (1 \sqrt{3} )$ 时,有唯一奇点 $z =1$,$I = 2 \pi i \cdot \frac{1}{6} = \frac{\pi}{3}i$。综上有:
\begin{equation}
I = I(r) = 
\begin{cases}
    0 &, r \in (0, 1) \\ 
    \displaystyle \frac{\pi}{3}i &, r \in (1, \sqrt{3}) \\ 
    -2\pi i &, r \in (\sqrt{3}, +\infty)
\end{cases}
\end{equation}

\item $\displaystyle \oint_{| z | = r} \frac{1}{z^3(z + 1)(z + 2)}  \ \mathrm{d}z $

先考虑 $r \in (2 , +\infty)$ 的情况,此时围道内有三个奇点 $0, -1, -2$,但我们不需要具体求解,直接由大圆弧定理:
\begin{equation}
\lim_{z \to \infty}  \left(z \cdot  \frac{1}{z^3(z + 1)(z + 2)} \right) = 0  \Longrightarrow I = 2 \pi i \cdot 0 = 0
\end{equation}

$r \in (1 ,2)$ 时,有两奇点 $0,\ -1$,于是:
\begin{equation}
I = 2 \pi i \left\{ \frac{1}{2!}\cdot \left[ \frac{1}{(z + 1)(z + 2)} \right]^{(2)}_{z = 0} + \left[ \frac{1}{z^3(z + 2)} \right]_{z = -1}\right\} = 2 \pi i \left[ \frac{7}{8} + (-1) \right] = -\frac{1}{4} \pi i
\end{equation}

再考虑上 $r \in (0, 1)$,综上有:
\begin{equation}
I = I(r) = 
\begin{cases}
    \displaystyle \frac{7}{4} \pi i &, r \in [0, 1) \\ 
    \displaystyle -\frac{1}{4} \pi i &, r \in (1, 2) \\ 
    0 &, r \in (2, +\infty)
\end{cases}
\end{equation}

\end{enumerate}

\section{设 $\displaystyle f(z) = \oint_{| \zeta  |= 2} \frac{3 \zeta^2 + 7 \zeta + 1}{\zeta - z} \ \mathrm{d}\zeta $,求 $f''(1+i)$}

由 Cauthy 积分公式:
\begin{equation}
f(z) = 2 \pi i \left[ 3z^2 + 7z + 1 \right],\quad \Longrightarrow  f''(1+i) = 2 \pi i \cdot 6 = 12 \pi i
\end{equation}

\section{计算积分 $\displaystyle f(z) = \oint_{| \zeta | = 1} \frac{\overline{\zeta}}{\zeta - z} \ \mathrm{d}\zeta$,其中 $| z | \ne 1$}

$| \zeta | = 1$ 时有 $\overline{\zeta} = \frac{1}{\zeta}$,$z = 0$ 的情况需单独计算,综合有:
\begin{equation}
f(z) = \oint_{| \zeta | = 1} \frac{1}{\zeta(\zeta - z)} \ \mathrm{d}\zeta =
\begin{cases}
    0 &, | z | \in [0, 1) \\ 
    -\frac{2\pi i}{z} &, | z | \in (1, +\infty)
\end{cases}
\end{equation}


\section{计算积分 $\displaystyle f(z) = \oint_{| \zeta | = 2} \frac{\overline{\zeta}^2 e^\zeta}{\zeta - z} \ \mathrm{d}\zeta $,其中 $| z | \ne 2$}

$| \zeta |  = 2$ 时 $\overline{\zeta} = \frac{4}{\zeta}$,于是 $| z | < 2$ 时:
\begin{align}
I 
&= 16 \oint_{| \zeta | = 2} \frac{e^\zeta}{\zeta^2(\zeta - z)} \ \mathrm{d}\zeta 
= 16 \cdot 2 \pi i \left\{ \left[\frac{e^\zeta}{\zeta - z}\right]^{(1)}_{\zeta = 0} + \left[\frac{e^\zeta}{\zeta^2}\right]_{\zeta = z}\right\} \\
&= 32 \pi i \left[ \left(-\frac{z+1}{z^2}\right) + \frac{e^z}{z^2}  \right] = 32 \pi i \cdot \frac{e^z - z - 1}{z^2} 
\end{align}
$| z | = 0$ 时 $I = 16 \oint_{| \zeta | = 2} \frac{e^\zeta}{\zeta^3} \ \mathrm{d}\zeta = 16 \pi i$,再考虑上 $| z | > 2$,综合有:
\begin{equation}
I = f(z) =
\begin{cases}
    16 \pi i &,  | z | = 0 \\ 
    32 \pi i \cdot \frac{e^z - z - 1}{z^2} &, | z | \in (0, 2) \vspace*{1mm} \\ 
    -32 \pi i \cdot \frac{z+1}{z^2} &, | z | \in (2, +\infty)
\end{cases}
\end{equation}

\section{计算积分 $\displaystyle \oint_{| z | = 1} \frac{e^z }{z^3} \ \mathrm{d}z$}
由高阶导数公式:
\begin{equation}
I = 2 \pi i \cdot \frac{1}{2!} \cdot\left[e^z \right]^{(2)}_{z = 0} =  \pi i
\end{equation}

\section{求 $a$ 的值使得函数 $\displaystyle F(z) = \int_{z_0}^z e^z \left( \frac{1}{z} + \frac{a}{z^3}\right) \ \mathrm{d}z$ 是单值的}

$F(z)$ 是单值的,也即积分与路径无关,这等价于被积函数是解析函数,由于没有限制 $z$ 的范围,也即 $z \in \C$,因此:
\begin{equation}
\oint_{\partial G} \left( \frac{e^z}{z} + a\frac{e^z}{z^3} \right) \ \mathrm{d}z = 0,\quad \forall\ G \subset \C 
\end{equation}
计算左边的积分:
\begin{equation}
I =  2 \pi i \left\{ \left[e^z\right]_{z=0} + \frac{a}{2!}\cdot\left[e^z\right]^{(2)}_{z=0} \right\} = 2 \pi i \left( 1 + \frac{a}{2} \right) = 0 \Longrightarrow a = -2
\end{equation}

\chapter{Homework 6: 2024.10.8 - 2024.10.14}\thispagestyle{fancy}

求幂级数的收敛半径有两个常用方法:
\begin{gather}
\frac{1}{R} = \overline{\lim_{n \to \infty}} \, | c_n |^{\frac{1}{n}} ,\quad 
\frac{1}{R} = \lim_{n \to \infty} \left| \frac{c_{n+1}}{c_{n}} \right|
\end{gather}

前者称为 Cauchy-Hadamard 公式,是普遍成立的,后者称为 d'Alembert 公式,在极限存在时成立,但通常计算更简单。

\section{确定下列幂级数的收敛半径}

\subsection{$\sum_{n=1}^{\infty} \frac{z^n}{n}$}
\noindent
\begin{equation}
    \frac{1}{R} = \lim_{n \to \infty} \frac{n}{n+1} = 1 \Longrightarrow R = 1
\end{equation}

\subsection{$\sum_{n=1}^{\infty} n^n z^n $}
\noindent
\begin{equation}
    \frac{1}{R} = \lim_{n \to \infty} \frac{(n+1)^{n+1}}{n^n} = \lim_{n \to \infty} (n+1)\cdot \left(1 + \frac{1}{n}\right)^n = \infty \cdot e \Longrightarrow R = 0
\end{equation}

\subsection{$\sum_{n=1}^{\infty} z^{n!},\quad 
???
$}

\subsection{$\sum_{n=1}^{\infty} z^{2n}$}
\noindent
\begin{equation}
\text{级数 $\sum_{n=1}^{\infty} z^{n} $ 的收敛半径 $r = 1 \Longrightarrow  \sum_{n=1}^{\infty} z^{2n}$ 的收敛半径为 $R = \sqrt{r} = 1$。}
\end{equation}



\subsection{$\sum_{n=1}^{\infty} \left[3 + (-1)^n\right]^n z^n$}
\noindent
\begin{equation}
\frac{1}{R} = \overline{\lim_{n \to \infty}} \left[3 + (-1)^n\right] = 4 \Longrightarrow R = \frac{1}{4}
\end{equation}


\subsection{$\sum_{n=1}^{\infty} cos(in) \cdot z^n$}
\noindent
\begin{equation}
\frac{1}{R} 
= \lim_{n \to \infty} \left| \frac{\cos (in+i)}{\cos (in)} \right| 
= \lim_{n \to \infty} \left| \cos i - \sin i \cdot \tan (in) \right|
= \left| \cos i - i\sin i \right|
= | e^{i(-i)} | = e \Longrightarrow R = \frac{1}{e}
\end{equation}

\subsection{$\sum_{n=1}^{\infty} (n + a^n) z^n$}
\noindent
\begin{equation}
\frac{1}{R} 
= \lim_{n \to \infty} \left| \frac{(n+1) + a^{n+1}}{n + a^n} \right| 
= \lim_{n \to \infty} \left|\frac{1 + a\cdot \left(\frac{a^n}{n}\right)}{1 + \left(\frac{a^n}{n}\right)} \right| 
= \begin{cases}
    1, & | a | \leqslant 1 \\
    a, & | a | > 1
\end{cases}\Longrightarrow 
R =
\begin{cases}
    1, & | a | \leqslant 1 \\
    \frac{1}{|a|}, & | a | > 1
\end{cases}
\end{equation}

\subsection{$\sum_{n=1}^{\infty} (1 - \frac{1}{n})^n z^n$}
\noindent
\begin{equation}
\frac{1}{R} = \overline{\lim_{n \to \infty}} \left(1 - \frac{1}{n}\right) = 1 \Longrightarrow R = 1
\end{equation}

\section{设幂级数 $\sum_{n=1}^{\infty} c_n z^n$ 的收敛半径为 $R \in (0,\infty)$,求下列幂级数的收敛半径}

\subsection{$\sum_{n=1}^{\infty} n^Rc_n z^n$}
\noindent
\begin{equation}
\frac{1}{R_1} 
= \lim_{n \to \infty} \left| \frac{(n+1)^R c_{n+1}}{n^R c_n} \right|
= \lim_{n \to \infty} \left(1 + \frac{1}{n}\right)^R \left| \frac{c_{n+1}}{c_n} \right|
= 1 \cdot \frac{1}{R} \Longrightarrow R_1 = R
\end{equation}

\subsection{$\sum_{n=1}^{\infty} (2^n - 1)c_n z^n$}
\noindent
\begin{equation}
\frac{1}{R_2} 
= \lim_{n \to \infty} \frac{2\cdot 2^n - 1}{2^n - 1} \left| \frac{c_{n+1}}{c_n} \right|
= 2\cdot \frac{1}{R} \Longrightarrow R_2 = \frac{R}{2}
\end{equation}

\subsection{$\sum_{n=1}^{\infty} (c_n)^k z^n$}
\noindent
\begin{equation}
\frac{1}{R_3}
= \lim_{n \to \infty} \left| \frac{c_{n+1}}{c_n} \right|^k
= \frac{1}{R^k} \Longrightarrow R_3 = R^k
\end{equation}

\section{证明级数 $\sum_{n=1}^{\infty} \frac{z^{n-1}}{(1 - z^n)(1 - z^{n+1})}$ 在 $ | z | \ne 1$ 上收敛,并求其和函数}
\begin{gather*}
    S_n(z) 
    = \sum_{k=1}^{n} \frac{z^{k-1}}{(1 - z^k)(1 - z^{k+1})} 
    = \frac{1}{z(1 - z)} \cdot \sum_{k=1}^{n} \left( \frac{1}{1 - z^k} - \frac{1}{1 - z^{k+1}} \right) 
    = \frac{1}{z(1 - z)} \cdot \left[ \frac{1}{z^{n+1} - 1} - \frac{1}{z - 1} \right]
    \\ 
    \Longrightarrow 
    S(z) = \lim_{n \to \infty}S_n(z) = 
    \begin{cases}
        \frac{1}{(1 - z)^2} &, | z | < 1 \vspace*{2mm} \\ 
        \frac{1}{z(1 - z)^2} &, | z | > 1
    \end{cases}
\end{gather*}
因此级数在 $ | z | \ne 1$ 上收敛。


\section{证明级数 $\sum_{n=0}^{\infty} \left( \frac{z^{n+1}}{n+1} - \frac{2z^{2n+3}}{2n+3} \right)$ 的和函数 $S = S(z)$ 在 $z = 1$ 不连续}
容易知道上面级数在 $| z | < 1$ 收敛而在 $| z | > 1$ 发散,因此在 $| z | = 1$ 处不连续 $\Longrightarrow $ 在 $z = 1$ 点不连续。但我们不妨求解一下和函数。

先求和函数 $S(z),\ | z | < 1$。级数 $\sum_{n=0}^{\infty} \frac{z^{n+1}}{n+1}$ 和 $\sum_{n=0}^{\infty} \frac{2z^{2n+3}}{2n+3}$ 的收敛半径都为 1,,因此当 $| z | < 1$ 时,由一致收敛性有:
\begin{gather}
    \sum_{n=0}^{\infty} \frac{z^{n+1}}{n+1} 
    = \int \left[ \sum_{n=0}^{\infty}  \frac{\mathrm{d} }{\mathrm{d} z }  \left(\frac{z^{n+1}}{n+1}\right) \right]\ \mathrm{d}z
    = \int \left[  \sum_{n=0}^{\infty} z^n  \right]\ \mathrm{d}z 
    = \int \frac{1}{1 - z} \ \mathrm{d}z
    = - \ln (z - 1) + C_1
\end{gather}
$z = 0$ 时级数为 0,因此 $C_1 = \ln (-1)$。同理可得:
\begin{gather}
    \sum_{n=0}^{\infty} \frac{2z^{2n+3}}{2n+3}
    = 2 \int \left[ \sum_{n=0}^{\infty}  \frac{\mathrm{d} }{\mathrm{d} z }  \left(\frac{z^{2n+3}}{2n+3}\right) \right]\ \mathrm{d}z
    = 2 \int \left[  \sum_{n=0}^{\infty} \left(z^2\right)^{n+1}  \right]\ \mathrm{d}z 
    = 2 \int \frac{z^2}{1 - z^2} \ \mathrm{d}z \\ 
    = 2 \int \left[ -1 - \frac{1}{2}\left( \frac{1}{z - 1} - \frac{1}{z + 1}\right) \right] \ \mathrm{d}z
    = - 2z - \ln \left(\frac{z - 1}{z + 1}\right)  + C_2
\end{gather}
$z = 0$ 时级数为 0,因此 $C_2 = 0$。由于原级数在 $| z | < 1$ 内绝对收敛,可以任意交换求和次序,因此有:
\begin{equation}
    \sum_{n=0}^{\infty} \left( \frac{z^{n+1}}{n+1} - \frac{2z^{2n+3}}{2n+3} \right) 
    = \sum_{n=0}^{\infty}\frac{z^{n+1}}{n+1} - \sum_{n=0}^{\infty}\frac{2z^{2n+3}}{2n+3}
    =  - \left[ \ln(z - 1) + 2z + \ln \left(\frac{z - 1}{z + 1}\right) \right] + \ln (-1)
\end{equation}
于是极限 $\lim_{z \to 1} S(z)$ 不存在,自然不可能连续。


\section{对 $| z | < 1$,求下列级数的和}



\subsection{$\sum_{n=1}^{\infty} n z^n$}
级数的收敛半径为 1,由绝对收敛性: 
\begin{equation}
    \sum_{n=1}^{\infty} n z^n = \sum_{n=1}^{\infty} (n+1) z^n - \sum_{n=1}^{\infty} z^n = \sum_{n=1}^{\infty} (n+1) z^n - \frac{z}{1 - z}
\end{equation}

级数 $\sum_{n=1}^{\infty} (n+1) z^n$ 的收敛半径仍为 1,由一致收敛性: 
\begin{equation}
    \sum_{n=1}^{\infty} (n+1) z^n
    = \frac{\mathrm{d} }{\mathrm{d} z } \left[ \sum_{n=1}^{\infty} \left( \int (n+1) z^n \ \mathrm{d}z\right) \right]
    = \frac{\mathrm{d} }{\mathrm{d} z } \left[ \sum_{n=1}^{\infty}  z^{n+1}  \right]
    = \frac{\mathrm{d} }{\mathrm{d} z } \left[ \frac{z^2}{1 - z} \right]
    = \frac{z(2 - z)}{(1 - z)^2}
\end{equation}
综上有:
\begin{equation}
    \sum_{n=1}^{\infty} n z^n 
    = \frac{z(2 - z)}{(1 - z)^2} - \frac{z}{1 - z} 
    = \frac{z}{(1 - z)^2}
\end{equation}

\subsection{$\sum_{n=1}^{\infty} \frac{z^{2n+1}}{2n+1}$}
由一致收敛性:
\begin{gather}
    \sum_{n=1}^{\infty} \frac{z^{2n+1}}{2n+1}
    =  \int \left[ \sum_{n=1}^{\infty}  \frac{\mathrm{d} }{\mathrm{d} z }  \left(\frac{z^{2n+1}}{2n+1}\right) \right]\ \mathrm{d}z
    =  \int \left[  \sum_{n=1}^{\infty} \left(z^2\right)^{n}  \right]\ \mathrm{d}z 
    =  \int \frac{z^2}{1 - z^2} \ \mathrm{d}z \\ 
    =  \int \left[ -1 - \frac{1}{2}\left( \frac{1}{z - 1} - \frac{1}{z + 1}\right) \right] \ \mathrm{d}z
    =  - z - \frac{1}{2}\ln \left(\frac{z - 1}{z + 1}\right)  + C
\end{gather}
$z = 0$ 时级数为 0,因此 $C = 0$。

\subsection{$\sum_{n=1}^{\infty} (-1)^{n+1}\frac{z^n}{n}$}
由一致收敛性:
\begin{gather}
    \sum_{n=0}^{\infty} \frac{z^{n+1}}{n+1} 
    = \int \left[ \sum_{n=0}^{\infty}  \frac{\mathrm{d} }{\mathrm{d} z }  \left(\frac{z^{n+1}}{n+1}\right) \right]\ \mathrm{d}z
    = \int \left[  \sum_{n=0}^{\infty} z^n  \right]\ \mathrm{d}z 
    = \int \frac{1}{1 - z} \ \mathrm{d}z
    = - \ln (z - 1) + C_1
\end{gather}
$z = 0$ 时级数为 0,因此 $C_1 = \ln (-1)$。

\section{证明级数 $\sum_{n=1}^{\infty} \frac{(-1)^{n-1}}{z + n}$ 在不包含负整数的任意闭圆上一致收敛}

\noindent 首先有两个引理:

\begin{LineTheorem}[Dirichlet 判别法]\label{Dirichlet 判别法}
    设 $\sum_{n=1}^{\infty} a_n $ 有界,$\sum_{n=1}^{\infty} \left(v_{n+1} - v_{n}\right)$ 绝对收敛且 $\lim v_n = 0$,则 $\sum_{n=1}^{\infty} a_n v_n$ 收敛。
\end{LineTheorem}
\begin{LineTheorem}[级数收敛]\label{级数收敛}
    级数 $\sum_{n=1}^{\infty} \frac{(-1)^{n-1}}{n}$ 收敛,但不绝对收敛。证明略。
\end{LineTheorem}

\noindent 在 Theorem.\ref{级数收敛} 的基础上,由 Theorem.\ref{Dirichlet 判别法} (Dirichlet 判别法) 知 $\sum_{n=1}^{\infty} \frac{(-1)^{n-1}}{z + n}$ 收敛,可以任意加括号。给定不包含负整数的任意闭圆 $G$,记 $r = \left| z \right| $,$N_0 = \sup_z \left \lceil \frac{| z | + 1}{2}  \right \rceil $,则有:
\begin{gather}
S(z) 
= \sum_{n=1}^{\infty} \frac{(-1)^{n-1}}{z + n}
= \sum_{n=1}^{\infty} \left( \frac{1}{z+ 2n -1} - \frac{1}{z + 2n } \right)
= \sum_{n=1}^{\infty} \frac{1}{ (z + 2n -1)(z + 2n) } 
\\
\Longrightarrow  
\left| S(z) \right| 
\leqslant \sum_{n=1}^{\infty} \frac{1}{\left| z + 2n -1 \right|  \cdot \left| z + 2n \right| } 
= \sum_{n=1}^{N_0 - 1} \frac{1}{\left| z + 2n -1 \right|  \cdot \left| z + 2n \right| }  + \sum_{n=N_0}^{\infty} \frac{1}{\left| z + 2n -1 \right|  \cdot \left| z + 2n \right| } 
\\
\leqslant \sum_{n=1}^{\infty} \frac{1}{\left| 2n - 1 -r \right| \cdot \left| 2n - r \right| }
\leqslant \sum_{n=1}^{\infty} \frac{1}{\left| 2n - 1 -r \right|^2} < \infty
\end{gather}

\begin{align}
S(z) 
&= \sum_{n=1}^{\infty} \frac{(-1)^{n-1}}{z + n}
= \sum_{n=1}^{\infty} \left( \frac{1}{z+ 2n -1} - \frac{1}{z + 2n } \right)
= \sum_{n=1}^{\infty} \frac{1}{ (z + 2n -1)(z + 2n) } 
\\
\Longrightarrow  
\left| S(z) \right| 
&\leqslant \sum_{n=1}^{\infty} \frac{1}{\left| z + 2n -1 \right|  \cdot \left| z + 2n \right| } 
= \sum_{n=1}^{N_0 - 1} \frac{1}{\left| z + 2n -1 \right|  \cdot \left| z + 2n \right| }  + \sum_{n=N_0}^{\infty} \frac{1}{\left| z + 2n -1 \right|  \cdot \left| z + 2n \right| } 
\\ 
& \leqslant \sum_{n=1}^{N_0 - 1} \frac{1}{\left| z + 2n -1 \right|  \cdot \left| z + 2n \right| }  + \sum_{n=N_0}^{\infty} \frac{1}{\left|2n -1 - r \right|  \cdot \left|2n - r\right| } 
\\
& \leqslant \sum_{n=1}^{N_0 - 1} \frac{1}{\left| z + 2n -1 \right|  \cdot \left| z + 2n \right| }  + \sum_{n=N_0}^{\infty} \frac{1}{\left|2n -1 - r \right|^2 } 
\end{align}
对给定的区域 $G$,前一项是有限和,自然收敛,后一项是收敛级数,因此原级数在 $G$ 上一致收敛。 $\quad\square$

\chapter{Homework 7: 2024.10.15 - 2024.10.16 (略)}\thispagestyle{fancy}



\chapter{Homework 8: 2024.10.15 - 2024.10.21}\thispagestyle{fancy}
\vspace*{-6mm}
\begin{graybox}
作业题目详见网址 \href{https://www.123865.com/s/0y0pTd-jKKj3}{https://www.123865.com/s/0y0pTd-jKKj3},后文不再重复叙述题目。
\end{graybox}

\section{}
\begin{enumerate}
\item $\frac{1}{z - z^3}$ : 
有 $z = 0, 1, -1$ 三个一阶极点
\item $\cos \frac{1}{\sqrt{z}}$ : 
由 $\lim_{z\to 0} z\cos \frac{1}{\sqrt{z}} = 0$ 知,有一阶极点 $z=0$
\item $\frac{\sqrt{z}}{\sin \sqrt{z}}$ : 
$\lim_{z\to 0} \frac{\sqrt{z}}{\sin \sqrt{z}} = \lim_{z\to 0}\frac{z}{\sin z} = 1$,因此有可去极点 $z = 0$ 
\item $\frac{1}{(z-1)\ln z}$ : 
$\lim_{z\to 1} (z-1)^2\cdot \frac{1}{(z-1)\ln z} = 1$,因此有二阶极点 $z=1$
\item $f(z) = \int_{0}^{z} \frac{\sin \sqrt{\zeta}}{\sqrt{\zeta}} \ \mathrm{d}\zeta$ : 
作换元 $t = \sqrt{\zeta}$,可得 $f(z) = 1 - \cos t = 1 - \cos \sqrt{z}$,在 $\C$ 上无奇点,在 $\overline{\C}$ 上有本性奇点 $z = \infty$
\item $\frac{1 - e^z}{2 + e^z}$ : 
$\lim_{z\to \infty} \frac{1 - e^z}{2 + e^z} = -1$,因此有且仅有可去奇点 $z = \infty$
\item $\frac{1}{z^3(2 - \cos z)}$ : 有三阶极点 $z = 0$
\end{enumerate}

\section{}
\begin{enumerate}
\item $\lim_{z\to \infty} \frac{\cos z}{z} = 0$,是可去奇点
\item 做换元 $t =  \frac{1}{z}$,有 $\lim_{t\to 0} t^2\cdot \frac{1}{t\cos \frac{1}{t}} = 0$,因此为二阶极点。
\item 做换元 $t =  \frac{1}{z}$,有 $\lim_{t \to 0} t \cdot \sqrt{\left(\frac{1}{t} - a\right)\left(\frac{1}{t} - b\right)} = \lim_{t \to 0} \sqrt{(1 - at)(1-bt)} = 1$,因此为一阶极点。
\end{enumerate}

\section{}
\begin{enumerate}
\item $f(z) = \frac{e^{z^2}}{z - 1}$,$z_0 = 1$ :

$z_0 = 1$ 为一阶极点,因此:
\begin{equation}
    \res f(1) = \lim_{z \to 1}(z-1)f(z) = e
\end{equation}

\item $f(z) = \frac{z^2 + z -1}{z^2(z-1)}$ :

$f(z)$ 有二阶极点 $z= 0$ 和一阶极点 $z = 1$,于是:
\begin{gather}
\res f(0) = \left[z^2 f(z)\right]^{(1)}_{z = 0} = \left[\frac{2\,z+1}{z-1}-\frac{z^2 +z-1}{{{\left(z-1\right)}}^2 }\right]_{z=0}= \left[-\frac{2\,z-z^2 }{{{\left(z-1\right)}}^2 }\right]_{z=0} = 0\\ 
\res f(1) = \lim_{z \to 1} (z-1)f(z) = 1
\end{gather}

\item $f(z) = \frac{e^z}{z^2(z^2 + 9)}$ :

有一阶极点 $z = \pm 3 i$ 和二阶极点 $z= 0$,因此:
\begin{equation}
    \res f(3i) = \lim_{z \to 3i} (z-3i)f(z) = -\frac{e^{3i}}{54 i}
    ,\quad \res f(-3i) = \lim_{z \to 3i} (z-3i)f(z) = \frac{e^{-3i}}{54 i}
\end{equation}
\begin{gather}
    \res f(0) =  \left[z^2 f(z)\right]^{(1)}_{z = 0} =  \left[ \frac{e^z(z^2-2z+9)}{(z^2 + 9)^2} \right]_{z = 0} = \frac{1}{9} 
\end{gather}

\item $\frac{1}{z^2 \sin z}$,$z_0 = 0$ :

$z_0 = 0$ 为三阶极点,因此:
\begin{equation}
\res f(0) =  \frac{1}{2!}\left[z^3 f(z)\right]^{(2)}_{z = 0}
= \frac{1}{2} \left[\frac{2\,z\,{\cos \left(z\right)}^2 -2\,\cos \left(z\right)\,\sin \left(z\right)+z\,{\sin \left(z\right)}^2 }{{\sin \left(z\right)}^3 }\right]_{z = 0} = \frac{1}{2}\cdot\frac{1}{3} = \frac{1}{6}
\end{equation}

\item $\frac{1}{\cosh \sqrt{z}}$,$z_0 = -\left(\frac{2n + 1}{2} \pi\right)^2$ :

考虑 $\sqrt{z}$ 满足 $\sqrt{z}\mid_{z = 0} = 0$ 的单值分支,我们有$\sqrt{-\left(\frac{2n + 1}{2} \pi\right)^2} = i\cdot \left(\frac{\pi}{2} + n\pi\right)$,因此本题相当于求 $f(z) = \frac{1}{\cosh z}$,$z_0 =i \cdot \left(\frac{\pi}{2} + n\pi\right)$ 处的留数,由于 
\begin{equation}
    \cosh z = \frac{e^z + e^{-z}}{2} = \frac{1}{2}\left[\left(e^x + \frac{1}{e^x}\right)\cos y + i\cdot\left(e^x - \frac{1}{e^x}\right)\sin y\right]
\end{equation}
我们有:
\begin{gather}
    z = z_0 \Longleftrightarrow 
    \begin{cases}
    x = 0 \\ 
    y = \pm \frac{\pi}{2} + n\pi
    \end{cases} \Longrightarrow \frac{1}{\cosh \sqrt{z_0}} = \frac{1}{\cos y} = \infty \\ 
    \lim_{z \to z_0} (z - z_o)f(z) = \lim_{z \to z_0} \frac{2(z - z_0)}{e^z - e^{-z}} \overset{\text{L'H}}{=} \lim_{z \to z_0} \frac{2}{e^z - e^{-z}} = \pm 1
\end{gather}
因此都是一阶极点,有:
\begin{equation}
\res f(z_0) = \lim_{z\to z_0}(z - z_0)f(z) = 
\begin{cases}
    -i &, n = 0, 2, 4, \dots \\
    i &, n = 1, 3, 5, \dots
\end{cases}
\end{equation}

\end{enumerate}








\end{document}



% VScode 常用快捷键:

% F2:                       变量重命名
% Ctrl + Enter:             行中换行
% Alt + up/down:            上下移行
% 鼠标中键 + 移动:           快速多光标
% Shift + Alt + up/down:    上下复制
% Ctrl + left/right:        左右跳单词
% Ctrl + Backspace/Delete:  左右删单词    
% Shift + Delete:           删除此行
% Ctrl + J:                 打开 VScode 下栏(输出栏)
% Ctrl + B:                 打开 VScode 左栏(目录栏)
% Ctrl + `:                 打开 VScode 终端栏
% Ctrl + 0:                 定位文件
% Ctrl + Tab:               切换已打开的文件(切标签)
% Ctrl + Shift + P:         打开全局命令(设置)

% Latex 常用快捷键

% Ctrl + Alt + J:           由代码定位到PDF
% 


% Git提交规范:
% update: Linear Algebra 2 notes
% add: Linear Algebra 2 notes
% import: Linear Algebra 2 notes
% delete: Linear Algebra 2 notes
