% 若编译失败,且生成 .synctex(busy) 辅助文件,可能有两个原因:
% 1. 需要插入的图片不存在:Ctrl + F 搜索 'figure' 将这些代码注释/删除掉即可
% 2. 路径/文件名含中文或空格:更改路径/文件名即可

% --------------------- 文章宏包及相关设置 --------------------- %
% >> ------------------ 文章宏包及相关设置 ------------------ << %
% 设定文章类型与编码格式
\documentclass[UTF8]{report}		

% 本文特殊宏定义
\def\Re{\mathrm{\,Re}\,}
\def\Im{\mathrm{\,Im}\,}
\def\sinc{\mathrm{\,sinc}\,}
\def\Jc{\mathrm{\,Jc}}


% 自定义宏定义
    \def\N{\mathbb{N}}
    \def\F{\mathbb{F}}
    \def\Z{\mathbb{Z}}
    \def\Q{\mathbb{Q}}
    \def\R{\mathbb{R}}
    \def\C{\mathbb{C}}
    \def\T{\mathbb{T}}
    \def\S{\mathbb{S}}
    \def\A{\mathbb{A}}
    \def\I{\mathscr{I}}
    \def\Im{\mathrm{Im\,}}
    \def\Re{\mathrm{Re\,}}
    \def\d{\mathrm{d}}
    \def\p{\partial}

% 导入基本宏包
    \usepackage[UTF8]{ctex}     % 设置文档为中文语言
        \usepackage{hyperref}  % 宏包:自动生成超链接 (此宏包与标题中的数学环境冲突)
    \hypersetup{
        colorlinks=true,    % false:边框链接 ; true:彩色链接
        citecolor={blue},    % 文献引用颜色
        linkcolor={blue},   % 目录 (我们在目录处单独设置),公式,图表,脚注等内部链接颜色
        urlcolor={magenta},    % 网页 URL 链接颜色,包括 \href 中的 text
        % cyan 浅蓝色 
        % magenta 洋红色
        % yellow 黄色
        % black 黑色
        % white 白色
        % red 红色
        % green 绿色
        % blue 蓝色
        % gray 灰色
        % darkgray 深灰色
        % lightgray 浅灰色
        % brown 棕色
        % lime 石灰色
        % olive 橄榄色
        % orange 橙色
        % pink 粉红色
        % purple 紫色
        % teal 蓝绿色
        % violet 紫罗兰色
    }
    % \usepackage{docmute}    % 宏包:子文件导入时自动去除导言区,用于主/子文件的写作方式,\include{./51单片机笔记}即可。注:启用此宏包会导致.tex文件capacity受限。
    \usepackage{amsmath}    % 宏包:数学公式
    \usepackage{mathrsfs}   % 宏包:提供更多数学符号
    \usepackage{amssymb}    % 宏包:提供更多数学符号
    \usepackage{pifont}     % 宏包:提供了特殊符号和字体
    \usepackage{extarrows}  % 宏包:更多箭头符号
    \usepackage{multicol}   % 宏包:支持多栏 


% 文章页面margin设置
    \usepackage[a4paper]{geometry}
        \geometry{top=1in}
        \geometry{bottom=1in}
        \geometry{left=0.75in}
        \geometry{right=0.75in}   % 设置上下左右页边距
        \geometry{marginparwidth=1.75cm}    % 设置边注距离(注释、标记等)

% 配置数学环境
    %\everymath{\displaystyle}   % 设置全文数学公式都为展示样式
    \usepackage{amsthm} % 宏包:数学环境配置
    % theorem-line 环境自定义
        \newtheoremstyle{MyLineTheoremStyle}% <name>
            {11pt}% <space above>
            {11pt}% <space below>
            {\kaishu}% <body font> 使用默认正文字体
            {}% <indent amount>
            {\bfseries}% <theorem head font> 设置标题项为加粗
            {:}% <punctuation after theorem head>
            {.5em}% <space after theorem head>
            {\textbf{#1}\thmnumber{#2}\ \ (\,\textbf{#3}\,)}% 设置标题内容顺序
        \theoremstyle{MyLineTheoremStyle} % 应用自定义的定理样式
        \newtheorem{LineTheorem}{Theorem.\,}
    % theorem-block 环境自定义
        \newtheoremstyle{MyBlockTheoremStyle}% <name>
            {11pt}% <space above>
            {11pt}% <space below>
            {\kaishu}% <body font> 使用默认正文字体
            {}% <indent amount>
            {\bfseries}% <theorem head font> 设置标题项为加粗
            {:\\ \indent}% <punctuation after theorem head>
            {.5em}% <space after theorem head>
            {\textbf{#1}\thmnumber{#2}\ \ (\,\textbf{#3}\,)}% 设置标题内容顺序
        \theoremstyle{MyBlockTheoremStyle} % 应用自定义的定理样式
        \newtheorem{BlockTheorem}[LineTheorem]{Theorem.\,} % 使用 LineTheorem 的计数器
    % definition 环境自定义
        \newtheoremstyle{MySubsubsectionStyle}% <name>
            {11pt}% <space above>
            {11pt}% <space below>
            {}% <body font> 使用默认正文字体
            {}% <indent amount>
            {\bfseries}% <theorem head font> 设置标题项为加粗
            {:\\ \indent}% <punctuation after theorem head>
            {0pt}% <space after theorem head>
            {\textbf{#3}}% 设置标题内容顺序
        \theoremstyle{MySubsubsectionStyle} % 应用自定义的定理样式
        \newtheorem{definition}{}

%宏包:有色文本框(proof环境)及其设置
    \usepackage[dvipsnames,svgnames]{xcolor}    %设置插入的文本框颜色
    \usepackage[strict]{changepage}     % 提供一个 adjustwidth 环境
    \usepackage{framed}     % 实现方框效果
        \definecolor{graybox_color}{rgb}{0.95,0.95,0.96} % 文本框颜色。修改此行中的 rgb 数值即可改变方框纹颜色,具体颜色的rgb数值可以在网站https://colordrop.io/ 中获得。(截止目前的尝试还没有成功过,感觉单位不一样)(找到喜欢的颜色,点击下方的小眼睛,找到rgb值,复制修改即可)
        \newenvironment{graybox}{%
        \def\FrameCommand{%
        \hspace{1pt}%
        {\color{gray}\small \vrule width 2pt}%
        {\color{graybox_color}\vrule width 4pt}%
        \colorbox{graybox_color}%
        }%
        \MakeFramed{\advance\hsize-\width\FrameRestore}%
        \noindent\hspace{-4.55pt}% disable indenting first paragraph
        \begin{adjustwidth}{}{7pt}%
        \vspace{2pt}\vspace{2pt}%
        }
        {%
        \vspace{2pt}\end{adjustwidth}\endMakeFramed%
        }

% 外源代码插入设置
    % matlab 代码插入设置
    \usepackage{matlab-prettifier}
        \lstset{style=Matlab-editor}    % 继承 matlab 代码高亮 , 此行不能删去
    \usepackage[most]{tcolorbox} % 引入tcolorbox包 
    \usepackage{listings} % 引入listings包
        \tcbuselibrary{listings, skins, breakable}
        \newfontfamily\codefont{Consolas} % 定义需要的 codefont 字体
        \lstdefinestyle{MatlabStyle_inc}{   % 插入代码的样式
            language=Matlab,
            basicstyle=\small\ttfamily\codefont,    % ttfamily 确保等宽 
            breakatwhitespace=false,
            breaklines=true,
            captionpos=b,
            keepspaces=true,
            numbers=left,
            numbersep=15pt,
            showspaces=false,
            showstringspaces=false,
            showtabs=false,
            tabsize=2,
            xleftmargin=15pt,   % 左边距
            %frame=single, % single 为包围式单线框
            frame=shadowbox,    % shadowbox 为带阴影包围式单线框效果
            %escapeinside=``,   % 允许在代码块中使用 LaTeX 命令 (此行无用)
            %frameround=tttt,    % tttt 表示四个角都是圆角
            framextopmargin=0pt,    % 边框上边距
            framexbottommargin=0pt, % 边框下边距
            framexleftmargin=5pt,   % 边框左边距
            framexrightmargin=5pt,  % 边框右边距
            rulesepcolor=\color{red!20!green!20!blue!20}, % 阴影框颜色设置
            %backgroundcolor=\color{blue!10}, % 背景颜色
        }
        \lstdefinestyle{MatlabStyle_src}{   % 插入代码的样式
            language=Matlab,
            basicstyle=\small\ttfamily\codefont,    % ttfamily 确保等宽 
            breakatwhitespace=false,
            breaklines=true,
            captionpos=b,
            keepspaces=true,
            numbers=left,
            numbersep=15pt,
            showspaces=false,
            showstringspaces=false,
            showtabs=false,
            tabsize=2,
        }
        \newtcblisting{matlablisting}{
            %arc=2pt,        % 圆角半径
            % 调整代码在 listing 中的位置以和引入文件时的格式相同
            top=0pt,
            bottom=0pt,
            left=-5pt,
            right=-5pt,
            listing only,   % 此句不能删去
            listing style=MatlabStyle_src,
            breakable,
            colback=white,   % 选一个合适的颜色
            colframe=black!0,   % 感叹号后跟不透明度 (为 0 时完全透明)
        }
        \lstset{
            style=MatlabStyle_inc,
        }

% table 支持
    \usepackage{booktabs}   % 宏包:三线表
    \usepackage{tabularray} % 宏包:表格排版
    \usepackage{longtable}  % 宏包:长表格

% figure 设置
    \usepackage{graphicx}  % 支持 jpg, png, eps, pdf 图片 
    \usepackage{subcaption} % 支持子图
    \usepackage{svg}       % 支持 svg 图片
        \svgsetup{
            % 指向 inkscape.exe 的路径
            inkscapeexe = C:/aa_MySame/inkscape/bin/inkscape.exe, 
            % 一定程度上修复导入后图片文字溢出几何图形的问题
            inkscapelatex = false                 
        }

% 图表进阶设置
    \usepackage{caption}    % 图注、表注
        \captionsetup[figure]{name=图}  
        \captionsetup[table]{name=表}
        \captionsetup{
            labelfont=bf, % 设置标签为粗体
            textfont=bf,  % 设置文本为粗体
            font=small  
        }
    \usepackage{float}     % 图表位置浮动设置 
    \usepackage{etoolbox} % 用于保证图注表注的数学字符为粗体
        \AtBeginEnvironment{figure}{\boldmath} % 图注中的数学字符为粗体
        \AtBeginEnvironment{table}{\boldmath}  % 表注中的数学字符为粗体
        \AtBeginEnvironment{tabular}{\unboldmath}   % 保证表格中的数学字符不受额外影响

% 圆圈序号自定义
    \newcommand*\circled[1]{\tikz[baseline=(char.base)]{\node[shape=circle,draw,inner sep=0.8pt, line width = 0.03em] (char) {\small \bfseries #1};}}   % TikZ solution

% 列表设置
\usepackage{enumitem}   % 宏包:列表环境设置
    \setlist[enumerate]{
        label=\bfseries(\arabic*) ,   % 设置序号样式为加粗的 (1) (2) (3)
        ref=\arabic*, % 如果需要引用列表项,这将决定引用格式(这里仍然使用数字)
        itemsep=0pt, parsep=0pt, topsep=0pt, partopsep=0pt, leftmargin=3.5em} 
    \setlist[itemize]{itemsep=0pt, parsep=0pt, topsep=0pt, partopsep=0pt, leftmargin=3.5em}
    \newlist{circledenum}{enumerate}{1} % 创建一个新的枚举环境  
    \setlist[circledenum,1]{  
        label=\protect\circled{\arabic*}, % 使用 \arabic* 来获取当前枚举计数器的值,并用 \circled 包装它  
        ref=\arabic*, % 如果需要引用列表项,这将决定引用格式(这里仍然使用数字)
        itemsep=0pt, parsep=0pt, topsep=0pt, partopsep=0pt, leftmargin=3.5em
    }  
    

% 其它设置
    % 脚注设置
    \renewcommand\thefootnote{\ding{\numexpr171+\value{footnote}}}
    % 参考文献引用设置
        \bibliographystyle{unsrt}   % 设置参考文献引用格式为unsrt
        \newcommand{\upcite}[1]{\textsuperscript{\cite{#1}}}     % 自定义上角标式引用
    % 文章序言设置
        \newcommand{\cnabstractname}{序言}
        \newenvironment{cnabstract}{%
            \par\Large
            \noindent\mbox{}\hfill{\bfseries \cnabstractname}\hfill\mbox{}\par
            \vskip 2.5ex
            }{\par\vskip 2.5ex}

% 文章默认字体设置
    \usepackage{fontspec}   % 宏包:字体设置
        \setmainfont{SimSun}    % 设置中文字体为宋体字体
        \setCJKmainfont[AutoFakeBold=3]{SimSun} % 设置加粗字体为 SimSun 族,AutoFakeBold 可以调整字体粗细
        \setmainfont{Times New Roman} % 设置英文字体为Times New Roman

% 各级标题自定义设置
    \usepackage{titlesec}   
    % chapter
        \titleformat{\chapter}[hang]{\normalfont\Large\bfseries\centering\boldmath}{Homework \thechapter :}{10pt}{}
        \titlespacing*{\chapter}{0pt}{-30pt}{10pt} % 控制上方空白的大小
    % section
        \titleformat{\section}[hang]{\normalfont\large\bfseries\boldmath}{\thesection}{8pt}{}
    % subsection
        % 设置 subsection 样式为 (1) (2) (3)
        \renewcommand{\thesubsection}{(\arabic{subsection})}    
        \titleformat{\subsection}[hang]{\normalfont\bfseries\boldmath}{\thesubsection}{8pt}{}    

% --------------------- 文章宏包及相关设置 --------------------- %
% >> ------------------ 文章宏包及相关设置 ------------------ << %



% ------------------------ 文章信息区 ------------------------ %
% >> --------------------- 文章信息区 --------------------- << %
% 页眉页脚设置

\usepackage{fancyhdr}   %宏包:页眉页脚设置
    \pagestyle{fancy}
    \fancyhf{}
    \cfoot{\thepage}
    \renewcommand\headrulewidth{1pt}
    \renewcommand\footrulewidth{0pt}
    \chead{光学课程作业,\ 丁毅,\ 2023K8009908031}
    \lhead{Homework \thechapter}
    \rhead{\small dingyi233@mails.ucas.ac.cn}

%文档信息设置
\title{光学课程作业\\ Homework of Optics}
\author{丁毅\\ \footnotesize 中国科学院大学,北京 100049\\ Yi Ding \\ \footnotesize University of Chinese Academy of Sciences, Beijing 100049, China}
\date{\footnotesize 2024.9 -- 2025.1}
% >> --------------------- 文章信息区 --------------------- << %
% ------------------------ 文章信息区 ------------------------ %

% 开始编辑文章

\begin{document}
\zihao{5}           % 设置全文字号大小

% ------------------------ 封面序言与目录 ------------------------ %
% >> --------------------- 封面序言与目录 --------------------- << %
% 封面
    \maketitle\newpage  
    \pagenumbering{Roman} % 页码为大写罗马数字
    \thispagestyle{fancy}   % 显示页码、页眉等

% 序言
    \begin{cnabstract}\normalsize 
        本文为笔者本科时的“光学”课程作业(Homework of Optics, 2024.9-2025.1)。本门课程笔记和其他科目的笔记、作业,例如热学、电磁学、电路原理和数学物理方法等,也可以在我的 \href{https://github.com/YiDingg/LatexNotes}{GitHub > LatexNotes} 仓库上找到。

        每次提交作业,老师给予批阅反馈之后,会对原作业内容进行修改、订正,以期达到满分作业的参考标准。但是,由于个人学识浅陋,认识有限,文中难免有不妥甚至错误之处,望读者不吝指正。读者可以将错误发送到我的邮箱 {\color{blue}\ dingyi233@mails.ucas.ac.cn},也可以到笔者的 \href{https://github.com/YiDingg/LatexNotes}{GitHub (https://github.com/YiDingg/LatexNotes)} 上提 \href{https://github.com/YiDingg/LatexNotes/issues}{issue},衷心感谢。
    \end{cnabstract}
    \addcontentsline{toc}{chapter}{序言} % 手动添加为目录

% 不换页目录
    \setcounter{tocdepth}{0}
    \noindent\rule{\textwidth}{0.1em}   % 分割线
    \noindent\begin{minipage}{\textwidth}\centering 
        \vspace{1cm}
        \tableofcontents\thispagestyle{fancy}   % 显示页码、页眉等   
    \end{minipage}  
    \addcontentsline{toc}{chapter}{目录} % 手动添加为目录

% 收尾工作
    \newpage    
    \pagenumbering{arabic} 

% >> --------------------- 封面序言与目录 --------------------- << %
% ------------------------ 封面序言与目录 ------------------------ %

\chapter{第一章作业}\thispagestyle{fancy}

\section{求入射到光纤的角度满足的条件}
\begin{equation}
    n_0 \sin i = n_g \sin i', \quad n_g \sin (\frac{\pi}{2} - i') = n_c \sin \frac{\pi}{2} 
     \Longrightarrow i \leqslant \arcsin \left( \frac{n_g}{n_0} \sqrt{1 - \frac{n_c^2}{n_g^2}}\,  \right)
\end{equation}
    

\begin{figure}[H]\centering
\includegraphics[width=0.55\textwidth]{assets/1/c6c9e6ef2f6755ead534508010ad0452 (2).jpg}
\caption{\textbf{求入射到光纤的角度满足的条件}}\label{求入射到光纤的角度满足的条件}
\end{figure}

\section{推导光线轨迹方程}

在 $x$-$y$ 平面中,设 $y = y(x)$ 表示光线的轨迹方程,$n = n(y)$ 表示介质的折射率。由几何关系,我们有:
\begin{equation}
\frac{\mathrm{d} y }{\mathrm{d} x } = \tan \theta = \frac{1}{\tan i} = \frac{\sqrt{1-\sin^2 i}}{\sin i} 
\end{equation}

由折射定律,记 $[n(y)\sin i(y)]_{y=0} = C$ ,则我们有:
\begin{equation}
n(y)\sin i(y) = C  \Longrightarrow \frac{\mathrm{d} y }{\mathrm{d} x } = \frac{\sqrt{n^2 - C^2}}{C^2}, \quad \left(\frac{\mathrm{d} y }{\mathrm{d} x }\right)^{2} = \frac{n^2}{C^2} - 1
\end{equation}
两边同时对 $x$ 求导,得到:
\begin{equation}
2 \left(\frac{\mathrm{d} y }{\mathrm{d} x }\right) \left(\frac{\mathrm{d}^2 y }{\mathrm{d} x^2 }\right) = \frac{1}{C^2} \left(\frac{\mathrm{d} n^2 }{\mathrm{d} y }\right) \left(\frac{\mathrm{d} y }{\mathrm{d} x }\right) \Longrightarrow \frac{\mathrm{d}^2 y }{\mathrm{d} x^2 } = \frac{1}{2C^2}\cdot \frac{\mathrm{d} n^2 }{\mathrm{d} y } 
\end{equation}
也即
\begin{equation}
    \frac{\mathrm{d}^2 y }{\mathrm{d} x^2 } = \frac{1}{2n_0^2\sin^2 i_0}\cdot \frac{\mathrm{d} n^2 }{\mathrm{d} y } = \frac{1}{2n_0^2\cos^2 \theta_0}\cdot \frac{\mathrm{d} n^2 }{\mathrm{d} y } \quad \square
\end{equation}

\begin{figure}[H]\centering
\includegraphics[width=0.8\textwidth]{assets/1/88b3f41951f9d3be9e2964935ebaf0f7.jpg}
\caption{\textbf{推导光线轨迹方程}}\label{推导光线轨迹方程}
\end{figure}

事实上,在三维坐标系中考虑上述过程,或者利用费马原理和变分法,又或考虑哈密顿光学,可以得到更一般的形式,称为光路方程,如下:
\begin{equation}
    \nabla n=\frac{\mathrm{d}}{\mathrm{d}s}\left(n\frac{\mathrm{d}\vec{r}}{\mathrm{d}s}\,\right)
\end{equation}

\section{(已被删去)}
\section{利用费马原理给出物像关系}

折射球面如图,由余弦定理可知:
\begin{equation}
\text{OPL} = np + n'p' 
= n \sqrt{r^2 + (s+r)^2 \,{\color{red} -}\, 2r(s+r)\cos \phi } + n' \sqrt{r^2 + (s'-r)^2 \,{\color{red} +}\, 2r(s'-r)\cos \phi } 
\end{equation}

由费马原理,$\frac{\mathrm{d} \text{OPL}  }{\mathrm{d} \phi } = 0$,于是:
\begin{equation}
\frac{-nr(s+r)\sin \phi }{p} + \frac{n'r(s'-r)\sin \phi }{p'} = 0 \Longrightarrow  \frac{n}{p} + \frac{n'}{p'} = \frac{1}{R\,}\left( \frac{n's'}{p'} - \frac{ns}{p} \right)
\end{equation}

在傍轴条件下,有 $s \approx p$,$s' \approx p'$,于是:
\begin{equation}
\frac{n}{s} + \frac{n'}{s'} = \frac{n' - n}{R}   \quad\square
\end{equation}
证毕。
\begin{figure}[H]\centering
\includegraphics[width=0.45\textwidth]{assets/1/7dd67844c3c8000268c32546583de193.png}
\caption{\textbf{折射球面物像关系}}\label{折射球面物像关系}
\end{figure}

\section{推导反射球面的物像公式}

这里要注意,由于像是虚像,$l_2$ 贡献虚光程(为负),且 $s_2<0$,因此圆心到像点的距离为 $r+s_2$ 而非 $r-s_2$。同由余弦定理,写出光程 \text{OPL},有:
\begin{equation}
\text{OPL} = n_1l_1 - n_2l_2 
=
n_1\sqrt{r^2 + (r+s_1)^2 - 2r(r+s_1)\cos \phi } \,{\color{red} -}\, n_2 \sqrt{r^2 + (r+s_2)^2 - 2r(r+s_2)\cos \phi } 
\end{equation}

由费马原理,$\frac{\mathrm{d} \text{OPL}  }{\mathrm{d} \phi } = 0$,于是有:
\begin{equation}
\frac{-n_1r(r+s_1)\sin \phi }{l_1} + \frac{n_2r(r+s_2)\sin \phi }{l_2} = 0 
\Longrightarrow 
\frac{n_2}{l_2} - \frac{n_1}{l_1} = \frac{1}{r}\left( \frac{n_1s_1}{l_1} - \frac{n_2s_2}{l_2} \right)
\end{equation}
傍轴时,有 $s_1 \approx l_1$,$s_2 \approx -l_2$,于是:
\begin{equation}
-\frac{n_2}{l_2} - \frac{n_1}{l_1} = \frac{1}{r}(n_1 + n_2)
\end{equation}
当反射球面两侧为相同介质时,$n_1 = n_2$,则:
\begin{equation}
\frac{1}{s_1} + \frac{1}{s_2} = -\frac{2}{r}  \quad\square
\end{equation}
证毕。

\section{画出图中的像点}
如下图所示,左侧为手绘图,右侧为光路仿真软件 \href{https://www.optico.app/en/start-en/}{Optico} 效果图。

\begin{figure}[H]\centering
\begin{subfigure}[t]{0.47\textwidth}\centering
    \includegraphics[height=130pt]{assets/1/图1.png}
    \caption{ 手绘图 }
\end{subfigure}\begin{subfigure}[t]{0.52\textwidth}\centering
    \includegraphics[height=130pt]{assets/1/1.png}
    \caption{ 光路仿真 }
\end{subfigure}
\caption{ 画出虚物 $Q$ 的像点 $Q'$}
\end{figure}

\begin{figure}[H]\centering
    \begin{subfigure}[t]{0.47\textwidth}\centering
        \includegraphics[height=130pt]{assets/1/图2.png}
        \caption{ 手绘图 }
    \end{subfigure}\begin{subfigure}[t]{0.52\textwidth}\centering
        \includegraphics[height=130pt]{assets/1/2.png}
        \caption{ 光路仿真 }
    \end{subfigure}
    \caption{ 画出实物 $Q$ 经凹透镜的像点 $Q'$}
\end{figure}

    \begin{figure}[H]\centering
\begin{subfigure}[t]{0.47\textwidth}\centering
    \includegraphics[height=130pt]{assets/1/图3.png}
    \caption{ 手绘图 }
\end{subfigure}\begin{subfigure}[t]{0.52\textwidth}\centering
    \includegraphics[height=130pt]{assets/1/3.png}
    \caption{ 光路仿真 }
\end{subfigure}
\caption{ 画出虚物 $Q$ 经凹透镜的像点 $Q'$}
\end{figure}



\chapter{第二章作业}\thispagestyle{fancy}
\vspace{-2mm}
\section{对于正入射的情况,写出菲涅尔公式}

菲涅尔公式如下:

\begin{table}[H]
    \centering
    \renewcommand{\arraystretch}{1.5} % 调整行间距为默认值的1.5倍 
    \begin{tabular}{|c|c|c|c|c|} 
    \hline
    类型 & \multicolumn{2}{c|}{振幅反射系数 $r$} & \multicolumn{2}{c|}{振幅透射系数 $t$ }  \\ 
    \hline
    s 波 & $\displaystyle r_s = \frac{n_i\cos \theta_i - n_t \cos \theta_t}{n_i\cos \theta_i + n_t \cos \theta_t} $ & $\displaystyle  - \frac{\sin (\theta_i - \theta_t) }{\sin (\theta_i + \theta_t)}$ & $\displaystyle t_s  = \frac{2n_i \cos \theta_i}{n_i\cos \theta_i + n_t \cos \theta_t} $ &   $\displaystyle  + \frac{2 \sin \theta_t \cos \theta_i}{\sin (\theta_i + \theta_t)}$   \\ 
    \hline
    p 波 & $\displaystyle r_p = \frac{n_t\cos \theta_i - n_i \cos \theta_t}{n_t\cos \theta_i + n_i \cos \theta_t} $ &     $ \displaystyle  + \frac{\tan (\theta_i - \theta_t)}{\tan (\theta_i + \theta_t)} $  &  $\displaystyle t_p  = \frac{2n_i \cos \theta_i}{n_i\cos \theta_t + n_t \cos \theta_i} $ &   $\displaystyle + \frac{2 \sin \theta_t \cos \theta_i}{\sin (\theta_i + \theta_t) \cos (\theta_i - \theta_t)}$                  \\
    \hline
    \end{tabular}
\end{table}

正入射时,$\theta_i = \theta_t = 0$,于是有:
\begin{gather}
    r_p = (-r_s)  = \frac{n_t - n_i}{n_t + n_i},\quad t_p = t_s = \frac{2n_i}{n_i + n_t} \\ 
    F = R_s = R_p = \left( \frac{n_t - n_i}{n_t + n_i} \right)^2
\end{gather}


不妨作出相关的图像,图 \ref{振幅系数随入射角的变化} 是 s 波、p 波振幅系数关于入射角 $\theta_i$ 的变化情况\footnote{源码见附录 \ref{图振幅系数随入射角的变化源码}}。

\begin{figure}[H]\centering
\begin{subfigure}[t]{0.49\textwidth}\centering
    \includegraphics[height=180pt]{assets/2/2024-09-15_10-53-31.pdf}
    \caption{ 由空气入射玻璃($n_i = 1,\ n_t = 1.5$) }
\end{subfigure}
\begin{subfigure}[t]{0.49\textwidth}\centering
    \includegraphics[height=180pt]{assets/2/2024-09-15_10-53-27.pdf}
    \caption{ 由玻璃入射空气($n_i = 1.5,\ n_t = 1$) }
\end{subfigure}
\caption{ 振幅系数 $r$ 随入射角 $\theta_i$ 的变化 }\label{振幅系数随入射角的变化}
\end{figure}

\vspace{-7mm}
\section{一自然光以 Brewster Angle 入射到空气中的一块玻璃,已知功率透射率为 0.86。}

%{\color{red} $\star $ 事实上,本题题设并不合理,是不符合实际的。我们先给出解题过程,再说明为何玻璃折射率。}

\textbf{(1)\ \ 求功率的反射率}

$T = 0.86$,由能量守恒,功率反射率 $R = 0.14$。

\textbf{(2)\ \ 若输入为 1000 W,求透射光 s 分量上的功率}

光束为自然光,因此 s 分量和 p 分量的功率相同,都为 500 W,也即 $\Phi_{e,i,s} = \Phi_{e,i,p} = 500 \ \mathrm{W}$。又由 Brewster Angle 入射,因此反射光的 p 分量为 0,也即 $R_p = 0$,于是:
\begin{gather}
T_p = 1 - R_p = 1,\quad T_s = 2T - T_p = 0.72 
\end{gather}
由此可求得透射光 s 分量上的辐射通量(即辐射功率):
\begin{equation}
\Phi_{e,t,s} = T_s \Phi_{e,i,s} = 0.72 \times 500 \ \mathrm{W} =  360 \ \mathrm{W}
\end{equation}

{\color{red} \textbf{(3)\ \ 求玻璃的折射率}}

虽然题目并未要求\footnote{查阅资料发现,此题来自于\textit{光学(尤金,第五版)Optics (Eugene Hecht)} 的 Page 152},但我们不妨求解一下玻璃的折射率 $n_t$。在题设条件下,$R = 0.14$,默认空气折射率为 1,则唯一的未知量是玻璃折射率 $n_t$,这是可以求解的,方程如下:
\begin{gather}\label{玻璃折射率}
    R = \frac{1}{2}(R_s + R_p) = 0.14,\quad \theta_i = \theta_B = \arctan\left(\frac{n_t}{n_i}\right) ,\quad n_i = 1
    \Longrightarrow \\ 
    \left[ \frac{ \cos (\arctan n_t) - \sqrt{n_{t}^2 - \sin^2 (\arctan n_t)} }{\cos (\arctan n_t) + \sqrt{n_{t}^2 - \sin^2 (\arctan n_t)}} \right]^2 + \left[ \frac{ n_{t}^2\cos (\arctan n_t) - \sqrt{n_{t}^2 - \sin^2 (\arctan n_t)} }{n_{t}^2\cos (\arctan n_t) + \sqrt{n_{t}^2 - \sin^2 (\arctan n_t)}} \right]^2  = 2\times 0.14
\end{gather}
此方程有唯一未知量 $n_t$,用 Matlab 解此非线性方程组\footnote{源码见附录 \ref{玻璃折射率源码}},得到玻璃折射率 $n_t$,以及其它参量\footnote{图 \ref{方程左边的变化情况} 源码见附录 \ref{方程左边的变化情况源码}}: 
\begin{gather}
\begin{cases}
    n_t = 0.554902 
    ,\quad 
    \theta_i = \theta_B  = 29.025970^\circ
    \\
    \theta_t = 60.974030^\circ
    ,\quad 
    \theta_C = 33.703947^\circ
    \\
    R = 0.1400,\quad   R_s = 0.280000,\    R_p = 0.000000 \\ 
    T = 0.8600,\quad   T_s = 0.720000,\    T_p = 1.000000 
\end{cases}
\begin{cases}
    n_t = 1.802121
    ,\quad 
    \theta_i = \theta_B  = 60.974030^\circ 
    \\
    \theta_t = 29.025970^\circ
    ,\quad 
    \theta_C = 90.000000^\circ
    \\
    R = 0.1400,\quad   R_s = 0.280000,\    R_p = 0.000000 \\ 
    T = 0.8600,\quad   T_s = 0.720000,\    T_p = 1.000000 
\end{cases}
\end{gather}



\begin{center}\noindent\begin{minipage}{0.50\textwidth}
\hspace*{2em} 也即上述方程有两解,考虑 $n_{ti} \in [0,\ 2]$,令方程左边为 $f(n_{ti})$,作出图像如右。图 \ref{方程左边的变化情况} 说明了我们并没有漏掉其它解。

\hspace*{2em} 一般玻璃的折射率在 1.5 左右,即使是特殊玻璃(例如高折射率镜片),也基本在 1.3 至 1.9 之间,0.5 的玻璃折射率显然是不合理的,即使是考虑介质折射率关于波长的变化(如 X 射线或 Gamma 射线),也不会达到如此低的折射率。因此舍去 $n_t = 0.554902$,最终得 $n_t = 1.802121$。
\end{minipage}\hfill\begin{minipage}{0.43\textwidth}
    \begin{figure}[H]\centering
        \includegraphics[width=\textwidth]{assets/2/2024-09-18_01-08-40.pdf}
        \vspace*{-9mm}
        \caption{ 方程 \ref{玻璃折射率} 左边函数值随 $n_{ti}$ 的变化情况}\label{方程左边的变化情况}
        \end{figure}
\end{minipage}\end{center}



\section*{上题改编:一自然光由空气入射玻璃,玻璃折射率为 1.5,已知功率透射率为 0.86。}


\textbf{(1)\ \ 求功率的反射率:}

$T = 0.86$,由能量守恒,功率反射率 $R = 0.14$。

\textbf{(2)\ \ 若输入为 1000 W,求透射光 s 分量上的功率}

光束为自然光,因此 s 分量和 p 分量的功率相同,都为 500 W。先求解入射角 $\theta_i$,由菲涅尔定理和能量关系:
\begin{equation}
R =  \frac{1}{2}(R_s + R_p),\  R_s =  \left[ \frac{ \cos \theta_i - \sqrt{n_{ti}^2 - \sin^2 \theta_i} }{\cos \theta_i + \sqrt{n_{ti}^2 - \sin^2 \theta_i}} \right]^2,\ R_p = \left[ \frac{ n_{ti}^2\cos \theta_i - \sqrt{n_{ti}^2 - \sin^2 \theta_i} }{n_{ti}^2\cos \theta_i + \sqrt{n_{ti}^2 - \sin^2 \theta_i}} \right]^2
\end{equation}
其中 $n_i = 1$,$n_t = 1.5$,因此 $n_{ti} = 1.5$,代入即得:
\begin{equation}
    \left[ \frac{ \cos \theta_i - \sqrt{1.5^2 - \sin^2 \theta_i} }{\cos \theta_i + \sqrt{1.5^2 - \sin^2 \theta_i}} \right]^2 + \left[ \frac{ 1.5^2\cos \theta_i - \sqrt{1.5^2 - \sin^2 \theta_i} }{1.5^2\cos \theta_i + \sqrt{1.5^2 - \sin^2 \theta_i}} \right]^2 = 2\times0.14
\end{equation}
用 Matlab 解此非线性方程组\footnote{源码见附录 \ref{公式解入射角源码}},得到入射角 $\theta_i$ 和其它参量:
\begin{gather}\label{解入射角}
\begin{matrix}
    \theta_i =  1.173220\ \ \mathrm{rad}  = 67.220559^\circ \\
    R = 0.140000,\quad   R_s = 0.256933,\    R_p = 0.023067 \\ 
    T = 0.860000,\quad   T_s = 0.743067,\    T_p = 0.976933
\end{matrix}
\end{gather}

于是透射光 s 分量上的辐射通量为:
\begin{equation}
\Phi_{e,t,s} = T_s \Phi_{e,i,s} = 0.743067 \times 500 \ \mathrm{W} =  371.5335 \ \mathrm{W}
\end{equation}


\section{光束垂直入射到玻璃-空气界面,玻璃折射率 1.5,求出能量反射率和透射率}

$\theta_i = 0$ 时,由菲涅尔定律和能量关系,有:
\begin{gather}
    R =  \frac{1}{2}(R_s + R_p),\quad  T = 1 - R\\ 
    R_s =  \left[ \frac{ \cos \theta_i - \sqrt{n_{ti}^2 - \sin^2 \theta_i} }{\cos \theta_i + \sqrt{n_{ti}^2 - \sin^2 \theta_i}} \right]^2 = \left[ \frac{1 - n_{ti}}{1 + n_{ti}} \right]^2,\ R_p = \left[ \frac{ n_{ti}^2\cos \theta_i - \sqrt{n_{ti}^2 - \sin^2 \theta_i} }{n_{ti}^2\cos \theta_i + \sqrt{n_{ti}^2 - \sin^2 \theta_i}} \right]^2 =  \left[ \frac{n_{ti}^2 - n_{ti}}{n_{ti}^2 + n_{ti}} \right]^2
\end{gather}
由空气入射玻璃时,$n_{ti} = 1.5$,由玻璃入射空气时,$n_{ti} = \frac{2}{3}$,代入得到:
\begin{gather*}
\text{空气入射玻璃: }\ R = 0.04,\quad  T = 0.96 \\ 
\text{玻璃入射空气: }\ R = 0.04,\quad  T = 0.96 
\end{gather*}
也即无论从哪边入射,能量反射率和透射率分别为 0.04 和 0.96。


\chapter{第三章作业}\thispagestyle{fancy}

\section{在杨氏双缝实验中,设两缝之间的距离为 0.2 mm,在距双缝 1 m 远的屏上观察干涉
条纹,若入射光是波长为 400 nm 至 760 nm 的白光,问屏上距零级明纹 20 mm 处,哪些波
长的光最大限度地加强?}

也即求哪些波长的光在 20 mm 处是亮条纹。杨氏干涉中,两相邻亮(暗)条纹的间距 $\Delta x = \frac{D \lambda}{d}$,其中 $D$ 是双缝屏与屏幕的距离,$d$ 是双缝间距,$\lambda$ 是波长。因此有:
\begin{gather}
k \Delta x = 20 \ \mathrm{mm} \Longrightarrow \lambda = \frac{d\cdot 20 \ \mathrm{mm}}{D} \cdot \frac{1 }{k} = \frac{4}{k}\times 10^{-6},\quad k = 1,2,3,\cdots
\end{gather}
而波长范围 $\lambda \in [400 \ \mathrm{nm} ,\ 760 \ \mathrm{nm}]$,于是:
\begin{equation}
k \in [5.2632,\ 10] \Longrightarrow k = 6,7,8,9,10 ,\quad \lambda = 400.0 \ \mathrm{nm},\ 444.4 \ \mathrm{nm},\ 500.0 \ \mathrm{nm},\ 571.4 \ \mathrm{nm},\ 666.7 \ \mathrm{nm}
\end{equation}

\section{在空气中用某单色光进行双缝干涉实验时,观察到干涉条纹相邻明条纹的间距为
1.33 mm,当把实验装置放在水中时(水的折射率为 1.33),则相邻明条纹的间距变为多
少?}

空气折射率近似为 1,设光在空气中的波长为 $\lambda$,则在水中的波长为 $\frac{\lambda}{n}$,其中 $n$ 为水的折射率。而双缝干涉中相邻亮条纹间距为:
\begin{equation}
\Delta x = \frac{D \lambda}{d} \Longrightarrow \Delta x' = \frac{\Delta x}{n} = 1\ \mathrm{mm}
\end{equation}

\section{用波长为 589.3 nm 的钠黄光垂直照射长 $\boldsymbol{L = 20 \ \mathrm{mm}}$  的空气尖劈,测得条纹间距为 $\boldsymbol{1.18}$ $\boldsymbol{ \times 10^{-4}\ \mathrm{m}}$,求钢球直径 $\boldsymbol{d}$。}

构成劈尖干涉,相邻亮条纹间距为 $\Delta x = \frac{\lambda}{2 \tan \theta} \approx  \frac{\lambda}{2 \theta} $,设劈尖长为 $L$,倾角为 $\theta$,钢球的直径为 $D$,则有:
\begin{equation}
\tan \theta = \frac{D}{L} \Longrightarrow D =  L \tan \theta \approx \theta L  = \frac{\lambda L}{2 \Delta x} = 4.9941 \times 10^{-5}\ \mathrm{m}
\end{equation}
即为所求直径。

\section{厚度为 0.050 mm 的玻璃片,其折射率为 1.520,插入迈克尔孙干涉仪的一条光路中,照明光为波长 587.56 nm 的氦黄线。求插入这片玻璃片移动了多少干涉条纹?}

改变两干涉光束的光程差,会使原干涉条纹发生移动。设 $n_f$ 为玻璃片折射率,$d$ 为玻璃片厚度,$\lambda_0$ 为氦黄线在空气中的波长,则有:
\begin{equation}
2 n_f d = N\lambda_0 \Longrightarrow N = \frac{2 n_f d}{\lambda_0} = 258.6970
\end{equation}

\section{迈克耳逊干涉仪两臂中分别加入 20 cm 长的玻璃管,一个抽成真空,一个充以一个大气压的氩气,今以汞光线(波长为 546.0 nm)入射干涉仪,如将氩气抽出,发现干涉仪
中条纹移动了 205 条,求氩气的折射率。}

抽成真空的玻璃管补偿了穿过玻璃管带来的光程,因此没有引入附加光程差。与上题同理,设 $n_f$ 为氩气折射率,$d$ 为玻璃管长度,$\lambda_0$ 为汞光线在空气中的波长,并近似空气折射率为 1,则有: 
\begin{equation}
2 (n_f-1) d = N\lambda_0 \Longrightarrow n_f = \frac{N \lambda_0}{2d} + 1 = 1.0002798
\end{equation}

\section{有一谱线结构,谱线范围是 500 nm 至 501 nm,若 F-P 标准具 $d = 0.5 \ \mathrm{mm}$,可否用它来分析这一谱线结构?}

波长的自由光谱宽度 $\left(\Delta\lambda\right)_{\text{fsr}}$、最小分辨率 $\left(\Delta\lambda\right)_{\min}$ 和极限分辨率 $\left(\Delta\lambda\right)_{\lim}$ 分别为:
\begin{equation}
    \left(\Delta\lambda\right)_{\text{fsr}} = \frac{\lambda_0^2}{2 n d}
    ,\quad 
    \left(\Delta\lambda\right)_{\min} = \frac{2\lambda_0}{\pi \sqrt{F}}
    ,\quad 
    \left(\Delta\lambda\right)_{\lim} = \frac{\lambda_0^2}{\pi n d \sqrt{F}}
\end{equation}
代入数据 $d = 0.5 \ \mathrm{mm}$,空气折射率近似 $n = 1$,并取能量反射率为典型值 $R = 0.90$,可以得到:
\begin{equation}
F = 80.0000 ,\quad 
\left(\Delta\lambda\right)_{\text{fsr}} = 0.2505 \ \mathrm{nm}
,\quad
\left(\Delta\lambda\right)_{\min} = 0.0011 \ \mathrm{nm}
,\quad 
\left(\Delta\lambda\right)_{\lim} = 5.6382 \times 10^{-7} \ \mathrm{nm}
\end{equation}
而谱线宽度 $\Delta \lambda = 1 \ \mathrm{nm} > \left(\Delta\lambda\right)_{\text{fsr}}$,因此,无论光谱是连续谱还是分立谱,虽然可以观察到明显的干涉条纹(对分立谱),或者在频谱分析仪中看到明显的频率纵模(对连续谱),但是都会出现严重的条纹越级,因此不能用它来分析这一谱线结构。

\chapter{第四章作业}\thispagestyle{fancy}

\section{对圆盘衍射,当圆盘恰好包含 $n$ 个半波带时(即 $n$ 个半波带被遮挡),为何中心为亮斑?}
\begin{graybox}
\textbf{我们先给出结论,再作具体的讨论:}\\
除了紧挨着圆盘后的一小段区域,{\bfseries 整个中轴线上处处呈亮态}(辐照度不为 0),{\bfseries 这与圆盘是否包含整数个半波带无关},但可能是(与圆盘平行的)平面上的极小值。如果是圆孔衍射,则当圆盘恰好包含偶数个半波带时,中心为暗斑,恰好包含奇数个半波带时,中心为亮斑(且是极大值)。
\end{graybox}

为了讨论圆盘衍射,我们需要先给出菲涅尔衍射的基本原理(菲涅尔波带法)。

\subsection{球面波的传播(菲涅尔波带法)}

在菲涅尔衍射中,之前的许多近似都不再成立,需要建立另外一套理论基础。

由菲涅尔原理,如果每个子波向一切方向都均匀地辐射,那么除了产生一个向前进的波以外,还会出现一个向波源后退的反向波。实验上并没有发现这样的波,因此我们必须对次级发射体的辐射图样作某些修改。更详细的理论\footnote{基尔霍夫理论,详见参考文献 \cite{Optics} 的 10.4 节}表明,次波源发射的光具有方向性,由倾斜因子 $K = K(\theta)$ 来描述,它是次波源在不同方向光场的振幅系数:
\begin{equation}
    K = K(\theta) = \frac{1}{2}\left(1 + \cos \theta\right),\quad E = K\frac{\varepsilon_A}{R} \,e^{i(kr - \omega t)}
\end{equation}

如图 \ref{球形波阵面的传播},由波带理论,第 $m$ 级半波带(后文简称“波带”)在点 $P$ 的电场为:
\begin{equation}
E_m = (-1)^{m+1} \frac{2 K_m \varepsilon_0}{\rho_0 + r_0} \,e^{i\left(k(\rho_0 + r_0) - \omega t\right)},\quad \varepsilon_0 = \varepsilon_A \rho_0 \lambda
\end{equation}
式中 $\varepsilon_0 = \varepsilon_A \rho_0 \lambda$ 是波源强度,即球面波表达式 $E = \frac{\varepsilon_0}{r} \,e^{i(kr - \omega t)}$ 中的 $\varepsilon_0$。

\begin{figure}[H]\centering
    \includegraphics[width=0.6\columnwidth]{assets/4/4.5 球形波阵面的传播.png}
    \caption{球形波阵面的传播}\label{球形波阵面的传播}
\end{figure}

\subsection{圆孔近场衍射}

我们指出,虽然在数学上将半波带分为了无限多个,但由于倾斜因子 $K(\theta)$ 的存在,认为小孔中只能“看到”有限个半波带是合适的。通过计算给定小孔上的半波带数目 $N_F$,可以得到中轴线上辐照度的一个良好近似。每个半波带的面积 $A$ 由下式给出:
\begin{equation}
A = \pi \frac{r_0 \rho}{r_0 + \rho} \lambda \approx \pi r_0 \lambda
\end{equation}
对圆形小孔,半波带数目为:
\begin{equation}
N_F = \frac{\pi a^2}{A} = \frac{(\rho_0 + r_0)a^2 }{r_0 \rho_0 \lambda} \approx \frac{a^2}{ r_0 \lambda}
\end{equation}
上式中 $\rho$ 和 $r_0$ 分别是小孔到光源和观察点的距离,$a$ 是小孔的半径。$N_F$ 常称为菲涅耳数。保持小孔半径不变,当点 $P$ 从无穷远处向小孔靠近时,$r_0$ 由无穷到 0,$N_F$ 会由 $0$ 逐渐增大为 $\infty$。

由图 \ref{振动曲线} (c) 可以看出在不同半波带数目下,中轴线上的振幅情况。角度还可看出,实际相位角与惠更斯-菲涅尔原理所预测的相位角的不同,$O_s$ 点的切线(向右)是惠更斯-菲涅尔原理的相位角,而相矢量 $\overrightarrow{O_sA_s}$ 的切线对应实际相位角。

\begin{figure}[H]\centering
\begin{subfigure}[b]{0.33\columnwidth}\centering
    \includegraphics[height=130pt]{assets/4/4.5 相矢量叠加.png}
    \caption{相矢量叠加}
\end{subfigure}
\begin{subfigure}[b]{0.33\columnwidth}\centering
    \includegraphics[height=130pt]{assets/4/4.5 总振幅随波带数目的变化.png}
    \caption{总振幅随波带数目的变化}
\end{subfigure}
\begin{subfigure}[b]{0.33\columnwidth}\centering
    \includegraphics[height=130pt]{assets/4/4.5 振动曲线.png}
    \caption{振动曲线}
\end{subfigure}
\caption{利用波带和振动曲线来判断中轴线上的振幅情况}
\label{振动曲线}
\end{figure}

在固定直径的孔内,由于 $A = \frac{\pi a^2}{N_F}$,随着 $N_F$ 的增大,每个波带的面积 $A$ 会减小,使得轴上辐照度的极大值将依 $\frac{1}{N_F^2}$ 减小(包络线)。一个定性的近似是 $I = I(0) \sinc^2 \left(\frac{\pi}{2}N_F\right)$,其中 $I(0)$ 是 $N_F = 0$ ($P$ 点离小孔无穷远),作出 $I$ 关于 $N_F$ 的变化情况,如图 \ref{中心振幅随波带数的变化} 所示:

\begin{figure}[H]\centering
\begin{subfigure}[b]{0.5\columnwidth}\centering
    \includegraphics[height=185pt]{assets/4/4.5 中心振幅随波带数的变化 2.pdf}
    \caption{$N_F \in [0, 10]$}
\end{subfigure}\hfill
\begin{subfigure}[b]{0.5\columnwidth}\centering
    \includegraphics[height=185pt]{assets/4/4.5 中心振幅随波带数的变化.pdf}
    \caption{$N_F \in [0, 4]$}
\end{subfigure}
\caption{中心振幅随波带数的变化}
\label{中心振幅随波带数的变化}
\end{figure}


另外,当观察点不在中轴线上时,随着点 $P$ 向外移动,“观察”到的波带也会发生变化,如图 \ref{圆孔向外移动},此时辐照度会有一系列极大与极小值,变化比较复杂。对整个观察平面而言,所得衍射图样随着 $N_F$ 的变化而变化,如图 \ref{不同菲涅尔数时的圆孔衍射图样} 所示。
\begin{figure}[H]\centering
    \includegraphics[width=0.75\columnwidth]{assets/4/4.5 圆孔向外移动.png}
    \caption{圆孔内“观察”到的波带}\label{圆孔向外移动}
\end{figure}
\begin{figure}[H]\centering
\begin{subfigure}[b]{0.59\columnwidth}\centering
    \includegraphics[height=185pt]{assets/4/4.5 菲涅尔衍射 不同波带 1.png}
\end{subfigure}
\begin{subfigure}[b]{0.39\columnwidth}\centering
    \includegraphics[height=185pt]{assets/4/4.5 菲涅尔衍射 不同波带.png}
\end{subfigure}
\caption{不同菲涅尔数 $N_F$ 时的圆孔衍射图样}
\label{不同菲涅尔数时的圆孔衍射图样}
\end{figure}



可以看到,当 $N_F \to 0 $ 时(即 $N_F \gg 1$),发生夫琅禾费衍射,这实质上是夫琅禾费衍射的另一种判别方法;当 $N_F \geqslant 1$ 时,发生菲涅尔衍射。特别地,对于环形孔,我们也可以借助振动曲线来分析,如下图:
\begin{figure}[H]\centering
    \includegraphics[width=0.65\columnwidth]{assets/4/4.5 菲涅尔圆环的轴上振幅.png}
    \caption{透光圆环中心轴上的菲涅尔衍射情况}\label{透光圆环中心轴上的菲涅尔衍射情况}
\end{figure}
图 \ref{透光圆环中心轴上的菲涅尔衍射情况} 是一个包含 $\frac{1}{3} + 3 + \frac{1}{3}$ 个波带的环形孔,中心波带(第一波带)被圆盘挡住大约 $\frac{2}{3}$(剩余 $\frac{1}{3}$),振动曲线的 $A_s$ 和 $B_s$ 分别对应图中 $A$ 点和 $B$ 点。由合成结果知道,相矢量 $\overrightarrow{A_sB_s}$ 给出了振幅的大小和相位。

\subsection{圆盘近场衍射}

我们知道,一个未受阻碍的波有无穷多个波带到达 $P$ 点(中轴线上一点),在此处产生一个大小约为第一波带一半的电场,即 $E \approx \frac{1}{2} E_1$。如果障碍物正好盖住第一波带,在振动曲线中减去第一波带的贡献,此时的电场 $E' = -\frac{1}{2}E_1$,这表明障碍物的加入不会改变点 $P$ 的亮暗状态(仍是亮斑)。

用类似的思想,如果障碍物从无开始,逐渐遮住 $1, 2, ..., n$ 个波带,这相当于对给定的圆屏,点 $P$ 由无穷远向圆盘靠近。由振动曲线可看出,除非 $n$ 非常大($P$ 离圆盘很近),相矢量 $\overrightarrow{A_sB_s}$ 的振幅始终不(近似)为 0。这表明 {\bfseries 除了紧挨着圆盘之后的一小段,中轴线上处处为亮点(辐照度始终不为 0),但可能是(与圆盘平行的)平面上的极小值}。遮住前 $n$ 个波带时的电场可写为:
\begin{equation}
E = \frac{1}{2} E_{n+1} = (-1)^{n} K_{n+1} \frac{\varepsilon_A \rho_0 \lambda}{\rho_0 + r_0} \,e^{i\left(k(\rho_0 + r_0) - \omega t\right)} = (-1)^{n} K_n E_0
\end{equation}
其中 $E_0$ 是无阻挡时的电场,$K_n$ 是第 $n$ 级波带法线与中轴线的夹角,随着 $n$ 的增大,夹角逐渐向 $\pi$ 靠近。放到图 \ref{圆盘障碍物衍射} (b) 中,便是点 $A_s$ 逆时针绕振动不断旋转,随着 $N_F$ 的增大,逐渐向中心点 $O_s'$ 靠近,直到 $A_s$ 和 $O_s'$ 重合,$| E | = 0 $。

\begin{figure}[H]\centering
\begin{subfigure}[b]{0.5\columnwidth}\centering
    \includegraphics[height=160pt]{assets/4/4.5 菲涅尔圆形障碍物.png}
    \caption{直径为 3 mm 滚珠的衍射图样}
\end{subfigure}\hfill
\begin{subfigure}[b]{0.5\columnwidth}\centering
    \includegraphics[height=160pt]{assets/4/4.5 圆盘衍射时的振动曲线.png}
    \caption{圆盘衍射时振动曲线上的相矢量}
\end{subfigure}
\caption{圆盘障碍物衍射}
\label{圆盘障碍物衍射}
\end{figure}



\section{讨论光栅的自由光谱范围。}
光栅方程为:
\begin{equation}
a \left(\sin \theta_m  - \sin \theta_i \right) = m \lambda,\quad m = 0, \pm1, \pm2, ...
\end{equation}
其中 $a$ 是光栅常数,表示两相邻狭缝的间距,即周期长度。以垂直于光栅平面的直线为法线,$\theta_i$ 是入射角而 $\theta_m$ 是第 $m$ 级(极大)衍射角。对于正入射的情况,$\theta_i = 0 \Longrightarrow m < \frac{a}{\lambda}$,对于自准直装置,$\theta_i = -\theta_m \Longleftarrow m < \frac{2a}{\lambda}$。

考虑光栅的自由光谱范围,由于波长小(频率高)的光谱线更密集,当 $\lambda_0 - \frac{\Delta \lambda}{2}$ 的第 $m+1$ 级与 $\lambda_0 + \frac{\Delta \lambda}{2}$ 的第 $m$ 级重合时,达到自由光谱范围 $\left(\Delta \lambda\right)_{\text{fsr}}$ :
\begin{equation}
\begin{cases}
    a(\sin \theta_m - \sin \theta_i ) = (m+1)\left(\lambda_0 - \frac{\Delta \lambda}{2}\right) \\ 
    a(\sin \theta_m - \sin \theta_i ) = m\left(\lambda_0 + \frac{\Delta \lambda}{2}\right)
\end{cases}
\Longrightarrow 
\left(\Delta \lambda\right)_{\text{fsr}} = \frac{\lambda_0}{m}
\end{equation}
上面所有公式中的级数 $m$ 在范围 $[0, \frac{2a}{\lambda})$ 内。由此可得到各参数在不同要求下的最值,这与当初 F-P 时的讨论类似,我们不多赘述。


\section{波长为 634.8 nm 的平行光射向直径为 2.76 mm 的圆孔,与孔相距 1 m 处放一屏。回答下面两个问题:(1) 屏上正对圆孔中心的 $P$ 点是亮点还是暗点?(2) 要使 $P$ 点变成与 (1) 相反的情况,至少要把屏幕分别向前或向后移动多少?}

\subsection{屏上正对圆孔中心的 $P$ 点是亮点还是暗点?}

计算菲涅尔数 $N_F$ : 
\begin{equation}
N_F = \frac{\pi a^2}{A} = \frac{(\rho_0 + r_0) a^2}{r_0 \rho_0 \lambda} \overset{\rho_0 \to \infty}{=} \frac{D^2}{4 r_0 \lambda} = \frac{(2.76 \ \mathrm{mm})^2}{4 \times 1 \ \mathrm{m} \times 634.8 \ \mathrm{nm}} = 3
\end{equation}
$N_F$ 为奇数,因此中心点是亮点。

\subsection{要使 $P$ 点变成与 (1) 相反的情况,至少要把屏幕分别向前或向后移动多少?}

$N_F$ 为偶数时中心成暗点,因此分别令 $N_F$ 为 2 和 4 ,得到:
\begin{equation}
r_0 = \frac{D^2}{4 N_F \lambda} = 1.5 \ \mathrm{m},\quad 0.75 \ \mathrm{m}
\end{equation}
因此至少要把屏幕向 $P$ 点移动 0.25 m,或者把屏幕远离 $P$ 点移动 0.5 m。

\section{一波带片由五个环带组成,第一环带是半径 $0 \sim r_1$ 的不透明圆盘,第二环带半径 $r_1\sim r_2$ 透明,第三环带半径 $r_2\sim r_3$ 不透明,第四环带半径 $r_3\sim r_4$ 透明,第五环带是 $r_4 \sim \infty$ 的不透明区域。用波长 500 nm 的平行单色光照明,最亮的像点在距波带片 1 m 的轴上,试求:(1) $r_1$;(2) 像点的光强;(3) 其它光强极大值出现在轴上的哪些位置?}

\subsection{求 $r_1$}

由题意知主焦距$f = 1 \ \mathrm{m}$ ,因此:
\begin{equation}
f = \frac{R_1^2}{\lambda} \Longrightarrow R_1 = \sqrt{f \lambda} = \sqrt{ 1 \ \mathrm{m} \times 500 \ \mathrm{nm} } = 0.7071 \ \mathrm{mm}
\end{equation}

\subsection{求像点的光强}

像点是第一焦点,波带片如图 \ref{题意中的波带片} 所示。
\begin{figure}[H]\centering
    \includegraphics[width=0.3\columnwidth]{assets/4/波带片.pdf}
    \caption{题意中的波带片}\label{题意中的波带片}
\end{figure}
设没有阻挡时的辐照度为 $I_0$,由于波带片透过第 2 和第 4 个(半)环带,因此振幅和辐照度为:
\begin{equation}
E = 2 E_1 = 4 E_0 \Longrightarrow I = 16 I_0
\end{equation}

\subsection{其它光强极大值出现在轴上的哪些位置?}

\begin{graybox}
\textbf{还是先给出结论,再做详细的讨论:}\\
记主焦点(第一焦点)位于 $r = f_1$ 处,辐照度为 $I_1$,则波带片的所有焦点及其辐照度大小是:
\begin{gather}
    f_1 = \frac{R_1^2}{\lambda},\quad   E_1 \approx n E_0 ,\quad   I_1 \approx n^2 I_0 \\
    f_k = \frac{f_1}{k},\quad I_k = \frac{I_1}{k^2},\quad k = 1, 3, 5, ...
\end{gather}\noindent
\end{graybox}


由图 \ref{波带片} (a),我们先来计算各波带的半径。
\begin{figure}[H]\centering
\begin{subfigure}[b]{0.53\columnwidth}\centering
    \includegraphics[height=160pt]{assets/4/4.5 波带半径计算.png}
    \caption{波带片各级圆环半径的计算}
\end{subfigure}\hfill
\begin{subfigure}[b]{0.46\columnwidth}\centering
    \includegraphics[height=160pt]{assets/4/4.5 波带片次焦点.png}
    \caption{波带片的各级焦点}
\end{subfigure}
\caption{波带片}
\label{波带片}
\end{figure}
将第 $m$ 个波带的外缘标以点 $A_m$,按定义,路程 $S$-$A_m$-$P$ 的光程应当比 $S$-$O$-$P$ 要大 $\frac{m\lambda}{2}$,也即:
\begin{equation}
(\rho_m + r_0) - (\rho_0 + r_0) = m \frac{\lambda}{2}
\end{equation}
作泰勒展开 $\rho_m = \rho_0 + \frac{R_m^2}{2\rho_0}$ 和 $r_m = r_0 + \frac{R_m^2}{2r_0}$,代入得到:
\begin{equation}
R_m = \sqrt{ \frac{m\lambda}{\frac{1}{\rho_0} + \frac{1}{r_0}} } \overset{\rho_0 \to \infty}{=}\sqrt{m r_0 \lambda}
\end{equation}
更精确的公式\footnote{由参考文献 \cite{波带片的设计及其衍射特性研究} Page 3 给出}是 $R_m = \sqrt{mr_0\lambda + \frac{m^2\lambda^2}{4}}$,式中 $\frac{m^2\lambda^2}{4}$ 代表球差。

是上面,我们依据“在点 $r_0$ 的各波带振幅相互加强”的原则,得到了波带片的各级半径,使得点 $P$ 是中轴线上光强最大的一点,此时点 $P$ 称为主焦点或一级焦点,距离 $r_0$ 也相应的记作 $f$ 或 $f_1$,有:
\begin{equation}
f_1 = \frac{R_1^2}{\lambda} \Longrightarrow  \frac{1}{\rho_0} + \frac{1}{r_0} = \frac{1}{f}
\end{equation}
与薄透镜公式有相同的形式,因此,用一光束准直入射给定的波带片(起到 $\rho_0 \to \infty$ 的作业),此时中轴线上最亮的点就是主焦距,它是辐照度分布中的一个极大值(也是最大值),因为在 $f$ 处波带片上的各圆环刚好和波阵面上的各波带重合。

需要注意,上面的“$n$ 级半径”并不是波带片的最大圆环半径,对有 $n_0$ 个圆环的挡光型波带片而言(中心圆算第一个圆环),无穷远处不透光,最外围的第 $n_0$ 级圆环 ($R_{n_0-1} \sim R_{n_0}$) 是透光的,则式中的 $n = n_0$;对透光型波带片而言(如图 \ref{菲涅尔波带片} 所示的正负菲涅尔波带片),无穷远处透光,最外围的第 $n$ 级圆环 ($R_{n_0-1} \sim R_{n_0}$) 是不透光的,因此需要再往外“扩张一个半径”,式中的 $n = n_0 + 1$。当然,实际中的 $n_0$ 一般都较大(100 以上),即使不考虑也几乎没有误差。


为什么我们要说是“一级”焦距?因为波带片本质上还是一个光栅,它(在中轴线上)的衍射图样是一系列的主焦点和次焦点交替出现(次焦点都比主焦点近),如图 \ref{波带片} (b) 所示。下面我们推导这些次焦点的位置和辐照度大小。

只需考虑 $r = \frac{f}{k}, k = 2, 3, 4, ...$ 时的情况,其余情况介于两者中间,可定性地判断辐照度的大小变化,且稍后能轻易知道它们都不是极值点。对于给定的 $k$,点 $P$ 与波带距离 $r = \frac{f}{k}$,我们保持波带片的直径 $D$ 不变,当距离变为原来的 $\frac{1}{k}$,由 $R_m = \sqrt{mr\lambda}$ 知道,$P$ 点“看到”的各波带半径变为原来的 $\frac{1}{\sqrt{k}}$。之前 $r = f$ 时波带片共有 $n$ 个半径,“孔”(波带片)的直径没变,而波带半径缩小为原来的 $\frac{1}{\sqrt{k}}$,因此在点 $P$ “看到” 的波带由 $n$ 个增长至 $kn$ 个。

记 $r = \frac{f}{k}$ 时,$P$ 点“看到”的一系列波带半径为 $R_j^{(k)}$,$k = 1, 2, 3, \ \ j = 1, 2, ..., kn$,将它们的相对大小($\frac{R_j^{(k)}}{R_1^(1)}$)如表 \ref{孔内不同波带的半径及位置关系} 一样列出,一切都会变得显然:
\begin{table}[H]\centering
    %\renewcommand{\arraystretch}{1.5} % 调整行间距为 1.5 倍
    %\setlength{\tabcolsep}{1.5mm} % 调整列间距
    \caption{$r = \frac{f}{k}, k = 1, 2, 3, ... $ 时孔内一系列波带的相对半径及位置关系}
    \label{孔内不同波带的半径及位置关系}
\begin{tabular}{cccccccccc}\toprule
    $k = 1$ & $1$ & $\sqrt{2}$ & $\sqrt{3}$  & ... & $\sqrt{n}$ \\
    \midrule
    $k = 2$ & $(\frac{\sqrt{1}}{\sqrt{2}}, 1)$ & $(\frac{\sqrt{3}}{\sqrt{2}}, 2)$ &  $(\frac{\sqrt{5}}{\sqrt{2}}, \sqrt{3})$  & ... & $(\frac{\sqrt{2n -1 }}{\sqrt{2}}, \sqrt{n})$  \\
    $k = 3$ &  $(\frac{\sqrt{1}}{\sqrt{3}}, \frac{\sqrt{2}}{\sqrt{3}}, 1)$ & $(\frac{\sqrt{4}}{\sqrt{3}}, \frac{\sqrt{5}}{\sqrt{3}}, \sqrt{2})$ &  $(\frac{\sqrt{7}}{\sqrt{3}}, \frac{\sqrt{8}}{\sqrt{3}}, \sqrt{3})$  & ... & $(\frac{\sqrt{3n -1 }}{\sqrt{3}}, \frac{\sqrt{3n-2}}{\sqrt{3}}, \sqrt{n})$   \\
    $k = 4$ &  $(\frac{\sqrt{1}}{\sqrt{4}}, \frac{\sqrt{2}}{\sqrt{4}}, \frac{\sqrt{3}}{\sqrt{4}}, 1)$ & $(\frac{\sqrt{5}}{\sqrt{4}}, \frac{\sqrt{6}}{\sqrt{4}}, \frac{7}{\sqrt{4}}, \sqrt{2})$ &  $(\frac{\sqrt{9}}{\sqrt{4}}, \frac{\sqrt{10}}{\sqrt{4}}, \frac{\sqrt{11}}{\sqrt{4}}, \sqrt{3})$  & ... & $(\frac{\sqrt{4n -1 }}{\sqrt{4}}, \frac{\sqrt{4n-2}}{\sqrt{4}}, \frac{\sqrt{4n-3}}{\sqrt{4}}, \sqrt{n})$   \\
    \bottomrule
\end{tabular}
\end{table}

可以看到,当 $k$ 是奇数时,波带片的一个圆环内透过奇数个相邻波带,抵消之后仍剩一个波带的振幅,呈现亮斑;当 $k$ 是偶数时,波带片的一个圆环内透过偶数个相邻波带,抵消之后剩下的是零振幅,呈现暗斑。因此,波带片的所有焦点位置是 $\frac{f}{1}, \frac{f}{3}, \frac{f}{5}, ...$。

再来看辐照度大小,在 $k$ 为奇数的情况下,每个波带片环相当于只透过一个波带,由 $A = \pi r_0 \lambda$ 知道波带面积缩小为原来的 $\frac{1}{k}$,因此电场振幅变为原来的 $\frac{1}{k}$,辐照度变为原来的 $\frac{1}{k^2}$。综上,波带片的所有焦点和辐照度大小是:
\begin{gather}
f_1 = \frac{R_1^2}{\lambda},\quad   E_1 \approx n E_0 ,\quad   I_1 \approx n^2 I_0 \\
f_k = \frac{f_1}{k},\quad I_k = \frac{I_1}{k^2},\quad k = 1, 3, 5, ...
\end{gather}
式中 $n$ 为波带片最外圆的半径级数,也即波带片圆环总个数,$E_0$ 和 $I_0$ 分别是无阻挡时的振幅和辐照度。

\section{一束波长为 500 nm 的平行光垂直照射在一个单缝上,如果所用的单缝的宽度 $a=0.5 \ \mathrm{mm}$,缝后紧挨着的薄透镜焦距 $f=1$ m ,试求: (1) 中央明条纹的角宽度; (2) 中央亮纹的线宽度; (3) 第一级与第二级暗纹的距离。}

\begin{graybox}
本题中的计算都可以借助近似来完成,但考虑到手边有计算机,我们还是取原公式来精确计算。从计算结果我们也可以看到,在小角近似下的误差是非常小的。
\end{graybox}

\subsection{中央明条纹的角宽度}

计算菲涅尔数 $N_F$ : 
\begin{equation}
N_F = \frac{b^2}{f \lambda} = \frac{(0.5 \ \mathrm{mm})^2}{1 \ \mathrm{m} \times 500 \ \mathrm{nm}} = 0.5
\end{equation}
可以近似认为发生的是夫琅禾费衍射。由夫琅禾费单缝衍射公式:
\begin{equation}
I = I(0) \sinc^2 \beta,\quad \beta = \frac{1}{2}kb \sin \theta
\end{equation}
可以推得全峰角宽度 $\xi_\theta$ : 
\begin{gather}
    I = I(0) \sinc^2 \beta =  I(0) \sinc^2\left( \frac{\pi b}{\lambda} \sin \theta \right) = I(0) \sinc^2\left( N \pi \sin \theta \right) \\ 
    b \sin \theta_0 = \lambda \Longrightarrow \xi_{\theta} = 2 \theta_0 = 2 \arcsin \left(\frac{\lambda}{b}\right) = 2 \arcsin \left(\frac{1}{N}\right)
\end{gather}
代入数据:
\begin{equation}
    \xi_\theta = 2 \arcsin \left( \frac{500 \ \mathrm{nm}}{ 0.5 \ \mathrm{mm} } \right) = 0.00200000033 \ \mathrm{rad} \approx 0.0020 \ \mathrm{rad}
\end{equation}

\subsection{中央亮纹的线宽度}
借助狭缝中心点在屏幕上的成像,可以帮助我们推导辐照度随位置 $z$ 的分布。可以计算中央极大的全峰线宽度 $\xi_z$ : 
\begin{gather}
    \sin \theta_0 = \frac{\lambda}{b} \Longrightarrow \xi_z = 2 f \tan \theta_0 = 2f\frac{\sin \theta}{\sqrt{1 - \sin^2 \theta}} = \frac{2f}{\sqrt{N^2 - 1}},\quad N = \frac{b}{\lambda}
\end{gather}
代入数据:
\begin{equation}
N = \frac{0.5 \ \mathrm{mm}}{0.5 \ \mathrm{\mu m}} = 1000 \Longrightarrow \xi_z = \frac{2 \times 1 \ \mathrm{m}}{\sqrt{1000^2 - 1}} = 0.0020000010 \ \mathrm{m} \approx 2.0 \ \mathrm{mm}
\end{equation}

\subsection{第一级与第二级暗纹的距离}
由单缝夫琅禾费衍射公式,可得极小值(暗斑)对应的角度:
\begin{equation}
b\sin \theta_m = m\lambda \Longrightarrow \sin \theta_m = \frac{m\lambda}{b},\quad m = 1, 2, ...
\end{equation}
于是一级和二级暗斑距离为:
\begin{equation}
\Delta x = f\tan \theta_2 - f \tan \theta_1 = f\left[ \frac{\sin \theta_2}{\sqrt{1 - \sin^2 \theta_2}} - \frac{\sin \theta_1}{\sqrt{1 - \sin^2 \theta_1}} \right],\quad \sin \theta_1 = \frac{\lambda}{b},\ \sin \theta_2 = \frac{2\lambda}{b}
\end{equation}
代入数据:
\begin{equation}
    \Delta x = 0.00100000350 \ \mathrm{m} \approx 1.0 \ \mathrm{mm}
\end{equation}

\section{一束白光垂直照射在一光栅上,在形成的同一级光栅光谱中,偏离中央明纹最远的是红光还是蓝光?为什么?}
\vspace*{-3mm}
由单缝夫琅禾费衍射公式,可得极大值(亮斑)对应的角度:
\begin{equation}
b\sin \theta_m = \left(m + \frac{1}{2}\right)\lambda,\quad m = 1, 2, ...
\end{equation}
从蓝光到红光,随着 $\lambda$ 的增大,同级下的 $\theta_m$ 也增大,因此红光散射角度更大,偏离中央明纹最远的是红光。

\vspace*{-5mm}
\section{波长为 600 nm 的单色光垂直入射在一光栅上,第二级明纹出现在 $\sin \theta_2 = 0.2$ 处,第 4 级为第一个缺级。求 (1) 光栅上相邻两缝的距离是多少? (2) 狭缝可能的最小宽度是多少? (3) 在该最小宽度下,实际上能观察到的全部明纹数是多少?}

\vspace*{-3.5mm}
\subsection{光栅上相邻两缝的距离是多少?}
要相邻两缝的中心距离,即求光栅常数 $a$。由光栅方程:
\begin{equation}
a(\sin \theta_m - \sin \theta_i) = m\lambda,\quad m = 0, \pm1, \pm2, ... \Longrightarrow a = \frac{2 \lambda}{\sin \theta_2} = 6 \ \mathrm{\mu m}
\end{equation}

\vspace*{-8mm}
\subsection{狭缝可能的最小宽度是多少?}
当光栅是多缝光栅时,可以考虑狭缝宽度的概念。设光栅有 $N_\text{s}$ 条多缝,有多缝夫琅禾费衍射公式:
\begin{equation}
I = I(0) \sinc^2 \beta\  \frac{\sin^2 N_\text{s} \alpha}{N_s^2 \sin^2 \alpha} ,\quad \beta = \frac{1}{2}kb \sin \theta,\ \alpha = \frac{1}{2}ka \sin \theta
\end{equation}
设 $m = \frac{a}{b}$,由于第 4 极大是第一个缺级,因此 $m = 4$,得到:
\begin{equation}
b = \frac{a}{4} = 1.5 \ \mathrm{\mu m}
\end{equation}

当然,$m$ 可以适当的有一些变化范围,我们取 $m \in [3.5, 4.5]$,可以得到:
\begin{equation}
b \in [1.33 \ \mathrm{\mu m}, 1.71 \ \mathrm{\mu m}] \Longrightarrow b_{\min} = 1.3  \ \mathrm{\mu m}
\end{equation}

\vspace*{-6.5mm}
\subsection{在该最小宽度下,实际上能观察到的全部明纹数是多少?}

若 $b = 1.5  \ \mathrm{\mu m}$,此时 $m = 4$,实际能观察到的全部明纹数(主衍射峰内的极大,不包括缺级部分)为 $2m - 1 = 7$ 条。但是在 $b_{\min} = 1.33  \ \mathrm{\mu m}$ 下,$m = 4.5$,不好确定能观察到的明纹数,因此我们直接作出图像:
\begin{figure}[H]\centering
\begin{subfigure}[b]{0.5\columnwidth}\centering
    \includegraphics[height=190pt]{assets/4/2024-11-08_01-24-51.pdf}
    \caption{光栅的衍射图样}
\end{subfigure}\hfill
\begin{subfigure}[b]{0.5\columnwidth}\centering
    \includegraphics[height=190pt]{assets/4/2024-11-08_01-24-54.pdf}
    \caption{光栅的对数衍射图样}
\end{subfigure}\vspace*{-5mm}
\caption{光栅的衍射图样}
\end{figure}
由图可知,$b_{\min} = 1.33  \ \mathrm{\mu m}$ 时 ($m = 4.5$) ,实际上能看到 7 条明纹,分别是 $0, \pm 1, \pm 2, \pm 3$ 级明纹。

% --------------------------- 附录 --------------------------- %
% >> ------------------------ 附录 ------------------------ << %

\newpage
\appendix
% chapter 标题自定义设置
\titleformat{\chapter}[hang]{\normalfont\huge\bfseries\centering}{}{20pt}{}
\titlespacing*{\chapter}{0pt}{-25pt}{8pt} % 控制上方空白的大小
% section 标题自定义设置 
\titleformat{\section}[hang]{\normalfont\centering\Large\bfseries}{\thesection}{8pt}{}
\lhead{附录 \thechapter}

% 附录 A
\chapter*{附录 A\hspace*{20pt}  Matlab 代码}\setcounter{chapter}{1} 
\setcounter{equation}{0}    % 重置公式计数器   
\addcontentsline{toc}{chapter}{附录 A\hspace*{6pt}  Matlab 代码}   
\thispagestyle{fancy} 
\setcounter{section}{0}   
\renewcommand\thesection{A.\arabic{section}}   
\renewcommand{\thefigure}{A.\arabic{figure}} 
\renewcommand{\thetable}{A.\arabic{table}}


\section{图 \ref{振幅系数随入射角的变化} 源码}\label{图振幅系数随入射角的变化源码}
\begin{matlablisting}
%%%%%%%%%% 空气入射玻璃 %%%%%%%%%%
global n_i n_t
n_i = 1;
n_t = 1.5;

theta_t = @(theta_i) asin(n_i/n_t*sin(theta_i));
r_s = @(theta_i, theta_t) - sin(theta_i - theta_t)./sin(theta_i + theta_t);
r_p = @(theta_i, theta_t) + tan(theta_i - theta_t)./tan(theta_i + theta_t);
t_s = @(theta_i, theta_t) 2*sin(theta_t).*cos(theta_i)./sin(theta_i + theta_t);
t_p = @(theta_i, theta_t) 2*sin(theta_t).*cos(theta_i) ./ ( sin(theta_i + theta_t).*cos(theta_i - theta_t) );
theta_B = atan(n_t/n_i);
theta_C = asin(n_t/n_i);

theta_array = linspace(-0.1, pi/2, 101);
Y = [
    r_s(theta_array, theta_t(theta_array))
    r_p(theta_array, theta_t(theta_array))
    t_s(theta_array, theta_t(theta_array))
    t_p(theta_array, theta_t(theta_array))
    ];
stc = MyPlot(theta_array, Y);
xline(theta_B, 'b')
yline(0)
xlim([0, pi/2])
ylim([-1, 1])
stc.leg.String = ["$r_s$"; "$r_p$"; "$t_s$"; "$t_p$"; "$\theta_i = \theta_B$"];
stc.leg.Interpreter = "latex";
stc.leg.FontSize = 14;
stc.leg.Location = "southwest";
stc.axes.Title.String = '$n_i = 1 < n_t = 1.5$';
stc.axes.Title.Interpreter = "latex";
stc.label.x.String = '$\theta_i$';
stc.label.y.String = '$r$';
stc.plot.plot_3.LineStyle = ":";
stc.plot.plot_3.Color = 'b';
stc.plot.plot_4.LineStyle = ":";
stc.plot.plot_4.Color = [1 0 1];
%MyExport_pdf

%%%%%%%%%% 玻璃入射空气 %%%%%%%%%%
n_i = 1.5;
n_t = 1;

theta_t = @(theta_i) asin(n_i/n_t*sin(theta_i));
r_s = @(theta_i, theta_t) - sin(theta_i - theta_t)./sin(theta_i + theta_t);
r_p = @(theta_i, theta_t) + tan(theta_i - theta_t)./tan(theta_i + theta_t);
t_s = @(theta_i, theta_t) 2*sin(theta_t).*cos(theta_i)./sin(theta_i + theta_t);
t_p = @(theta_i, theta_t) 2*sin(theta_t).*cos(theta_i) ./ ( sin(theta_i + theta_t).*cos(theta_i - theta_t) );
theta_B = atan(n_t/n_i);
theta_C = asin(n_t/n_i);


theta_array = linspace(0, theta_C, 101);
Y = [
    r_s(theta_array, theta_t(theta_array))
    r_p(theta_array, theta_t(theta_array))
    t_s(theta_array, theta_t(theta_array))
    t_p(theta_array, theta_t(theta_array))
    ];
stc = MyPlot(theta_array, Y);
xline(theta_B, 'b')
xline(theta_C, 'r')
yline(0)
xlim([0, pi/2])
ylim([-0.5, 3])
stc.leg.String = ["$r_s$"; "$r_p$"; "$t_s$"; "$t_p$"; "$\theta_i = \theta_B$"; "$\theta_i = \theta_C$"];
stc.leg.Interpreter = "latex";
stc.axes.Title.String = '$n_i = 1.5 > n_t = 1$';
stc.axes.Title.Interpreter = "latex";
stc.label.x.String = '$\theta_i$';
stc.label.y.String = '$r$';
stc.plot.plot_3.LineStyle = ":";
stc.plot.plot_3.Color = 'b';
stc.plot.plot_4.LineStyle = ":";
stc.plot.plot_4.Color = [1 0 1];
%MyExport_pdf
\end{matlablisting}

\section{公式 \ref{玻璃折射率} 源码}\label{玻璃折射率源码}

\begin{matlablisting}
R_s = @(n_ti, t) ( (cos(t) - sqrt(n_ti^2 - sin(t)^2)) / (cos(t) + sqrt(n_ti^2 - sin(t)^2)) )^2;
R_p = @(n_ti, t) ( (n_ti^2*cos(t) - sqrt(n_ti^2 - sin(t)^2)) / (n_ti^2*cos(t) + sqrt(n_ti^2 - sin(t)^2)) )^2;

theta_B = @(n_ti) atan(n_ti);
n_ti = fzero(@(n_ti) 0.5*(R_s(n_ti, theta_B(n_ti)) + R_p(n_ti, theta_B(n_ti))) - 0.14, 1);

theta_C = @(n_ti) asin(n_ti);
theta_B = @(n_ti) atan(n_ti);
theta_t = @(n_ti, theta_i) asin(sin(theta_i)/n_ti);

disp(['n_ti = ', num2str(n_ti, '%.6f')])
disp(['theta_i = theta_B = ', num2str(theta_B(n_ti), '%.6f') ' rad = ', num2str(rad2deg(theta_B(n_ti)), '%.6f'), ' deg'])
disp(['theta_t = ', num2str(theta_t(n_ti, theta_B(n_ti)), '%.6f') ' rad = ', num2str(rad2deg(theta_t(n_ti, theta_B(n_ti))), '%.6f'), ' deg'])
disp(['theta_C = ', num2str(theta_C(n_ti), '%.6f') ' rad = ', num2str(rad2deg(theta_C(n_ti)), '%.6f'), ' deg'])
disp(['R_s = ', num2str(R_s(n_ti, theta_B(n_ti)), '%.6f')])
disp(['R_p = ', num2str(R_p(n_ti, theta_B(n_ti)), '%.6f')])
disp(['T_s = ', num2str(1 - R_s(n_ti, theta_B(n_ti)), '%.6f')])
disp(['T_p = ', num2str(1 - R_p(n_ti, theta_B(n_ti)), '%.6f')])
disp(['R = ', num2str( 0.5*(R_s(n_ti, theta_B(n_ti)) + R_p(n_ti, theta_B(n_ti))) , '%.6f')])
disp(['T = ', num2str( 0.5*(2 - R_s(n_ti, theta_B(n_ti)) - R_p(n_ti, theta_B(n_ti))), '%.6f')])


%{
>> Output:
n_ti = 0.554902
theta_i = theta_B = 0.506599 rad = 29.025970 deg
theta_t = 1.064198 rad = 60.974030 deg
theta_C = 0.588245 rad = 33.703947 deg
R_s = 0.280000
R_p = 0.000000
T_s = 0.720000
T_p = 1.000000
R = 0.140000
T = 0.860000
%}
\end{matlablisting}

\section{公式 \ref{解入射角} 源码}\label{公式解入射角源码}

\begin{matlablisting}
R_s = @(n_ti, t) ( (cos(t) - sqrt(n_ti^2 - sin(t)^2)) / (cos(t) + sqrt(n_ti^2 - sin(t)^2)) )^2;
R_p = @(n_ti, t) ( (n_ti^2*cos(t) - sqrt(n_ti^2 - sin(t)^2)) / (n_ti^2*cos(t) + sqrt(n_ti^2 - sin(t)^2)) )^2;

theta_i = fzero(@(t) ( R_s(1.5, t) + R_p(1.5, t) - 2*0.14) , deg2rad(45));
disp(['theta_i = ', num2str(theta_i, '%.6f'), ' rad'])
disp(['theta_i = ', num2str(rad2deg(theta_i), '%.6f'), ' deg'])
disp(['R_s = ', num2str(R_s(1.5, theta_i), '%.6f')])
disp(['R_p = ', num2str(R_p(1.5, theta_i), '%.6f')])
disp(['T_s = ', num2str(1 - R_s(1.5, theta_i), '%.6f')])
disp(['T_p = ', num2str(1 - R_p(1.5, theta_i), '%.6f')])
disp(['R = ', num2str( 0.5*(R_s(1.5, theta_i) + R_p(1.5, theta_i)) , '%.6f')])
disp(['T = ', num2str( 0.5*(2 - R_s(1.5, theta_i) - R_p(1.5, theta_i)), '%.6f')])

%{
>> Output:
theta_i = 1.173220 rad
theta_i = 67.220559 deg
R_s = 0.256933
R_p = 0.023067
T_s = 0.743067
T_p = 0.976933
R = 0.140000
T = 0.860000
%}
\end{matlablisting}

\section{图 \ref{方程左边的变化情况} 源码}\label{方程左边的变化情况源码}

\begin{matlablisting}
R_s = @(n_ti, t) ( (cos(t) - sqrt(n_ti^2 - sin(t)^2)) / (cos(t) + sqrt(n_ti^2 - sin(t)^2)) )^2;
R_p = @(n_ti, t) ( (n_ti^2*cos(t) - sqrt(n_ti^2 - sin(t)^2)) / (n_ti^2*cos(t) + sqrt(n_ti^2 - sin(t)^2)) )^2;
theta_B = @(n_ti) atan(n_ti);

eq_left = @(n_ti) 0.5*(R_s(n_ti, theta_B(n_ti)) + R_p(n_ti, theta_B(n_ti)))

X = linspace(0, 2, 100);
Y = zeros(size(X));
for i = 1:length(X)
    Y(i) = eq_left(X(i));
end

stc = MyPlot(X, Y);
yline(0.14, 'Color', 'r', 'LineWidth', 0.4);
xline(0.554902, 'Color', [0.1, 0.1, 0.1], 'LineWidth', 0.4)
xline(1.802121, 'Color', [0.3, 0.3, 0.3], 'LineWidth', 0.4)
stc.axes.Title.String = '';
stc.label.x.String = '$n_{ti}$';
stc.leg.Location = 'northeast';
hold on
scatter([0.554902, 1.802121], [0.14, 0.14], 180, '.r')
stc.leg.String = ["$y =  0.5\left[ R_s(n_{ti}) + R_p(n_{ti}) \right]$"; "$y = 0.14$"; "$n_{ti} = 0.554902$"; "$n_{ti} = 1.802121$"];
%MyExport_pdf
\end{matlablisting}

% >> ------------------------ 附录 ------------------------ << %
% --------------------------- 附录 --------------------------- %


\end{document}

% VScode 常用快捷键:

% Ctrl + R:                 打开最近的文件夹
% F2:                       变量重命名
% Ctrl + Enter:             行中换行
% Alt + up/down:            上下移行
% 鼠标中键 + 移动:           快速多光标
% Shift + Alt + up/down:    上下复制
% Ctrl + left/right:        左右跳单词
% Ctrl + Backspace/Delete:  左右删单词    
% Shift + Delete:           删除此行
% Ctrl + J:                 打开 VScode 下栏(输出栏)
% Ctrl + B:                 打开 VScode 左栏(目录栏)
% Ctrl + `:                 打开 VScode 终端栏
% Ctrl + 0:                 定位文件
% Ctrl + Tab:               切换已打开的文件(切标签)
% Ctrl + Shift + P:         打开全局命令(设置)

% Latex 常用快捷键

% Ctrl + Alt + J:           由代码定位到PDF
% 


% Git提交规范:
% update: Linear Algebra 2 notes
% add: Linear Algebra 2 notes
% import: Linear Algebra 2 notes
% delete: Linear Algebra 2 notes
