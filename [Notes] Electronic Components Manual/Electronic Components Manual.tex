% 若编译失败,且生成 .synctex(busy) 辅助文件,可能有两个原因:
% 1. 需要插入的图片不存在:Ctrl + F 搜索 'figure' 将这些代码注释/删除掉即可
% 2. 路径/文件名含中文或空格:更改路径/文件名即可

% --------------------- 文章宏包及相关设置 --------------------- %
% >> ------------------ 文章宏包及相关设置 ------------------ << %
% 设定文章类型与编码格式
\documentclass[UTF8]{report}		

% 本 .tex 专属的宏定义
    \def\uV{\ \mathrm{uV}}
    \def\mV{\ \mathrm{mV}}
    \def\V{\ \mathrm{V}}
    \def\kV{\ \mathrm{KV}}
    \def\KV{\ \mathrm{KV}}
    \def\MV{\ \mathrm{MV}}
    \def\uF{\ \mathrm{uF}}
    \def\nF{\ \mathrm{nF}}
    \def\pF{\ \mathrm{pF}}
    \def\uA{\ \mathrm{uA}}
    \def\mA{\ \mathrm{mA}}
    \def\A{\ \mathrm{A}}
    \def\kA{\ \mathrm{KA}}
    \def\KA{\ \mathrm{KA}}
    \def\MA{\ \mathrm{MA}}
    \def\uO{\ \mu\Omega}
    \def\mO{\ \mathrm{m}\Omega}
    \def\O{\ \Omega}
    \def\kO{\ \mathrm{K}\Omega}
    \def\KO{\ \mathrm{K}\Omega}
    \def\MO{\ \mathrm{M}\Omega}
    \def\Hz{\ \mathrm{Hz}}
    \def\Res{\,\mathrm{Res}\,}
    \def\Im{\,\mathrm{Im}\,}
    \def\Re{\,\mathrm{Re}\,}

% 自定义宏定义
    \def\N{\mathbb{N}}
    \def\F{\mathbb{F}}
    \def\Z{\mathbb{Z}}
    \def\Q{\mathbb{Q}}
    \def\R{\mathbb{R}}
    \def\C{\mathbb{C}}
    \def\T{\mathbb{T}}
    \def\S{\mathbb{S}}
    %\def\A{\mathbb{A}}
    \def\I{\mathscr{I}}
    \def\d{\mathrm{d}}
    \def\p{\partial}


% 导入基本宏包
    \usepackage[UTF8]{ctex}     % 设置文档为中文语言
    \usepackage{hyperref}  % 宏包:自动生成超链接 (此宏包与标题中的数学环境冲突)
    \hypersetup{
        % 超链接颜色设置
        colorlinks=true,    % false:边框链接 ; true:彩色链接
        citecolor={blue},    % 文献引用颜色
        linkcolor={blue},   % 目录 (我们在目录处单独设置),公式,图表,脚注等内部链接颜色
        urlcolor={magenta},    % 网页 URL 链接颜色,包括 \href 中的 text
        % pdf 相关设置
        pdfauthor={丁毅},% PDF 作者
        pdftitle={Notes of Principles of Electric Circuits},% PDF 标题
        pdfproducer={LaTeX with hyperref}, % PDF 制作软件
        pdfcreator={xelatex},       % PDF 创建者
        pdfpagelayout=TwoColumn,    % 双栏布局
        pdfstartview=FitH,          % 页面宽度适合窗口
        %pdfpagemode=UseNone,       % 不显示书签 (默认会显示)
        pdfpagemode=UseOutlines,    % 显示书签
        pdfnewwindow=true,          % 在新窗口中打开链接
        pdfencoding=auto,           % 自动编码
        pdfborder={0 0 0},          % pdf 无边框
        pdfhighlight=/I,            % 链接高亮样式
        % magenta 洋红色
        % cyan 浅蓝色 
        % magenta 洋红色
        % yellow 黄色
        % black 黑色
        % white 白色
        % red 红色
        % green 绿色
        % blue 蓝色
        % gray 灰色
        % darkgray 深灰色
        % lightgray 浅灰色
        % brown 棕色
        % lime 石灰色
        % olive 橄榄色
        % orange 橙色
        % pink 粉红色
        % purple 紫色
        % teal 蓝绿色
        % violet 紫罗兰色
    }
    % \usepackage{docmute}    % 宏包:子文件导入时自动去除导言区,用于主/子文件的写作方式,\include{./51单片机笔记}即可。注:启用此宏包会导致.tex文件capacity受限。
    \usepackage{amsmath}    % 宏包:数学公式
    \usepackage{mathrsfs}   % 宏包:提供更多数学符号
    \usepackage{amssymb}    % 宏包:提供更多数学符号
    \usepackage{pifont}     % 宏包:提供了特殊符号和字体
    \usepackage{extarrows}  % 宏包:更多箭头符号 
    \usepackage{multicol}   % 宏包:支持多栏 

% 文章页面margin设置
    \usepackage[a4paper]{geometry}
        \geometry{top=0.75in}
        \geometry{bottom=0.75in}
        \geometry{left=0.75in}
        \geometry{right=0.75in}   % 设置上下左右页边距
        \geometry{marginparwidth=1.75cm}    % 设置边注距离(注释、标记等)


% table 支持
    \usepackage{booktabs}   % 宏包:三线表
    \usepackage{tabularray} % 宏包:表格排版
    \usepackage{longtable}  % 宏包:长表格

% figure 设置
    \usepackage{graphicx}  % 支持 jpg, png, eps, pdf 图片 
    \usepackage{svg}       % 支持 svg 图片
        \svgsetup{
            % 指向 inkscape.exe 的路径
            inkscapeexe = C:/aa_MySame/inkscape/bin/inkscape.exe, 
            % 一定程度上修复导入后图片文字溢出几何图形的问题
            inkscapelatex = false                 
        }
    \usepackage{subcaption} % 用于子图和小图注  

% 图表进阶设置
    \usepackage{caption}    % 图注、表注
        \captionsetup[figure]{name=图}  
        \captionsetup[table]{name=表}
        \captionsetup{
            labelfont=bf, % 设置标签为粗体
            textfont=bf,  % 设置文本为粗体
            font=small  
        }
    \usepackage{float}     % 图表位置浮动设置 
    \usepackage{etoolbox} % 用于保证图注表注的数学字符为粗体
        \AtBeginEnvironment{figure}{\boldmath} % 图注中的数学字符为粗体
        \AtBeginEnvironment{table}{\boldmath}  % 表注中的数学字符为粗体
        \AtBeginEnvironment{tabular}{\unboldmath}   % 保证表格中的数学字符不受额外影响

% 圆圈序号自定义
    \newcommand*\circled[1]{\tikz[baseline=(char.base)]{\node[shape=circle,draw,inner sep=0.8pt, line width = 0.03em] (char) {\small \bfseries #1};}}   % TikZ solution

% 列表设置
    \usepackage{enumitem}   % 宏包:列表环境设置
        \setlist[enumerate]{
            label=(\arabic*) ,   % 设置序号样式为加粗的 (1) (2) (3)
            ref=\arabic*, % 如果需要引用列表项,这将决定引用格式(这里仍然使用数字)
            itemsep=0pt, parsep=0pt, topsep=0pt, partopsep=0pt, leftmargin=3.5em} 
        \setlist[itemize]{itemsep=0pt, parsep=0pt, topsep=0pt, partopsep=0pt, leftmargin=3.5em}
        \newlist{circledenum}{enumerate}{1} % 创建一个新的枚举环境  
        \setlist[circledenum,1]{  
            label=\protect\circled{\arabic*}, % 使用 \arabic* 来获取当前枚举计数器的值,并用 \circled 包装它  
            ref=\arabic*, % 如果需要引用列表项,这将决定引用格式(这里仍然使用数字)
            itemsep=0pt, parsep=0pt, topsep=0pt, partopsep=0pt, leftmargin=3.5em
        }  

% 其它设置
    % 脚注设置
        \renewcommand\thefootnote{\ding{\numexpr171+\value{footnote}}}
    % 参考文献引用设置
        \bibliographystyle{unsrt}   % 设置参考文献引用格式为unsrt
        \newcommand{\upcite}[1]{\textsuperscript{\cite{#1}}}     % 自定义上角标式引用
    % 文章序言设置
        \newcommand{\cnabstractname}{序言}
        \newenvironment{cnabstract}{%
            \par\Large
            \noindent\mbox{}\hfill{\bfseries \cnabstractname}\hfill\mbox{}\par
            \vskip 2.5ex
            }{\par\vskip 2.5ex}

% 文章默认字体设置
    \usepackage{fontspec}   % 宏包:字体设置
        \setmainfont{SimSun}    % 设置中文字体为宋体字体
        \setCJKmainfont[AutoFakeBold=3]{SimSun} % 设置加粗字体为 SimSun 族,AutoFakeBold 可以调整字体粗细
        \setmainfont{Times New Roman} % 设置英文字体为Times New Roman

% 各级标题自定义设置
    \usepackage{titlesec}   
        % chapter 标题自定义设置
        \titleformat{\chapter}[hang]{\normalfont\huge\bfseries\centering\boldmath}{第\,\thechapter\,章}{20pt}{}
        \titlespacing*{\chapter}{0pt}{-20pt}{20pt} % 控制上下间距
        % section标题自定义设置 
        \titleformat{\section}[hang]{\normalfont\Large\bfseries\boldmath}{§\,\thesection\,}{8pt}{}
        % subsubsection标题自定义设置
        \titlespacing*{\subsubsection}{0pt}{3pt}{0pt} % 控制上下间距


% --------------------- 文章宏包及相关设置 --------------------- %
% >> ------------------ 文章宏包及相关设置 ------------------ << %

% ------------------------ 文章信息区 ------------------------ %
% ------------------------ 文章信息区 ------------------------ %
% 页眉页脚设置
    \usepackage{fancyhdr}   % 宏包:页眉页脚设置
        \pagestyle{fancy}
        \fancyhf{}
        \cfoot{\thepage}
        \renewcommand\headrulewidth{1pt}
        \renewcommand\footrulewidth{0pt}
        \usepackage{fontawesome}    % 宏包:更多符号与图标 (用于插入 GitHub 图标等)
        \lhead{\small
        \href{https://github.com/YiDingg/LatexNotes}{\color{black}\faGithub\ https://github.com/YiDingg/LatexNotes}
        }
        \chead{Electronic Components Manual}
        \rhead{\small dingyi233@mails.ucas.ac.cn}

%文档信息设置
    \title{电子元件手册\\ Electronic Components Manual}
    \author{丁毅\\ \footnotesize {\kaishu (中国科学院大学,北京 100049)} \\ Yi Ding \\ \footnotesize \textit{(University of Chinese Academy of Sciences, Beijing 100049, China)}}
    \date{\footnotesize 2024.11 -- ...}
% ------------------------ 文章信息区 ------------------------ %
% ------------------------ 文章信息区 ------------------------ %

% 开始编辑文章

\begin{document} 
\zihao{5}           % 设置全文字号大小

% ------------------------ 封面序言与目录 ------------------------ %
% >> --------------------- 封面序言与目录 --------------------- << %
% 封面
    \maketitle\newpage  
    \pagenumbering{Roman} % 页码为大写罗马数字
    \thispagestyle{fancy}   % 显示页码、页眉等

% 序言
\begin{cnabstract}\normalsize 
本文为笔者本科时的电子元件手册,对学习过程中接触较多的元件作了系统性的介绍和总结,同时对部分元件进行了仿真或实际电路测试,给出了相应的参数曲线或波形图。

随着学习的不断深入,笔者对某些元件的认识一定会更加深刻,也会认识新的元件。因此,笔者打算将本手册作为一份长期更新的(真的很长期)电子元件手册,既是我个人对元件知识的总结,也是便于以后自己参考。读者可到我的 GitHub 下载手册的最新版本 \href{https://github.com/YiDingg/LatexNotes}{https://github.com/YiDingg/LatexNotes},也可以在我的个人网站 \href{URL}{待更新}  上找到相关资料,包括各元件的 Data Sheet、学习电子元件及其相关电路的优秀网站等(例如 \href{https://www.electronics-tutorials.ws/}{Electornics Tutorials})。

另外,有相当一部分元件(例如 Operational Amplifier)其实是基础元件所构成的模块化电路。在本书,我们不会详谈如何通过基本元件搭建出这些进阶元件,而是讨论如何对这些元件(模块化的电路)建立正确的认识,探究在不同情况下应该选用怎样的模型进行分析,以达到足够高的精度,同时尽可能地降低模型复杂度。实现它们的具体电路,以及 Oscillators、Amplifiers 和 Power Supplies 等模块化电路,会放到另一本书“电子电路手册”中,读者可到网址 \href{URL}{待更新} 自行下载和阅读。为了提高读者的自学效率,笔者在这里推荐几个免费且优秀的电子学习网站:
\begin{enumerate}
\item \href{https://www.electronics-tutorials.ws/}{Electornics Tutorials} : 这个网站提供了丰富的电子学习资料,包括 Transistors、Amplifiers、Diodes、Filters 等,还提供了很多实用的电子工程师工具,例如在线电阻电感电容计算器等;
\item \href{https://www.learnabout-electronics.org/}{https://www.learnabout-electronics.org/} : 这个网站提供了电子学习的基础和进阶知识,包括 Semiconductors, Amplifiers, Oscillators, Power Supplies 等,最令人惊喜的是几乎所有资料都提供了 PDF 下载,方便读者下载后自行学习;
\item \href{https://www.electronics-lab.com}{https://www.electronics-lab.com} : 这个网站不仅开源了很多电子电路设计实例,包括基础电路、模拟电路、数字电路等,还提供了丰富的学习资源(文章)在 \href{https://www.electronics-lab.com/articles/}{here};
\item \href{https://developerhelp.microchip.com/}{https://developerhelp.microchip.com/} : 这是 Microchip 官方的开发者帮助网站,提供了大量的电子元件的 Data Sheet、Application Notes、Reference Manuals 等,是学习 Microchip 产品的重要参考网站,当然,在这里也可以找到与模拟电路、数字电路相关的的学习资料;
\end{enumerate}


由于个人学识浅陋,认识有限,文中难免有不妥甚至错误之处,望读者不吝指正。读者可以将错误发送到我的邮箱 {\color{blue}\ dingyi233@mails.ucas.ac.cn},也可以到笔者的 \href{https://github.com/YiDingg/LatexNotes}{GitHub (https://github.com/YiDingg/LatexNotes)} 上提 issue,衷心感谢。

\end{cnabstract}
\addcontentsline{toc}{chapter}{序言} % 手动添加为目录

% 目录
    \setcounter{tocdepth}{2}    % 目录深度 (为 1 时显示到 section)
    % 目录页
        \tableofcontents
        \addcontentsline{toc}{chapter}{目录}
        \thispagestyle{fancy}                   % 显示页码、页眉等
    % 插图页
        %\cleardoublepage\listoffigures
        %\addcontentsline{toc}{chapter}{插图}
        %\thispagestyle{fancy}                   % 显示页码、页眉等
    % 表格页
        %\cleardoublepage\listoftables
        %\addcontentsline{toc}{chapter}{表格}
        %\thispagestyle{fancy}                   % 显示页码、页眉等 

% 收尾工作
    \newpage    
    \pagenumbering{arabic} 


% >> --------------------- 封面序言与目录 --------------------- << %
% ------------------------ 封面序言与目录 ------------------------ %




\chapter{Resistors (电阻)}\thispagestyle{fancy}


\chapter{Capacitors (电容)}\thispagestyle{fancy}

\chapter{Inductors (电感)}\thispagestyle{fancy}


\chapter{Diodes (二极管)}\thispagestyle{fancy}
二极管分为...

\section{Power/Rectifier/General Diode (功率/整流/通用二极管)}
\subsection{1N4007 [GOODWORK (固得沃克)]}

由 GOODWORK (固得沃克, 中国江苏) 生产的 1N4007 是一种功率/整流/通用二极管,常应用于整流电路等(同类型的有 1N4000, 1N4001, ..., 1N4006),可以在 \href{https://item.szlcsc.com/3428711.html}{立创商城} 或 \href{http://www.gk-goodwork.com/cn/product/1n4007.html}{GOODWORK 官网} 上找到它。
\begin{figure}[H]\centering
\begin{subfigure}[b]{0.5\columnwidth}\centering
    \includegraphics[height=180pt]{assets/Diodes/1N4007 [GOODWORK (固得沃克)]/1N4007 正面.png}
    \caption{The front side}
\end{subfigure}\hfill
\begin{subfigure}[b]{0.5\columnwidth}\centering
    \includegraphics[height=180pt]{assets/Diodes/1N4007 [GOODWORK (固得沃克)]/1N4007 背面.png}
    \caption{The back side}
\end{subfigure}
\caption{1N4007 [GOODWORK (固得沃克)]}
\end{figure}

Data Sheet 中的具体参数我们不再重复了,可到数据手册中自行查找,这里给出 1N4007 在经典 NMOS 反相器电路下的实际 $V$-$I$ 特性曲线。

\begin{figure}[H]\centering
\begin{subfigure}[b]{0.5\columnwidth}\centering
    \includegraphics[height=175pt]{assets/Diodes/1N4007 [GOODWORK (固得沃克)]/1N4007 Voltage-Current Characteristics.pdf}
    \caption{Voltage-Current Characteristics}
\end{subfigure}\hfill
\begin{subfigure}[b]{0.5\columnwidth}\centering
    \includegraphics[height=175pt]{assets/Diodes/1N4007 [GOODWORK (固得沃克)]/1N4007 Resistance Characteristics.pdf}
    \caption{Resistance Characteristics}
\end{subfigure}
\caption{Characteristics of 1N4007}
\end{figure}





\section{Voltage regulator diode (稳压二极管)}
\section{LED (Light Emitting Diode) (发光二极管)}
\subsection{XL-302 [XINGLIGHT (成兴光)]}
XL-302SURD (red)、XL-302SURC (white)、XL-302UYD (yellow)、XL-302UYC (green)、是由 XINGLIGHT (成兴光, 中国广东) 生产的 LED,可以在 \href{https://so.szlcsc.com/global.html?k=XL-302U}{立创商城} 或 \href{http://www.xinglight.cn/index.php?s=cpzx&c=search&keyword=XL-302}{XINGLIGHT 官网} 上找到它们。其实物图如下:

\subsection{LED [某宝]}
在某宝中购买大小和外观与 XL-302 相同的 LED,其电气特性如下:
\begin{figure}[H]\centering
    \includegraphics[width=0.9\columnwidth]{assets/Diodes/LED (Taobao)/Voltage-Current Characteristics.pdf}
    \caption{Voltage-Current Characteristics}
\end{figure}
\begin{figure}[H]\centering
    \includegraphics[width=0.9\columnwidth]{assets/Diodes/LED (Taobao)/Equivalent Conductance Characteristics.pdf}
    \caption{Equivalent Conductance Characteristics}
\end{figure}
由图可以看到,黄绿红三色导通电压 $V_\text{th}$ 和 20 mA 工作压降 $V_\text{D}$ 分别为:
\begin{equation}
\boldmath
V_\text{th} = 1.8 \ \mathrm{V},\quad V_\text{D} = 2.3 \ \mathrm{V} \quad (\text{Yellow, Green, Red})
\end{equation}
而蓝色和白色 LED 的导通电压 $V_\text{th}$ 和 20 mA 工作压降 $V_\text{D}$ 分别为:
\begin{equation}
\boldmath
V_\text{th} = 2.5 \ \mathrm{V},\quad V_\text{D} = 3.2 \ \mathrm{V} \quad (\text{Blue, White})
\end{equation}

\section{Schottky Diode (肖特基二极管)}

\chapter{Transistors (晶体管)}\thispagestyle{fancy}
\section{BTJ (Bipolar Transistor Junction) (双极型晶体管)}
\section{JFET (Junction Field Effect Transistor) (结型场效应晶体管)}
\section{MOSFET (Metal-Oxide -Semiconductor Field Effect Transistor) (金属氧化物半导体场效应晶体管)}


\subsection{2N7000 [onsemi (安森美)]}

2N7000 [onsemi (安森美)] 是由 onsemi 公司生产的 N-Channel Enhancement Mode Field Effect Transistor (N-Channel Enhancement MOSFET),可以在 \href{https://item.szlcsc.com/232636.html}{立创商城} 或  \href{https://www.GOODWORK.com/products/discrete-power-modules/mosfets/small-signal-mosfets/1N4007}{GOODWORK 官网} 上找到它。其实物图如下:

\begin{figure}[H]\centering
    \begin{subfigure}[b]{0.5\columnwidth}\centering
        \includegraphics[height=160pt]{assets/Transistors/2N7000 [onsemi (安森美)]/2N7000 正面.png}
        \caption{The front side}
    \end{subfigure}\hfill
    \begin{subfigure}[b]{0.5\columnwidth}\centering
        \includegraphics[height=160pt]{assets/Transistors/2N7000 [onsemi (安森美)]/2N7000 背面.png}
        \caption{The back side}
    \end{subfigure}
    \caption{2N7000 [onsemi (安森美)]}
\end{figure}


用经典反相器结构,取合适的电阻 $R_L$ 用作电流表(我们这里取 $R_L = 5.2 \Omega$),固定 $V_\text{GS}$,同时改变 $V_\text{DS}$,可以得到 MOS 的工作特性 (Operation Characteristics) $I_\text{DS} = I_\text{DS}(V_\text{DS})$,这包括了导通特性 (On-Region Characteristics) 和体二极管特性 (Body Diode Characteristics)。
\begin{figure}[H]\centering
    \includegraphics[width=0.9\columnwidth]{assets/Transistors/2N7000 [onsemi (安森美)]/2N7000 Operation Characteristics.pdf}
    \caption{Operation Characteristics of 2N7000}
\end{figure}
依据实际测得的数据,可以计算出 MOS 的等效电导特性 (Equivalent Conductance Characteristics),如下图所示:
\begin{figure}[H]\centering
    \includegraphics[width=0.9\columnwidth]{assets/Transistors/2N7000 [onsemi (安森美)]/2N7000 Cunductance Characteristics.pdf}
    \caption{Equivalent Conductance Characteristics of 2N7000}
\end{figure}
\begin{figure}[H]\centering
    \includegraphics[width=0.9\columnwidth]{assets/Transistors/2N7000 [onsemi (安森美)]/2N7000 Cunductance Characteristics (filtered).pdf}
    \caption{Filtered Equivalent Conductance Characteristics of 2N7000}
\end{figure}

固定 $V_{\text{DS}}$,改变 $V_\text{GS}$,串联万用表以测量 $I_\text{DS}$,得到 2N7000 的转移特性曲线 $I_\text{DS} = I_\text{DS}(V_\text{GS})$ 如下:
\begin{figure}[H]\centering
    \includegraphics[width=0.9\columnwidth]{assets/Transistors/2N7000 [onsemi (安森美)]/2N7000 Transfer Characteristics.pdf}
    \caption{Transfer Characteristics of 2N7000}
\end{figure}
上图中 $V_\text{DS} = 15$ V 时仅测到 $V_\text{GS} = 2.7$ V,这是因为往后 MOS 管发热严重,温度不断上升,无法得到稳定示数。用 $V_\text{DS} = 15$ V 时测得的数据拟合饱和电流公式 $I_\text{DS} = \frac{K}{2} \left( V_\text{GS} - V_\text{T} \right)^2$,其中 $K$ 和 $V_\text{T}$ 是待定参量,得到:
\begin{equation}
\boldmath V_\text{T} = 2.109 \ \mathrm{V},\quad K = 0.182 \ \mathrm{A\cdot V^{-2}} = 182 \ \mathrm{mA\cdot V^{-2}}
\end{equation}

\begin{figure}[H]\centering
    \includegraphics[width=0.9\columnwidth]{assets/Transistors/2N7000 [onsemi (安森美)]/2N7000 拟合 K 值与 V_T.png}
    \caption{Threshold Voltage of 2N7000}
\end{figure}

\section{MODFET (Modulation Doped Field Effect Transistor) (调制掺杂场效应晶体管) }
\section{Darlington Transistors (达林顿晶体管)}

\chapter{Amplifiers}\thispagestyle{fancy}
\section{Common Emitter Amplifier (共射极放大器)}
\section{Common Base Amplifier (共基极放大器)}
\section{Common Collector Amplifier (共集极放大器)}
\section{Common Source Amplifier (共源极放大器)}
\section{MOSFET Amplifier (MOSFET 放大器)}

\subsection{LM258P}
\subsection{NE5532P}
\subsection{LM318N}

\section{Class AB Amplifier (AB 类放大器)}

\chapter{}\thispagestyle{fancy}



















\end{document}



% VScode 常用快捷键:

% F2:                       变量重命名
% Ctrl + Enter:             行中换行
% Alt + up/down:            上下移行
% 鼠标中键 + 移动:           快速多光标
% Shift + Alt + up/down:    上下复制
% Ctrl + left/right:        左右跳单词
% Ctrl + Backspace/Delete:  左右删单词    
% Shift + Delete:           删除此行
% Ctrl + J:                 打开 VScode 下栏(输出栏)
% Ctrl + B:                 打开 VScode 左栏(目录栏)
% Ctrl + `:                 打开 VScode 终端栏
% Ctrl + 0:                 定位文件
% Ctrl + Tab:               切换已打开的文件(切标签)
% Ctrl + Shift + P:         打开全局命令(设置)

% Latex 常用快捷键

% Ctrl + Alt + J:           由代码定位到PDF
% 


% Git提交规范:
% update: Linear Algebra 2 notes
% add: Linear Algebra 2 notes
% import: Linear Algebra 2 notes
% delete: Linear Algebra 2 notes