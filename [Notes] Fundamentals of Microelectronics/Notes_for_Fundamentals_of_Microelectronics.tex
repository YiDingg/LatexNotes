% 设定文章类型与编码格式
% 主文件和 input 的文件都不能含空格或中文

\documentclass[UTF8]{report}		
\input{D:/a_RemoteRepo/GH.LatexNotes/.config/ConfigForNotes_TwoColumns.tex}  % 引用导言区设置




% ------------------------ 文章信息区 ------------------------ %
% ------------------------ 文章信息区 ------------------------ %

\lhead{\small
\href{https://github.com/YiDingg/LatexNotes}{\footnotesize \color{black}\faGithub\ https://github.com/YiDingg/LatexNotes}
}
\chead{Notes for \textit{Fundamentals of Microelectronics (Razavi)}}
\rhead{\small dingyi233@mails.ucas.ac.cn}


% ------------------------ 文章信息区 ------------------------ %
% ------------------------ 文章信息区 ------------------------ %


% 开始编辑文章
\begin{document} 
\zihao{5}           % 设置全文字号大小


% ------------------------ 封面序言与目录 ------------------------ %
% >> --------------------- 封面序言与目录 --------------------- << %
% 封面
%\maketitle
\begin{titlepage}
\begin{center}
\topskip0pt % 创建垂直居中环境
\vspace*{\fill}



% 标题
{\huge 
    Notes for \textit{Fundamentals of Microelectronics \\ (Razavi) (2nd edition, 2014)}
    \\ ~ \\
    《微电子基础》笔记
}
\\\vspace*{1.2cm}
% 作者
{\large Yi Ding \\ \small\textit{(University of Chinese Academy of Sciences, Beijing 100049, China)} \\ ~ \\ 丁毅\\ \small {\kaishu (中国科学院大学,北京 100049)} }
\\\vspace*{0.8cm}
% 日期
{\small
2024.11 -- ...
}
\end{center}
% 封面图
\vspace*{1cm}
\begin{figure}[H]\centering
    %\includegraphics[width=0.75\columnwidth]{[Notes] Design of Analog CMOS Integrated Circuits/assets/Chapter 1/(Page-25) Razavi CMOS (2nd edition, 2015).pdf}
\end{figure}
\vspace*{5mm}
\begin{figure}[H]\centering
    %\includegraphics[width=0.75\columnwidth]{[Notes] Design of Analog CMOS Integrated Circuits/assets/Chapter 2/2.1/(Page-27) Razavi CMOS (2nd edition, 2015).pdf}
\end{figure}


\vspace*{\fill}
%
\end{titlepage}
    \newpage  
    \pagenumbering{Roman} % 页码为大写罗马数字
    \thispagestyle{fancy}   % 显示页码、页眉等

% 序言
\begin{enabstract} \normalsize
    to be completed
\end{enabstract}
\addcontentsline{toc}{chapter}{Preface} % 手动添加为目录




\newpage
\begin{cnabstract}\normalsize 
    待完成
\end{cnabstract}
\addcontentsline{toc}{chapter}{序言} % 手动添加为目录



% 目录
    \setcounter{tocdepth}{2}    % 目录深度 (为 1 时显示到 section)
    % 目录页
        \tableofcontents
        \addcontentsline{toc}{chapter}{Contents}
        \thispagestyle{fancy}                   % 显示页码、页眉等
    % 插图页
        %\cleardoublepage\listoffigures
        %\addcontentsline{toc}{chapter}{插图}
        %\thispagestyle{fancy}                   % 显示页码、页眉等
    % 表格页
        %\cleardoublepage\listoftables
        %\addcontentsline{toc}{chapter}{表格}
        %\thispagestyle{fancy}                   % 显示页码、页眉等 

% 收尾工作
    \newpage    
    \pagenumbering{arabic} 


% >> --------------------- 封面序言与目录 --------------------- << %
% ------------------------ 封面序言与目录 ------------------------ %

% >> --------------------- 下面是正文 --------------------- << %
% ------------------------ 下面是正文 ------------------------ %




\rhead{\nouppercase{\rightmark}}  % \rightmark 是 section 标题
\subfile{contents/Chapter 1 Intro/Notes for Microelectronics (Chapter 1).tex}
\subfile{contents/Chapter 2 Semiconductors/Notes for Microelectronics (Chapter 2).tex}
\subfile{contents/Chapter 3 Diode Circuits/Notes for Microelectronics (Chapter 3).tex}
\subfile{contents/Chapter 4 Physics of BJT/Notes for Microelectronics (Chapter 4).tex}
\subfile{contents/Chapter 5 Bipolar Amplifiers/Notes for Microelectronics (Chapter 5).tex}
\subfile{contents/Chapter 6 MOS Transistors/Notes for Microelectronics (Chapter 6).tex}
\subfile{contents/Chapter 7 CMOS Amplifiers/Notes for Microelectronics (Chapter 7).tex}
\subfile{contents/Chapter 8 Op Amp/Notes for Microelectronics (Chapter 8).tex}
\subfile{contents/Chapter 9 Cascode and Mirrors/Notes for Microelectronics (Chapter 9).tex}
\subfile{contents/Chapter 10 Differential Amp/Notes for Microelectronics (Chapter 10).tex}
\subfile{contents/Chapter 11 Frequency Response/Notes for Microelectronics (Chapter 11).tex}
\subfile{contents/Chapter 12 Feedback/Notes for Microelectronics (Chapter 12).tex}
\subfile{contents/Chapter 13 Oscillators/Notes for Microelectronics (Chapter 13).tex}
\subfile{contents/Chapter 14 Output and Power Amp/Notes for Microelectronics (Chapter 14).tex}
\subfile{contents/Chapter 15 Analog Filters/Notes for Microelectronics (Chapter 15).tex}
\subfile{contents/Chapter 16 Digital CMOS Circuits/Notes for Microelectronics (Chapter 16).tex}


\renewcommand{\bibname}{Reference}  % 将参考文献标题从 bibliography 改为 Reference
\nocite{*}
\bibliography{bibtex}
\thispagestyle{fancy} 
\addcontentsline{toc}{chapter}{Reference}


\end{document}



% VScode 常用快捷键:

% F2:                       变量重命名
% Ctrl + Enter:             行中换行
% Alt + up/down:            上下移行
% 鼠标中键 + 移动:           快速多光标
% Shift + Alt + up/down:    上下复制
% Ctrl + left/right:        左右跳单词
% Ctrl + Backspace/Delete:  左右删单词    
% Shift + Delete:           删除此行
% Ctrl + J:                 打开 VScode 下栏(输出栏)
% Ctrl + B:                 打开 VScode 左栏(目录栏)
% Ctrl + `:                 打开 VScode 终端栏
% Ctrl + 0:                 定位文件
% Ctrl + Tab:               切换已打开的文件(切标签)
% Ctrl + Shift + P:         打开全局命令(设置)

% Latex 常用快捷键

% Ctrl + Alt + J:           由代码定位到PDF
% 


% Git提交规范:
% update: Linear Algebra 2 notes
% add: Linear Algebra 2 notes
% import: Linear Algebra 2 notes
% delete: Linear Algebra 2 notes