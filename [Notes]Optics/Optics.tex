% 若编译失败,且生成 .synctex(busy) 辅助文件,可能有两个原因:
% 1. 需要插入的图片不存在:Ctrl + F 搜索 'figure' 将这些代码注释/删除掉即可
% 2. 路径/文件名含中文或空格:更改路径/文件名即可

% --------------------- 文章宏包及相关设置 --------------------- %
% >> ------------------ 文章宏包及相关设置 ------------------ << %
% 设定文章类型与编码格式
\documentclass[UTF8]{report}		
\usepackage{media9}
%\usepackage{multimedia}
% 本文特殊宏包
\usepackage{siunitx} % 埃米单位

% 本文特殊宏定义
\def\Re{\mathrm{\,Re}\,}
\def\Im{\mathrm{\,Im}\,}
\def\sinc{\mathrm{\,sinc}\,}
\def\Jc{\mathrm{\,Jc}}

% 自定义宏定义
\def\N{\mathbb{N}}
\def\F{\mathbb{F}}
\def\Z{\mathbb{Z}}
\def\Q{\mathbb{Q}}
\def\R{\mathbb{R}}
\def\C{\mathbb{C}}
\def\T{\mathbb{T}}
\def\S{\mathbb{S}}
\def\A{\mathbb{A}}
\def\I{\mathscr{I}}
\def\Arg{\mathrm{\,Arg\,}}
\def\d{\mathrm{d}}
\def\p{\partial}


% 导入基本宏包
\usepackage[UTF8]{ctex}     % 设置文档为中文语言
\usepackage{hyperref}  % 宏包:自动生成超链接 (此宏包与标题中的数学环境冲突)
\hypersetup{
    colorlinks=true,    % false:边框链接 ; true:彩色链接
    citecolor={blue},    % 文献引用颜色
    linkcolor={blue},   % 目录 (我们在目录处单独设置),公式,图表,脚注等内部链接颜色
    urlcolor={magenta},    % 网页 URL 链接颜色,包括 \href 中的 text
    % magenta 洋红色
    % cyan 浅蓝色 
    % magenta 洋红色
    % yellow 黄色
    % black 黑色
    % white 白色
    % red 红色
    % green 绿色
    % blue 蓝色
    % gray 灰色
    % darkgray 深灰色
    % lightgray 浅灰色
    % brown 棕色
    % lime 石灰色
    % olive 橄榄色
    % orange 橙色
    % pink 粉红色
    % purple 紫色
    % teal 蓝绿色
    % violet 紫罗兰色
}

% \usepackage{docmute}    % 宏包:子文件导入时自动去除导言区,用于主/子文件的写作方式,\include{./51单片机笔记}即可。注:启用此宏包会导致.tex文件capacity受限。
\usepackage{amsmath}    % 宏包:数学公式
\usepackage{mathrsfs}   % 宏包:提供更多数学符号
\usepackage{amssymb}    % 宏包:提供更多数学符号
\usepackage{pifont}     % 宏包:提供了特殊符号和字体
\usepackage{extarrows}  % 宏包:更多箭头符号
\usepackage{multicol}   % 宏包:支持多栏 


% 文章页面margin设置
\usepackage[a4paper]{geometry}
    \geometry{top=1in}  % 1 inch= 2.46 cm, 0.75 inch = 1.85 cm
    \geometry{bottom=1in}
    \geometry{left=0.75in}
    \geometry{right=0.75in}   % 设置上下左右页边距
    \geometry{marginparwidth=1.75cm}    % 设置边注距离(注释、标记等)

% 配置数学环境
\usepackage{amsthm} % 宏包:数学环境配置
% theorem-line 环境自定义
    \newtheoremstyle{MyLineTheoremStyle}% <name>
        {11pt}% <space above>
        {11pt}% <space below>
        {\kaishu}% <body font> 使用默认正文字体
        {}% <indent amount>
        {\bfseries}% <theorem head font> 设置标题项为加粗
        {:}% <punctuation after theorem head>
        {.5em}% <space after theorem head>
        {\textbf{#1}\thmnumber{#2}\ \ (\,\textbf{#3}\,)}% 设置标题内容顺序
    \theoremstyle{MyLineTheoremStyle} % 应用自定义的定理样式
    \newtheorem{LineTheorem}{Theorem.\,}
% theorem-block 环境自定义
    \newtheoremstyle{MyBlockTheoremStyle}% <name>
        {11pt}% <space above>
        {11pt}% <space below>
        {\kaishu}% <body font> 使用默认正文字体
        {}% <indent amount>
        {\bfseries}% <theorem head font> 设置标题项为加粗
        {:\\ \indent}% <punctuation after theorem head>
        {.5em}% <space after theorem head>
        {\textbf{#1}\thmnumber{#2}\ \ (\,\textbf{#3}\,)}% 设置标题内容顺序
    \theoremstyle{MyBlockTheoremStyle} % 应用自定义的定理样式
    \newtheorem{BlockTheorem}[LineTheorem]{Theorem.\,} % 使用 LineTheorem 的计数器
% definition 环境自定义
    \newtheoremstyle{MySubsubsectionStyle}% <name>
        {11pt}% <space above>
        {11pt}% <space below>
        {}% <body font> 使用默认正文字体
        {}% <indent amount>
        {\bfseries}% <theorem head font> 设置标题项为加粗
        {:\\ \indent}% <punctuation after theorem head>
        {0pt}% <space after theorem head>
        {\textbf{#3}}% 设置标题内容顺序
    \theoremstyle{MySubsubsectionStyle} % 应用自定义的定理样式
    \newtheorem{definition}{}

%宏包:有色文本框(proof环境)及其设置
\usepackage[dvipsnames,svgnames]{xcolor}    %设置插入的文本框颜色
\usepackage[strict]{changepage}     % 提供一个 adjustwidth 环境
\usepackage{framed}     % 实现方框效果
    \definecolor{graybox_color}{rgb}{0.95,0.95,0.96} % 文本框颜色。修改此行中的 rgb 数值即可改变方框纹颜色,具体颜色的rgb数值可以在网站https://colordrop.io/ 中获得。(截止目前的尝试还没有成功过,感觉单位不一样)(找到喜欢的颜色,点击下方的小眼睛,找到rgb值,复制修改即可)
    \newenvironment{graybox}{%
    \def\FrameCommand{%
    \hspace{1pt}%
    {\color{gray}\small \vrule width 2pt}%
    {\color{graybox_color}\vrule width 4pt}%
    \colorbox{graybox_color}%
    }%
    \MakeFramed{\advance\hsize-\width\FrameRestore}%
    \noindent\hspace{-4.55pt}% disable indenting first paragraph
    \begin{adjustwidth}{}{7pt}%
    \vspace{2pt}\vspace{2pt}%
    }
    {%
    \vspace{2pt}\end{adjustwidth}\endMakeFramed%
    }

% 外源代码插入设置
% matlab 代码插入设置
\usepackage{matlab-prettifier}
    \lstset{style=Matlab-editor}    % 继承 matlab 代码高亮 , 此行不能删去
\usepackage[most]{tcolorbox} % 引入tcolorbox包 
\usepackage{listings} % 引入listings包
    \tcbuselibrary{listings, skins, breakable}
    \newfontfamily\codefont{Consolas} % 定义需要的 codefont 字体
    \lstdefinestyle{MatlabStyle_inc}{   % 插入代码的样式
        language=Matlab,
        basicstyle=\small\ttfamily\codefont,    % ttfamily 确保等宽 
        breakatwhitespace=false,
        breaklines=true,
        captionpos=b,
        keepspaces=true,
        numbers=left,
        numbersep=15pt,
        showspaces=false,
        showstringspaces=false,
        showtabs=false,
        tabsize=2,
        xleftmargin=15pt,   % 左边距
        %frame=single, % single 为包围式单线框
        frame=shadowbox,    % shadowbox 为带阴影包围式单线框效果
        %escapeinside=``,   % 允许在代码块中使用 LaTeX 命令 (此行无用)
        %frameround=tttt,    % tttt 表示四个角都是圆角
        framextopmargin=0pt,    % 边框上边距
        framexbottommargin=0pt, % 边框下边距
        framexleftmargin=5pt,   % 边框左边距
        framexrightmargin=5pt,  % 边框右边距
        rulesepcolor=\color{red!20!green!20!blue!20}, % 阴影框颜色设置
        %backgroundcolor=\color{blue!10}, % 背景颜色
    }
    \lstdefinestyle{MatlabStyle_src}{   % 插入代码的样式
        language=Matlab,
        basicstyle=\small\ttfamily\codefont,    % ttfamily 确保等宽 
        breakatwhitespace=false,
        breaklines=true,
        captionpos=b,
        keepspaces=true,
        numbers=left,
        numbersep=15pt,
        showspaces=false,
        showstringspaces=false,
        showtabs=false,
        tabsize=2,
    }
    \newtcblisting{matlablisting}{
        %arc=2pt,        % 圆角半径
        % 调整代码在 listing 中的位置以和引入文件时的格式相同
        top=0pt,
        bottom=0pt,
        left=-5pt,
        right=-5pt,
        listing only,   % 此句不能删去
        listing style=MatlabStyle_src,
        breakable,
        colback=white,   % 选一个合适的颜色
        colframe=black!0,   % 感叹号后跟不透明度 (为 0 时完全透明)
    }
    \lstset{
        style=MatlabStyle_inc,
    }

% table 支持
\usepackage{booktabs}   % 宏包:三线表
\usepackage{tabularray} % 宏包:表格排版
\usepackage{longtable}  % 宏包:长表格
\usepackage{diagbox}    % 宏包:表头斜线

% figure 设置
\usepackage{graphicx}  % 支持 jpg, png, eps, pdf 图片 
\usepackage{svg}       % 支持 svg 图片
    \svgsetup{
        % 指向 inkscape.exe 的路径
        %inkscapeexe = C:/aa_MySame/inkscape/bin/inkscape.exe, 
        inkscapeexe = C:/aa_MySame/inkscape/bin/inkscape.exe,  
        % 一定程度上修复导入后图片文字溢出几何图形的问题
        inkscapelatex = false                 
    }

% 图表进阶设置
\usepackage{caption}    % 图注、表注
    \captionsetup[figure]{name=图}  
    \captionsetup[table]{name=表}
    \captionsetup{
        labelfont=bf, % 设置标签为粗体
        textfont=bf,  % 设置文本为粗体
        font=small  
    }
\usepackage{float}     % 图表位置浮动设置 
\usepackage{subcaption} % subfigure 子图支持
\usepackage{etoolbox} % 用于保证图注表注的数学字符为粗体
    \AtBeginEnvironment{figure}{\boldmath} % 图注中的数学字符为粗体
    \AtBeginEnvironment{table}{\boldmath}  % 表注中的数学字符为粗体
    \AtBeginEnvironment{tabular}{\unboldmath}   % 保证表格中的数学字符不受额外影响

% 圆圈序号自定义
\newcommand*\circled[1]{\tikz[baseline=(char.base)]{\node[shape=circle,draw,inner sep=0.8pt, line width = 0.03em] (char) {\small \bfseries #1};}}   % TikZ solution

% 列表设置
\usepackage{enumitem}   % 宏包:列表环境设置
    \setlist[enumerate]{
        label=(\arabic*) ,   % 设置序号样式为 (1) (2) (3)
        ref=\arabic*, % 如果需要引用列表项,这将决定引用格式(这里仍然使用数字)
        itemsep=0pt, parsep=0pt, topsep=0pt, partopsep=0pt, leftmargin=3.5em} 
    \setlist[itemize]{itemsep=0pt, parsep=0pt, topsep=0pt, partopsep=0pt, leftmargin=3.5em}
    \newlist{circledenum}{enumerate}{1} % 创建一个新的枚举环境  
    \setlist[circledenum,1]{  
        label=\protect\circled{\arabic*}, % 使用 \arabic* 来获取当前枚举计数器的值,并用 \circled 包装它  
        ref=\arabic*, % 如果需要引用列表项,这将决定引用格式(这里仍然使用数字)
        itemsep=0pt, parsep=0pt, topsep=0pt, partopsep=0pt, leftmargin=3.5em
    }  

% 其它设置
% 脚注设置
    \usepackage[perpage]{footmisc} % 每页脚注重新计数
    \renewcommand\thefootnote{\ding{\numexpr171+\value{footnote}}}
% 参考文献引用设置
    \bibliographystyle{unsrt}   % 设置参考文献引用格式为unsrt
    \newcommand{\upcite}[1]{\textsuperscript{\cite{#1}}}     % 自定义上角标式引用
% 文章序言设置
    \newcommand{\cnabstractname}{序言}
    \newenvironment{cnabstract}{%
        \par\Large
        \noindent\mbox{}\hfill{\bfseries \cnabstractname}\hfill\mbox{}\par
        \vskip 2.5ex
        }{\par\vskip 2.5ex}

% 文章默认字体设置
\usepackage{fontspec}   % 宏包:字体设置
    \setmainfont{SimSun}    % 设置中文字体为宋体字体
    \setCJKmainfont[AutoFakeBold=3]{SimSun} % 设置加粗字体为 SimSun 族,AutoFakeBold 可以调整字体粗细
    \setmainfont{Times New Roman} % 设置英文字体为 Times New Roman

% 各级标题自定义设置
\usepackage{titlesec}   
    % chapter 标题自定义设置
    \titleformat{\chapter}[hang]{\normalfont\huge\bfseries\centering}{第\,\thechapter\,章}{20pt}{}
    \titlespacing*{\chapter}{0pt}{-20pt}{20pt} % 控制上方空白的大小
    % section 标题自定义设置 
    \titleformat{\section}[hang]{\normalfont\Large\bfseries}{§\,\thesection\,}{8pt}{}
% subsubsection 标题自定义设置
%\titleformat{\subsubsection}[hang]{\normalfont\bfseries}{}{8pt}{}

% --------------------- 文章宏包及相关设置 --------------------- %
% >> ------------------ 文章宏包及相关设置 ------------------ << %

% ------------------------ 文章信息区 ------------------------ %
% ------------------------ 文章信息区 ------------------------ %
% 页眉页脚设置
\usepackage{fancyhdr}   %宏包:页眉页脚设置
    \pagestyle{fancy}
    \fancyhf{}
    \cfoot{\thepage}
    \renewcommand\headrulewidth{1pt}
    \renewcommand\footrulewidth{0pt}
    \usepackage{fontawesome}    % 宏包:更多符号与图标 (用于插入 GitHub 图标等)
    \lhead{\small
    \href{https://github.com/YiDingg/LatexNotes}{\color{black}\faGithub\ https://github.com/YiDingg/LatexNotes}
    } 
    \chead{Optics Notes}    
    \rhead{\small dingyi233@mails.ucas.ac.cn}
%文档信息设置
\title{光学笔记\\Optics Notes}
\author{丁毅\\ \footnotesize 中国科学院大学,北京 100049\\ Yi Ding \\ \footnotesize University of Chinese Academy of Sciences, Beijing 100049, China}
\date{\footnotesize 2024.8 -- 2025.1}
% ------------------------ 文章信息区 ------------------------ %
% ------------------------ 文章信息区 ------------------------ %

% 开始编辑文章

\begin{document} 
\zihao{5}             % 设置全文字号大小, -4 为小四, 5 为五号

% ------------------------ 封面序言与目录 ------------------------ %
% >> --------------------- 封面序言与目录 --------------------- << %
% 封面
\maketitle\newpage  
\pagenumbering{Roman} % 页码为大写罗马数字
\thispagestyle{fancy}   % 显示页码、页眉等

% 目录
\setcounter{tocdepth}{2}                % 目录深度(chapter 下 为 1 时显示到section)
\tableofcontents                        % 目录页
\addcontentsline{toc}{chapter}{目录}    % 手动添加此页为目录
\thispagestyle{fancy}                   % 显示页码、页眉等 

% 收尾工作
\newpage    
\pagenumbering{arabic} 

% >> --------------------- 封面序言与目录 --------------------- << %
% ------------------------ 封面序言与目录 ------------------------ %



%%%%%%%%%%%%%%%%%%%%%%%%%%%%%%%
% 下面是应直接复制到总文件的内容 %
%%%%%%%%%%%%%%%%%%%%%%%%%%%%%%%

\chapter{偏振}\thispagestyle{fancy}
在这一章,我们将要讨论光会以什么样的状态(即偏振态)进行传播、合成,如何观察、产生和改变光的偏振态,以及如何利用它。相比于干涉和衍射两章,本章内容较为简短,因为许多介绍性的东西都没有放在这里(抄教材不符合我们的初衷),读者若感兴趣,可阅读文献 \cite{Optics} 的 Page 425-497。

\section{偏振光的性质}
偏振光可粗略地分为自然光(非偏振光),部分偏振光,(椭)圆偏振和线偏振。其中,线偏振态又记为 $\mathscr{P}$ 态,左圆和右圆分别记为 $\mathscr{L}$ 和 $\mathscr{R}$ 态。
\subsection{椭圆偏振光}
由于 $\boldsymbol{E}$ 的矢量性,所有偏振光的电矢量 $\boldsymbol{E}$ 都可以分解为两个互相垂直(正交)的光扰动:
\begin{equation}
\boldsymbol{E_x} =  E_{0x} \cos(kz - \omega t) \cdot \hat{i},\quad \boldsymbol{E_y} =  E_{0y} \cos(kz - \omega t + \varepsilon) \cdot \hat{j}
\end{equation}
其中 $\hat{i}$ 和 $\hat{j}$ 表示分解的方向,$E_{0x}$ 和 $E_{0y}$ 是分量的振幅(可能是时间的函数),$\varepsilon$ 为相位差(可能是时间的函数)。对于 $E_{0x}$、$E_{0y}$ 和 $\varepsilon$ 都为常量的情况,合成的光多为椭圆偏振光:
\begin{gather}
    E_y = E_{0y} \cos(kz - \omega t + \varepsilon) \Longrightarrow 
    \frac{E_y}{E_{0y}} = 
    \left[\cos(kz - \omega t) \cos \varepsilon - \sin(kz - \omega t) \sin \varepsilon\right] 
\end{gather}
由 $\sin^2 (kz - \omega t) = 1 - \cos^2 (kz - \omega t) = 1 - \frac{E_x^2}{E_{0x}^2}$,可以得到:
\begin{gather}
\left( \frac{E_y}{E_{0y}} -  \cos \varepsilon\frac{E_x}{E_{0x}} \right)^2 = \sin^2 \varepsilon \left( 1 - \frac{E_x^2}{E_{0x}^2} \right) \\ 
\Longrightarrow 
\frac{E_x^2}{E_{0x}^2} + \frac{E_y^2}{E_{0y}^2} - 2 \frac{E_xE_y}{E_{0x}E_{0y}} \cos \varepsilon = \sin^2 \varepsilon 
\\ \label{eq:椭圆偏振光}
\Longleftrightarrow \frac{E_x^2}{(E_{0x} \sin \varepsilon)^2} + \frac{E_y^2}{(E_{0y} \sin \varepsilon)^2} - 2 \frac{E_xE_y}{(E_{0x} \sin \varepsilon )(E_{0y} \sin \varepsilon)} = 1
\end{gather}
由高中的知识可知,当 $\varepsilon \ne k \pi,\  k \in \Z$ 时,这是一个椭圆(或圆)方程,因为 $\Delta = B^2 - 4AC = 4 \cos^2 \varepsilon - 4 < 0$。这样的的偏振光称为椭圆偏振光。

椭圆偏振光也分左旋和右旋,这是因为从观察点向光源看去时(光指向“眼睛”),若 $\varepsilon \in (0, \pi)$,我们“看到”的由 $E_x$ 和 $E_y$ 合成后的 $\boldsymbol{E}$ 在逆时针旋转(左旋)。一个典型的例子是 $\varepsilon = \frac{\pi}{2}$ 时,也即 $E_x$ “领先” $E_y$ 相位 $\frac{\pi}{2}$ :
\begin{gather}
\boldsymbol{E_x} = E_{0x} \cos (kz - \omega t) = E_{0x} \cos (\omega t - kz) \cdot \hat{i},\quad \\ 
\boldsymbol{E_y} = E_{0y} \cos (kz - \omega t + \frac{\pi}{2}) 
= E_{0y} \sin (\omega t - kz) \cdot \hat{j}
\end{gather}
由椭圆的参数方程知道,$\boldsymbol{E} = \boldsymbol{E_x} + \boldsymbol{E_y}$ 的极角 $\theta = (\omega t - kz)$ 随时间 $t$ 增大,$\boldsymbol{E}$ 逆时针旋转,称为左圆光。相反,当 $\varepsilon \in (\pi, 2\pi)$ 时(也可以说是 $(-\pi, 0)$),$\boldsymbol{E}$ 顺时针旋转(右旋),称为右圆光。

随着 $\varepsilon$ 不同,$\boldsymbol{E}$ 的形状和方向也不同,但总的来讲,椭圆主轴夹角满足:
\begin{gather}
\tan \left(2 \alpha\right) = 2 \cos \varepsilon \cdot \frac{E_{0x}E_{0y}}{E_{0x}^2 - E_{0y}^2} \\ 
\Longrightarrow \alpha = \frac{1}{2} \arctan \left(2 \cos \varepsilon \cdot \frac{E_{0x}E_{0y}}{E_{0x}^2 - E_{0y}^2}\right)
\end{gather}
主轴,是指长轴,也即 $2E_{0x}$ 和 $2E_{0y}$ 中更长的轴{\color{red} 与对应的 $x$ 或 $y$ 轴夹角}。举个例子,当 $E_{0x} > E_{0y}$ 时,$x$ 为主轴(长轴),$\alpha$ 为长轴与 $x$ 轴的夹角,此时 $\alpha$ 随 $\varepsilon$ 的变化如图 \ref{fig:alpha} (a) 所示;当 $E_{0x} < E_{0y}$ 时,情况则相反,$\alpha$ 是长轴与 $y$ 轴的夹角,如图 \ref{fig:alpha} (b) 所示。

从图中可以看出,随着 $\varepsilon$ 不断变化,椭圆会在主轴附近“摆动”,而不是转动。
\begin{figure}[H]\centering
\begin{subfigure}[b]{0.5\columnwidth}\centering
    \includegraphics[height=175pt]{assets/5/5.1 长轴为 x.pdf}
    \caption{$E_{0x} > E_{0y}$ 时 $x$ 为主轴}
\end{subfigure}\hfill
\begin{subfigure}[b]{0.5\columnwidth}\centering
    \includegraphics[height=175pt]{assets/5/5.1 长轴为 y.pdf}
    \caption{$E_{0x} < E_{0y}$ 时 $y$ 为主轴}
\end{subfigure}
\caption{$\alpha$ 随 $\varepsilon$ 的变化情况}
\label{fig:alpha}
\end{figure}
\begin{figure}[H]\centering
    \includegraphics[width=0.9\columnwidth]{assets/5/5.1 主轴为 y.png}
    \caption{主轴为 $y$ 时椭圆的“摆动”情况}
\end{figure}
特别地,我们指出,若 $x$ 为主轴($E_{0x} > E_{0y}$),则 $\alpha$ 一定在 $(-45^\circ, 45^\circ)$ 之间,这表明椭圆的长轴更“贴近” $x$ 轴,$y$ 的情况也同理。另外,当 $E_{0x} = E_{0y}$ 时,椭圆退化为圆,不存在 $\alpha$ 的概念,但左旋和右旋仍然存在。

为了方便参考,我们给出平面椭圆的一般公式,设椭圆中心为 $(x_0, y_0)$,长轴与 $x$ 轴夹角为 $\alpha$,则椭圆方程为:
\begin{equation}
\frac{\left[(x - x_0) \cos \alpha + (y - y_0) \sin \alpha \right]^2}{a^2} + \frac{\left[(x - x_0) \sin \alpha + (y - y_0) \cos \alpha  \right]^2}{b^2} = 1
\end{equation}
对比系数,可以得到公式 \ref{eq:椭圆偏振光} 对应椭圆的半长轴 $a$ 和半短轴 $b$ :
\begin{equation}
a^2 = \frac{E_{0x}^2E_{0y}^2 \left(\cos^2 \alpha - \sin^2 \alpha\right)}{E_{0y}^2\cos^2 \alpha  -  E_{0x}^2\sin^2 \alpha},\quad 
b^2 = \frac{E_{0x}^2E_{0y}^2 \left(\cos^2 \alpha - \sin^2 \alpha\right)}{E_{0x}^2\cos^2 \alpha  -  E_{0y}^2\sin^2 \alpha}
\end{equation}

\subsection{线偏振光}
当 $\varepsilon = k \pi,\  k \in \Z$ 时,方程退化为:
\begin{equation}
\left( \frac{E_x}{E_{0x}} \pm \frac{E_y}{E_{0y}} \right)^2 = 0 \Longleftrightarrow \frac{E_x}{E_{0x}} \pm \frac{E_y}{E_{0y}} = 0
\end{equation}
这是一个直线方程,表示线偏振光。

\subsection{自然光}
特别地,对于自然光,我们可以用两个振幅相等、非相干(即相位差 $\varepsilon$ 迅速且无规变化)、正交的线偏振光的合成来表示自然光,这是数学上是一种非常方便的处理。

\section{偏振光的数学表示}

偏振光的状态可以用向量来描述,常见的有斯托克斯(四维)参量和琼斯矢量(二维)。前者可以描述所有偏振光(包括完全和不完全)和非偏振光,但参数较多,后者仅可以描述偏振光,但较为简洁。相应地,偏振光器件对偏振光的作用可以用矩阵来表示,分别对应穆勒矩阵(四维)和琼斯矩阵(二维)。

\subsection{斯托克斯参量}
设想有四个滤波片,它们都只能透过一半(强度)的入射光。第一个是简单各向同性地,允许各个方向的偏振通过;第二个是(通光轴)水平的线偏振器,第三个是(通光轴)$45^\circ$ 的线偏振器;最后一个是右圆起偏器(对 $\mathscr{L}$ 不透明)。

把每个滤光片分别放在要研究的光束的光路上,也即光路上每次只有一个偏振器,测量到的辐照度分别记为 $I_0, I_1, I_2, I_3$,则斯托克斯参量的定义为:
\begin{gather}
\delta_0 = 2I_0,\quad 
\begin{cases}
    \delta_1 = 2I_1 - \delta_0 \\ 
    \delta_2 = 2I_2 - \delta_0 \\ 
    \delta_3 = 2I_3 - \delta_0 
\end{cases}
\end{gather}
$\delta_1, \delta_2, \delta_3$ 直接反映了光束的偏振态。具体而言:
\begin{enumerate}
\item $\delta_1$ 反映光束更接近水平 $\mathscr{P}$ 态 ($\delta_1 \to \delta_0$) 还是竖直 $\mathscr{P}$ 态 ($\delta_1 \to -\delta_0$) ;
\item $\delta_2$ 反映光束更接近 $+45^\circ$ $\mathscr{P}$ 态 ($\delta_2 \to \delta_0$) 还是 $-45^\circ$ $\mathscr{P}$ 态 ($\delta_2 \to -\delta_0$) ;
\item $\delta_3$ 反映光束更接近右旋 ($\delta_3 \to \delta_0$) 还是左旋 ($\delta_3 \to -\delta_0$) 
\end{enumerate}
对于准单色光 $\boldsymbol{E}$,将其分解为 $\boldsymbol{E_x}$ 和 $\boldsymbol{E_y}$,可将斯托克斯参量进一步写为:
\begin{equation}
    \delta_0 = \langle E_{0x}^2 \rangle_T + \langle E_{0y}^2 \rangle_T ,\quad 
    \delta_1 = \langle E_{0x}^2 \rangle_T - \langle E_{0y}^2 \rangle_T \\ 
    \delta_2 = \langle 2 E_{0x}E_{0y} \cos \varepsilon \rangle_T ,\quad 
    \delta_3 = \langle 2 E_{0x}E_{0y} \sin \varepsilon \rangle_T 
\end{equation}
我们在上式中略去了常数 $\frac{\varepsilon_0c}{2}$,因此这些参量现在正比于辐照度。把每个参量都除以 $\delta_0$ 以归一化常常带来很大的方便,此时 $\delta_{k} \in [-1, 1], \ k = 1, 2, 3$。

描述一束光偏振程度的量,称为偏振度 $\mathbf{V}$,定义为:
\begin{equation}
\mathbf{V} = \frac{\sqrt{\delta_1^2 + \delta_2^2 + \delta_3^2}}{\delta_0}
\end{equation}

对两束{\color{red} 不相干}的光 $(\delta'_0, \delta'_1, \delta'_2, \delta'_3)$ 和 $(\delta''_0, \delta''_1, \delta''_2, \delta''_3)$,在斯托克斯参量下,可以直接将它们的偏振态相加,得到合成的光的偏振态为 $(\delta'_0 + \delta''_0, \delta'_1 + \delta''_1, \delta'_2 + \delta''_2, \delta'_3 + \delta''_3)$,用公式表示为:
\begin{gather}
\begin{bmatrix}
    \delta'_0 \\ 
    \delta'_1 \\
    \delta'_2 \\
    \delta'_3
\end{bmatrix}
+
\begin{bmatrix}
    \delta''_0 \\ 
    \delta''_1 \\
    \delta''_2 \\
    \delta''_3
\end{bmatrix}
=
\begin{bmatrix}
    \delta'_0 + \delta''_0 \\ 
    \delta'_1 + \delta''_1 \\
    \delta'_2 + \delta''_2 \\
    \delta'_3 + \delta''_3
\end{bmatrix}
\end{gather}

\subsection{琼斯矢量}

琼斯矢量是直接用 $E_x$ 和 $E_y$ 来表示光的偏振态,它是一个二维复矢量。对一束光 $\boldsymbol{E} = E_{0x}\cos (kz - \omega t +\varphi_x) \cdot \hat{i} + E_{0y}\cos (kz - \omega t +\varphi_y) \cdot \hat{j}$,它的琼斯矢量定义为:
\begin{equation}
\boldsymbol{E} = 
\begin{bmatrix}
    E_{0x}e^{i\varphi_x} \\ 
    E_{0y}e^{i\varphi_y}
\end{bmatrix} = 
E_{0x}e^{i\varphi_x}
\begin{bmatrix}
    1 \\ 
    \frac{E_{0y}}{E_{0x}}e^{i\varepsilon}
\end{bmatrix}
\end{equation}
实际应用中常常不需要知道具体的振幅的相位,只需要知道相对相位差 $\varepsilon = \varphi_y - \varphi_x$ 即可,因此琼斯矢量也常用归一化的方式来表达。下图列出了常见偏振态的斯托克斯和琼斯矢量表示:
\begin{figure}[H]\centering
    \includegraphics[width=0.95\columnwidth]{assets/5/5.1 琼斯矢量和斯托克斯参量.png}
    \caption{常见偏振态的斯托克斯和琼斯矢量表示}
\end{figure}
上图的圆偏振可以轻松的扩展到椭圆,例如一个右旋椭圆的偏振态可表示为 $\boldsymbol{E} = \frac{1}{\sqrt{5}} \begin{bmatrix}
    2 \\ 
    -i
\end{bmatrix} $。

与斯托克斯参量类似,光偏振态的合成也可以直接在琼斯矢量下进行,即直接将两个琼斯矢量相加即可。例如相同振幅的 $\mathscr{R}$ 态和 $\mathscr{L}$ 态可以合成为水平的 $\mathscr{P}$ 态:
\begin{equation}
\boldsymbol{E}_{\mathscr{R}} + \boldsymbol{E}_{\mathscr{L}} =
\frac{1}{\sqrt{2}}
\begin{bmatrix}
    1 \\ 
    -i
\end{bmatrix} 
+
\frac{1}{\sqrt{2}}
\begin{bmatrix}
    1 \\ 
    i
\end{bmatrix} =
\sqrt{2}
\begin{bmatrix}
    1 \\ 
    0
\end{bmatrix}
\end{equation}

特别地,当两束光的琼斯矢量相互垂直时,称两个偏振态正交。由于琼斯矢量是复矢量,因此正交不是内积而是 Hermitian 内积,即 $\langle \boldsymbol{E_1} \mid \boldsymbol{E_2} \rangle = \boldsymbol{E}_1 \cdot \boldsymbol{E}_2^*$。例如 $\mathscr{R}$ 态和 $\mathscr{L}$ 态是正交的、水平 $\mathscr{P}$ 态 (记作 $\mathscr{H}$) 和垂直 $\mathscr{P}$ 态 (记作 $\mathscr{V}$) 也是正交的:
\begin{equation}
\boldsymbol{E_{\mathscr{R}}} \cdot \boldsymbol{E_{\mathscr{L}}^*} =
\left\langle \begin{bmatrix}
    1 \\ 
    -i
\end{bmatrix} 
\mid 
\begin{bmatrix}
    1 \\ i
\end{bmatrix}^*
\right\rangle
= 1 + i^2 = 0
,\quad 
\boldsymbol{E_{\mathscr{H}}} \cdot \boldsymbol{E_{\mathscr{V}}^*} =
\left\langle \begin{bmatrix}
    1 \\ 
    0
\end{bmatrix}
\mid
\begin{bmatrix}
    0 \\ 1
\end{bmatrix}^*
\right\rangle
= 0
\end{equation}
由线性代数的知识知道,任何偏振态(即琼斯矢量)都可以由这样的一组正交偏振态合成得到,这也验证了我们之前对自然光“可分解为相位差迅速随机变化的两线偏振光”表述的合理性。

\subsection{琼斯矩阵和穆勒矩阵}
偏振器件对光的作用可以直接由矩阵来描述,常记作 $\mathscr{A}$(或 $\boldsymbol{A}$): 
\begin{equation}
\boldsymbol{E_t} = \mathscr{A} \boldsymbol{E_i}
\end{equation}
相应地,多个偏振器作用于同一光束时,设第一个通过的是 $\mathscr{A_1}$,按矩阵乘法有:
\begin{equation}
    \boldsymbol{E_t} = \mathscr{A}_n\mathscr{A}_{n-1}\cdots\mathscr{A}_{2}\mathscr{A}_{1} \boldsymbol{E_i}
\end{equation}
图 \ref{琼斯矩阵} 给出了常见偏振器的琼斯矩阵和穆勒矩阵。需要指出,我们指介绍了矩阵方法较重要的一些内容,对这个专题的完备讨论远远超出了本课程的范围。
\begin{figure}[H]\centering
    \includegraphics[width=0.6\columnwidth]{assets/5/5.1 琼斯矩阵和穆勒矩阵.png}
    \caption{常见偏振器的琼斯矩阵和穆勒矩阵}
    \label{琼斯矩阵}
\end{figure}

\section{双折射}

\subsection{双折射现象}
一些特殊的晶体是光学各向异性的,最直接的表现是双折射。双折射是指光在晶体中传播时,不同偏振态(即不同电矢量)的光有不同的折射率,因此会有不同的折射角。这种具有两个(两套)折射率的性质称为双折射。

这样的晶体一般都有一个特殊方向(称为光轴),当光沿此方向入射时,无论偏振态如何,都不会发生双折射,退化到普通入射现象,称为单轴晶体。当光线的传播方向 $\boldsymbol{k}$ 确定时,与光轴垂直的分量称为 $o$ 光,平行的分量称为 $e$ 光($o$ 和 $e$ 正交)。对于单轴晶体,$o$ 光和 $e$ 光的折射率是不同的,准确的说,$o$ 光的折射率是一个常数,而 $e$ 光的折射率是与光轴夹角 $\theta$ 的函数(在光轴方向上 $n_e(\theta) = n_o$)。由不同方向上折射率大小构成的曲面是一个椭球面,称为折射率椭球。$n_{e0} < n_o$的单轴晶体称为负晶体(等价于 $v_o < v_e$),$n_{e0} > n_o$($v_o > v_e$)的称为正晶体。

以光轴为 $z$ 轴,则单轴晶体的折射率椭球可写为:
\begin{equation}
\frac{x^2 + y^2}{n_o^2} + \frac{z^2}{n_e^2} = 1
\end{equation}

为了研究较一般的情况,我们先给出主截面和入射面的概念:
\begin{enumerate}
\item 主截面:介质表面法线与光轴共同构成的平面;
\item 入射面:介质表面法线与入射光线构成的平面。
\end{enumerate}
当入射面和主截面重合时(入射光线在主截面)内,折射的 $o$ 光和 $e$ 光都在主截面内,可由惠更斯原理推出 $o$ 光和 $e$ 光各自的折射方向,如下图所示:
\begin{figure}[H]\centering
    \includegraphics[width=0.75\columnwidth]{assets/5/5.4 惠更斯作图法.png}
    \caption{用惠更斯作图法求折射线}
\end{figure}
下面讨论几种特殊的情况:
\begin{enumerate}
\item 光轴垂直于表面,光线正入射(图 \ref{fig:oe} (a)):$n_o = n_e(\theta)$,没有发生双折射。
\item 光轴平行于表面,光线正入射(图 \ref{fig:oe} (b)):$n_o \ne n_e(\theta)$,尽管两光方向相同,但波速不同(这会引起相位差,将在后文提到),发生了双折射。
\item 光轴垂直于入射面,光线斜入射(图 \ref{fig:oe} (c)):$n_o \ne n_e(\theta)$,两光方向和波速都不同。
\end{enumerate}
\begin{figure}[H]\centering
\begin{subfigure}[b]{0.39\columnwidth}\centering
    \includegraphics[height=100pt]{assets/5/光轴垂直于界面.png}
    \caption{光轴垂直于介质表面}
\end{subfigure}\hfill
\begin{subfigure}[b]{0.34\columnwidth}\centering
    \includegraphics[height=100pt]{assets/5/光轴平行于界面.png}
    \caption{光轴平行于介质表面}
\end{subfigure}
\begin{subfigure}[b]{0.26\columnwidth}\centering
    \includegraphics[height=100pt]{assets/5/光轴垂直于入射面.png}
    \caption{光轴垂直于入射面}
\end{subfigure}
\caption{不同情形下 $o$ 光与 $e$ 光的行为}
\label{fig:oe}
\end{figure}



设 $\theta$ 是折射 $o$ 光与光轴的夹角,$\xi$ 是折射 $e$ 光与光轴的夹角,则有法向折射率 $n_N$ : 
\begin{gather}
    \cot \xi = \frac{n_e^2}{n_o^2} \cot \theta,\quad  n_N = n_N(\theta) = \sqrt{ \frac{n_o^2n_e^2}{n_o^2\sin^2 \theta + n_e^2\cos^2 \theta} }
\end{gather}
由上面两个公式可以分别确定法向折射率 $n$ 和 $\xi$,而 $\xi$ 又可以确定 $e$ 光的折射方向。需要注意,上式中的 $n_N$ 是法向折射率,如图 \ref{fig: e光} (d),对于正入射,$o$ 光和 $e$ 光的光程差是 $\Delta L =  n_N(\theta)d - n_od$,而不是 $n_N(\theta) \frac{d}{\cos \alpha} - n_od$。后一种应该用射线折射率 $n_r = n_r(\xi)$。

\begin{figure}[H]\centering
\begin{subfigure}[b]{0.2\columnwidth}\centering
    \includegraphics[height=120pt]{assets/5/5.4 1.png}
    \caption{法向速度 $v_N$ 与射线速度 $v_r$}
\end{subfigure}\hfill
\begin{subfigure}[b]{0.22\columnwidth}\centering
    \includegraphics[height=120pt]{assets/5/5.4 2.png}
    \caption{$e$ 光的射线面与法向面}
\end{subfigure}
\begin{subfigure}[b]{0.25\columnwidth}\centering
    \includegraphics[height=120pt]{assets/5/5.4 3.png}
    \caption{$v_r$ 夹角 $\xi$ 与 $v_N$ 夹角 $\theta$ 的关系}
\end{subfigure}
\begin{subfigure}[b]{0.32\columnwidth}\centering
    \includegraphics[height=120pt]{assets/5/5.4 4.png}
    \caption{光线正入射时的相位差}
\end{subfigure}
\caption{$e$ 光折射方向与光程差}
\label{fig: e光}
\end{figure}
为了继承原有的惯性思维(沿光线传播方向上的折射率),我们推荐使用 $n_r$ 而不是 $n_N$,如下:
\begin{equation}
    \cot \xi = \frac{n_e^2}{n_o^2} \cot \theta,\quad n_r = n_r(\theta) = {\color{red} \cos \alpha}\sqrt{ \frac{n_o^2n_e^2}{n_o^2\sin^2 \theta + n_e^2\cos^2 \theta} } = 
    {\color{red} \cos (\xi - \theta)}\sqrt{ \frac{n_o^2n_e^2}{n_o^2\sin^2 \theta + n_e^2\cos^2 \theta} } 
\end{equation}
此时图 \ref{fig: e光} (d) 中的光程差便是直觉上的 $\Delta L = n_r\frac{d}{\cos \alpha} - n_od$。\footnote{这一节如果不好理解,可以到网址 \href{https://www.bilibili.com/video/BV1VDpUeCEvv}{here} 观看视频的 11:00 - 12:00 部分,在动画的帮助下很快能懂。}


\subsection{相位延迟片}
再回来思考图 \ref{fig:oe} (b) 中发生的情况:$e$ 光和 $o$ 光方向都不变,但是它们的光程却不同,相位延迟片便是这样构成的。$o$ 光相对于 $e$ 光的相位增量是:
\begin{equation}
\Delta \phi_o = \phi_o - \phi_e = \frac{2\pi}{\lambda} (n_o - n_e)d
\end{equation}
这样便可以用先前的 $\varepsilon$ 来判断相位延迟片(又称波晶片)对偏振态的影响。最常用的波晶片是四分之一波片(简称 $\frac{\lambda}{4}$ 片),对应 $\Delta \phi_o = \frac{\pi}{2}$。

不能混淆的是,一些教材喜欢用“落后”或“领先”来表达这样的相位关系,尽管我们不提倡,但还是要指明,在$(kz - \omega t)$ 的情形中,相位增加 $\Delta \phi_o$ 意味着延迟(落后),即从 $(kz - \omega t)$ 变为 $(kz - \omega t + \Delta \phi_o)$。

现在我们把各种光经过四分之一波片后偏振态的变化做一个总结,如下图所示:
\begin{figure}[H]\centering
    \includegraphics[width=0.7\columnwidth]{assets/5/5.5 四分之一波片.png}
    \caption{偏振光经过四分之一波片后偏振态的变化}
\end{figure}

\subsection{偏振光的检验}
检验入射光到底是哪种偏振态,只需要一个偏振片和一个四分之一波片,具体方法如下图所示:
\begin{figure}[H]\centering
    \includegraphics[width=0.7\columnwidth]{assets/5/5.3 偏振光检验.png}
    \caption{偏振光的检验}
\end{figure}

\section{偏振态的计算}
本小节我们讨论不同偏振态经过不同厚度的波片(相位延迟片)后,会得到怎样的偏振态。一般有矩阵和相位两种方法,前者是利用偏振态和波片的琼斯矢量(矩阵),直接作矩阵乘法,后者是利用波片对 o 光的相位延迟作用。从数学地角度上,前者更直接,计算也更简单,但后者更能体现物理上的直观,有助于对偏振态的理解。

\subsection{矩阵法}
在玻片平面上建立平面直角坐标系,将光轴所在角度记为 $\phi$,称为波片的方位角。入射光束由琼斯矩阵 $\boldsymbol{E} = (E_{0x}, E_{0y})$ 来描述。考虑光线垂直入射到厚度为 $d$ 的波片(光线与光轴垂直),波片对 o 光的相位延迟是 $\Delta \varepsilon = \frac{2\pi}{\lambda}(n_o - n_e) d$,所以其琼斯矩阵 $\boldsymbol{W}$ 可写为:
\begin{equation}
\boldsymbol{W} = \boldsymbol{R}^{-1} \boldsymbol{W_0} \boldsymbol{R},\quad \boldsymbol{R} = 
\begin{bmatrix}
    \cos \phi & \sin \phi \\ 
    - \sin \phi & \cos \phi
\end{bmatrix},\quad \boldsymbol{W} =
{\color{red} e^{i\varphi}}\begin{bmatrix}
    e^{i\frac{\Delta \varepsilon}{2}} & 0 \\ 
    0 & e^{-i\frac{\Delta \varepsilon}{2}}
\end{bmatrix}
\end{equation}
其中 $\boldsymbol{R}$ 是二维旋转矩阵,满足 $\boldsymbol{R}^{-1} = \boldsymbol{R}^T$,而 $\boldsymbol{W_0}$ 是波片在 oe 坐标系(e 为横轴)下的琼斯矩阵,$\varphi = \frac{1}{2}\cdot \frac{2 \pi }{\lambda} (n_o + n_e) d$ 称为平均相位变化。由于 $e^{i\varphi}$ 一项同时作用在 $E_x$ 和 $E_y$,不会影响出射光的偏振态。在绝大多数情况下,我们仅关心相对相位差 $\varepsilon_0 =  \varepsilon + \Delta \varepsilon$,因此这一项常常被略去,此时波片的琼斯矩阵写为:
\begin{equation}
    \boldsymbol{W} = 
    \begin{bmatrix}
        \cos \phi & {\color{red} - \sin \phi } \\ 
         \sin \phi & \cos \phi
    \end{bmatrix}\cdot 
    \begin{bmatrix}
        e^{i\frac{\Delta \varepsilon}{2}} & 0 \\ 
        0 & e^{-i\frac{\Delta \varepsilon}{2}}
    \end{bmatrix}
    \cdot
    \begin{bmatrix}
        \cos \phi & \sin \phi \\ 
        {\color{red} - \sin  \phi} & \cos \phi
    \end{bmatrix}
\end{equation}

\subsection{相位法}

利用相位法,可以在脑海中形成偏振态关于 $\varepsilon$ 变化的“动图”,便于理解偏振态的变化过程。下面的讨论都默认波片的  $n_o > n_e$,也即 e 轴是快轴。
一束光入射波片,分别以玻片的 e 轴、o 轴为 $x$ 和 $y$ 轴建系(注意 e 是横轴),则入射光可在 eo 坐标系下分解为:
\begin{equation}
E_e = E_x = E_{0x}\cos (kr - \omega t),\quad E_o = E_y = E_{0y}\cos (kr - \omega t + \varepsilon_0)
\end{equation}
注意不要与矩阵法中统一的 $x$、$y$ 轴混淆。波片对光的作用,等价于 o 光发生了相位增量 $\Delta \varepsilon$,出射光变为:
\begin{equation}
    E_x = E_{0x}\cos (kr - \omega t ),\quad E_y = E_{0y}\cos (kr - \omega t + \varepsilon_0 + \Delta \varepsilon)
\end{equation}
此时相位差 $\varepsilon_0$ 变为 $\varepsilon = \phi_y - \phi_x = \varepsilon_0 + \Delta \varepsilon$,其中 $\Delta \varepsilon = \frac{2 \pi }{\lambda} (n_o - n_e)d$。回到最开始我们讨论 $\varepsilon$ 对偏振态“形状”的影响,这相当于 $\varepsilon$ 在 $[0, 2\pi]$ 上的周期性变化,引起偏振态的周期性变化。

设入射光线是线偏振光(等价于 $\varepsilon_0 = 0$),且与 $x$ 轴(e 轴)夹角为 $\alpha \in [0, \frac{\pi}{2}]$(否则作对称变换),夹角 $\alpha$ 即确定了 $E_{0x}$ 和 $E_{0y}$(准确的说是比值)。当 $E_{0x}E_{0y} \ne 0$ 且 $E_{0x} \ne E_{0y}$ 时,波片对线偏振光的作用,可以用下图来直观表示(以 $E_y > E_x$ 为例):

\begin{figure}[H]\centering
\begin{subfigure}[b]{0.24\columnwidth}\centering
    \includegraphics[height=110pt]{assets/5/5.5/2024-11-20_15-43-12.pdf}
    \caption{$\varepsilon \in [0, \frac{\pi}{2}]$, Left Rotation}
\end{subfigure}
\begin{subfigure}[b]{0.24\columnwidth}\centering
    \includegraphics[height=110pt]{assets/5/5.5/2024-11-20_15-43-14.pdf}
    \caption{$\varepsilon \in [\frac{\pi}{2}, \pi]$ , Left Rotation}
\end{subfigure}
\begin{subfigure}[b]{0.24\columnwidth}\centering
    \includegraphics[height=110pt]{assets/5/5.5/2024-11-20_15-43-16.pdf}
    \caption{$\varepsilon \in [\pi, \frac{3\pi}{2}]$ , Right Rotation}
\end{subfigure}
\begin{subfigure}[b]{0.24\columnwidth}\centering
    \includegraphics[height=110pt]{assets/5/5.5/2024-11-20_15-43-18.pdf}
    \caption{$\varepsilon \in [\frac{3\pi}{2}, 2\pi]$ , Right Rotation}
\end{subfigure}
\caption{波片对线偏振光的作用,$E_y > E_x > 0$}
\end{figure}

当 $E_{0x}E_{0y} \ne 0$ 且 $E_{0x} = E_{0y}$ 时,随着 $\varepsilon$ 的变化,偏振态依次经过右线、右椭圆、圆、左椭圆、左线,然后又依次返回到右线。且线、椭圆(和圆)都在 $\pm 45^\circ$ 线上,如下图所示:

\begin{figure}[H]\centering
    \begin{subfigure}[b]{0.24\columnwidth}\centering
        \includegraphics[height=110pt]{assets/5/5.6/2024-11-20_16-06-30.pdf}
        \caption{$\varepsilon \in [0, \frac{\pi}{2}]$, Left Rotation}
    \end{subfigure}
    \begin{subfigure}[b]{0.24\columnwidth}\centering
        \includegraphics[height=110pt]{assets/5/5.6/2024-11-20_16-06-32.pdf}
        \caption{$\varepsilon \in [\frac{\pi}{2}, \pi]$ , Left Rotation}
    \end{subfigure}
    \begin{subfigure}[b]{0.24\columnwidth}\centering
        \includegraphics[height=110pt]{assets/5/5.6/2024-11-20_16-06-34.pdf}
        \caption{$\varepsilon \in [\pi, \frac{3\pi}{2}]$ , Right Rotation}
    \end{subfigure}
    \begin{subfigure}[b]{0.24\columnwidth}\centering
        \includegraphics[height=110pt]{assets/5/5.6/2024-11-20_16-06-36.pdf}
        \caption{$\varepsilon \in [\frac{3\pi}{2}, 2\pi]$ , Right Rotation}
    \end{subfigure}
    \caption{波片对线偏振光的作用,$E_y = E_x > 0$}
\end{figure}

特别地,当  $E_{0x}E_{0y} = 0$ 时(至少有一个为零),由于不存在两分量的合成,偏振态不会发生任何变化(除了固有的相位增量)。

\section{旋光性}
\subsection{旋光性物质}
按我们之前的理论,一束线偏振光沿石英晶体的光轴传播时,不会发生双折射,因此不会发生偏振态的变化。但实际上,石英晶体具有旋光性的,线偏振光在沿其光轴传播时,振动方向不断转动,这种现象称为光的旋光性,而引起这种现象的物质称为旋光性物质。

旋光分为左旋和右旋,实验表面,振动面旋转的角度 $\delta \psi$ 与旋光晶体的厚度 $d$ 成正比,即 $\delta \psi = \alpha d$,$\alpha$ 称为旋光率,例如石英对 546.1 nm 的光的旋光率为 $25.538^\circ \cdot \ \mathrm{mm^{-1}}$。具体而言,定义逆时针旋转为正(向光源看去),继承之前的思想,我们可以有:
\begin{equation}\label{旋光性}
\Delta \psi = {\color{red} \frac{1}{2}}\cdot  \frac{2\pi}{\lambda} (n_R - n_L) d
\end{equation}
其中 $n_R$ 是物质对 $\mathscr{R}$ 态的折射率,$n_L$ 是对 $\mathscr{L}$ 态的折射率,所以旋光性也称为“圆双折射”。$n_R >n_L$ 时发生左旋(顺时针转动的光走得慢),$n_R < n_L$ 时发生右旋(逆时针转动的光走得慢)。

由公式 (\ref{旋光性}) 可以看出,系数 $\frac{\pi}{\lambda}$ 是波长的函数。因此,当一束线偏振白光沿光轴入射旋光性物质时,各波长的光转动不同的角度,在观察屏前放一检偏器,不断旋转,即可出现漂亮的色彩,这种现象称为旋光色散。

当然,我们也可以用琼斯矩阵的方法计算旋光,设入射偏振光的琼斯矢量为 $\boldsymbol{E} = (E_{0x}, E_{0y})$,则旋光性晶体的琼斯矩阵为:
\begin{equation}
\boldsymbol{W} = 
\begin{bmatrix}
    (x - iy) + e^{2 i \Delta \psi} (x + iy)  & 0\\ 
    -i \left[ (x - iy) - e^{2 i \Delta \psi} (x + iy) \right] & 0
\end{bmatrix}
\end{equation}
由于线性方程组解空间维数限制,上面的写法不唯一。当然,为了方便计算,我们更希望关注下面的结论:
\begin{equation}
    \boldsymbol{E'} = 
    \boldsymbol{W}\cdot\boldsymbol{E} = 
    \boldsymbol{W}\cdot \begin{bmatrix}
        E_{0x} \\
        E_{0y}
    \end{bmatrix}
    = 
    \boldsymbol{W} \cdot\left( a \begin{bmatrix}
        1 \\
        i
    \end{bmatrix} + 
    b \begin{bmatrix}
        1 \\
        -i
    \end{bmatrix} 
    \right)
    = 
    a \begin{bmatrix}
        1 \\
        i
    \end{bmatrix} +
    b \, e^{2 i \Delta \psi}\begin{bmatrix}
        1 \\
        -i
    \end{bmatrix}
\end{equation}
注意 $e^{2 i \Delta \psi}$ 一项应加在 $\mathscr{R}$ 光上,这是因为 $n_R > n_L$ 时,$v_R < v_L$ 更慢,相当于 $\mathscr{R}$ 发生了延迟。上式中 $a$、$b$ 是 $\boldsymbol{E}$ 在 $\mathscr{R}$、$\mathscr{L}$ 上的投影分量,具体而言:
\begin{equation}
a = \frac{E_{0x} - i E_{0y}}{2},\quad b = \frac{E_{0x} + i E_{0y}}{2}
\end{equation}
特别地,当 $\boldsymbol{E} = (1, 0)$ 为水平线偏振时:
\begin{equation}
a = b = \frac{1}{2},\quad \boldsymbol{E'} = \boldsymbol{W}\cdot \boldsymbol{E} = 
\begin{bmatrix}
    \frac{1}{2}\left(e^{2 i \Delta \psi} + 1\right) \\
    \frac{1}{2i}(e^{2 i \Delta \psi} - 1)
\end{bmatrix}
=
e^{i \Delta \psi}
\begin{bmatrix}
    \cos \Delta \psi \\
    \sin \Delta \psi
\end{bmatrix}
\end{equation}
当 $\boldsymbol{E} = (0, 1)$ 为垂直线偏振时:
\begin{equation}
a = -\frac{i}{2},\ b = \frac{i}{2},\quad \boldsymbol{E'} = \boldsymbol{W}\cdot \boldsymbol{E} =
\begin{bmatrix}
    \frac{i}{2}\left(e^{2 i \Delta \psi} + 1\right) \\
    \frac{i}{2i}(e^{2 i \Delta \psi} - 1)
\end{bmatrix}
= 
i\, e^{i \Delta \psi}
\begin{bmatrix}
    -\cos \Delta \psi \\
    \sin \Delta \psi
\end{bmatrix}
\end{equation}

\subsection{磁致旋光}
当光在磁场 $\boldsymbol{B}$ 中,沿磁场方向传播时,同样会出现旋光现象,称为“磁致旋光”或“法拉第旋转效应”,光通过长度为 $l$ 的样品时:
\begin{equation}
\Delta \psi = VBl
\end{equation}
其中 $V$ 称为韦尔代常量,一般物质的韦尔代常量都很小。在磁场边界放一反射镜,当光线进入磁场、反射、再出磁场时,光的振动面会旋转 $2\Delta \psi$,可以利用这一原理制造光隔离器。

%%%%%%%%%%%%%%%%%%%%%%%%%%%%%%%
% 上面是应直接复制到总文件的内容 %
%%%%%%%%%%%%%%%%%%%%%%%%%%%%%%%

\end{document}



% VScode 常用快捷键:

% F2:                       变量重命名
% Ctrl + Enter:             行中换行
% Alt + up/down:            上下移行
% 鼠标中键 + 移动:           快速多光标
% Shift + Alt + up/down:    上下复制
% Ctrl + left/right:        左右跳单词
% Ctrl + Backspace/Delete:  左右删�