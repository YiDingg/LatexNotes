% 若编译失败,且生成 .synctex(busy) 辅助文件,可能有两个原因:
% 1. 需要插入的图片不存在:Ctrl + F 搜索 'figure' 将这些代码注释/删除掉即可
% 2. 路径/文件名含中文或空格:更改路径/文件名即可

% --------------------- 文章宏包及相关设置 --------------------- %
% >> ------------------ 文章宏包及相关设置 ------------------ << %
% 设定文章类型与编码格式
    \documentclass[UTF8]{report}		

% 本文特殊宏包
    \usepackage{siunitx} % 埃米单位

% 本文特殊宏定义
    \def\Re{\mathrm{Re}\,}
    \def\Im{\mathrm{Im}\,}

% 自定义宏定义
    \def\N{\mathbb{N}}
    \def\F{\mathbb{F}}
    \def\Z{\mathbb{Z}}
    \def\Q{\mathbb{Q}}
    \def\R{\mathbb{R}}
    \def\C{\mathbb{C}}
    \def\T{\mathbb{T}}
    \def\S{\mathbb{S}}
    \def\A{\mathbb{A}}
    \def\I{\mathscr{I}}
    \def\Arg{\mathrm{Arg}\,}
    \def\d{\mathrm{d}}
    \def\p{\partial}


% 导入基本宏包
    \usepackage[UTF8]{ctex}     % 设置文档为中文语言
    \usepackage{hyperref}  % 宏包:自动生成超链接 (此宏包与标题中的数学环境冲突)
    \hypersetup{
        colorlinks=true,    % false:边框链接 ; true:彩色链接
        linkcolor={blue},   % 目录,公式,图表,脚注等内部链接颜色
        citecolor={blue},   % 文献引用颜色
        urlcolor={blue}     % 网页 URL 链接颜色,包括 \href 中的 text
    }
    
    % \usepackage{docmute}    % 宏包:子文件导入时自动去除导言区,用于主/子文件的写作方式,\include{./51单片机笔记}即可。注:启用此宏包会导致.tex文件capacity受限。
    \usepackage{amsmath}    % 宏包:数学公式
    \usepackage{mathrsfs}   % 宏包:提供更多数学符号
    \usepackage{amssymb}    % 宏包:提供更多数学符号
    \usepackage{pifont}     % 宏包:提供了特殊符号和字体
    \usepackage{extarrows}  % 宏包:更多箭头符号
    \usepackage{multicol}   % 宏包:支持多栏 


% 文章页面margin设置
    \usepackage[a4paper]{geometry}
        \geometry{top=1in}  % 1 inch= 2.46 cm, 0.75 inch = 1.85 cm
        \geometry{bottom=1in}
        \geometry{left=0.75in}
        \geometry{right=0.75in}   % 设置上下左右页边距
        \geometry{marginparwidth=1.75cm}    % 设置边注距离(注释、标记等)

% 配置数学环境
    \usepackage{amsthm} % 宏包:数学环境配置
    % theorem-line 环境自定义
        \newtheoremstyle{MyLineTheoremStyle}% <name>
            {11pt}% <space above>
            {11pt}% <space below>
            {}% <body font> 使用默认正文字体
            {}% <indent amount>
            {\bfseries}% <theorem head font> 设置标题项为加粗
            {:}% <punctuation after theorem head>
            {.5em}% <space after theorem head>
            {\textbf{#1}\thmnumber{#2}\ \ (\,\textbf{#3}\,)}% 设置标题内容顺序
        \theoremstyle{MyLineTheoremStyle} % 应用自定义的定理样式
        \newtheorem{LineTheorem}{Theorem.\,}
    % theorem-block 环境自定义
        \newtheoremstyle{MyBlockTheoremStyle}% <name>
            {11pt}% <space above>
            {11pt}% <space below>
            {}% <body font> 使用默认正文字体
            {}% <indent amount>
            {\bfseries}% <theorem head font> 设置标题项为加粗
            {:\\ \indent}% <punctuation after theorem head>
            {.5em}% <space after theorem head>
            {\textbf{#1}\thmnumber{#2}\ \ (\,\textbf{#3}\,)}% 设置标题内容顺序
        \theoremstyle{MyBlockTheoremStyle} % 应用自定义的定理样式
        \newtheorem{BlockTheorem}[LineTheorem]{Theorem.\,} % 使用 LineTheorem 的计数器
    % definition 环境自定义
        \newtheoremstyle{MySubsubsectionStyle}% <name>
            {11pt}% <space above>
            {11pt}% <space below>
            {}% <body font> 使用默认正文字体
            {}% <indent amount>
            {\bfseries}% <theorem head font> 设置标题项为加粗
            {:\\ \indent}% <punctuation after theorem head>
            {0pt}% <space after theorem head>
            {\textbf{#3}}% 设置标题内容顺序
        \theoremstyle{MySubsubsectionStyle} % 应用自定义的定理样式
        \newtheorem{definition}{}

%宏包:有色文本框(proof环境)及其设置
    \usepackage[dvipsnames,svgnames]{xcolor}    %设置插入的文本框颜色
    \usepackage[strict]{changepage}     % 提供一个 adjustwidth 环境
    \usepackage{framed}     % 实现方框效果
        \definecolor{graybox_color}{rgb}{0.95,0.95,0.96} % 文本框颜色。修改此行中的 rgb 数值即可改变方框纹颜色,具体颜色的rgb数值可以在网站https://colordrop.io/ 中获得。(截止目前的尝试还没有成功过,感觉单位不一样)(找到喜欢的颜色,点击下方的小眼睛,找到rgb值,复制修改即可)
        \newenvironment{graybox}{%
        \def\FrameCommand{%
        \hspace{1pt}%
        {\color{gray}\small \vrule width 2pt}%
        {\color{graybox_color}\vrule width 4pt}%
        \colorbox{graybox_color}%
        }%
        \MakeFramed{\advance\hsize-\width\FrameRestore}%
        \noindent\hspace{-4.55pt}% disable indenting first paragraph
        \begin{adjustwidth}{}{7pt}%
        \vspace{2pt}\vspace{2pt}%
        }
        {%
        \vspace{2pt}\end{adjustwidth}\endMakeFramed%
        }

% 外源代码插入设置
    % matlab 代码插入设置
    \usepackage{matlab-prettifier}
        \lstset{
            style=Matlab-editor,  % 继承matlab代码颜色等
        }
    \usepackage[most]{tcolorbox} % 引入tcolorbox包 
    \usepackage{listings} % 引入listings包
        \tcbuselibrary{listings, skins, breakable}
        \newfontfamily\codefont{Consolas} % 定义需要的 codefont 字体
        \lstdefinestyle{matlabstyle}{
            language=Matlab,
            basicstyle=\small\ttfamily\codefont,    % ttfamily 确保等宽 
            breakatwhitespace=false,
            breaklines=true,
            captionpos=b,
            keepspaces=true,
            numbers=left,
            numbersep=15pt,
            showspaces=false,
            showstringspaces=false,
            showtabs=false,
            tabsize=2
        }
        \newtcblisting{matlablisting}{
            arc=2pt,        % 圆角半径
            top=-5pt,
            bottom=-5pt,
            left=1mm,
            listing only,
            listing style=matlabstyle,
            breakable,
            colback=white   % 选一个合适的颜色
        }

% table 支持
    \usepackage{booktabs}   % 宏包:三线表
    \usepackage{tabularray} % 宏包:表格排版
    \usepackage{longtable}  % 宏包:长表格

% figure 设置
    \usepackage{graphicx}  % 支持 jpg, png, eps, pdf 图片 
    \usepackage{svg}       % 支持 svg 图片
        \svgsetup{
            % 指向 inkscape.exe 的路径
            %inkscapeexe = C:/aa_MySame/inkscape/bin/inkscape.exe, 
            inkscapeexe = C:/aa_MySame/inkscape/bin/inkscape.exe,  
            % 一定程度上修复导入后图片文字溢出几何图形的问题
            inkscapelatex = false                 
        }

% 图表进阶设置
    \usepackage{caption}    % 图注、表注
        \captionsetup[figure]{name=图}  
        \captionsetup[table]{name=表}
        \captionsetup{labelfont=bf, font=small}
    \usepackage{float}     % 图表位置浮动设置 
    \usepackage{subcaption} % subfigure 子图支持

% 圆圈序号自定义
    \newcommand*\circled[1]{\tikz[baseline=(char.base)]{\node[shape=circle,draw,inner sep=0.8pt, line width = 0.03em] (char) {\small \bfseries #1};}}   % TikZ solution

% 列表设置
    \usepackage{enumitem}   % 宏包:列表环境设置
        \setlist[enumerate]{
            label=(\arabic*) ,   % 设置序号样式为 (1) (2) (3)
            ref=\arabic*, % 如果需要引用列表项,这将决定引用格式(这里仍然使用数字)
            itemsep=0pt, parsep=0pt, topsep=0pt, partopsep=0pt, leftmargin=3.5em} 
        \setlist[itemize]{itemsep=0pt, parsep=0pt, topsep=0pt, partopsep=0pt, leftmargin=3.5em}
        \newlist{circledenum}{enumerate}{1} % 创建一个新的枚举环境  
        \setlist[circledenum,1]{  
            label=\protect\circled{\arabic*}, % 使用 \arabic* 来获取当前枚举计数器的值,并用 \circled 包装它  
            ref=\arabic*, % 如果需要引用列表项,这将决定引用格式(这里仍然使用数字)
            itemsep=0pt, parsep=0pt, topsep=0pt, partopsep=0pt, leftmargin=3.5em
        }  

% 其它设置
    % 脚注设置
        \renewcommand\thefootnote{\ding{\numexpr171+\value{footnote}}}
    % 参考文献引用设置
        \bibliographystyle{unsrt}   % 设置参考文献引用格式为unsrt
        \newcommand{\upcite}[1]{\textsuperscript{\cite{#1}}}     % 自定义上角标式引用
    % 文章序言设置
        \newcommand{\cnabstractname}{序言}
        \newenvironment{cnabstract}{%
            \par\Large
            \noindent\mbox{}\hfill{\bfseries \cnabstractname}\hfill\mbox{}\par
            \vskip 2.5ex
            }{\par\vskip 2.5ex}

% 文章默认字体设置
    \usepackage{fontspec}   % 宏包:字体设置
        \setmainfont{SimSun}    % 设置中文字体为宋体字体
        \setCJKmainfont[AutoFakeBold=3]{SimSun} % 设置加粗字体为 SimSun 族,AutoFakeBold 可以调整字体粗细
        \setmainfont{Times New Roman} % 设置英文字体为Times New Roman

% 各级标题自定义设置
    \usepackage{titlesec}   
        \titleformat{\chapter}[hang]{\normalfont\huge\bfseries\centering}{第\,\thechapter\,章}{20pt}{}
        \titlespacing*{\chapter}{0pt}{-20pt}{20pt} % 控制上方空白的大小
        % section标题自定义设置 
        \titleformat{\section}[hang]{\normalfont\Large\bfseries}{§\,\thesection\,}{8pt}{}
% subsubsection标题自定义设置
%\titleformat{\subsubsection}[hang]{\normalfont\bfseries}{}{8pt}{}

% --------------------- 文章宏包及相关设置 --------------------- %
% >> ------------------ 文章宏包及相关设置 ------------------ << %

% ------------------------ 文章信息区 ------------------------ %
% ------------------------ 文章信息区 ------------------------ %
% 页眉页脚设置
    \usepackage{fancyhdr}   %宏包:页眉页脚设置
        \pagestyle{fancy}
        \fancyhf{}
        \cfoot{\thepage}
        \renewcommand\headrulewidth{1pt}
        \renewcommand\footrulewidth{0pt}
        \lhead{2024.8 -- 2025.1} 
        \chead{光学笔记(Optics Notes)}    
        \rhead{dingyi233@mails.ucas.ac.cn}
%文档信息设置
    \title{光学笔记\\Optics Notes}
    \author{丁毅\\ \footnotesize 中国科学院大学,北京 100049\\ Yi Ding \\ \footnotesize University of Chinese Academy of Sciences, Beijing 100049, China}
    \date{\footnotesize 2024.8 -- 2025.1}
% ------------------------ 文章信息区 ------------------------ %
% ------------------------ 文章信息区 ------------------------ %

% 开始编辑文章

\begin{document} 
\zihao{5}             % 设置全文字号大小, -4 为小四, 5 为五号

% ------------------------ 封面序言与目录 ------------------------ %
% >> --------------------- 封面序言与目录 --------------------- << %
% 封面
\maketitle\newpage  
\pagenumbering{Roman} % 页码为大写罗马数字
\thispagestyle{fancy}   % 显示页码、页眉等

% 序言
\begin{cnabstract}\normalsize 
    本文为笔者本科时的“光学”课程笔记(Notes of Optics, 2024.8-2025.1)。由于个人学识浅陋,认识有限,文中难免有不妥甚至错误之处,望读者不吝指正,在此感谢。我的邮箱是 dingyi233@mails.ucas.ac.cn。\par
    为了更好地学习光学,建议先跳转至附录部分,了解相关理论知识。\par
\end{cnabstract}
\addcontentsline{toc}{chapter}{序言} % 手动添加为目录

% 目录
\setcounter{tocdepth}{1}                % 目录深度(为1时显示到section)
\tableofcontents                        % 目录页
\addcontentsline{toc}{chapter}{目录}    % 手动添加此页为目录
\thispagestyle{fancy}                   % 显示页码、页眉等 

% 收尾工作
\newpage    
\pagenumbering{arabic} 


% >> --------------------- 封面序言与目录 --------------------- << %
% ------------------------ 封面序言与目录 ------------------------ %

\chapter{光学导言}\thispagestyle{fancy}

\section{光学发展简史(略)}


\section{光的几何传播规律}

\begin{definition}[光传播的基本原理]
光传播的常见基本原理:
\begin{enumerate}
    \item 直线传播:光在均匀介质里沿直线传播 \footnote{对高功率激光,此定律不成立}
    \item 反射定律:光线入射到两种不同的均匀介质的分界面上反射线位于入射面内,反射线和入射线分居法线两侧,反射角等于入射角
    \item 折射定律(斯涅尔定律):折射线位于入射面内,折射线与入射线分居法线两侧,入射角的正弦与折射角的正弦
    之比为一与入射角无关的常数 \footnote{折射率较大的一侧称为光密介质;较小的一侧称为光疏介质}
    \begin{equation}
    n_1\sin i_1 = n_2 \sin i_2
    \end{equation}
    
    \begin{figure}[H]\centering
    \includegraphics[width=0.4\textwidth]{assets/1,2/image (44).jpg}
    \caption{\textbf{反射与折射}}\label{反射与折射}
    \end{figure}
    
    \item 光路可逆性:光沿反方向传播时,必定沿原光路返回 \footnote{也即在几何光学中,任何光路都是可逆的}
    \item 独立传播:光在传播过程中与其他光束相遇时,各光束都各自独立传播,不改变其传播方向
    \item 全反射:光线从光密介质入射到光疏介质,当入射角大于某临界值时,折射光完
    全消失,只剩下反射光。该临界角度称为全反射临界角。
    \begin{equation}
    i_C = \arcsin (\frac{n_1}{n_2})\quad n_1<n_2
    \end{equation}
\end{enumerate}
\end{definition}


\begin{definition}[彩虹]
\end{definition}


\begin{definition}[三棱镜最小偏向角]

最小偏向角 $\theta_0 = (i_1 - i_1')_{\text{min}}$ 满足:
\begin{equation}
    \theta_0 = 2i_1 - A, \quad \frac{n_2}{n_1} = \frac{\sin\frac{\theta_0+A}2}{\sin\frac A2}
\end{equation}

\begin{figure}[H]\centering
\includegraphics[width=0.4\textwidth]{assets/1,2/image (45).jpg}
\caption{\textbf{三棱镜最小偏向角}}\label{三棱镜最小偏向角}
\end{figure}

\end{definition}

\section{惠更斯原理与费马原理}

\begin{LineTheorem}[惠更斯原理]\label{LineTheorem: 惠更斯原理}
    由振源发出的波动在 $t$ 时刻传播到一个波面 $S$,波面上的每一个面元可认为是次波的波源。由面元发出的次波向四面八方传播。在以后的时刻 $t'$ 形成次波面。这些次波面的包络面 $S'$ 就是 $t'$ 时刻总扰动的波面。
\end{LineTheorem}

\noindent 其中:
\begin{enumerate}
    \item 波面:在同一振源的波场中,扰动同时到达的各点具有相同的相位,这些点的轨迹构成一个曲面,称为波面(也称为等相位面)。
    \item 波线:与波面处处正交的曲线称为波线,其切线方向为光的传播方向
\end{enumerate}

\noindent 几何光学的定律需要前提条件:
\begin{enumerate}
\item 必须是均匀介质,即同一介质的折射率处处相等,折射率不是位置的函数。
\item 必须是各向同性介质,即光在介质中传播时各个方向的折射率相等,折射率不是方向的函数。
\item 光强不能太强,否则巨大的光能量会使线性叠加原理不再成立而出现非线性情况。
\item 光学元件的线度应比光的波长大得多,否则不能把光束简化为光线。
\end{enumerate}

\begin{figure}[H]\centering
\includegraphics[width=0.5\textwidth]{assets/1,2/image (46).jpg}
\caption{\textbf{惠更斯原理}}\label{惠更斯原理}
\end{figure}

\begin{BlockTheorem}[费马原理]\label{费马原理}
光从空间中一点传播到另一点时,总是沿光程(optical length, OPL)取极值的路径传播 \footnote{这里的“极值”可以是极小值、极大值或常数,一般情况下,实际光程大多取极小值。极大值(如凹面镜成像)、拐点(如椭球面镜、凸透镜)的例子,可以参考 \href{https://zhuanlan.zhihu.com/p/107739173}{知乎:浅谈几何光学(1)——费马原理}},公式:
\begin{equation}
    \mathrm{d}\ \mathrm{OPL} =  \mathrm{d}\left(\int\limits_{Q}^{P}ndl\right)=0 \Longrightarrow \frac{\mathrm{d}\  \mathrm{OPL} }{\mathrm{d} \varphi } = \frac{\mathrm{d} OPL }{\mathrm{d} s } = 0 
\end{equation}
\end{BlockTheorem}
由费马原理可以导出诸多推论,包括我们熟知的几条基本原理,还有物像之间的等光程性(例如凸透镜):
在物点Q与像点Q’之间,不管光线经何路径,凡是由Q通过同样的光学系统到达Q’的光线,都是等光程的。

\section{成像}

理想的像与物体在形状上一致,大小成比例。物与像之间的关系:本质上是一系列物点与像点的点点对应,推广至线线、面面对应。

同心光束:各光线本身或其延长线交于同一点的光束称为同心光束,在各向同性介质中,它对应于球面波。

由若干反射面或折射面组成的光学系统称为光具组

\begin{enumerate}
\item 实物:发散的同心入射光束的“心”
\item 虚物:汇聚的同心入射光束的“心”
\item 实像:发散的同心出射光束的“心”
\item 虚像:汇聚的同心出射光束的“心”
\end{enumerate}

\textbf{物像的共轭性(可逆性):}
若 $P$ 为 物体 $P$(可实可虚)的像点,则反之,当物点为 $P$ 时,像点必在点 $P'$(实际光路可能不同)。是光路可逆性的必然结果。 

计算由物到像的 OPL 时,若为实线(实物、实像)则取正,称为实光程,若为虚线(虚物、虚像)则取负,称为虚光程。

\begin{figure}[H]\centering
\includegraphics[width=0.7\textwidth]{assets/1,2/path2.pdf}
\caption{\textbf{光程恒定的例子}}\label{光程恒定的例子}
\end{figure}


\begin{definition}[折射球面与反射球面]
的
对于折射球面,存在一对恰好成像的共轭点,称为齐明点。在齐明点处,可以证明 $Q$ 到 $Q'$ 的光程(即物像间的OPL)$l_{QQ'}$。

折射球面公式:
\begin{gather}
\frac{n_1}{l_1} + \frac{n_2}{l_2} = \frac{1}{R\,}\left( \frac{n_{{\color{red} 2}}s_{{\color{red} 2}}}{l_{{\color{red} 2}}} - \frac{n_1s_1}{l_1} \right)
\end{gather}
{\color{gray}\small
\begin{equation}
    \frac{n'}{s'}  +  \frac{n}{s} = \frac{n'-n}{r}\quad  \text{(傍轴)}
\end{equation}
}

反射球面公式:
\begin{equation}
\frac{1}{l_1} + \frac{1}{l_2} = {\color{red} -}\frac{2}{R\,}\left(\frac{s_1}{l_1} + \frac{s_2}{l_2}  \right)
\end{equation}
{\par\color{gray}\small
\begin{equation}
 \frac{1}{s_1} + \frac{1}{s_2} = -\frac{2}{R\,} \quad  \text{(傍轴)}
\end{equation}
\par}

\begin{figure}[H]\centering
\begin{subfigure}[t]{0.45\textwidth}\centering
    \includegraphics[height=100pt]{assets/1,2/image.jpg}
    \caption{\bfseries 折射球面 }
\end{subfigure}\begin{subfigure}[t]{0.4\textwidth}\centering
    \includegraphics[height=100pt]{assets/1,2/image (1).jpg}
    \caption{\bfseries 反射球面 }
\end{subfigure}
\caption{\bfseries 折射球面与反射球面 }
\end{figure}

\end{definition}


\begin{definition}[像的放大率]
放大率公式:
\begin{equation}
\frac{n_1 | y_1 |}{s_1} = \frac{n_2 | y_2 |}{s_2}
\end{equation}

Lagrange-Helmholtz 恒等式:
\begin{equation}
n_1u_1y_1 = n_2u_2y_2
\end{equation}

上式的 $u$ 和 $y$ 是有正负的,例如折射球面中 $u_1 > 0,\ y_1 >0$ 而 $u_2 <0,\ y_2 < 0$。

\end{definition}





\section{光学仪器}


\subsection{薄透镜}

透镜是由两个共轴折射球面构成的光具组,球面间距远远小于球面半径和物距像距的透镜称为薄透镜,也即 $d \ll | R_1 |, | R_2 |, | s |, | s' |$。此时可以认为两球面顶点重合,称为光心。

薄透镜成像公式(物像距公式):
\begin{gather} 
\frac{n}{s} + \frac{n'}{s'} = \frac{n_L - n}{r_1} + \frac{n' - n_L}{r_2} \label{薄透镜成像公式} \\
s' = \infty \Longrightarrow  f = \frac{n}{\frac{n_L - n}{r_1} + \frac{n' - n_L}{r_2}}\quad \text{物方焦距} \label{物方焦距} \\ 
s = \infty \Longrightarrow  f' = \frac{n'}{\frac{n_L - n}{r_1} + \frac{n' - n_L}{r_2}}\quad \text{像方焦距} \label{像方焦距}
\end{gather}
故物像焦距满足 $\frac{f}{n} = \frac{f'}{n'}$。
特别地,当物像方折射率都为 1 时(真空),我们有磨镜者公式和像的横向放大率:
\begin{equation}
f =f' = \frac{1}{(n_L - 1)(\frac{1}{r_1} - \frac{1}{r_2})},\quad  V = -\frac{\frac{s'}{n'}}{\frac{s}{n}} = -\frac{fs'}{f's} =  - \frac{s'}{s}
\end{equation}


将公式 \ref{物方焦距} 和公式 \ref{像方焦距} 代入式 \ref{薄透镜成像公式} 中,可以得到 Gauss 物像公式:
\begin{equation}
\frac{f}{s} + \frac{f'}{s'} = 1 \overset{n = n'}{\ \ \ \Longrightarrow\ \ \  } \frac{1}{s} + \frac{1}{s'} = \frac{1}{f}
\label{Gauss物像公式}
\end{equation}

令 $s = x + f$,$s' = x' + f'$,代入公式 \ref{Gauss物像公式},可以得到 Newton 物像公式:
\begin{equation}
xx = ff'
\end{equation}


\subsection{其它仪器}
投影仪器、照相机、眼睛、放大镜、显微镜、望远镜

\section{光波的描述}
\section{光度学基本概念}

在学习光度学之前,需要区分辐射度学与光度学中的基本概念。辐射度学研究的是辐射能量对实际物体的影响,而光度学研究的是辐射能量对人眼的影响,是基于人眼实验数据的学科,例如 Luminous Efficiency Function。它们的概念相互对应(可以相互转化)但并不相同,如下表所示: 
\begin{table}[H]\centering
    \caption{\textbf{光度学与辐射度学概念对应关系}}
    \label{光度学与辐射度学概念对应关系}
    \resizebox{\linewidth}{!}{   % 设置宽度为 \linewidth 等比例缩放
    \begin{tabular}{|c|c|c|c|c|c|c|c|}
        \hline
        学科范围 & \multicolumn{7}{c|}{基本概念} \\
        \hline
        辐射度学 & 辐射能 $Q_e$  & 辐射通量 $\Phi_e$  & 辐射强度 $I_e$ & 辐射亮度 $L_e$ &  辐射照度 $E_e$ & 辐射出射度 $M_e$ & 辐射通量谱密度 $\Phi_{e,\lambda}$ \\
        \hline
        光度学   & 光量 $Q_v$ &  光通量 $\Phi_v$  & 光强度 $I_v$ & 光亮度 $L_v$ & 光照度 $E_v$ & 光出射度 $M_v$ & 光通量谱密度 $\Phi_{v,\lambda}$\\
        \hline
    \end{tabular}
    }
\end{table}



\subsection{辐射度学基本概念}

\begin{table}[H]\centering
    \caption{\textbf{辐射度学基本概念}}
    \label{辐射度学基本概念}
    \renewcommand{\arraystretch}{1.1} % 调整行间距为默认值的1.5倍
%\resizebox{0.9\linewidth}{!}{   % 设置宽度为 \linewidth 等比例缩放
%}
\begin{tabular}{|c|c|c|c|c|c|c|c|c|c|}\hline
    名称 & 符号& 定义式 & 单位 & 概念描述\\
    \hline
    辐射能 & $Q_e$ & - & J & 以辐射形式传播的能量 \\
    \hline
    辐通量 & $\Phi_e$ & $\Phi_e = \frac{\mathrm{d}Q_e}{\mathrm{d}t}$ & W & 单位时间内流过某截面的辐射能量 \\
    \hline
    辐强度 & $\boldsymbol{I}_e$ & $\boldsymbol{I}_e = \frac{\mathrm{d}\Phi_e }{\mathrm{d} \boldsymbol{\Omega} } $ & $\mathrm{W\cdot sr^{-1}}$ & 点辐射源在某方向上单位立体角\footnote{立体角的定义参考 \href{https://www.zhihu.com/question/611533175/answer/3244345528}{辐亮度和辐照度是如何计算的}}内的辐射通量 \\
    \hline
    辐照度 & $\boldsymbol{E}_e$ & $ \boldsymbol{E}_e = \frac{\mathrm{d}\Phi_e}{\mathrm{d}\boldsymbol{A}}$ & $\mathrm{W\cdot m^{-2}}$ & 被辐射体单位面积上的辐射通量 \\
    \hline
    辐亮度 & $\boldsymbol{L}_e$ & $\boldsymbol{L}_e = \frac{\mathrm{d}\boldsymbol{I}_e}{\mathrm{d}A \cos \theta} $ & $\mathrm{\mathrm{W}\cdot sr^{-1}\cdot m^{-2}}$ & 单位面积的面辐射源在某方向上的辐射强度\\
    \hline
    辐出射度 & $M_e$ & $ M_e = \frac{\mathrm{d}\Phi_e}{\mathrm{d}A}$ & $\mathrm{W\cdot m^{-2}} $ & 辐射体单位面积向半球空间发射的辐射通量 \\
    \hline
    辐谱密度 & $\phi_e$ & $\phi_e = \frac{\Delta \Phi_{e,\lambda}}{ \Delta\lambda} $ & $\mathrm{W\cdot m^{-1}} $ & 辐射能(通量)在频谱中的分布 \\
    \hline
\end{tabular}
\end{table}

其中 $\Delta \Phi_{e,\lambda}$ 表示波长为 $\lambda$(也可认为是 $[\lambda,\ \lambda + \Delta \lambda]$)的部分所贡献的辐射通量。

\subsection{明视觉曲线}

人眼对不同波长的光具有不同的明亮感觉程度\footnote{参考 \href{https://www.writebug.com/static/uploads/2024/9/16/09153f4fce1d2b99e38c19ef9deeda44.pdf}{新旧明视觉光谱光视效率曲线.pdf}。},称为明视觉光谱光视效率曲线\footnote{参考 \href{https://webstore.ansi.org/preview-pages/ESTA/preview_E1-48_2014.pdf}{ANSI E1.48 - 2014 (A Recommended Luminous Efficiency Function for Stage and Studio Luminaire Photometry)},国际照明委员会(CIE)规定的标准光谱光视效率函数 \href{https://rdrr.io/cran/colorSpec/man/luminsivity.html}{Luminous Efficiency Functions} 或者 \href{https://www.zhihu.com/question/400643965/answer/2727547334}{知乎:光通量与光辐照度之间的换算}。}(函数),常简称为“明视觉曲线”或“视觉曲线”,记为 $V = V(\lambda)$。

光谱光效能 $K$,表示在某一波长上每一瓦辐射通量可以产生多少流明的光通量。光谱光视效率 $V = V(\lambda)$,就是归一化的光谱光效能:
\begin{equation}\label{光谱光效能}
    K = \frac{\Delta \Phi_{v,\lambda}}{\Delta \Phi_{e,\lambda}} = \frac{\phi_v(\lambda)}{\phi_e(\lambda)}, \quad V(\lambda) = \frac{K(\lambda)}{K_{\text{max}}} = \frac{1}{K_{\text{max}}}\cdot \frac{\phi_v(\lambda)}{\phi_e(\lambda)}
\end{equation}
$K_{\text{max}} = 683\ \mathrm{lm \cdot W^{-1}}$ 在波长约 555.0 nm 取到,因此 $V = V(\lambda)$ 也表示在相同辐射通量下,波长为 $\lambda$ 的光与 555.0 nm 的光所产生的亮暗感觉比值。

另外,公式 \ref{光谱光效能} 建立了辐射度学参量与光度学参量之间的转化关系:
\begin{equation}
\Phi_v(\lambda) 
= \int \phi_v(\lambda) \mathrm{d} \lambda 
= \int K_{\text{max}}V(\lambda)\phi_e(\lambda) \mathrm{d} \lambda  
\end{equation}

\subsection{光度学基本概念}

\begin{table}[H]\centering
    \caption{\textbf{光度学基本概念}}
    \label{光度学基本概念}
    \renewcommand{\arraystretch}{1.15} % 调整行间距为默认值的1.5倍
%\resizebox{0.9\linewidth}{!}{   % 设置宽度为 \linewidth 等比例缩放
%}
\begin{tabular}{|c|c|c|c|c|c|c|c|c|c|}\hline
    名称 & 符号& 定义式 & 单位 & 概念描述\\
    \hline
    光量 & $Q_v$ & $Q_v(\lambda) = V(\lambda)\cdot Q_e(\lambda)$ & $\mathrm{cd \cdot sr \cdot s}$ & 辐射能的光度量大小 \\
    \hline
    光通量 & $\Phi_v$ & $\Phi_v = \frac{\mathrm{d}Q_v}{\mathrm{d}t}$ & $\mathrm{lm} = \mathrm{cd \cdot sr}$ & 单位时间内流过某截面的光度学光量 \\
    \hline
    光强度 & $\boldsymbol{I}_v$ & $\boldsymbol{I}_v = \frac{\mathrm{d}\Phi_v }{\mathrm{d} \boldsymbol{\Omega} } $ & $\mathrm{cd}$ & 点辐射源在某方向上单位立体角内的光通量 \\
    \hline
    光照度 & $\boldsymbol{E}_v$ & $ \boldsymbol{E}_v = \frac{\mathrm{d}\Phi_v}{\mathrm{d}\boldsymbol{A}}$ & $\mathrm{lm \cdot m^{-2}}$ & 被辐射体单位面积上的光通量 \\
    \hline
    光亮度 & $\boldsymbol{L}_v$ & $\boldsymbol{L}_v = \frac{\mathrm{d}\boldsymbol{I}_v}{\mathrm{d}A \cos \theta} $ & $\mathrm{cd \cdot m^{-2}}$ & 单位面积的面辐射源在某方向上的光强度 \\
    \hline
    光出射度 & $M_v$ & $ M_v = \frac{\mathrm{d}\Phi_v}{\mathrm{d}A}$ & $\mathrm{lm \cdot m^{-2}} $ & 辐射体单位面积向半球空间发射的光通量 \\
    \hline
    光谱密度 & $\phi_v$ & $\phi_v = \frac{\Delta \Phi_{v,\lambda}}{ \Delta\lambda} $ & $\mathrm{lm \cdot m^{-1}}$ & 光量(光通量)在频谱中的分布 \\
    \hline
\end{tabular}
\end{table}

它们\footnote{参考 \href{https://www.zhihu.com/question/53080536/answer/133398317}{知乎:如何区分并记忆光度、照度、发光强度、光强、亮度等}}之间的转化关系:
\begin{gather}
\text{与光通量的转换:} \Phi_v = \int \boldsymbol{E}_v \mathrm{d}\boldsymbol{A} = \int \boldsymbol{I}_v \mathrm{d}\boldsymbol{\Omega} = \iint \boldsymbol{L}_v \cos \theta\, \mathrm{d}A \,\mathrm{d} \boldsymbol{\Omega} \\ 
\text{与光强的转换:} \boldsymbol{I}_v = r^2\boldsymbol{E}_v = \int \boldsymbol{L}_v \cos \theta\, \mathrm{d}A = \int \boldsymbol{L}_v  \mathrm{d}A_{\perp}
\end{gather}

计算时的常用微分:
\begin{gather}
\begin{aligned}
    & \text{直角坐标系:}&& \mathrm{d}A = \mathrm{d}x \mathrm{d}y &&,  \mathrm{d}\boldsymbol{\Omega} = \frac{\mathrm{d}\boldsymbol{A}}{r^2} \\ 
    & \text{球坐标系:}&& \mathrm{d}A = r^2 \sin \theta \mathrm{d}\theta \mathrm{d}\phi &&,\mathrm{d}\Omega = \sin \theta \mathrm{d}\theta \mathrm{d}\phi \\ 
\end{aligned}
\end{gather}

\section{特殊发光体}

\subsection{余弦发光体(朗伯发光体)}

\subsection{定向发光体}

\chapter{光的反射与折射}\thispagestyle{fancy}

在本章,我们先以一定的顺序,依次对反射折射过程中所出现的现象或相关物理量进行讨论,最后给出所有现象的总结。

\section{菲涅尔公式}

\begin{BlockTheorem}[菲涅尔公式, Fresnel Formula]\label{菲涅尔公式}
光线在通过两介质分界面时通常会同时发生折射(透射)和反射现象,设入射光(incident ray)介质折射率 $\eta_i$,入射角 $\theta_i$,透射光(transmitted ray)介质折射率 $\eta_t$,透射角(折射角)$\theta_t$,则有\footnote{对于金属材质(非绝缘材质),需要引入消光系数 $k_t$ 来修正菲涅尔公式(绝缘材质等价于 $k_t = 0$),具体参见 \href{https://zhuanlan.zhihu.com/p/480405520?utm_psn=1818236176659771392}{知乎: 菲涅尔公式}}:


\begin{table}[H]
\centering
\renewcommand{\arraystretch}{1.6} % 调整行间距为默认值的1.5倍 
\begin{tabular}{|c|c|c|c|c|} 
\hline
类型 & \multicolumn{2}{c|}{振幅反射系数 $r$} & \multicolumn{2}{c|}{振幅透射系数 $t$ }  \\ 
\hline
$s$ 波 & $\displaystyle r_s = \frac{n_i\cos \theta_i - n_t \cos \theta_t}{n_i\cos \theta_i + n_t \cos \theta_t} $ & $\displaystyle  - \frac{\sin (\theta_i - \theta_t) }{\sin (\theta_i + \theta_t)}$ & $\displaystyle t_s  = \frac{2n_i \cos \theta_i}{n_i\cos \theta_i + n_t \cos \theta_t} $ &   $\displaystyle  + \frac{2 \sin \theta_t \cos \theta_i}{\sin (\theta_i + \theta_t)}$   \\ 
\hline
$p$ 波 & $\displaystyle r_p = \frac{n_t\cos \theta_i - n_i \cos \theta_t}{n_t\cos \theta_i + n_i \cos \theta_t} $ &     $ \displaystyle  + \frac{\tan (\theta_i - \theta_t)}{\tan (\theta_i + \theta_t)} $  &  $\displaystyle t_p  = \frac{2n_i \cos \theta_i}{n_i\cos \theta_t + n_t \cos \theta_i} $ &   $\displaystyle + \frac{2 \sin \theta_t \cos \theta_i}{\sin (\theta_i + \theta_t) \cos (\theta_i - \theta_t)}$                  \\
\hline
\end{tabular}
\end{table}

折射角 $\theta_t$、$s$ 波通量反射率 $R_s$、$p$ 波通量反射率 $R_p$ 和总通量反射率 $R$ 为:
\begin{equation}
    \cos \theta_t = \sqrt{1 - \left( \frac{\eta_i}{\eta_t} \sin \theta_i\right)^2},\quad R_s = r_s^2,\ R_p = r_p^2, \quad  R = \frac{1}{2}\left( R_s + R_p \right)
\end{equation}

总强度反射率 $R$ 的严格证明见下一节。特别地,若 $1 - \left( \frac{\eta_i}{\eta_t} \sin \theta_i\right)^2 < 0$,则发生全反射,此时 $R = 1$。另外,需要指出菲涅尔公式的适用条件,也即推导时所做的一些假设,如下:
\begin{enumerate}
\item 介质为绝缘介质,无表面自由电荷或传导电流
\item 介质为各向同性的光学线性介质(弱光强)
\item 介质磁导率(约)等于真空磁导率\footnote{对于介质磁导率不等于真空磁导率的情况,参考 \href{https://www.writebug.com/static/uploads/2024/9/2/3ed06af7e4f074f1964feb480a541a6b.pdf}{Optics (Eugene Hecht, 尤金) Page 144}} $\mu_i = \mu_t = \mu_0$,其中 $\mu_0$ 为真空磁导率。
\end{enumerate}
\end{BlockTheorem}

\section{反射时的相位变化}

菲涅尔公式的推导以矢量分析为基础,因此公式中系数 $r_s$ 的正负具有明确物理意义,它标识着方向。若为负,则反射后的方向与原方向相反,否则相同。各系数正负情况见表 \ref{振幅系数的正负情况},其中 o 表示可正可负。

%\footnote{拓展阅读 \href{https://zhuanlan.zhihu.com/p/607510257}{知乎:你一直没搞懂的半波损失(机械波、光波)}}

\begin{center}\noindent\begin{minipage}{0.65\columnwidth}
    \hspace*{2em} 从波的角度,方向相反可以等价地视为相位发生了 $\pi$ 的前移(或后移),称为相位突变。$n_i < n_t$ 时,相位突变要么是 0,要么是 $\pi$,$n_i > n_t$ 时的相位变化比较复杂,我们不深究。在 $\theta_i + \theta_t = \frac{\pi}{2}$ 时,$r_p$ 的正负发生变化,$p$ 波的反射波相位也发生突变,称此时 $\theta_i$ 的角度为布儒斯特角(Brewster angle),记为 $\theta_B$,也称为偏振角或起偏角。
\end{minipage}\hfill\begin{minipage}{0.3\columnwidth}
    \begin{table}[H]\centering
        % \setlength{\tabcolsep}{1.5mm} % 调整列间距
            \caption{\textbf{振幅系数的正负情况}}
            \label{振幅系数的正负情况}
        \begin{tabular}{cccccccccc}\toprule
            折射率 & $r_s$& $r_p$ & $t_s$ & $t_p$\\
            \midrule                        
            $n_i < n_t$ & $-$ & o & $+$ & $+$\\
            $n_i > n_t$ & o & o & $+$ & $+$\\
            \bottomrule
        \end{tabular}
    \end{table}
\end{minipage}\end{center}
可以推得 Brewster angle 的值为:
\begin{equation}
    \theta_B = \arctan \left( \frac{n_2}{n_1} \right)
\end{equation}



具体的振幅系数变化见图 \ref{振幅系数随入射角的变化},见图 \ref{反射时 s 波和 p 波的相位变化},$n_i < n_t$ 时的反射示意图见图 \ref{反射示意图}。

\begin{figure}[H]\centering
\begin{subfigure}[t]{0.49\textwidth}\centering
    \includegraphics[height=190pt]{assets/1,2/2024-09-15_10-53-31.pdf}
    \caption{\bfseries 由空气入射玻璃($n_i = 1,\ n_t = 1.5$) }
\end{subfigure}
\begin{subfigure}[t]{0.49\textwidth}\centering
    \includegraphics[height=190pt]{assets/1,2/2024-09-15_10-53-27.pdf}
    \caption{\bfseries 由玻璃入射空气($n_i = 1.5,\ n_t = 1$) }
\end{subfigure}
\caption{\bfseries 振幅系数 $r$ 随入射角 $\theta_i$ 的变化 }\label{振幅系数随入射角的变化}
\end{figure}

\begin{figure}[ht]\centering
    \includesvg[width=0.93\textwidth]{assets/1,2/相位突变.svg}
    \caption{\bfseries $s$ 波和 $p$ 波在反射时的相位变化}\label{反射时 s 波和 p 波的相位变化}
\end{figure}

\begin{figure}[H]\centering
\includesvg[width=0.95\textwidth]{assets/1,2/反射后的方向情况.svg}
\caption{\bfseries 由空气入射玻璃的光线示意图}\label{反射示意图}
\end{figure}


由菲涅尔公式,当 $n_i < n_t$ 时,我们还有如下结论:
\begin{gather}
    \begin{aligned}
        &\text{$\theta_i = 0$ 时:} &&r_p = (-r_s)  = \frac{n_t - n_i}{n_t + n_i}, &&t_p = t_s = \frac{2n_i}{n_i + n_t} \\ 
        &\text{$\theta_i = \frac{\pi}{2}$ 时:} &&r_p = r_s  = -1,&&t_p = t_s  =0
    \end{aligned}
\end{gather}
这表明,即使是正射(垂直于介质分界面的入射,$\theta_i = 0$),一般也存在部分反射光。总之,当 $n_i < n_t$ 时,入射光的 s 分量在反射中一定会相位跃变,p 分量都有可能。

另外,菲涅尔公式还可写成:
\begin{gather}
\boxed{
\begin{aligned}
    &\quad\quad \quad \quad   (-r_s) + t_s  = 1, &&\quad 
    \quad \quad\quad   r_p + t_p = 1 
    \\ 
    &r_s = \frac{\cos \theta_i - \sqrt{n_{ti}^2 - \sin^2 \theta_i} }{\cos \theta_i + \sqrt{n_{ti}^2 - \sin^2 \theta_i}}, && r_p = \frac{n_{ti}^2\cos \theta_i - \sqrt{n_{ti}^2 - \sin^2 \theta_i} }{n_{ti}^2\cos \theta_i + \sqrt{n_{ti}^2 - \sin^2 \theta_i}}
\end{aligned}
}
\end{gather}

\section{完全偏振反射光}

当光波由布儒斯特角 $\theta_B$ 入射时,由 Fresnel Formula,$r_p = \frac{\tan (\theta_i - \theta_t)}{\tan (\theta_i + \theta_t)} = 0$,也即反射光的 p 分量为 0,仅存在 s 分量。这说明反射光是完全偏振光,$\boldsymbol{E}$ 的方向(称为振动方向)垂直于入射面。
{\par\color{gray}\small
但此时反射光能量占比 $F$ 很小\footnote{可以使用玻璃片堆得到强度较大的偏振光},例如,空气($n=1$)入射玻璃($n = 1.5$)时,$\theta_B = 56.310^\circ$,$F =0.0740 $;玻璃入射空气时,$\theta_B = 33.690^\circ$,$F =0.0740 $。
\par}

    % \begin{matlablisting}
    % n_i = 1;
    % n_t = 1.5;
    % 
    % theta_B = atan(n_t/n_i);
    % theta_t = @(theta_i) asin(n_i/n_t*sin(theta_i));
    % r_s = @(theta_i, theta_t) - sin(theta_i - theta_t)./sin(theta_i + theta_t);
    % r_p = @(theta_i, theta_t) + tan(theta_i - theta_t)./tan(theta_i + theta_t);
    % F = @(theta_i, theta_t) 0.5*( r_s(theta_i, theta_t)^2 + r_p(theta_i, theta_t)^2 );
    % 
    % theta_i = theta_B;
    % disp(['theta_B = ', num2str(rad2deg(theta_B))])
    % disp(['F = ', num2str(F(theta_i, theta_t(theta_i)))])
    % 
    % %{ 
    % Output:
    % theta_B = 56.3099
    % F = 0.073964
    % %}
    % \end{matlablisting}


\section{反射折射时的能量关系}

在 Fresnel Formula 中可以发现,$r_s^2 + t_s^2 \ne 1$,$r_p^2 + t_p^2 \ne 1$,是能量不守恒了吗?显然不是。那么,反射光和透射光的能量关系是怎样的?这需要借助辐射度学的相关概念。

如图,圆形光束从空气入射到分界面上的一个面元 $\boldsymbol{A}$(界面下是玻璃),以此面元为研究对象。考虑玻印亭矢量 $\boldsymbol{S} = \boldsymbol{E} \times \boldsymbol{B}$,即单位时间内通过单位面积的电磁辐射能量(单位面积辐射功率),于是瞬时辐射照度 $\boldsymbol{E}_e$: 
\begin{equation}
\boldsymbol{E}_e = \boldsymbol{S} = c^2 \varepsilon_0 \boldsymbol{E} \times \boldsymbol{B},\quad  E_e = \varepsilon_0 cE^2 \cdot \sqrt{\frac{\varepsilon_r}{\mu_r}} = \frac{\varepsilon_0 c}{\mu_r}\cdot  nE^2
\end{equation}
其中 $\varepsilon_r$,$\mu_r$ 分别为相对介电常量、相对磁导率,对空气近似有 $\varepsilon_r = \mu_r = 1$,于是

核心思想是 $\mathrm{d} Q_e = (\boldsymbol{E}_e \cdot \boldsymbol{A})\, \mathrm{d}t $。入射、反射、透射光束的截面面积分别为 $A \cos \theta_i,\ A \cos \theta_r,\ A \cos \theta_t$,设其瞬时辐射照度分别为$\boldsymbol{E}_i,\ \boldsymbol{E}_r,\ \boldsymbol{E}_t$ ,则辐射通量为:
\begin{equation}
\Phi_{e, k} = E_{e, k} A \cos \theta_k = \frac{\varepsilon_0 cA }{\mu_r}\cdot  n_k\cos \theta_k\,E_k^2,\quad  k = i, r, t
\end{equation}

分别写出入射、反射、透射光的辐射通量:
\begin{align}
\Phi_{e,i} &= \frac{\varepsilon_0 cA }{\mu_r} \cdot  n_i\cos \theta_i\,E_i^2 
= \frac{\varepsilon_0 cA }{\mu_r} \cdot  n_i\cos \theta_i\, \left( E_{i,s}^2 + E_{i,p}^2 \right) 
\\ 
\Phi_{e,r} 
&= \frac{\varepsilon_0 cA }{\mu_r} \cdot  n_i\cos \theta_i\,E_r^2 
= \frac{\varepsilon_0 cA }{\mu_r} \cdot  n_i\cos \theta_i\, \left( r_s^2E_{i,s}^2 + r_p^2E_{i,p}^2 \right) 
\\ 
\Phi_{e,t} 
&= \frac{\varepsilon_0 cA }{\mu_r} \cdot  n_t\cos \theta_t\,E_t^2 
= \frac{\varepsilon_0 cA }{\mu_r} \cdot  n_t \cos \theta_t \, \left( t_s^2E_{i,s}^2 + t_p^2E_{i,p}^2 \right) 
\end{align}

由于入射光可分解为 $s$ 波与 $p$ 波,我们自然想到它们俩在入射前后应该是能量守恒的,这指导我们分别作数学上的处理。对 $s$ 波,由菲涅尔定律(这说明已经做了近似 $\mu_r = 1$),做减法得到:
\begin{align*}
&n_i \cos \theta_i E_{i,s}^2 - n_i \cos \theta_i r_{i,s}^2E_{i,s}^2 -   n_t \cos \theta_t t_s^2E_{i,s}^2 \\
&= E_{i,s}^2 \left[ n_i\cos \theta_i \left( 1 - \frac{(n_i\cos \theta_i - n_t \cos \theta_t)^2}{(n_i\cos \theta_i + n_t \cos \theta_t)^2} \right) - n_t \cos \theta_t \cdot \frac{(2n_i \cos \theta_i)^2}{(n_i\cos \theta_i + n_t \cos \theta_t)^2} \right] \\ 
& = \frac{E_{i,s}^2}{(n_i\cos \theta_i + n_t \cos \theta_t)^2} \left[ n_i \cos \theta_i \cdot (4 n_i\cos \theta_i \cdot n_t \cos \theta_t) - 4 n_t \cos \theta_t \cdot n_i^2 \cos^2 \theta_i\right] \\ 
& = 0
\end{align*}

同理,考虑 p 分量,作减法可得到:
\begin{equation*}
n_i \cos \theta_i E_{i,p}^2 - n_i \cos \theta_i r_{i,p}^2E_{i,p}^2 -   n_t \cos \theta_t t_p^2E_{i,p}^2 = 0
\end{equation*}

代入即得:
\begin{equation}
    \Phi_{e,i} - \Phi_{e,r} - \Phi_{e,t} = 0 \Longrightarrow  \Phi_{e,i} = \Phi_{e,r} + \Phi_{e,t}
\end{equation}
这便验证了入射前后的能量是守恒的。

由此,我们可以定义一些能量系数:
\begin{gather}
\begin{aligned}
    &\text{强度反射率 $R$: }\ &&R = \frac{1}{2}(R_s + R_p), &&R_s = r_s^2, && R_p = r_p^2 \\
    &\text{强度透射率 $T$: }\ &&T = \frac{1}{2}(T_s + T_p), &&T_s = \left( \frac{n_t \cos \theta_t}{n_i \cos \theta_i} \right)^2 t_s^2, && T_p = \left( \frac{n_t \cos \theta_t}{n_i \cos \theta_i} \right)^2 t_p^2 \\
\end{aligned}
\end{gather}

这样,它们具有下面的性质,方便我们计算能量关系:
\begin{gather}
\boxed{
\begin{aligned}
    &R_s + T_s = 1, &&\quad  R_p + T_p = 1,&&\ \  R + T = 1 \\ 
    &\Phi_{e,r} = R\Phi_{e,i}, && 
    \Phi_{e,r,s} = R_s\Phi_{e,i,s}, &&  \Phi_{e,r,p} = R_p\Phi_{e,i,s} 
    \\ 
    &\Phi_{e,t} = T\Phi_{e,i}, && \Phi_{e,t,s} = T_s\Phi_{e,i,s}, && \Phi_{e,t,p} = T_p\Phi_{e,i,s}
\end{aligned}
}
\end{gather}
{\color{red} 其中 $\Phi_{e,r} = R\Phi_{e,i}$ 和 $\Phi_{e,t} = T\Phi_{e,i}$ 是怎么来的?}




\section{全反射时的隐失波与穿透深度}

假设现在由光密介质射向光疏介质,即 $n_i > n_t$,则有临界角 $\theta_C = \arcsin n_{ti}$。当 $\theta_i > \theta_C$ 时,发生全反射,$R = 1,\ T = 0$,若简单地认为没有任何透射光,是不满足电磁场边界条件的。具体来讲,$\boldsymbol{E}$ 的切向分量连续告诉我们,在透射介质中一定存在振荡场,它在平行于界面上的分量具有时间频率 $\omega$(与入射光相同)。

进一步的推导表明\footnote{详见参考文献 \cite{Optics} 的 Page 158},在透射介质中存在一种波(称为隐失波),其波函数如下:
\begin{equation}
\boldsymbol{E} = \left( e^{-\beta y} \boldsymbol{E_{t,0}}\right)\cdot e^{i\left( \frac{\sin \theta_i}{n_{ti}}  k_t x - \omega t \right)},\quad \text{衰减系数}\  \beta = k_t \sqrt{\frac{\sin^2 \theta_i}{n_{ti}^2} - 1} = k_i\sqrt{\sin^2 \theta_i - n_{ti}^2} 
\end{equation}

这是一个不均匀波,其振幅在 $y$ 方向上极速衰减,只在几个波长的距离上就可以忽略不计。且它同时有纵波成分和横波成分,不是简单的简谐横波。

我们将振幅下降到 $\frac{1}{e}$ 的深度称为\textbf{穿透深度},记为 $\delta = \frac{1}{\beta} $,它通常在一个波长以内。

对于此过程的能量守恒问题,更详尽广泛的讨论表明(利用波印廷矢量 $\boldsymbol{S}$),能量实际上是跨过界面往复循环,最终使透向第二介质的净流量为零。就现阶段,可以理解为能量从入射波流到隐失波再回到反射波,或者说隐失波沿入射波又绕回了反射波。

\section{古斯-亨欣位移(Goos-Hanchen Shift)}

一束被全反射的光,入射点会与(反射后的)出射点存在微小偏移(事实上既有平行偏移也有垂直偏移),称为 Goos-Hanchen Shift。较为严谨的推导表明\footnote{详见参考文献 \cite{GHShift},或者 \href{https://www.zhihu.com/question/446676895/answer/3407740051}{知乎:古斯汉欣位移产生的原因 (https://www.zhihu.com/question/446676895/answer/3407740051)},以及 \href{https://www.zhihu.com/question/620522351/answer/3209865128}{知乎:古斯汉森位移的原理是什么 (https://www.zhihu.com/question/620522351/answer/3209865128)}},沿入射方向、与分界线平行的偏移量如下(又称为侧向偏移):
\begin{equation}
\delta_{\perp} = \frac{\lambda_i \sin \theta_i}{\pi \sqrt{\sin^2 \theta_i - n_{ti}^2} },\quad \Delta x =  \frac{\lambda_i \tan \theta_i}{\pi \sqrt{\sin^2 \theta_i - n_{ti}^2} } = 2 \delta \tan \theta_i 
\end{equation}

\begin{figure}[H]\centering
\includegraphics[width=0.8\columnwidth]{assets/1,2/GHShift.pdf}
\caption{\bfseries Goos-Hanchen Shift}\label{Goos-Hanchen Shift}
\end{figure}

\section{全反射时的相位变化}\label{全反射时的相位变化}


发生全内反射时\footnote{全内反射是指,由光疏介质射向光密介质且入射角大于临界角时发生的全反射现象},入射波 s 分量、p 分量的相位变化并非简单的 0 或 $\pi$,下面作推导。

对入射光的波函数 $\boldsymbol{E_i} = \boldsymbol{E_{i,0}} \cdot e^{i \theta} =\boldsymbol{E_{i,0}} \cdot e^{i(k x - \omega t)}$,若反射光满足 $\boldsymbol{E_{r}} = \boldsymbol{E_i}\cdot \lambda e^{i \delta}$,则表明相对于入射光,反射光的振幅变为了原来的 $\lambda$倍,且相位增加了 $\delta$。特别地,$\lambda < 0$ 时,可以等价于 $\lambda > 0$ 且相位增加 $\delta + \pi$ 或 $\delta - \pi$。

由菲涅尔定律,我们有 $\boldsymbol{E_{r,s}} = r_s \boldsymbol{E_{i,s}},\quad \boldsymbol{E_{r,p}} = r_p \boldsymbol{E_{i,p}}$。可以发现,在全反射时,$r_s, r_p \in \C \setminus \R$,并且 $| r_s | = | r_p | = 1$,振幅不变,于是可以令 $r = e^{i \delta }$。为了反解相位增量 $\delta $,一种自然的想法是考虑 
\begin{equation}
    e^{i \delta} = \cos \delta + i \lambda \sin \delta = a + ib \Longrightarrow  \delta =  \arctan \left( \frac{b}{a} \right)
\end{equation}

这样做虽然可行,但由于 $\arctan $ 函数的局限性,其值域范围在 $[-\frac{\pi}{2}, \frac{\pi}{2}]$,而 $\delta $ 的取值范围在 $[0, \pi]$ 或者 $[-\pi, 0]$。因此,最终得到的 $\delta$ 仅在部分区域上正确,对另一部分需做数学上的平移修正。因此,我们考虑另一种方法。在全反射时,注意到 $r_s$ 和 $r_p$ 的形式为 $r = \frac{a - bi}{a +bi}$,其中 $a, b \in \R$,有如下过程:
\begin{equation}
\frac{a - bi}{a + bi} = e^{i \theta} \Longrightarrow e^{i \frac{\theta}{2}} = \pm \frac{a - bi}{\sqrt{a^2 + b^2} },\ \tan \frac{\delta}{2} = - \frac{b}{a},\ \frac{\delta}{2} = \arctan \left( - \frac{b}{a} \right)
\end{equation}
这样得到的 $\frac{\delta}{2}$ 便是全范围正确的,无需修正。分别令 $r = r_s, r_p$,代入即得:
\begin{gather}
\delta_{r,s} = - 2 \arctan \left( \frac{\sqrt{\sin^2 \theta_i - n_{ti}^2} }{\cos \theta_i} \right) 
,\quad 
\delta_{r,p} = - 2 \arctan \left( \frac{\sqrt{\sin^2 \theta_i - n_{ti}^2} }{n_{21}^2 \cos \theta_i} \right)
\end{gather}

\section{折射时的相位变化}

入射光不发生全反射时,由菲涅尔定律,$t_s, t_p \in (0, \frac{2n_i}{n_i + n_t}) \subset \R$,恒为正实数,因此相位不发生任何变化。当入射光发生全反射时,折射光(透射光)以隐失波的形式存在,我们前面已经提过,隐失波同时含有横波纵波成分,它与入射光不再是同一种波,此时谈论相位变化自然没有意义。


\section{反射折射总结}

到此,我们已经讨论了目前可能接触到的所有情况,包括光疏射向光密、光密射向光疏、小于或大于临界角时的折射反射振幅、相位以及能量关系等,终于可以给出反射折射的总结。

整个结论由菲涅尔定律推导而来,基于电磁场的边界条件和麦克斯韦方程组,从波的角度揭示了光在反射折射时发生的变化,包括振幅、相位、能量、位移等关系。
\begin{align}
&\text{反射波:} \quad \boldsymbol{E_r} = \boldsymbol{E_{r,s}} + \boldsymbol{E_{r,p}} = r_s \boldsymbol{E_{i,s}} + r_p \boldsymbol{E_{i,p}},\quad  r \in \C 
\\ 
&\text{透射波:}\quad \boldsymbol{E_t} = \boldsymbol{E_{t,s}} + \boldsymbol{E_{t,p}} = t_s \boldsymbol{E_{i,s}} + t_p \boldsymbol{E_{i,p}},\quad  t \in \R,\ \theta_i < \theta_C
\\ 
& \text{反射系数:}
    r_s = \frac{\cos \theta_i - \sqrt{n_{ti}^2 - \sin^2 \theta_i} }{\cos \theta_i + \sqrt{n_{ti}^2 - \sin^2 \theta_i}},\quad  r_p = \frac{n_{ti}^2\cos \theta_i - \sqrt{n_{ti}^2 - \sin^2 \theta_i} }{n_{ti}^2\cos \theta_i + \sqrt{n_{ti}^2 - \sin^2 \theta_i}}
\\ 
& \text{透射系数:}(-r_s) + t_s  = 1,\quad r_p + t_p = 1 
\\
& \text{能量关系:}
\begin{cases}
    R = \frac{1}{2}(R_s + R_p)\quad R_s = | r_s |^2,\ R_p = | r_p |^2 \\ 
    T = \frac{1}{2}(T_s + T_p),\quad 
    T_s = \left( \frac{n_t \cos \theta_t}{n_i \cos \theta_i} \right)^2 t_s^2, T_p = \left( \frac{n_t \cos \theta_t}{n_i \cos \theta_i} \right)^2 t_p^2 \\
    R + T = 1 ,\ R_s + T_s = 1,\ R_p + T_p = 1 \\ 
    \Phi_{e,r} = R\Phi_{e,i},\ \Phi_{e,t} = T\Phi_{e,i}
\end{cases}
\\
& \text{$s$ 波反射相位增量:} 
\delta_{r,s} = 
\left\{\begin{matrix}
    -\pi, & n_i < n_t \\ 
    \begin{Bmatrix}
        0, & \theta_i \in (0, \theta_C)\\
        - 2 \arctan \left( \frac{\sqrt{\sin^2 \theta_i - n_{ti}^2} }{\cos \theta_i} \right), & \theta_i > \theta_C
    \end{Bmatrix}, & n_i > n_t
\end{matrix}\right.
\\
& \text{$p$ 波反射相位增量:} 
\delta_{r,p} = 
\left\{\begin{matrix}
    \begin{Bmatrix}
        0, & \theta_i \in (0, \theta_B) \\
        - \pi & \theta_i \in (\theta_B, \frac{\pi}{2})
    \end{Bmatrix}, & n_i < n_t 
    \\ 
    \begin{Bmatrix}
        -\pi, & \theta_i \in (0, \theta_B) \\
        0, & \theta_i \in (\theta_B, \theta_C) \\
        - 2 \arctan \left( \frac{\sqrt{\sin^2 \theta_i - n_{ti}^2} }{n_{ti}^2 \cos \theta_i} \right), & \theta_i \in (\theta_C, \frac{\pi}{2})
    \end{Bmatrix}, & n_i > n_t
\end{matrix}\right.
\\
&\text{隐失波:} \boldsymbol{E_{t}} = \left( e^{-\beta y} \boldsymbol{E_{t,0}}\right)\cdot e^{i\left( \frac{\sin \theta_i}{n_{ti}}  k_t x - \omega t \right)},\quad \beta  = k_i\sqrt{\sin^2 \theta_i - n_{ti}^2} = \frac{2 \pi}{\lambda_i} \sqrt{\sin^2 \theta_i - n_{ti}^2},\quad \delta = \frac{1}{\beta}
\\ 
& \text{Goos-Hanchen Shift: \ } \Delta x = 2 \delta \tan \theta_i = \frac{2 \tan \theta_i}{k_i\sqrt{\sin^2 \theta_i - n_{ti}^2}}
= \frac{\lambda_i \tan \theta_i }{\pi \sqrt{\sin^2 \theta_i - n_{ti}^2} }
\end{align}

\newpage
\begin{figure}[H]\centering
\begin{subfigure}[t]{0.44\columnwidth}\centering
    \includegraphics[height=170pt]{assets/1,2/2024-09-21_12-26-37.pdf}
    \caption{\bfseries 由空气入射玻璃 }
\end{subfigure}
\subcaptionsetup{
    margin={2cm,0cm}, % 右边距不变,左边距为 2cm
    %labelsep=period % 标签与标题内容之间的分隔符为点号
    }
\begin{subfigure}[t]{0.54\columnwidth}\centering
    \includegraphics[height=170pt]{assets/1,2/2024-09-21_12-31-00.pdf}
    \caption{\bfseries 由玻璃入射空气 }
\end{subfigure}
\caption{\bfseries 反射折射光的振幅与能量变化 }
\label{反射折射光的振幅与能量变化}
\end{figure}

\begin{figure}[H]\centering
\begin{subfigure}[t]{0.5\columnwidth}\centering
    \includegraphics[height=175pt]{assets/1,2/2024-09-21_13-19-43.pdf}
    \caption{\bfseries 由空气入射玻璃 }
\end{subfigure}\begin{subfigure}[t]{0.5\columnwidth}\centering
    \includegraphics[height=175pt]{assets/1,2/2024-09-21_13-19-45.pdf}
    \caption{\bfseries 由玻璃入射空气 }
\end{subfigure}
\caption{\bfseries 反射光 s 分量与 p 分量的相位增量 }
\label{反射光 s 分量与 p 分量的相位增量}
\end{figure}

\begin{figure}[H]\centering
\includegraphics[width=0.70\columnwidth]{assets/1,2/2024-09-21_13-29-49.pdf}
\caption{\bfseries 隐失波穿透深度与 GH Shift(玻璃入射空气)}\label{隐失波穿透深度与 GH Shift 玻璃入射空气}
\end{figure}

图 \ref{反射折射光的振幅与能量变化} 展示了反射折射光的振幅 $r_s, r_p, t_s, t_p$、能量 $R_s, R_p, R$ 随入射角 $\tan \theta_i$ 的变化\footnote{源码见附录 \ref{反射折射光的振幅与能量变化 源码}},其中 (a) 图为空气入射玻璃($n_i = 1, n_t = 1.5$),(b) 图为玻璃入射空气($n_i = 1.5, n_t = 1$)。

图 \ref{反射光 s 分量与 p 分量的相位增量} 展示了反射光的 s 分量与 p 分量的相位增量 $\delta_{r,s}, \delta_{r,p}$ 随入射角 $\theta_i$ 的变化\footnote{源码见附录 \ref{反射光 s 分量与 p 分量的相位增量 源码}},其中 (a) 图为空气入射玻璃($n_i = 1, n_t = 1.5$),(b) 图为玻璃入射空气($n_i = 1.5, n_t = 1$)。特别地,当 (b) 图中 $\theta_i > \theta_C$ 时,发生全(内)反射,此时 $r_s, r_p, t_s, t_p \in \C \setminus \R$ ,图中展示的是它们的模长,即 $|r_s|, |r_p|, |t_s|, |t_p|$。

图 \ref{隐失波穿透深度与 GH Shift 玻璃入射空气} 展示了隐失波穿透深度 $\delta$ 和 GH SHift $\Delta x$ 随入射角 $\theta_i$ 的变化\footnote{源码见附录 \ref{隐失波穿透深度与 GH Shift 玻璃入射空气 源码}}。

\chapter{光的干涉}\thispagestyle{fancy}

\section{光波的叠加和干涉}

平面波与球面波\footnote{也即平面电磁波与球面电磁波,两者概念详见 \href{https://www.zhihu.com/question/267133640/answer/319531458}{知乎:球面光波与平面光波 (https://www.zhihu.com/question/267133640/answer/319531458)} 和 \href{https://www.zhihu.com/question/534511391/answer/2501271591}{知乎:高斯光束,平面波,球面波三者间有什么关系 (https://www.zhihu.com/question/534511391/answer/2501271591)}}是光波的的基元,当两个光源(或两束光波)间存在某种关联,波的叠加会引起强度的重新分布,这种现象称为光的干涉。












\section{分波前干涉}
\section{分振幅干涉}
\section{等倾干涉与等厚干涉}
\section{迈克尔逊干涉与}
\section{光场的空间相干性与时间相干性}
\section{多光束干涉}
\section{激光}







































































































\nocite{*}
\bibliography{re}
\thispagestyle{fancy} 
\addcontentsline{toc}{chapter}{参考文献}





















% --------------------------- 附录 --------------------------- %
% >> ------------------------ 附录 ------------------------ << %


\newpage
\appendix
% chapter 标题自定义设置
\titleformat{\chapter}[hang]{\normalfont\huge\bfseries\centering}{}{20pt}{}
\titlespacing*{\chapter}{0pt}{-25pt}{8pt} % 控制上方空白的大小
% section 标题自定义设置 
\titleformat{\section}[hang]{\normalfont\centering\Large\bfseries}{\thesection}{8pt}{}

% 附录 A
\chapter*{附录 A\hspace*{20pt} 波理论}\addcontentsline{toc}{chapter}{附录 A\hspace*{6pt} 波理论}   
\thispagestyle{fancy} 
\setcounter{section}{0}   
\renewcommand\thesection{A.\arabic{section}}   
\renewcommand{\thefigure}{A.\arabic{figure}} 
\renewcommand{\thetable}{A.\arabic{table}}


光的真实本性是光学的全部讨论的中心问题, 在本书中我们从头到尾都得对待这个问题。“光究竟是一种波动现象还是一种粒子现象?” 这个似乎干脆利索的问题, 远比它初看之下复杂得多。

因为对光的经典讨论和量子力学讨论都要用到波的数学描述, 本章要为这两种表述所需
要的东西打好基础。下面叙说的想法将用于一切物理波, 从一杯茶的表面张力皱波, 到从某个遥远的星系照到我们的光脉冲。

\section{一维波}

\subsection{$n$ 维波的概念}

一维波指的是在一维空间中传播的波,或者可以看作在一维空间中传播的波。例如一束光在空间中传播,沿其传播方向建立 $x$ 轴,则有 $E = E_0 e^{kx - \omega t}$(具有正负),这束光便可视为一维波。

一维波函数的最一般的形式:
\begin{equation}
\psi(x,t) = f(x-vt) = g(kx - \omega t)
\end{equation}

具体而言,对于给定的波形(波的形状),我们只需令 $t=0$,拍一张“照片”(例如 $\psi(x) = \frac{3}{10x^2+1}$),得到 $\psi(x,0) = f(x)$,然后将 $f(x)$ 中的 $x$ 换为 $x-vt$,即可得到一个以速度 $v$(可为负) 向 $x$ 轴正方向运动的波 $\psi(x,t) = f(x - vt) = g(kx - \omega t)$。
{\par\color{gray}\small
绳索的上下振动是在第二个维度上的,但振动导出的波仍是一维波。
\par}


\subsection{波动方程}

线性、各向同性、无损耗介质中的波动方程(也称波动微分方程)为:
\begin{equation}
    \frac{\partial^{2}\psi}{\partial x^{2}}=\frac{1}{v^{2}}\frac{\partial^{2}\psi}{\partial t^{2}} 
\end{equation}

如果代表一个波的函数 $\psi$ 是这个方程的解, 它将同时是 $(x-vt)$ 的函数(即 $kx - \omega t$ 的函数),它还是一个可以同时对 $x$ 和 $t$ 以非平庸方式求二次微商的函数。特别地,我们有结论:$\psi$ 是一维波函数 $\Longleftrightarrow$ $\psi$ 是 $(x-vt)$ 的二次可微函数 $\Longleftrightarrow$ $\psi$ 是 $(kx - \omega t)$ 的二次可微函数。

三维空间中的波动方程可以写为:
\begin{equation}
    \Delta \boldsymbol{\psi} =\frac{1}{v^{2}}\frac{\partial^{2}\boldsymbol{\psi}}{\partial t^{2}}
\end{equation}
其中拉普拉斯算子 $\Delta$ 和矢量微分定义为:
\begin{equation}
    \Delta = \frac{\partial^2  }{\partial x^2 } +\frac{\partial^2  }{\partial y^2 } + \frac{\partial^2  }{\partial z^2 },\quad 
    \frac{\partial \boldsymbol{A} }{\partial x } =\left( \frac{\partial A_x }{\partial x }, \frac{\partial A_y }{\partial x }, \frac{\partial A_z }{\partial x } \right)  ,\quad 
    \frac{\partial^2 \boldsymbol{A} }{\partial x^2 } = \left( \frac{\partial^2 A_x }{\partial x^2 }, \frac{\partial^2 A_y }{\partial x^2 }, \frac{\partial^2 A_z }{\partial x^2 } \right)
\end{equation}


\section{谐波}

\subsection{相位和相速度}
考虑任何一个一维波函数 $\psi(x,t) = A \cos(\phi(x,t)) = A \cos (kx - \omega t + \phi_0) $,其中 $\phi = kx+vt + \phi_0$ 称为相位,$\phi_0$ 称为初相(也常用 $\varepsilon$ 表示)。只要相位中的 $kx$ 与 $\omega t$ 符号相反,即 $(kx - \omega t)$ 或 $(\omega t - kx)$,则波沿 $x$ 轴正方向传播,否则沿 $x$ 轴负方向。


\subsection{谐波的概念} 
谐波,指简谐波、正弦波,其轮廓图是正弦曲线,是最简单的波形。在后续的傅里叶变换一节我们可以看到,任何波形都可以由谐波叠加合成,因此谐波具有特殊的意义。
考虑如下波形:
\begin{equation}
    \psi(x,\:t)\big|_{t=0}=\psi(x)=A\:\sin kx=f(x)
\end{equation}
其中 $k>0$ 是一个常数,称为传播数(空间角频率),且 $k = \frac{2\pi}{\lambda} $($\lambda$ 为波长),$A$ 称为波的振幅。\par

谐波函数有多种等价形式,其中最常见的是:
\begin{equation}
    \psi(x,t)=A\sin(kx\mp\omega t) ,\quad \psi(x,t)=A\sin \left(  \kappa (x\mp vt) \right)
\end{equation}
在本书中,如无特殊需求,我们都采用前者,也即 $\psi = A\sin(kx\mp\omega t)$,有时也采用 $\psi = A\cos(kx\mp\omega t)$。当然,三维谐波(在三维空间中传播的谐波)可写为:
\begin{equation}
    \boldsymbol{\psi}=\boldsymbol{\psi_0}\sin(\boldsymbol{k}\cdot \boldsymbol{x} \mp \omega t)
\end{equation}

\subsection{空间频率 $\kappa$ 与空间角频率 $k$}

光学中常用的长度单位是纳米 $\mathrm{nm}$、微米 $\mathrm{\mu m}$ 和埃米 $1\ \si{\angstrom}$ $ = 10^{-10}\ \mathrm{m}$。本文规定,若无特殊情况,一般用 $\lambda$ 表示波长,$\tau $ 或 $T$ 表示周期,$\nu = \frac{1}{\tau}$ 表示时间频率,$\omega = 2\pi\nu$ 表示时间角频率,空间频率(波数)$\kappa = \frac{1}{\lambda}$,空间角频率(传播数)$k = 2\pi \kappa$。

光学信息可以以一种周期性方式散布在空间里,很像一个波的截图,我们可以将其视作静止($v=0$)的波,并用空间频率 $\kappa$ 来描述它们。

\begin{figure}[H]\centering
\begin{subfigure}[t]{0.62\columnwidth}\centering
    \includegraphics[height=120pt]{assets/1,2/image (41).jpg}
    \caption{\bfseries 空间频率较低的正弦亮度分布 }
\end{subfigure}\begin{subfigure}[t]{0.37\columnwidth}\centering
    \includegraphics[height=120pt]{assets/1,2/image (42).jpg}
    \caption{\bfseries 空间频率较高的正弦亮度分布 }
\end{subfigure}
\caption{\bfseries 正弦亮度分布 }
\end{figure}

\section{复数表示}
在之后的学习会看到,用余弦或正弦函数描述波函数会带来很多不便,而复数表示在大多时候显得尤为有效,因此引入复数表示是极有必要的。在本书中,为表示某个变量(物理量)是复数,我们在其上加一波浪号,例如 $\tilde{z}$ 或 $\tilde{E}$。

习惯上,我们用复数的实部来描述谐波,例如将 $\psi = A \cos(kx - wt + \varepsilon)$ 写为 $\psi = \Re [A e^{i (kx - wt + \varepsilon)}]$。
为了方便,常常把 $\Re$ 省略不写,即:
\begin{equation}
    \psi(x,t) = A e^{i \theta} = A e^{i (kx - wt + \varepsilon)}
\end{equation}
在后文,我们也采用此简写。需要时刻谨记,真实的波是实部,虚部没有物理意义。

另外,虽然复数表示在物理中十分常见,但应用它时需要时刻小心,只有运算限于加法、减法、乘除实数、对实变量进行微分和积分时,才能恢复实部。乘法运算(包括数乘、点乘和叉乘)必须仅与实数进行,否则会得到错误结论\footnote{这里有一个疑问,在 \ref{全反射时的相位变化} 节(全反射时的相位变化),推导反射光相位变化时,我们利用了 $\boldsymbol{E_r} = \boldsymbol{E_i}\cdot \lambda e^{i \delta} $ 所带来的相位变化,如何保证或说明这样做能得到正确的结果?}。例如 $\Re \tilde{z}_1 \cdot \Re \tilde{z}_2 \ne \Re (\tilde{z}_1\cdot \tilde{z}_2)$,$\Re \boldsymbol{\tilde{A}_1 }\cdot \Re \boldsymbol{\tilde{A}_2 }\ne \Re (\boldsymbol{\tilde{A}_1} \cdot \boldsymbol{\tilde{A}_2})$。

\section{相矢量和波的相加}

相矢量(也称复振幅、旋转矢量)是将谐波 $\psi = A e^{i (kx - wt + \varepsilon)}$ 中的位置变量 $x$ 或时间变量 $t$ 分离出来,以得到复平面上的矢量,常用于计算振幅\footnote{我们将在 3.1 节讨论波的叠加时使用相矢量,并讨论相矢量相加时所代表的意义}等。

\subsection{分离 $x$ 并随 $t$ 旋转}

考虑谐波 $\psi = \psi_0 e^{i(kx - \omega t + \varepsilon)}$,对于任意给定的 $x$,令 $\alpha = kx + \varepsilon$,谐波可写为 $\psi = \psi_0 e^{i(- wt + \alpha)} = (\psi_0 e^{i \alpha})\cdot e^{i(- wt)} $ 是 $t$ 的函数,则此时的相矢量定义为 $ \psi_0 \measuredangle \alpha = \psi_0 e^{i \alpha}$,也常记为 $\psi_0 \angle \alpha$。

相矢量是复平面中的一个矢量(即一个复数),$\psi_0$ 表示其模长,$\alpha$ 表示其幅角,真实的波是它在实轴上的投影。对于 $\psi = \psi_0 e^{i(- wt + \alpha)} $,随着 $t$ 增大,波的相位减小,代表相矢量在复平面中顺时针旋转,$\omega t$ 即为沿顺时针旋转的角度。对于 $\psi = \psi_0 e^{i( wt + \alpha)}$(也即沿 $x$ 轴负方向传播的波),相矢量在复平面中逆时针旋转,$\omega t$ 即为沿逆时针转过的角度。也就是说,将 $x$(以及初相 $\varepsilon$)分离为相矢量后,我们可以方便的研究 $x$ 这一点上,波关于时间 $t$ 的变化情况。

当然,对于波的正弦表示 $\psi = A \sin (kx - wt + \varepsilon)$,也可令  $\alpha = kx + \varepsilon$,得到相矢量 $\psi_0 \measuredangle \alpha = \psi_0 e^{i \alpha}$,只不过此时真实的波是它在虚轴上的投影。

例如,振动 $E_1 = 5 \cos (-\omega t)$,$E_2 = 10 \sin (\omega t + \frac{\pi}{3} )$ 的相矢量分别为 $5 \measuredangle 0$,$10 \measuredangle \frac{\pi}{3} $,前者顺时针旋转,向实轴投影,后者逆时针旋转,向虚轴投影。

\subsection{分离 $t$ 并随 $x$ 旋转}

类似地,考虑谐波 $\psi = \psi_0 e^{i(kx \pm \omega t + \varepsilon)}$。对于任意给定的 $t$,令 $\alpha = \pm \omega t + \varepsilon$,谐波可写为 $\psi = \psi_0 e^{i(kx + \alpha)} = (\psi_0 e^{i \alpha})\cdot e^{i(kx)} $ 是 $x$ 的函数,则此时的相矢量定义为 $ \psi_0 \measuredangle \alpha = \psi_0 e^{i \alpha}$。将 $t$ 分离为相矢量后,我们可以方便的研究 $t$ 这一时刻,波关于位置 $x$ 的变化情况。

习惯上,我们只考虑 $\psi_0 e^{i(kx + \alpha)}$,而不考虑 $\psi_0 e^{i(-kx + \alpha)}$ 的情况,后者可以通过三角变换,等价的改变初相 $\phi_0$ 的值转化为前者。 

例如,对振动 $E_3 = 5 \cos (kx)$,$E_4 = 10 \sin (kx + \frac{\pi}{2} )$,其相矢量分别为 $5 \measuredangle 0$,$10 \measuredangle \frac{\pi}{2} $,两者都逆时针旋转,前者向实轴投影,后者向虚轴投影。


\section{平面波}

我们常用 \footnote{平面波概念的引入详见参考文献 \cite{Optics} 的 Page 30,这里不再赘述}













\chapter*{附录 B\hspace*{20pt}  Matlab 代码}\addcontentsline{toc}{chapter}{附录 B\hspace*{6pt}  Matlab 代码}   
\thispagestyle{fancy} 
\setcounter{section}{0}   
\renewcommand\thesection{B.\arabic{section}}   
\renewcommand{\thefigure}{B.\arabic{figure}} 
\renewcommand{\thetable}{B.\arabic{table}}

\section{图 \ref{振幅系数随入射角的变化} 源码}
\begin{matlablisting}
%%%%%%%%%% 空气入射玻璃 %%%%%%%%%%
global n_i n_t
n_i = 1;
n_t = 1.5;

theta_t = @(theta_i) asin(n_i/n_t*sin(theta_i));
r_s = @(theta_i, theta_t) - sin(theta_i - theta_t)./sin(theta_i + theta_t);
r_p = @(theta_i, theta_t) + tan(theta_i - theta_t)./tan(theta_i + theta_t);
t_s = @(theta_i, theta_t) 2*sin(theta_t).*cos(theta_i)./sin(theta_i + theta_t);
t_p = @(theta_i, theta_t) 2*sin(theta_t).*cos(theta_i) ./ ( sin(theta_i + theta_t).*cos(theta_i - theta_t) );
theta_B = atan(n_t/n_i);
theta_C = asin(n_t/n_i);

theta_array = linspace(-0.1, pi/2, 101);
Y = [
    r_s(theta_array, theta_t(theta_array))
    r_p(theta_array, theta_t(theta_array))
    t_s(theta_array, theta_t(theta_array))
    t_p(theta_array, theta_t(theta_array))
    ];
stc = MyPlot(theta_array, Y);
xline(theta_B, 'b')
yline(0)
xlim([0, pi/2])
ylim([-1, 1])
stc.leg.String = ["$r_s$"; "$r_p$"; "$t_s$"; "$t_p$"; "$\theta_i = \theta_B$"];
stc.leg.Interpreter = "latex";
stc.leg.FontSize = 14;
stc.leg.Location = "southwest";
stc.axes.Title.String = '$n_i = 1 < n_t = 1.5$';
stc.axes.Title.Interpreter = "latex";
stc.label.x.String = '$\theta_i$';
stc.label.y.String = '$r$';
stc.plot.plot_3.LineStyle = ":";
stc.plot.plot_3.Color = 'b';
stc.plot.plot_4.LineStyle = ":";
stc.plot.plot_4.Color = [1 0 1];
%MyExport_pdf

%%%%%%%%%% 玻璃入射空气 %%%%%%%%%%
n_i = 1.5;
n_t = 1;

theta_t = @(theta_i) asin(n_i/n_t*sin(theta_i));
r_s = @(theta_i, theta_t) - sin(theta_i - theta_t)./sin(theta_i + theta_t);
r_p = @(theta_i, theta_t) + tan(theta_i - theta_t)./tan(theta_i + theta_t);
t_s = @(theta_i, theta_t) 2*sin(theta_t).*cos(theta_i)./sin(theta_i + theta_t);
t_p = @(theta_i, theta_t) 2*sin(theta_t).*cos(theta_i) ./ ( sin(theta_i + theta_t).*cos(theta_i - theta_t) );
theta_B = atan(n_t/n_i);
theta_C = asin(n_t/n_i);


theta_array = linspace(0, theta_C, 101);
Y = [
    r_s(theta_array, theta_t(theta_array))
    r_p(theta_array, theta_t(theta_array))
    t_s(theta_array, theta_t(theta_array))
    t_p(theta_array, theta_t(theta_array))
    ];
stc = MyPlot(theta_array, Y);
xline(theta_B, 'b')
xline(theta_C, 'r')
yline(0)
xlim([0, pi/2])
ylim([-0.5, 3])
stc.leg.String = ["$r_s$"; "$r_p$"; "$t_s$"; "$t_p$"; "$\theta_i = \theta_B$"; "$\theta_i = \theta_C$"];
stc.leg.Interpreter = "latex";
stc.axes.Title.String = '$n_i = 1.5 > n_t = 1$';
stc.axes.Title.Interpreter = "latex";
stc.label.x.String = '$\theta_i$';
stc.label.y.String = '$r$';
stc.plot.plot_3.LineStyle = ":";
stc.plot.plot_3.Color = 'b';
stc.plot.plot_4.LineStyle = ":";
stc.plot.plot_4.Color = [1 0 1];
%MyExport_pdf
\end{matlablisting}

\section{图 \ref{反射折射光的振幅与能量变化} 源码}
\label{反射折射光的振幅与能量变化 源码}
\begin{matlablisting}
global n_i n_t
%%%%%%%%%% 反射折射光振幅与能量变化 (空气入射玻璃) %%%%%%%%%%
MyColor = num2cell( ...
    [
"#ff8080" "#ff0000" "#990000" "#190000"
"#80ff80" "#00ff00" "#009900" "#001900"
"#8080ff" "#0000ff" "#000099" "#000019"
"#ff80ff" "#ff00ff" "#990099" "#190019"
"#ffff80" "#ffff00" "#999900" "#191900"
"#80ffff" "#00ffff" "#009999" "#001919"
"#ffffff" "#bbbbbb" "#999999" "#191919"
    ]...
);
n_i = 1;
n_t = 1.5;

theta_t = @(theta_i) asin(n_i/n_t*sin(theta_i));
r_s = @(theta_i, theta_t) - sin(theta_i - theta_t)./sin(theta_i + theta_t);
r_p = @(theta_i, theta_t) + tan(theta_i - theta_t)./tan(theta_i + theta_t);
t_s = @(theta_i, theta_t) 2*sin(theta_t).*cos(theta_i)./sin(theta_i + theta_t);
t_p = @(theta_i, theta_t) 2*sin(theta_t).*cos(theta_i) ./ ( sin(theta_i + theta_t).*cos(theta_i - theta_t) );
theta_B = atan(n_t/n_i);
theta_C = asin(n_t/n_i);

theta_array_2 = linspace(-0.1, pi/2, 101);
Y = [
    r_s(theta_array_2, theta_t(theta_array_2))
    r_p(theta_array_2, theta_t(theta_array_2))
    t_s(theta_array_2, theta_t(theta_array_2))
    t_p(theta_array_2, theta_t(theta_array_2))
    r_s(theta_array_2, theta_t(theta_array_2)).^2
    r_p(theta_array_2, theta_t(theta_array_2)).^2
    0.5 * ( r_s(theta_array_2, theta_t(theta_array_2)).^2 + r_p(theta_array_2, theta_t(theta_array_2)).^2 )
    ];

stc = MyPlot(theta_array_2, Y);
yline(0, 'black', 'Alpha', 1, 'LineWidth', 1)
xline(theta_B,'Color', [0 1 1], 'Alpha', 1, 'LineWidth', 0.7)
xlim([0, pi/2])
ylim([-1, 1])
    stc.leg.Interpreter = 'latex';
    stc.leg.FontSize = 15;
    stc.leg.Location = 'southwest';
    stc.axes.Title.String = '$n_i = 1 < n_t = 1.5$';
    stc.axes.Title.Interpreter = 'latex';
    stc.label.x.String = '$\theta_i$';
    stc.label.y.String = '$y$';
    %stc.leg.String = ["$y=r_s$"; "$y=r_p$"; "$y=t_s$"; "$y=t_p$"; "$y=R_s$"; "$y=R_p$"; "$y=R$"; "$y=0$";  "$\theta_i = \theta_B$";];
    stc.leg.Visible = 'off';

    stc.plot.plot_2.LineStyle = "-";
    stc.plot.plot_3.LineStyle = ":";
    stc.plot.plot_4.LineStyle = ":";
    stc.plot.plot_5.LineStyle = "--";
    %stc.plot.plot_5.LineWidth = 0.7;
    stc.plot.plot_6.LineStyle = "--";
    %stc.plot.plot_6.LineWidth = 0.7;
    stc.plot.plot_7.LineStyle = "-";

    stc.plot.plot_1.Color = MyColor{4, 2};
    stc.plot.plot_3.Color = MyColor{4, 1};
    stc.plot.plot_5.Color = MyColor{4, 3};
    stc.plot.plot_2.Color = MyColor{3, 2};
    stc.plot.plot_4.Color = MyColor{3, 1};
    stc.plot.plot_6.Color = MyColor{3, 3}; 
    stc.plot.plot_7.Color = [1 0 0];   
%MyExport_pdf
%MyExport_pdf_docked
%MyExport_svg_docked


%%%%%%%%%% 反射折射光振幅与能量变化 (玻璃入射空气) %%%%%%%%%%
n_i = 1.5;
n_t = 1;

theta_t = @(theta_i) asin(n_i/n_t*sin(theta_i));
r_s = @(theta_i, theta_t) - sin(theta_i - theta_t)./sin(theta_i + theta_t);
r_p = @(theta_i, theta_t) + tan(theta_i - theta_t)./tan(theta_i + theta_t);
t_s = @(theta_i, theta_t) 2*sin(theta_t).*cos(theta_i)./sin(theta_i + theta_t);
t_p = @(theta_i, theta_t) 2*sin(theta_t).*cos(theta_i) ./ ( sin(theta_i + theta_t).*cos(theta_i - theta_t) );
theta_B = atan(n_t/n_i);
theta_C = asin(n_t/n_i);

theta_array_2 = linspace(-0.1, theta_C, 250);
theta_array_all = [linspace(-0.1, 0.65, 100), linspace(0.65, 0.74, 50), linspace(0.74, pi/2, 100)];

X = [
    theta_array_all
    theta_array_all
    theta_array_all
    theta_array_all
    theta_array_all
    theta_array_all
    theta_array_all
];

Y = [
    (theta_array_all < theta_C).*r_s(theta_array_all, theta_t(theta_array_all)) + (theta_array_all > theta_C).*abs(r_s(theta_array_all, theta_t(theta_array_all)))
    (theta_array_all < theta_C).*r_p(theta_array_all, theta_t(theta_array_all)) + (theta_array_all > theta_C).*abs(r_p(theta_array_all, theta_t(theta_array_all)))
    abs( t_s(theta_array_all, theta_t(theta_array_all)) )
    abs( t_p(theta_array_all, theta_t(theta_array_all)) )
    abs(r_s(theta_array_all, theta_t(theta_array_all))).^2
    abs(r_p(theta_array_all, theta_t(theta_array_all))).^2
    0.5 * (  abs(r_s(theta_array_all, theta_t(theta_array_all))).^2 + abs(r_p(theta_array_all, theta_t(theta_array_all))).^2   )
    ];

stc = MyPlot(X, Y);
yline(0, 'black', 'Alpha', 1, 'LineWidth', 1)
xline(theta_B,'Color', [0 1 1], 'Alpha', 1, 'LineWidth', 0.7)
xline(theta_C,'Color', [0 1 0], 'Alpha', 1, 'LineWidth', 0.7)
xlim([0, pi/2])
ylim([-0.5, 3])
    stc.leg.Interpreter = 'latex';
    stc.leg.FontSize = 14;
    stc.leg.Location = 'northwestoutside';
    stc.axes.Title.String = '$n_i = 1 < n_t = 1.5$';
    stc.axes.Title.Interpreter = "latex";
    stc.label.x.String = '$\theta_i$';
    stc.label.y.String = '$y$';
    stc.leg.String = ["$y=r_s$"; "$y=r_p$"; "$y=t_s$"; "$y=t_p$"; "$y=R_s$"; "$y=R_p$"; "$y=R$"; "$y=0$";  "$\theta_i = \theta_B$"; "$\theta_i = \theta_C$";];
    
    stc.plot.plot_2.LineStyle = "-";
    stc.plot.plot_3.LineStyle = ":";
    stc.plot.plot_4.LineStyle = ":";
    stc.plot.plot_5.LineStyle = "--";
    %stc.plot.plot_5.LineWidth = 0.7;
    stc.plot.plot_6.LineStyle = "--";
    %stc.plot.plot_6.LineWidth = 0.7;
    stc.plot.plot_7.LineStyle = "-";

    stc.plot.plot_1.Color = MyColor{4, 2};
    stc.plot.plot_3.Color = MyColor{4, 1};
    stc.plot.plot_5.Color = MyColor{4, 3};
    stc.plot.plot_2.Color = MyColor{3, 2};
    stc.plot.plot_4.Color = MyColor{3, 1};
    stc.plot.plot_6.Color = MyColor{3, 3}; 
    stc.plot.plot_7.Color = [1 0 0];   
%MyExport_pdf
%MyExport_pdf_docked
%MyExport_svg_docked
\end{matlablisting}

\section{图 \ref{反射光 s 分量与 p 分量的相位增量} 源码}
\label{反射光 s 分量与 p 分量的相位增量 源码}
\begin{matlablisting}
global n_i n_t n_ti theta_B theta_C 

%%%%%%%%%% 反射光相位增量 (空气入射玻璃) %%%%%%%%%%
n_i = 1;
n_t = 1.5;
n_ti = n_t/n_i;
theta_B = atan(n_ti);

theta_array_2 = linspace(0, pi/2-0.001, 200);

delta_r_s = @(t) -pi ;
delta_r_p = @(t) (-pi).*(t > theta_B).*( t <pi/2);

delta_r_s_kongqi = delta_r_s(theta_array_2);
delta_r_p_kongqi = delta_r_p(theta_array_2);

Y = [
    zeros(size(theta_array_2)) - pi;
    delta_r_p_kongqi;    
];

stc1 = MyPlot(theta_array_2, Y([1 2], :));
xlim([0, pi/2])
yline(0, 'black', 'Alpha', 1, 'LineWidth', 0.5)
xline(theta_B,'Color', [0 1 1], 'Alpha', 1, 'LineWidth', 0.5)
xline(pi/2, '--');
stc1.plot.plot_2.LineStyle = '--';
stc1.leg.String = ["$\delta = \delta_{r,s}$"; "$\delta = \delta_{r,p}$"; "$\delta = 0$"; "$\theta_i = \theta_B$";];
stc1.label.x.String = '$\theta_i$';
stc1.label.y.String = '$\delta$';
stc1.axes.Title.Interpreter = 'latex';
stc1.axes.Title.String = '$n_i = 1 < n_t = 1.5$';
%MyExport_pdf

%%%%%%%%%% 反射光相位增量 (玻璃入射空气) %%%%%%%%%%
n_i = 1.5;
n_t = 1;
n_ti = n_t/n_i;
theta_B = atan(n_ti);
theta_C = asin(n_ti);


delta_r_s = @(t) (t>theta_C).*2.*atan( -(sqrt(sin(t).^2 - n_ti^2))./cos(t) ) ;
delta_r_p = @(t) ...
     (t<theta_B).*(-pi) ... 
   + (theta_B<t).*(t<theta_C).*0 ... 
   + (theta_C<t).*( -2*atan( (sqrt(sin(t).^2 - n_ti^2))./(n_ti^2.*cos(t)) ) );

Y = [
    zeros(size(theta_array_2)) - pi;
    delta_r_p_kongqi;
    delta_r_s(theta_array_2);
    delta_r_p(theta_array_2);
];


stc2 = MyPlot(theta_array_2, Y([3 4], :));
xlim([0, pi/2])
yline(0, 'black', 'Alpha', 1, 'LineWidth', 0.5)
xline(theta_B,'Color', [0 1 1], 'Alpha', 1, 'LineWidth', 0.5)
xline(theta_C,'Color', [0 1 0], 'Alpha', 1, 'LineWidth', 0.5)
xline(pi/2, '--');
stc2.plot.plot_2.LineStyle = '--';
stc2.leg.String = ["$\delta = \delta_{r,s}$"; "$\delta = \delta_{r,p}$"; "$\delta = 0$"; "$\theta_i = \theta_B$"; "$\theta_i = \theta_C$";];
stc2.leg.Location = 'northeast';
stc2.label.x.String = '$\theta_i$';
stc2.label.y.String = '$\delta$';
stc2.axes.Title.Interpreter = 'latex';
stc2.axes.Title.String = '$n_i = 1.5 > n_t = 1$';
%MyExport_pdf
\end{matlablisting}

\section{图 \ref{隐失波穿透深度与 GH Shift 玻璃入射空气} 源码}
\label{隐失波穿透深度与 GH Shift 玻璃入射空气 源码}
\begin{matlablisting}
%%%%%%%%%% 隐失波的穿透深度和 GH Shift (玻璃入射空气) %%%%%%%%%%
global lambda n_i n_t
n_i = 1.5;
n_t = 1;
lambda = 550 * 10^(-9);     % 550.0 nm 的绿色光
delta = @(t) 1 ./ ( 2*pi*sqrt( sin(t).^2 - n_ti^2 )/lambda );
Delta_x = @(t) 2*delta(t).*tan(t);

theta_array_1 = linspace(theta_C, pi/2, 200);
theta_array_2 = linspace(theta_C, pi/2-0.05, 200);


X = [
    theta_array_1
    theta_array_2
];
Y = [
    delta(theta_array_1)/lambda
    Delta_x(theta_array_2)/lambda
];

stc = MyPlot(X, Y);
xlim([theta_C - 0.05, pi/2+0.02])
yline(1, 'black', 'Alpha', 1, 'LineWidth', 0.5)
xline(theta_C,'Color', [0 1 0], 'Alpha', 1, 'LineWidth', 0.5)
stc.leg.String = ["$y = \delta / \lambda$"; "$y = \Delta x / \lambda$"; "$y = 1$"; "$\theta_i = \theta_C$"; "$\theta_i = \frac{\pi}{2}$"];
stc.label.x.String = '$\theta_i$';
stc.label.y.String = '$y$';
xlim([theta_C - 0.05, pi/2])

%MyExport_pdf_docked
\end{matlablisting}



% --------------------------- 附录 --------------------------- %
% >> ------------------------ 附录 ------------------------ << %

\end{document}



% VScode 常用快捷键:

% F2:                       变量重命名
% Ctrl + Enter:             行中换行
% Alt + up/down:            上下移行
% 鼠标中键 + 移动:           快速多光标
% Shift + Alt + up/down:    上下复制
% Ctrl + left/right:        左右跳单词
% Ctrl + Backspace/Delete:  左右删单词    
% Shift + Delete:           删除此行
% Ctrl + J:                 打开 VScode 下栏(输出栏)
% Ctrl + B:                 打开 VScode 左栏(目录栏)
% Ctrl + `:                 打开 VScode 终端栏
% Ctrl + 0:                 定位文件
% Ctrl + Tab:               切换已打开的文件(切标签)
% Ctrl + Shift + P:         打开全局命令(设置)

% Latex 常用快捷键

% Ctrl + Alt + J:           由代码定位到PDF
% 


% Git提交规范:
% update: Linear Algebra 2 notes
% add: Linear Algebra 2 notes
% import: Linear Algebra 2 notes
% delete: Linear Algebra 2 notes

